\documentclass[acmsmall,screen]{acmart}

\ifPDFTeX
\usepackage[T1]{fontenc}
\usepackage[utf8]{inputenc}
\usepackage[british]{babel}
\else
\usepackage{polyglossia}
\setdefaultlanguage{british}
\fi

\usepackage{bbold}
\newcommand{\hmmax}{0}
\newcommand{\bmmax}{0}
\usepackage{bm}
\usepackage{mathtools}
\input{mathlig}
\usepackage{mathpartir}
\usepackage{xspace}
\usepackage[capitalise,noabbrev]{cleveref}

\usepackage[inline]{enumitem} % for environment enumerate*
%\setlist[enumerate]{mode=unboxed}

\setlength {\marginparwidth }{2cm}
\usepackage[color=yellow,textsize=small]{todonotes}

\usepackage{microtype}

\citestyle{acmauthoryear}
\AtEndPreamble{%
\theoremstyle{acmdefinition}
\newtheorem{remark}[theorem]{Remark}}

% Shamelessly copied from the HoTT book

%%%% MACROS FOR NOTATION %%%%
% Use these for any notation where there are multiple options.

%%% Definitional equality (used infix) %%%
\newcommand{\jdeq}{\equiv}      % An equality judgment
\let\judgeq\jdeq
%\newcommand{\defeq}{\coloneqq}  % An equality currently being defined
\newcommand{\defeq}{\vcentcolon\equiv}  % A judgmental equality currently being defined

%%% Term being defined
\newcommand{\define}[1]{\textbf{#1}}

%%% Vec (for example)

\newcommand{\Vect}{\ensuremath{\mathsf{Vec}}}
\newcommand{\Fin}{\ensuremath{\mathsf{Fin}}}
\newcommand{\fmax}{\ensuremath{\mathsf{fmax}}}
\newcommand{\seq}[1]{\langle #1\rangle}

%%% Dependent products %%%
\def\prdsym{\textstyle\prod}
%% Call the macro like \prd{x,y:A}{p:x=y} with any number of
%% arguments.  Make sure that whatever comes *after* the call doesn't
%% begin with an open-brace, or it will be parsed as another argument.
\makeatletter
% Currently the macro is configured to produce
%     {\textstyle\prod}(x:A) \; {\textstyle\prod}(y:B),{\ }
% in display-math mode, and
%     \prod_{(x:A)} \prod_{y:B}
% in text-math mode.
% \def\prd#1{\@ifnextchar\bgroup{\prd@parens{#1}}{%
%     \@ifnextchar\sm{\prd@parens{#1}\@eatsm}{%
%         \prd@noparens{#1}}}}
\def\prd#1{\@ifnextchar\bgroup{\prd@parens{#1}}{%
    \@ifnextchar\sm{\prd@parens{#1}\@eatsm}{%
    \@ifnextchar\prd{\prd@parens{#1}\@eatprd}{%
    \@ifnextchar\;{\prd@parens{#1}\@eatsemicolonspace}{%
    \@ifnextchar\\{\prd@parens{#1}\@eatlinebreak}{%
    \@ifnextchar\narrowbreak{\prd@parens{#1}\@eatnarrowbreak}{%
      \prd@noparens{#1}}}}}}}}
\def\prd@parens#1{\@ifnextchar\bgroup%
  {\mathchoice{\@dprd{#1}}{\@tprd{#1}}{\@tprd{#1}}{\@tprd{#1}}\prd@parens}%
  {\@ifnextchar\sm%
    {\mathchoice{\@dprd{#1}}{\@tprd{#1}}{\@tprd{#1}}{\@tprd{#1}}\@eatsm}%
    {\mathchoice{\@dprd{#1}}{\@tprd{#1}}{\@tprd{#1}}{\@tprd{#1}}}}}
\def\@eatsm\sm{\sm@parens}
\def\prd@noparens#1{\mathchoice{\@dprd@noparens{#1}}{\@tprd{#1}}{\@tprd{#1}}{\@tprd{#1}}}
% Helper macros for three styles
\def\lprd#1{\@ifnextchar\bgroup{\@lprd{#1}\lprd}{\@@lprd{#1}}}
\def\@lprd#1{\mathchoice{{\textstyle\prod}}{\prod}{\prod}{\prod}({\textstyle #1})\;}
\def\@@lprd#1{\mathchoice{{\textstyle\prod}}{\prod}{\prod}{\prod}({\textstyle #1}),\ }
\def\tprd#1{\@tprd{#1}\@ifnextchar\bgroup{\tprd}{}}
\def\@tprd#1{\mathchoice{{\textstyle\prod_{(#1)}}}{\prod_{(#1)}}{\prod_{(#1)}}{\prod_{(#1)}}}
\def\dprd#1{\@dprd{#1}\@ifnextchar\bgroup{\dprd}{}}
\def\@dprd#1{\prod_{(#1)}\,}
\def\@dprd@noparens#1{\prod_{#1}\,}

% Look through spaces and linebreaks
\def\@eatnarrowbreak\narrowbreak{%
  \@ifnextchar\prd{\narrowbreak\@eatprd}{%
    \@ifnextchar\sm{\narrowbreak\@eatsm}{%
      \narrowbreak}}}
\def\@eatlinebreak\\{%
  \@ifnextchar\prd{\\\@eatprd}{%
    \@ifnextchar\sm{\\\@eatsm}{%
      \\}}}
\def\@eatsemicolonspace\;{%
  \@ifnextchar\prd{\;\@eatprd}{%
    \@ifnextchar\sm{\;\@eatsm}{%
      \;}}}

%%% Lambda abstractions.
% Each variable being abstracted over is a separate argument.  If
% there is more than one such argument, they *must* be enclosed in
% braces.  Arguments can be untyped, as in \lam{x}{y}, or typed with a
% colon, as in \lam{x:A}{y:B}. In the latter case, the colons are
% automatically noticed and (with current implementation) the space
% around the colon is reduced.  You can even give more than one variable
% the same type, as in \lam{x,y:A}.
\def\lam#1{{\lambda}\@lamarg#1:\@endlamarg\@ifnextchar\bgroup{.\,\lam}{.\,}}
\def\@lamarg#1:#2\@endlamarg{\if\relax\detokenize{#2}\relax #1\else\@lamvar{\@lameatcolon#2},#1\@endlamvar\fi}
\def\@lamvar#1,#2\@endlamvar{(#2\,{:}\,#1)}
% \def\@lamvar#1,#2{{#2}^{#1}\@ifnextchar,{.\,{\lambda}\@lamvar{#1}}{\let\@endlamvar\relax}}
\def\@lameatcolon#1:{#1}
\let\lamt\lam
% This version silently eats any typing annotation.
\def\lamu#1{{\lambda}\@lamuarg#1:\@endlamuarg\@ifnextchar\bgroup{.\,\lamu}{.\,}}
\def\@lamuarg#1:#2\@endlamuarg{#1}

%%% Dependent products written with \forall, in the same style
\def\fall#1{\forall (#1)\@ifnextchar\bgroup{.\,\fall}{.\,}}

%%% Existential quantifier %%%
\def\exis#1{\exists (#1)\@ifnextchar\bgroup{.\,\exis}{.\,}}

%%% Dependent sums %%%
\def\smsym{\textstyle\sum}
% Use in the same way as \prd
\def\sm#1{\@ifnextchar\bgroup{\sm@parens{#1}}{%
    \@ifnextchar\prd{\sm@parens{#1}\@eatprd}{%
    \@ifnextchar\sm{\sm@parens{#1}\@eatsm}{%
    \@ifnextchar\;{\sm@parens{#1}\@eatsemicolonspace}{%
    \@ifnextchar\\{\sm@parens{#1}\@eatlinebreak}{%
    \@ifnextchar\narrowbreak{\sm@parens{#1}\@eatnarrowbreak}{%
        \sm@noparens{#1}}}}}}}}
\def\sm@parens#1{\@ifnextchar\bgroup%
  {\mathchoice{\@dsm{#1}}{\@tsm{#1}}{\@tsm{#1}}{\@tsm{#1}}\sm@parens}%
  {\@ifnextchar\prd%
    {\mathchoice{\@dsm{#1}}{\@tsm{#1}}{\@tsm{#1}}{\@tsm{#1}}\@eatprd}%
    {\mathchoice{\@dsm{#1}}{\@tsm{#1}}{\@tsm{#1}}{\@tsm{#1}}}}}
\def\@eatprd\prd{\prd@parens}
\def\sm@noparens#1{\mathchoice{\@dsm@noparens{#1}}{\@tsm{#1}}{\@tsm{#1}}{\@tsm{#1}}}
\def\lsm#1{\@ifnextchar\bgroup{\@lsm{#1}\lsm}{\@@lsm{#1}}}
\def\@lsm#1{\mathchoice{{\textstyle\sum}}{\sum}{\sum}{\sum}({\textstyle #1})\;}
\def\@@lsm#1{\mathchoice{{\textstyle\sum}}{\sum}{\sum}{\sum}({\textstyle #1}),\ }
\def\tsm#1{\@tsm{#1}\@ifnextchar\bgroup{\tsm}{}}
\def\@tsm#1{\mathchoice{{\textstyle\sum_{(#1)}}}{\sum_{(#1)}}{\sum_{(#1)}}{\sum_{(#1)}}}
\def\dsm#1{\@dsm{#1}\@ifnextchar\bgroup{\dsm}{}}
\def\@dsm#1{\sum_{(#1)}\,}
\def\@dsm@noparens#1{\sum_{#1}\,}

%%% W-types
\def\wtypesym{{\mathsf{W}}}
\def\wtype#1{\@ifnextchar\bgroup%
  {\mathchoice{\@twtype{#1}}{\@twtype{#1}}{\@twtype{#1}}{\@twtype{#1}}\wtype}%
  {\mathchoice{\@twtype{#1}}{\@twtype{#1}}{\@twtype{#1}}{\@twtype{#1}}}}
\def\lwtype#1{\@ifnextchar\bgroup{\@lwtype{#1}\lwtype}{\@@lwtype{#1}}}
\def\@lwtype#1{\mathchoice{{\textstyle\mathsf{W}}}{\mathsf{W}}{\mathsf{W}}{\mathsf{W}}({\textstyle #1})\;}
\def\@@lwtype#1{\mathchoice{{\textstyle\mathsf{W}}}{\mathsf{W}}{\mathsf{W}}{\mathsf{W}}({\textstyle #1}),\ }
\def\twtype#1{\@twtype{#1}\@ifnextchar\bgroup{\twtype}{}}
\def\@twtype#1{\mathchoice{{\textstyle\mathsf{W}_{(#1)}}}{\mathsf{W}_{(#1)}}{\mathsf{W}_{(#1)}}{\mathsf{W}_{(#1)}}}
\def\dwtype#1{\@dwtype{#1}\@ifnextchar\bgroup{\dwtype}{}}
\def\@dwtype#1{\mathsf{W}_{(#1)}\,}

\newcommand{\suppsym}{{\mathsf{sup}}}
\newcommand{\supp}{\ensuremath\suppsym\xspace}

\def\wtypeh#1{\@ifnextchar\bgroup%
  {\mathchoice{\@lwtypeh{#1}}{\@twtypeh{#1}}{\@twtypeh{#1}}{\@twtypeh{#1}}\wtypeh}%
  {\mathchoice{\@@lwtypeh{#1}}{\@twtypeh{#1}}{\@twtypeh{#1}}{\@twtypeh{#1}}}}
\def\lwtypeh#1{\@ifnextchar\bgroup{\@lwtypeh{#1}\lwtypeh}{\@@lwtypeh{#1}}}
\def\@lwtypeh#1{\mathchoice{{\textstyle\mathsf{W}^h}}{\mathsf{W}^h}{\mathsf{W}^h}{\mathsf{W}^h}({\textstyle #1})\;}
\def\@@lwtypeh#1{\mathchoice{{\textstyle\mathsf{W}^h}}{\mathsf{W}^h}{\mathsf{W}^h}{\mathsf{W}^h}({\textstyle #1}),\ }
\def\twtypeh#1{\@twtypeh{#1}\@ifnextchar\bgroup{\twtypeh}{}}
\def\@twtypeh#1{\mathchoice{{\textstyle\mathsf{W}^h_{(#1)}}}{\mathsf{W}^h_{(#1)}}{\mathsf{W}^h_{(#1)}}{\mathsf{W}^h_{(#1)}}}
\def\dwtypeh#1{\@dwtypeh{#1}\@ifnextchar\bgroup{\dwtypeh}{}}
\def\@dwtypeh#1{\mathsf{W}^h_{(#1)}\,}


\makeatother

% Other notations related to dependent sums
\let\setof\Set    % from package 'braket', write \setof{ x:A | P(x) }.
\newcommand{\pair}{\ensuremath{\mathsf{pair}}\xspace}
\newcommand{\tup}[2]{(#1,#2)}
\newcommand{\proj}[1]{\ensuremath{\mathsf{pr}_{#1}}\xspace}
\newcommand{\fst}{\ensuremath{\proj1}\xspace}
\newcommand{\snd}{\ensuremath{\proj2}\xspace}
\newcommand{\ac}{\ensuremath{\mathsf{ac}}\xspace} % not needed in symbol index

%%% recursor and induction
\newcommand{\rec}[1]{\mathsf{rec}_{#1}}
\newcommand{\ind}[1]{\mathsf{ind}_{#1}}
\newcommand{\indid}[1]{\ind{=_{#1}}} % (Martin-Lof) path induction principle for identity types
\newcommand{\indidb}[1]{\ind{=_{#1}}'} % (Paulin-Mohring) based path induction principle for identity types

%%% Uniqueness principles
\newcommand{\uniq}[1]{\mathsf{uniq}_{#1}}

% Paths in pairs
\newcommand{\pairpath}{\ensuremath{\mathsf{pair}^{\mathord{=}}}\xspace}
% \newcommand{\projpath}[1]{\proj{#1}^{\mathord{=}}}
\newcommand{\projpath}[1]{\ensuremath{\apfunc{\proj{#1}}}\xspace}
\newcommand{\pairct}{\ensuremath{\mathsf{pair}^{\mathord{\ct}}}\xspace}

%%% For quotients %%%
%\newcommand{\pairr}[1]{{\langle #1\rangle}}
\newcommand{\pairr}[1]{{\mathopen{}(#1)\mathclose{}}}
\newcommand{\Pairr}[1]{{\mathopen{}\left(#1\right)\mathclose{}}}

\newcommand{\im}{\ensuremath{\mathsf{im}}} % the image

%%% 2D path operations
\newcommand{\leftwhisker}{\mathbin{{\ct}_{\mathsf{l}}}}  % was \ell
\newcommand{\rightwhisker}{\mathbin{{\ct}_{\mathsf{r}}}} % was r
\newcommand{\hct}{\star}

%%% Identity types %%%
\newcommand{\idsym}{{=}}
\newcommand{\id}[3][]{\ensuremath{#2 =_{#1} #3}\xspace}
\newcommand{\idtype}[3][]{\ensuremath{\mathsf{Id}_{#1}(#2,#3)}\xspace}
\newcommand{\idtypevar}[1]{\ensuremath{\mathsf{Id}_{#1}}\xspace}
% A propositional equality currently being defined
\newcommand{\defid}{\coloneqq}

%%% Dependent paths
\newcommand{\dpath}[4]{#3 =^{#1}_{#2} #4}

%%% singleton
% \newcommand{\sgl}{\ensuremath{\mathsf{sgl}}\xspace}
% \newcommand{\sctr}{\ensuremath{\mathsf{sctr}}\xspace}

%%% Reflexivity terms %%%
% \newcommand{\reflsym}{{\mathsf{refl}}}
\newcommand{\refl}[1]{\ensuremath{\mathsf{refl}_{#1}}\xspace}

%%% Path concatenation (used infix, in diagrammatic order) %%%
\newcommand{\ct}{%
  \mathchoice{\mathbin{\raisebox{0.5ex}{$\displaystyle\centerdot$}}}%
             {\mathbin{\raisebox{0.5ex}{$\centerdot$}}}%
             {\mathbin{\raisebox{0.25ex}{$\scriptstyle\,\centerdot\,$}}}%
             {\mathbin{\raisebox{0.1ex}{$\scriptscriptstyle\,\centerdot\,$}}}
}

%%% Path reversal %%%
\newcommand{\opp}[1]{\mathord{{#1}^{-1}}}
\let\rev\opp

%%% Coherence paths %%%
\newcommand{\ctassoc}{\mathsf{assoc}} % associativity law

%%% Transport (covariant) %%%
\newcommand{\trans}[2]{\ensuremath{{#1}_{*}\mathopen{}\left({#2}\right)\mathclose{}}\xspace}
\let\Trans\trans
%\newcommand{\Trans}[2]{\ensuremath{{#1}_{*}\left({#2}\right)}\xspace}
\newcommand{\transf}[1]{\ensuremath{{#1}_{*}}\xspace} % Without argument
%\newcommand{\transport}[2]{\ensuremath{\mathsf{transport}_{*} \: {#2}\xspace}}
\newcommand{\transfib}[3]{\ensuremath{\mathsf{transport}^{#1}(#2,#3)\xspace}}
\newcommand{\Transfib}[3]{\ensuremath{\mathsf{transport}^{#1}\Big(#2,\, #3\Big)\xspace}}
\newcommand{\transfibf}[1]{\ensuremath{\mathsf{transport}^{#1}\xspace}}

%%% 2D transport
\newcommand{\transtwo}[2]{\ensuremath{\mathsf{transport}^2\mathopen{}\left({#1},{#2}\right)\mathclose{}}\xspace}

%%% Constant transport
\newcommand{\transconst}[3]{\ensuremath{\mathsf{transportconst}}^{#1}_{#2}(#3)\xspace}
\newcommand{\transconstf}{\ensuremath{\mathsf{transportconst}}\xspace}

%%% Map on paths %%%
\newcommand{\mapfunc}[1]{\ensuremath{\mathsf{ap}_{#1}}\xspace} % Without argument
\newcommand{\map}[2]{\ensuremath{{#1}\mathopen{}\left({#2}\right)\mathclose{}}\xspace}
\let\Ap\map
%\newcommand{\Ap}[2]{\ensuremath{{#1}\left({#2}\right)}\xspace}
\newcommand{\mapdepfunc}[1]{\ensuremath{\mathsf{apd}_{#1}}\xspace} % Without argument
% \newcommand{\mapdep}[2]{\ensuremath{{#1}\llparenthesis{#2}\rrparenthesis}\xspace}
\newcommand{\mapdep}[2]{\ensuremath{\mapdepfunc{#1}\mathopen{}\left(#2\right)\mathclose{}}\xspace}
\let\apfunc\mapfunc
\let\ap\map
\let\apdfunc\mapdepfunc
\let\apd\mapdep

%%% 2D map on paths
\newcommand{\aptwofunc}[1]{\ensuremath{\mathsf{ap}^2_{#1}}\xspace}
\newcommand{\aptwo}[2]{\ensuremath{\aptwofunc{#1}\mathopen{}\left({#2}\right)\mathclose{}}\xspace}
\newcommand{\apdtwofunc}[1]{\ensuremath{\mathsf{apd}^2_{#1}}\xspace}
\newcommand{\apdtwo}[2]{\ensuremath{\apdtwofunc{#1}\mathopen{}\left(#2\right)\mathclose{}}\xspace}

%%% Identity functions %%%
\newcommand{\idfunc}[1][]{\ensuremath{\mathsf{id}_{#1}}\xspace}

%%% Homotopies (written infix) %%%
\newcommand{\htpy}{\sim}

%%% Other meanings of \sim
\newcommand{\bisim}{\sim}       % bisimulation
\newcommand{\eqr}{\sim}         % an equivalence relation

%%% Equivalence types %%%
\newcommand{\eqv}[2]{\ensuremath{#1 \simeq #2}\xspace}
\newcommand{\eqvspaced}[2]{\ensuremath{#1 \;\simeq\; #2}\xspace}
\newcommand{\eqvsym}{\simeq}    % infix symbol
\newcommand{\texteqv}[2]{\ensuremath{\mathsf{Equiv}(#1,#2)}\xspace}
\newcommand{\isequiv}{\ensuremath{\mathsf{isequiv}}}
\newcommand{\qinv}{\ensuremath{\mathsf{qinv}}}
\newcommand{\ishae}{\ensuremath{\mathsf{ishae}}}
\newcommand{\linv}{\ensuremath{\mathsf{linv}}}
\newcommand{\rinv}{\ensuremath{\mathsf{rinv}}}
\newcommand{\biinv}{\ensuremath{\mathsf{biinv}}}
\newcommand{\lcoh}[3]{\mathsf{lcoh}_{#1}(#2,#3)}
\newcommand{\rcoh}[3]{\mathsf{rcoh}_{#1}(#2,#3)}
\newcommand{\hfib}[2]{{\mathsf{fib}}_{#1}(#2)}

%%% Map on total spaces %%%
\newcommand{\total}[1]{\ensuremath{\mathsf{total}(#1)}}

%%% Universe types %%%
%\newcommand{\type}{\ensuremath{\mathsf{Type}}\xspace}
\newcommand{\UU}{\ensuremath{\mathcal{U}}\xspace}
% Universes of truncated types
\newcommand{\typele}[1]{\ensuremath{{#1}\text-\mathsf{Type}}\xspace}
\newcommand{\typeleU}[1]{\ensuremath{{#1}\text-\mathsf{Type}_\UU}\xspace}
\newcommand{\typelep}[1]{\ensuremath{{(#1)}\text-\mathsf{Type}}\xspace}
\newcommand{\typelepU}[1]{\ensuremath{{(#1)}\text-\mathsf{Type}_\UU}\xspace}
\let\ntype\typele
\let\ntypeU\typeleU
\let\ntypep\typelep
\let\ntypepU\typelepU
\renewcommand{\set}{\ensuremath{\mathsf{Set}}\xspace}
\newcommand{\setU}{\ensuremath{\mathsf{Set}_\UU}\xspace}
\newcommand{\prop}{\ensuremath{\mathsf{Prop}}\xspace}
\newcommand{\propU}{\ensuremath{\mathsf{Prop}_\UU}\xspace}
%Pointed types
\newcommand{\pointed}[1]{\ensuremath{#1_\bullet}}

%%% Ordinals and cardinals
\newcommand{\card}{\ensuremath{\mathsf{Card}}\xspace}
\newcommand{\ord}{\ensuremath{\mathsf{Ord}}\xspace}
\newcommand{\ordsl}[2]{{#1}_{/#2}}

%%% Univalence
\newcommand{\ua}{\ensuremath{\mathsf{ua}}\xspace} % the inverse of idtoeqv
\newcommand{\idtoeqv}{\ensuremath{\mathsf{idtoeqv}}\xspace}
\newcommand{\univalence}{\ensuremath{\mathsf{univalence}}\xspace} % the full axiom

%%% Truncation levels
\newcommand{\iscontr}{\ensuremath{\mathsf{isContr}}}
\newcommand{\contr}{\ensuremath{\mathsf{contr}}} % The path to the center of contraction
\newcommand{\isset}{\ensuremath{\mathsf{isSet}}}
\newcommand{\isprop}{\ensuremath{\mathsf{isProp}}}
% h-propositions
% \newcommand{\anhprop}{a mere proposition\xspace}
% \newcommand{\hprops}{mere propositions\xspace}

%%% Homotopy fibers %%%
%\newcommand{\hfiber}[2]{\ensuremath{\mathsf{hFiber}(#1,#2)}\xspace}
\let\hfiber\hfib

%%% Bracket/squash/truncation types %%%
% \newcommand{\brck}[1]{\textsf{mere}(#1)}
% \newcommand{\Brck}[1]{\textsf{mere}\Big(#1\Big)}
% \newcommand{\trunc}[2]{\tau_{#1}(#2)}
% \newcommand{\Trunc}[2]{\tau_{#1}\Big(#2\Big)}
% \newcommand{\truncf}[1]{\tau_{#1}}
%\newcommand{\trunc}[2]{\Vert #2\Vert_{#1}}
\newcommand{\trunc}[2]{\mathopen{}\left\Vert #2\right\Vert_{#1}\mathclose{}}
\newcommand{\ttrunc}[2]{\bigl\Vert #2\bigr\Vert_{#1}}
\newcommand{\Trunc}[2]{\Bigl\Vert #2\Bigr\Vert_{#1}}
\newcommand{\truncf}[1]{\Vert \blank \Vert_{#1}}
\newcommand{\tproj}[3][]{\mathopen{}\left|#3\right|_{#2}^{#1}\mathclose{}}
\newcommand{\tprojf}[2][]{|\blank|_{#2}^{#1}}
\def\pizero{\trunc0}
%\newcommand{\brck}[1]{\trunc{-1}{#1}}
%\newcommand{\Brck}[1]{\Trunc{-1}{#1}}
%\newcommand{\bproj}[1]{\tproj{-1}{#1}}
%\newcommand{\bprojf}{\tprojf{-1}}

\newcommand{\brck}[1]{\trunc{}{#1}}
\newcommand{\bbrck}[1]{\ttrunc{}{#1}}
\newcommand{\Brck}[1]{\Trunc{}{#1}}
\newcommand{\bproj}[1]{\tproj{}{#1}}
\newcommand{\bprojf}{\tprojf{}}

% Big parentheses
\newcommand{\Parens}[1]{\Bigl(#1\Bigr)}

% Projection and extension for truncations
\let\extendsmb\ext
\newcommand{\extend}[1]{\extendsmb(#1)}

%
%%% The empty type
\newcommand{\emptyt}{\ensuremath{\mathbf{0}}\xspace}

%%% The unit type
\newcommand{\unit}{\ensuremath{\mathbf{1}}\xspace}
\newcommand{\ttt}{\ensuremath{\star}\xspace}

%%% The two-element type
\newcommand{\bool}{\ensuremath{\mathbf{2}}\xspace}
\newcommand{\btrue}{{1_{\bool}}}
\newcommand{\bfalse}{{0_{\bool}}}

%%% Injections into binary sums and pushouts
\newcommand{\inlsym}{{\mathsf{inl}}}
\newcommand{\inrsym}{{\mathsf{inr}}}
\newcommand{\inl}{\ensuremath\inlsym\xspace}
\newcommand{\inr}{\ensuremath\inrsym\xspace}

%%% The segment of the interval
\newcommand{\seg}{\ensuremath{\mathsf{seg}}\xspace}

%%% Free groups
\newcommand{\freegroup}[1]{F(#1)}
\newcommand{\freegroupx}[1]{F'(#1)} % the "other" free group

%%% Glue of a pushout
\newcommand{\glue}{\mathsf{glue}}

%%% Colimits
\newcommand{\colim}{\mathsf{colim}}
\newcommand{\inc}{\mathsf{inc}}
\newcommand{\cmp}{\mathsf{cmp}}

%%% Circles and spheres
\newcommand{\Sn}{\mathbb{S}}
\newcommand{\base}{\ensuremath{\mathsf{base}}\xspace}
\newcommand{\lloop}{\ensuremath{\mathsf{loop}}\xspace}
\newcommand{\surf}{\ensuremath{\mathsf{surf}}\xspace}

%%% Suspension
\newcommand{\susp}{\Sigma}
\newcommand{\north}{\mathsf{N}}
\newcommand{\south}{\mathsf{S}}
\newcommand{\merid}{\mathsf{merid}}

%%% Blanks (shorthand for lambda abstractions)
\newcommand{\blank}{\mathord{\hspace{1pt}\text{--}\hspace{1pt}}}

%%% Nameless objects
\newcommand{\nameless}{\mathord{\hspace{1pt}\underline{\hspace{1ex}}\hspace{1pt}}}

%%% Some decorations
%\newcommand{\bbU}{\ensuremath{\mathbb{U}}\xspace}
% \newcommand{\bbB}{\ensuremath{\mathbb{B}}\xspace}
\newcommand{\bbP}{\ensuremath{\mathbb{P}}\xspace}

%%% Some categories
\newcommand{\uset}{\ensuremath{\mathcal{S}et}\xspace}
\newcommand{\ucat}{\ensuremath{{\mathcal{C}at}}\xspace}
\newcommand{\urel}{\ensuremath{\mathcal{R}el}\xspace}
\newcommand{\uhilb}{\ensuremath{\mathcal{H}ilb}\xspace}
\newcommand{\utype}{\ensuremath{\mathcal{T}\!ype}\xspace}

% Pullback corner
\newbox\pbbox
\setbox\pbbox=\hbox{\xy \POS(65,0)\ar@{-} (0,0) \ar@{-} (65,65)\endxy}
\def\pb{\save[]+<3.5mm,-3.5mm>*{\copy\pbbox} \restore}

% Macros for the categories chapter
\newcommand{\inv}[1]{{#1}^{-1}}
\newcommand{\idtoiso}{\ensuremath{\mathsf{idtoiso}}\xspace}
\newcommand{\isotoid}{\ensuremath{\mathsf{isotoid}}\xspace}
\newcommand{\op}{^{\mathrm{op}}}
\newcommand{\y}{\ensuremath{\mathbf{y}}\xspace}
\newcommand{\dgr}[1]{{#1}^{\dagger}}
\newcommand{\unitaryiso}{\mathrel{\cong^\dagger}}
\newcommand{\cteqv}[2]{\ensuremath{#1 \simeq #2}\xspace}
\newcommand{\cteqvsym}{\simeq}     % Symbol for equivalence of categories

%%% Natural numbers
\newcommand{\N}{\ensuremath{\mathbb{N}}\xspace}
%\newcommand{\N}{\textbf{N}}
\let\nat\N
\newcommand{\natp}{\ensuremath{\nat'}\xspace} % alternative nat in induction chapter

\newcommand{\zerop}{\ensuremath{0'}\xspace}   % alternative zero in induction chapter
\newcommand{\suc}{\mathsf{succ}}
\newcommand{\sucp}{\ensuremath{\suc'}\xspace} % alternative suc in induction chapter
\newcommand{\add}{\mathsf{add}}
\newcommand{\ack}{\mathsf{ack}}
\newcommand{\ite}{\mathsf{iter}}
\newcommand{\assoc}{\mathsf{assoc}}
\newcommand{\dbl}{\ensuremath{\mathsf{double}}}
\newcommand{\dblp}{\ensuremath{\dbl'}\xspace} % alternative double in induction chapter


%%% Lists
\newcommand{\lst}[1]{\mathsf{List}(#1)}
\newcommand{\nil}{\mathsf{nil}}
\newcommand{\cons}{\mathsf{cons}}
\newcommand{\lost}[1]{\mathsf{Lost}(#1)}

%%% Vectors of given length, used in induction chapter
\newcommand{\vect}[2]{\ensuremath{\mathsf{Vec}_{#1}(#2)}\xspace}

%%% Integers
\newcommand{\Z}{\ensuremath{\mathbb{Z}}\xspace}
\newcommand{\Zsuc}{\mathsf{succ}}
\newcommand{\Zpred}{\mathsf{pred}}

%%% Rationals
\newcommand{\Q}{\ensuremath{\mathbb{Q}}\xspace}

%%% Function extensionality
\newcommand{\funext}{\mathsf{funext}}
\newcommand{\happly}{\mathsf{happly}}

%%% A naturality lemma
\newcommand{\com}[3]{\mathsf{swap}_{#1,#2}(#3)}

%%% Code/encode/decode
\newcommand{\code}{\ensuremath{\mathsf{code}}\xspace}
\newcommand{\encode}{\ensuremath{\mathsf{encode}}\xspace}
\newcommand{\decode}{\ensuremath{\mathsf{decode}}\xspace}

% Function definition with domain and codomain
\newcommand{\function}[4]{\left\{\begin{array}{rcl}#1 &
      \longrightarrow & #2 \\ #3 & \longmapsto & #4 \end{array}\right.}

%%% Cones and cocones
\newcommand{\cone}[2]{\mathsf{cone}_{#1}(#2)}
\newcommand{\cocone}[2]{\mathsf{cocone}_{#1}(#2)}
% Apply a function to a cocone
\newcommand{\composecocone}[2]{#1\circ#2}
\newcommand{\composecone}[2]{#2\circ#1}
%%% Diagrams
\newcommand{\Ddiag}{\mathscr{D}}

%%% (pointed) mapping spaces
\newcommand{\Map}{\mathsf{Map}}

%%% The interval
\newcommand{\interval}{\ensuremath{I}\xspace}
\newcommand{\izero}{\ensuremath{0_{\interval}}\xspace}
\newcommand{\ione}{\ensuremath{1_{\interval}}\xspace}

%%% Arrows
\newcommand{\epi}{\ensuremath{\twoheadrightarrow}}
\newcommand{\mono}{\ensuremath{\rightarrowtail}}

%%% Sets
\newcommand{\bin}{\ensuremath{\mathrel{\widetilde{\in}}}}

%%% Semigroup structure
\newcommand{\semigroupstrsym}{\ensuremath{\mathsf{SemigroupStr}}}
\newcommand{\semigroupstr}[1]{\ensuremath{\mathsf{SemigroupStr}}(#1)}
\newcommand{\semigroup}[0]{\ensuremath{\mathsf{Semigroup}}}

%%% Macros for the formal type theory
\newcommand{\emptyctx}{\ensuremath{\cdot}}
\newcommand{\production}{\vcentcolon\vcentcolon=}
\newcommand{\conv}{\downarrow}
\newcommand{\ctx}{\ensuremath{\mathsf{ctx}}}
\newcommand{\wfctx}[1]{#1\ \ctx}
\newcommand{\oftp}[3]{#1 \vdash #2 : #3}
\newcommand{\jdeqtp}[4]{#1 \vdash #2 \jdeq #3 : #4}
\newcommand{\judg}[2]{#1 \vdash #2}
\newcommand{\tmtp}[2]{#1 \mathord{:} #2}

% rule names
\newcommand{\rform}{\textsc{form}}
\newcommand{\rintro}{\textsc{intro}}
\newcommand{\relim}{\textsc{elim}}
\newcommand{\rcomp}{\textsc{comp}}
\newcommand{\runiq}{\textsc{uniq}}
\newcommand{\Weak}{\mathsf{Wkg}}
\newcommand{\Vble}{\mathsf{Vble}}
\newcommand{\Exch}{\mathsf{Exch}}
\newcommand{\Subst}{\mathsf{Subst}}

%%% Macros for HITs
\newcommand{\cc}{\mathsf{c}}
\newcommand{\pp}{\mathsf{p}}
\newcommand{\cct}{\widetilde{\mathsf{c}}}
\newcommand{\ppt}{\widetilde{\mathsf{p}}}
\newcommand{\Wtil}{\ensuremath{\widetilde{W}}\xspace}

%%% Macros for n-types
\newcommand{\istype}[1]{\mathsf{is}\mbox{-}{#1}\mbox{-}\mathsf{type}}
\newcommand{\nplusone}{\ensuremath{(n+1)}}
\newcommand{\nminusone}{\ensuremath{(n-1)}}
\newcommand{\fact}{\mathsf{fact}}

%%% Macros for homotopy
\newcommand{\kbar}{\overline{k}} % Used in van Kampen's theorem

%%% Macros for induction
\newcommand{\natw}{\ensuremath{\mathbf{N^w}}\xspace}
\newcommand{\zerow}{\ensuremath{0^\mathbf{w}}\xspace}
\newcommand{\sucw}{\ensuremath{\mathsf{succ}^{\mathbf{w}}}\xspace}
\newcommand{\nalg}{\nat\mathsf{Alg}}
\newcommand{\nhom}{\nat\mathsf{Hom}}
\newcommand{\ishinitw}{\mathsf{isHinit}_{\mathsf{W}}}
\newcommand{\ishinitn}{\mathsf{isHinit}_\nat}
\newcommand{\w}{\mathsf{W}}
\newcommand{\walg}{\w\mathsf{Alg}}
\newcommand{\whom}{\w\mathsf{Hom}}

%%% Macros for real numbers
\newcommand{\RC}{\ensuremath{\mathbb{R}_\mathsf{c}}\xspace} % Cauchy
\newcommand{\RD}{\ensuremath{\mathbb{R}_\mathsf{d}}\xspace} % Dedekind
\newcommand{\R}{\ensuremath{\mathbb{R}}\xspace}           % Either
\newcommand{\barRD}{\ensuremath{\bar{\mathbb{R}}_\mathsf{d}}\xspace} % Dedekind completion of Dedekind

\newcommand{\close}[1]{\sim_{#1}} % Relation of closeness
\newcommand{\closesym}{\mathord\sim}
\newcommand{\rclim}{\mathsf{lim}} % HIT constructor for Cauchy reals
\newcommand{\rcrat}{\mathsf{rat}} % Embedding of rationals into Cauchy reals
\newcommand{\rceq}{\mathsf{eq}_{\RC}} % HIT path constructor
\newcommand{\CAP}{\mathcal{C}}    % The type of Cauchy approximations
\newcommand{\Qp}{\Q_{+}}
\newcommand{\apart}{\mathrel{\#}}  % apartness
\newcommand{\dcut}{\mathsf{isCut}}  % Dedekind cut
\newcommand{\cover}{\triangleleft} % inductive cover
\newcommand{\intfam}[3]{(#2, \lam{#1} #3)} % family of rational intervals

% Macros for the Cauchy reals construction
\newcommand{\bsim}{\frown}
\newcommand{\bbsim}{\smile}

\newcommand{\hapx}{\diamondsuit\approx}
\newcommand{\hapname}{\diamondsuit}
\newcommand{\hapxb}{\heartsuit\approx}
\newcommand{\hapbname}{\heartsuit}
\newcommand{\tap}[1]{\bullet\approx_{#1}\triangle}
\newcommand{\tapname}{\triangle}
\newcommand{\tapb}[1]{\bullet\approx_{#1}\square}
\newcommand{\tapbname}{\square}

%%% Macros for set theory
\newcommand{\vset}{\mathsf{set}}  % point constructor for cummulative hierarchy V
\def\cd{\tproj0}
\newcommand{\inj}{\ensuremath{\mathsf{inj}}} % type of injections
\newcommand{\acc}{\ensuremath{\mathsf{acc}}} % accessibility

\newcommand{\atMostOne}{\mathsf{atMostOne}}

\newcommand{\power}[1]{\mathcal{P}(#1)} % power set
\newcommand{\powerp}[1]{\mathcal{P}_+(#1)} % inhabited power set
 %% for type theory notation copied from HoTT Book
%!TEX root = BiSig.tex

% requires bbold package
\DeclareSymbolFont{bbsymbol}{U}{bbold}{m}{n}
\DeclareMathSymbol{\bbcolon}{\mathrel}{bbsymbol}{"3A}

\newcommand{\mathsc}[1]{\textnormal{\textsc{#1}}}

\newcommand{\Agda}{\textsc{Agda}\xspace}
\newcommand{\Coq}{\textsc{Coq}\xspace}
\newcommand{\SystemF}{{System~\textsf{F}}\xspace}
\newcommand{\SystemFsub}{{System~$\mathsf{F}_{\mathsf{<:}}$}\xspace}
\newcommand{\Dedukti}{\textsf{Dedukti}\xspace}
\newcommand{\PCF}{\textsf{PCF}\xspace}
\newcommand{\PoplMark}{\textsc{PoplMark}\xspace}
\newcommand{\AxiomK}{Axiom~\textsf{K}\xspace}
\newcommand{\LF}{\textsf{LF}\xspace}



\newcommand{\arity}{\mathit{ar}}

\newcommand{\xto}[1]{\xrightarrow{#1}}
  
\newcommand{\fv}{\mathit{fv}}
\newcommand{\dom}{\mathit{dom}}

\newcommand{\tmOp}{\mathsf{op}}
\newcommand{\tyOp}{\mathsf{op}}

\mathchardef\mhyphen="2D % hyphen in mathmode

\newcommand{\Type}{\mathsf{Ty}}
\newcommand{\Term}{\mathsf{Tm}}
\newcommand{\Cxt}{\mathsf{Cxt}}
\newcommand{\Mode}{\mathsf{Mode}}

\newcommand{\bto}{\mathbin{\bm{\supset}}}
\newcommand{\btimes}{\mathbin{\bm{\wedge}}}

\newcommand{\simsub}[2]{{#1}\!\left<{#2}\right>}

\newcommand{\Identifier}{\mathsf{Id}}

\newcommand{\annotate}{\bbcolon}
\newcommand{\subsum}{\mathord{\uparrow}}

\newcommand{\Sub}[2]{\mathsf{Sub}_{\Sigma}(#1,#2)}
\newcommand{\PSub}[2]{\mathsf{PSub}_{\Sigma}(#1,#2)}
\newcommand{\Ren}[2]{\mathsf{Ren}(#1, #2)}

\definecolor{dRed}{rgb}{0.45, 0.0, 0.0}
\definecolor{dBlue}{rgb}{0.0, 0.0, 0.65}
\definecolor{dPurple}{rgb}{0.45, 0.0, 0.65}
\definecolor{dDark}{rgb}{0.2, 0.2, 0.2}

\newcommand{\isTerm}[1]{{\textcolor{dDark}{#1}}}
\newcommand{\isType}[1]{#1}
\newcommand{\isCxt}[1]{#1}
\newcommand{\isDir}[1]{\mathrel{#1}}

\newcommand{\dir}[1]{{\color{dPurple}{#1}}}
\newcommand{\chk}{\mathrel{\color{dBlue}{\Leftarrow}}}
\newcommand{\syn}{\mathrel{\color{dRed}{\Rightarrow}}}
\newcommand{\Rule}[1]{\ensuremath{\mathsc{#1}}}
\newcommand{\SynRule}[1]{\ensuremath{\mathsc{#1}^{\syn}}}
\newcommand{\ChkRule}[1]{\ensuremath{\mathsc{#1}^{\chk}}}
\newcommand{\MC}{\mathsf{MC}}
\newcommand{\MCas}{\MC_{\mathit{as}}}

\newcommand{\ext}[1]{\bar{#1}}
\newcommand{\erase}[1]{\left|#1\right|}

\newcommand{\synvar}{\fv^{\syn}}

\newcommand{\abs}{\mathsf{abs}}
\newcommand{\app}{\mathsf{app}}

\mathlig{|-}{\mathrel{\vdash}} 
\mathlig{|/-}{\mathrel{\nvdash}} 
\newcommand{\judgbox}[2]{{\raggedright $\boxed{#1}$ \quad \text{#2}}}


\newcommand{\BTrue}{\ensuremath{\mathbf{T}}}
\newcommand{\BFalse}{\ensuremath{\mathbf{F}}}

% Some shortcuts
\newcommand{\vars}[1]{\vec{\isTerm{#1}}}
\newcommand{\tmOpts}{\isTerm{\tmOp}_o(\vars{x}_\isTerm{1}.\, \isTerm{t_1}; \ldots;\vars{x}_\isTerm{n}.\, \isTerm{t_n})}
\newcommand{\tmOpus}{\isTerm{\tmOp}_o(\vars{x}_\isTerm{1}.\, \isTerm{u_1}; \ldots;\vars{x}_\isTerm{n}.\, \isTerm{u_n})}


\newcommand{\aritysymbol}[4]{#1\colon #2 \mathrel{\rhd} #3 \to #4}
\newcommand{\formatarg}[2]{[#1]#2}
\newcommand{\formatbiarg}[3]{\formatarg{#1}{#2}^{\dir{#3}}}

\newcommand{\barg}[1][]{\formatarg{\Delta_{#1}}{A_{#1}}}
\newcommand{\bargs}{\barg[1], \ldots, \barg[n]}
\newcommand{\bop}{\aritysymbol{o}{\Xi}{\bargs}{A_0}}

\newcommand{\as}{\mathit{as}}
\newcommand{\chkbiarg}[1][]{\formatbiarg{\Delta_{#1}}{A_{#1}}{\chk}}
\newcommand{\synbiarg}[1][]{\formatbiarg{\Delta_{#1}}{A_{#1}}{\syn}}
\newcommand{\biarg}[1][]{\formatbiarg{\Delta_{#1}}{A_{#1}}{d_{#1}}}
\newcommand{\biargs}{\biarg[1], \ldots, \biarg[n]}
\newcommand{\biarity}{\biargs\to A_0^{\dir{d}}}
\newcommand{\biargvec}{\overrightarrow{\biarg[i]}}
\newcommand{\synop}{\aritysymbol{o}{\Xi}{\biargs}{A_0^{\syn}}}
\newcommand{\chkop}{\aritysymbol{o}{\Xi}{\biargs}{A_0^{\chk}}}
\newcommand{\biop}{\aritysymbol{o}{\Xi}{\biargs}{A_0^{\dir{d}}}}


\newcommand{\LT}[1]{\todo[author=LT,inline,color=orange!40]{{#1}}}
\newcommand{\Josh}[1]{\todo[author=Josh,inline,color=yellow!40]{{#1}}}

\begin{document}

\author{Liang-Ting Chen}
\email{liang.ting.chen.tw@gmail.com}
\orcid{0000-0002-3250-1331}
\author{Hsiang-Shang Ko}
\orcid{0000-0002-2439-1048}
\email{joshko@iis.sinica.edu.tw}
\affiliation{%
  \institution{Academia Sinica}
  \streetaddress{128 Academia Road}
  \city{Taipei}
  \country{Taiwan}
  \postcode{115}
}

\title{A Theory of Bidirectional Type Synthesis for Simple Types}
\begin{abstract}
  Type checking and inference serve as the transition from parsed abstract syntax trees to well-typed programs.
  While progress has been made in developing type-checking techniques on a case-by-case basis, the lack of a general theory poses a difficulty in implementing a robust `type-checker generator' for diverse languages.

  In response to this difficulty, we develop a simple yet general and constructive theory of bidirectional type synthesis where bidirectional type synthesis refers to the mutual use of type checking and inference based on bidirectional typing rules to determine the type of a program with type annotations.
  First, our theory is simple yet general as it deals with any simple type theory and its bidirectional type system that can be specified by a signature, analogous to the role of grammar in parsing.
  Second, our theory is constructive because it is not only formalised in a proof assistant based on constructive type theory but also formulated positively in order to compute.
  Specifically, the `proof' of logical decidability of type synthesis is instantiated to a synthesiser for the simple type theory specified by a valid signature, analogous to a parser generator.
  We introduce the notion of mode preprocessing to refine annotatability in bidirectional typing.
  Type synthesis and mode preprocessing result in the trichotomy of raw terms, which models a practical type synthesiser that may throw an exception and locate missing annotations.
  By formalizing our general theory of bidirectional type synthesis constructively, we provide a correct-by-construction `type-checker generator' for simple type theories, taking the first step towards a more general theory of bidirectional type synthesis.
\end{abstract}

\begin{CCSXML}
<ccs2012>
   <concept>
       <concept_id>10011007.10011006.10011041</concept_id>
       <concept_desc>Software and its engineering~Compilers</concept_desc>
       <concept_significance>500</concept_significance>
       </concept>
   <concept>
       <concept_id>10011007.10011006.10011039.10011040</concept_id>
       <concept_desc>Software and its engineering~Syntax</concept_desc>
       <concept_significance>500</concept_significance>
       </concept>
   <concept>
       <concept_id>10011007.10011074.10011099.10011692</concept_id>
       <concept_desc>Software and its engineering~Formal software verification</concept_desc>
       <concept_significance>300</concept_significance>
       </concept>
   <concept>
       <concept_id>10003752.10003790.10011740</concept_id>
       <concept_desc>Theory of computation~Type theory</concept_desc>
       <concept_significance>500</concept_significance>
       </concept>
 </ccs2012>
\end{CCSXML}

\ccsdesc[500]{Software and its engineering~Compilers}
\ccsdesc[500]{Software and its engineering~Syntax}
\ccsdesc[500]{Theory of computation~Type theory}
\ccsdesc[300]{Software and its engineering~Formal software verification}

\keywords{bidirectional typing, type synthesis, type checking, language formalisation}

\maketitle

%!TEX root = BiSig.tex

\section{Introduction}\label{sec:intro}

Type checking and inference serve as the transition to type-checked programs, also known as \emph{well-typed terms}, from parsed abstract syntax trees, referred as \emph{raw terms} by contrast.
Type inference algorithms were conceived to ascertain the type of any raw term without any type annotations.
However, it was later found that full parametric polymorphism leads to undecidability in type inference, as do dependent types~\citep{Wells1999,Dowek1993}.
In light of these limitations, bidirectional type synthesis emerged as a viable alternative, providing decidable algorithms for determining the types of \emph{suitably annotated} programs in languages with simple types, polymorphic types, dependent types, gradual types, among others. 
\citet{Dunfield2021} summarised its design principles and applications that we know so far. 

The idea of bidirectional type synthesis begins with extending typing judgements with two modes that direct the flow of types:
\begin{enumerate*}
  \item $\Gamma |- \isTerm{t} \syn A$ for synthesis and 
  \item $\Gamma |- \isTerm{t} \chk A$ for checking.
\end{enumerate*}
The former means that the type of a term is synthesised, using both the term and its context as input, while for the latter a term is to be checked against a given type.
Then, a type synthesis algorithm can be `read off' from a bidirectional type system, provided that a conditional called \emph{mode-correctness} --- every synthesised type in a typing rule is determined by previously synthesised types from its premises and its input if checking --- is satisfied for every bidirectional typing rule.


Contrary to the classic Damas--Milner type inference, bidirectional type synthesis does \emph{not} require unification.
\citet{Pierce2000} noted that bidirectional type synthesis propagates annotations locally within adjacent nodes of a syntax tree, so introducing long-distance unification constraints undermines the essence of locality in bidirectional type synthesis.
Further, annotations for, say, top-level definitions improve clarity, making the purpose of a program easier to understand, so they should not be considered redundant.
In light of these considerations, bidirectional type synthesis can be deemed as a technique that is more fundamental than unification and is a design paradigm capable of handling a broad spectrum of programming languages.

Unlike parsing, nevertheless, which has a comprehensive theory and widely applicable practical techniques, bidirectional typing has only been developed on a case-by-case basis without a theory.
While it is straightforward to derive a type synthesis algorithm, a general theory providing logical specifications and rigorously proven properties for a class of systems is still lacking.
As a result, we can only present design principles that we learned from individual systems loosely.
Moreover, unlike the plethora of available parser generators, `type-checker generators' rarely exist, so each type checker, based on bidirectional typing or not, has to be independently built from scratch.

To tackle these challenges, this paper presents a theory of bidirectional typing and a verified implementation of bidirectional type synthesis for simple type theories.
There are many ways and extensions for designing bidirectional typing rules, such as those related to polymorphic types~\citep{Pierce2000,Peyton-Jones2007,Dunfield2013,Xie2018}.
Yet, once the definition of a bidirectional type system is in place, deriving an algorithm becomes straightforward.
Accordingly, our theory's objective is not to formulate rules for various language features, but to introduce a formalism for bidirectional typing, analogous to grammar in parsing.
We introduce the notion of bidirectional binding signatures to specify bidirectional type systems and characterise essential criteria including \emph{soundness}, \emph{completeness}, and \emph{mode-correctness} that are sufficient to derive a type synthesis algorithm, as informally outlined by~\citet{Dunfield2021}.
For simplicity, we confine our discussion to bidirectional simple type systems that have \emph{syntax-directed} typing rules and leave the problem of presenting bidirectional typing for more general and advanced type theories as future work.

\subsection{Related work}\label{sec:related-work}
As our theory is developed formally not only for simple type theories but also in a manner distinct from typical formulations in bidirectional typing, it is pertinent to discuss related work upon which this work is built, apart from bidirectional typing, in order to explain our contributions.

\subsubsection{Language formalisation and its frameworks} \label{sec:language-formalisation}
The vision of formalising the metatheory of every programming language was initiated by the \PoplMark challenge~\citep{Aydemir2005}.
This has been exemplified in the textbook by~\citet{Wadler2022} where language concepts are formally addressed using the proof assistant~\Agda, including bidirectional type synthesis for~\PCF.

\begin{remark}\label{re:type-synthesis-as-decidability-proof}
The bidirectional type synthesis formulated by~\citeauthor{Wadler2022} differs from existing literature:
\begin{enumerate*}
  \item Its algorithm is presented as a proof of \emph{logical decidability} as to whether a `bidirectionally decorated' raw term~$\isTerm{t}$ can be bidirectionally typed, equivalent to a program that returns a typing derivation for $\isTerm{t}$ or otherwise the proof that such derivation is impossible.
  \item In contrast, an algorithm typically found in the literature is presented as \emph{algorithmic system} relations such as $\Gamma |- \isTerm{t} \syn A \mapsto \isTerm{t}'$, denoting that annotations can be added to $\isTerm{t}$ in the external language to produce $\isTerm{t}'$ of type $A$ in the internal language.
    Such an algorithm is then accompanied with \emph{soundness} and \emph{completeness} assertions such that the algorithm correctly synthesises the type of a raw term and every typable term can be synthesised if there are enough annotations.
\end{enumerate*}
\end{remark}

Earlier than the \PoplMark challenge, \citet{Altenkirch1993} commented that rudimentary meta-operations and their meta-properties constitute the bulk of formalisation, motivating a number of frameworks~\citep{Ahrens2018,Fiore2022,Gheri2020,Ahrens2022,Allais2021} to define the type of well-scoped/typed terms, substitution, term traversal and their meta-properties universally.
One of the core gadgets of these frameworks is the concept of \emph{descriptions} or \emph{binding signatures} (coined by~\citet{Aczel1978} in line with the term \emph{signature} in universal algebra) for specifying typing rules of a language.
It is noteworthy that these state-of-the-art frameworks cannot specify polymorphic type theories or dependent type theories yet and only \citeauthor{Allais2021} discussed meta-operations beyond substitution.


\subsubsection{Theories of abstract syntax with variable binding}\label{sec:theory-of-syntax}
The aforementioned frameworks except \citeauthor{Gheri2020}'s are at least inspired by \citet{Fiore1999}'s initial semantics for abstract syntax with variable binding using category theory.
The main idea is that the set of (untyped) abstract syntax trees for a language consists of
\begin{enumerate*}
  \item a family $\Term_{\Gamma}$ of well-scoped terms under the context~$\Gamma$ with
  \item variable renaming for a function $\sigma\colon \Gamma \to \Delta$ between variables acting as a functorial map from $X_{\Gamma}$ to $X_{\Delta}$, thus a presheaf $\Term\colon \mathbb{F} \to \mathsf{Set}$, and
  \item an initial algebra structure given by the variable rule as a map from the embedding $V\colon \mathbb{F} \hookrightarrow \mathsf{Set}$ to $\Term$ and other language constructs as $\mathbb{\Sigma}\Term \to \Term$ where $\mathbb{F}$ is the category of contexts, the functor $\mathbb{\Sigma}\colon \mathsf{Set}^\mathbb{F} \to \mathsf{Set}^\mathbb{F}$ encapsulates language constructs (except variables), and the initiality amounts to structural recursion, i.e.\ \emph{term traversal}.
\end{enumerate*}
To put it succinctly, it is the free $\mathbb{\Sigma}$-algebra over the presheaf of variables $V\colon \mathbb{F} \hookrightarrow \mathsf{Set}$.

Fortunately, constructing the initial algebra of terms in type theory boils down to defining an inductive type with a few constructors that align with the variable rule and a rule scheme for language constructs specified by a signature~\citep{Allais2021,Fiore2022}.

Substitution is also modelled categorically, but it does not play a role in this paper, though.

\begin{remark} \label{re:type-signature}
Most of existing theories treat types independently of terms, thereby excluding them from signatures.
To the best of our knowledge, the only exception to this approach is found in the work of~\citet{Arkor2020}, which incorporates signatures for both terms and types.
Interestingly, this inclusion is also critical for type synthesis for comparing a concrete type $N \bto N$ with an abstract type $A \bto B$, where $A$ and $B$ are type variables in a typing rule.
\end{remark}

\subsubsection{Type Checker Generation}
While there are some efforts to generate type checkers grounded in unification~\citep{Gast2004,Grewe2015}, it should be noted that unification-based approaches are not suited to more complex type theories.
Moreover, their algorithms are not proved complete.

\subsection{Contributions and Plan of this paper}
The major contributions of paper, extending upon previous discussions, are explained as follows.

Our theory can be viewed as a counterpart to those theories in \Cref{sec:theory-of-syntax} and is tailored towards \emph{extrinsically-typed}, \emph{syntax-directed} bidirectional type systems of simple types (\Cref{fig:raw-terms,fig:bidirectional-typing-derivations}) specified by a \emph{signature} for simple types and a \emph{bidirectional binding signature} for terms (\Cref{def:bidirectional-binding-signature} and c.f.\ \Cref{re:type-signature}). 
The extrinsic typing is needed for type synthesis and our adherence to syntax-directed systems aligns with the standard idea of re-casting non-syntax-directed typing rules into their syntax-directed form to derive type synthesis algorithm~\citep{Peyton-Jones2007}.
As substitution is not needed for synthesising simple types, it is not addressed.
Instead, our focus is on:
\begin{enumerate*}
  \item establishing the link between the specified bidirectional type system and its simple type theory, i.e.\ \emph{soundness} and \emph{completeness} (\Cref{sec:soundness-and-completeness}); 
  \item defining the \emph{mode-correctness} condition (\Cref{def:mode-correctness}) which suffices to establish the uniqueness of synthesised types (\Cref{thm:unique-syn}) and a decidable bidirectional type synthesis and checking (\Cref{thm:bidirectional-type-synthesis-checking}).
\end{enumerate*}

Our theory is based on Martin-L\"of type theory and formalised in \Agda (\Cref{sec:formalisation}), akin to those frameworks in \Cref{sec:language-formalisation}.
More importantly, we exploit the computational and logical aspects of type theory to formulate algorithmic soundness, completeness, and decidability. 
Recall that the law of excluded middle does not hold universally as an axiom in Martin-L\"of type theory, so a \emph{decidable statement} $P \vee \neg P$ is non-trivial.
Given that all proofs as programs terminate, logical decidability implies \emph{algorithmic decidability}.
Further, suppose that we have a proof of the following statement:
\begin{quote}
  `For a context $\Gamma$ and a raw term $t$, either a typing derivation of\, $\Gamma |- t : A$ exists for some $A$ or any derivation of\, $\Gamma |- t : A$ for some $A$ leads to a contradiction'
\end{quote}
or rephrased succinctly as 
\begin{quote}
  `A context $\Gamma$ and a raw term $t$ \underline{\emph{decide}} whether $\Gamma |- t : A$ has a derivation for some $A$',
\end{quote}
echoing the algorithm given by \citeauthor{Wadler2022} as noted in \Cref{re:type-synthesis-as-decidability-proof}.
Both algorithmic soundness and completeness are implied:
the proof either yields a typing derivation for the given raw term~$t$ or a proof that such a derivation is impossible where the former case is algorithmic soundness and the latter is algorithmic completeness in contrapositive form.
That is, our decidable bidirectional type synthesis (\Cref{thm:bidirectional-type-synthesis-checking}) provides algorithmic soundness, completeness, and decidability. 

In contrast to \citeauthor{Wadler2022}, our theory starts with raw terms and a general type synthesis problem, independent of bidirectional typing.
This starting point uncovers nuances regarding the lack of enough type annotations in bidirectional typing, often overlooked in literature, and thus distinguishes completeness (with respect to a type assignment system) from annotatability, concepts conflated by \citet{Dunfield2021}. 
To clarify the difference (\Cref{sec:annotatability}), we propose \emph{mode derivations} (\Cref{fig:mode-derivations,fig:generalised-mode-derivations}) and \emph{mode preprocessing} (\Cref{sec:mode-preprocessing}) assigning a mode first to a raw term with enough annotations or otherwise pinpoints missing annotations.
By combining bidirectional type synthesis, soundness, completeness, and mode preprocessing, we achieve our main result (\Cref{cor:trichotomy}) --- a \emph{trichotomy} of raw terms: a type synthesiser which checks if a given raw term is \emph{suitably annotated} and if a suitably annotated raw term has a typing derivation with respect to the specified simple type theory.
Our bidirectional type synthesis can be instantiated for any system specified by a mode-correct signature, effectively a verified type-checker generator. 

In summary, we contribute a theory of bidirectional type synthesis that is:
\begin{enumerate}
  \item \emph{simple yet general} for any system that can be specified by a bidirectional binding signature;
  \item \emph{constructive}, based on Martin-L\"of theory and preferring logical decidability over algorithmic soundness, completeness, and decidability; mode preprocessing over annotatability.
\end{enumerate}
Moreover, all these concepts have been formally developed with \Agda.

This paper is structured in logical order as follows.
We first present an overview of our theory using simply typed $\lambda$-calculus in \Cref{sec:key-ideas}, prior to developing a general framework for specifying bidirectional type systems in \Cref{sec:defs}.
Following this, we discuss the connection between a specified bidirectional type system and its associated simple type theory in \Cref{sec:pre-synthesis}.
In \Cref{sec:type-synthesis}, we introduce mode-correctness and bidirectional type synthesis.
In \Cref{sec:formalisation}, we sketch the formalisation of our theory in \Agda and some examples other than simply typed $\lambda$-calculus.
We conclude in \Cref{sec:future}, where we reflect on related techniques that can be used to improve the formal development and potential challenges in extending our theory to a more general setting.

\todo[inline,caption={}]{
\begin{enumerate}
  \item Introduction (\Cref{sec:intro}) 
    \LT{4 pp}
  \item Key ideas (\Cref{sec:key-ideas})
    \Josh{2.5 pp}
  \item Definitions for bidirectional type systems (bidirectional binding signature, bidirectional type systems, signature erasure and mode annotation) (\Cref{sec:defs})
    \LT{4 pp}
  \item Soundness, completeness, bidirectionalisation, and annotatability (\Cref{sec:pre-synthesis})
    \Josh{Erasure from bidirectional typing derivation to raw terms with a mode}
    \Josh{3 pp}
  \item Bidirectional type inference (\Cref{sec:type-synthesis})
    \LT{4 pp}
  \item Formalisation and further examples (\Cref{sec:formalisation})
    \LT{5.5pp}
  \item Concluding remarks (\Cref{sec:future})
    \LT{1 p} \Josh{1p}
\end{enumerate}
}

%!TEX root = BiSig.tex

\section{Bidirectional Type Synthesis for Simply Typed \texorpdfstring{$\lambda$}{λ}-Calculus} \label{sec:key-ideas}

\begin{figure}
  \small
  \bgroup
  \renewcommand{\arraystretch}{1.5}
  \begin{tabular}{ r l }
    $\boxed{V |- \isTerm{t}}$ & Given a list~$V$ of variables, $t$~is a raw term with free variables in~$V$
  \end{tabular}
  \egroup
  \centering
  \begin{mathpar}
    \inferrule{x \in V}{V \vdash \isTerm{x}}\,\Rule{Var}
    \and
    \inferrule{V \vdash \isTerm{t} \\ \isTerm{A}\text{ is a closed type}}{V \vdash \isTerm{(t \annotate A)}}\,\Rule{Anno}
    \and
    \inferrule{V, \isTerm{x} \vdash \isTerm{t}}{V \vdash \isTerm{\lam{x}t}}\,\Rule{Abs}
    \and
    \inferrule{V \vdash \isTerm{t} \\ V \vdash \isTerm{u}}{V \vdash \isTerm{t\;u}}\,\Rule{App}
  \end{mathpar}
  \caption{Raw terms for simply typed $\lambda$-calculus}
  \label{fig:STLC-raw-terms}
\end{figure}

\begin{figure}
  \small
  \bgroup
  \renewcommand{\arraystretch}{1.5}
  \begin{tabular}{ r l }
    $\boxed{\Gamma |- \isTerm{t} : A}$ & A raw term~$t$ has type~$A$ under context~$\Gamma$
  \end{tabular}
  \egroup
  \centering
  \begin{mathpar}
    \inferrule{(x : A) \in \Gamma}{\Gamma \vdash \isTerm{x} : A}\,\Rule{Var}
    \and
    \inferrule{\Gamma \vdash \isTerm{t} : A}{\Gamma \vdash \isTerm{(t \annotate A)} : A}\,\Rule{Anno}
    \and
    \inferrule{\Gamma, \isTerm{x} : A \vdash \isTerm{t} : B}{\Gamma \vdash \isTerm{\lam{x}t} : A \bto B}\,\Rule{Abs}
    \and
    \inferrule{\Gamma \vdash \isTerm{t} : A \bto B \\ \Gamma \vdash \isTerm{u} : A}{\Gamma \vdash \isTerm{t\;u} : B}\,\Rule{App}
  \end{mathpar}
  \caption{Typing derivations for simply typed $\lambda$-calculus}
  \label{fig:STLC-typing-derivations}
\end{figure}

We start with an overview of our theory by instantiating it to simply typed $\lambda$-calculus.
Roughly speaking, the problem of type synthesis requires us to take an untyped abstract syntax tree ---which we call a \emph{raw term}--- as input, and produce a typing derivation for the term if possible.
To give more precise definitions:
The raw terms for simply typed $\lambda$-calculus are defined in \cref{fig:STLC-raw-terms};%
\footnote{We omit the usual conditions about named representations of variables throughout this paper.}
besides the standard constructs, there is an \textsc{Anno} rule that allows the user to insert type annotations to help with type synthesis.
Correspondingly, the definition of typing derivations in \cref{fig:STLC-typing-derivations} also includes an \textsc{Anno} rule enforcing that the type of an annotated term does match the annotation.
%\footnote{The presence of type annotations allows type synthesis to subsume type checking: to check whether a term~$t$ has type~$A$, synthesise the type of $t \bbcolon A$.}
Now we can define what it means to solve the type synthesis problem.

\begin{definition}
\label{def:STLC-type-synthesiser}
Parametrised by an `excuse' predicate~$E$ on raw terms, a \emph{type synthesiser} takes a context~$\Gamma$ and a raw term $|\Gamma| \vdash t$ (where $|\Gamma|$~is the list of variables appearing in~$\Gamma$) as input, and establishes one of the following outcomes:
\begin{enumerate}
\item there exists a derivation of $\Gamma \vdash t : A$ for some type~$A$,
\item there does not exist a derivation $\Gamma \vdash t : A$ for any type~$A$, or
\item $E$~holds for~$t$.
\end{enumerate}
\end{definition}

It is crucial to allow the third outcome, without which we would be requiring the type synthesis problem to be decidable, but this requirement would quickly become impossible to meet when the theory is extended to handle more complex types.
If a type synthesiser cannot decide whether there is a typing derivation, it is allowed to give an excuse instead of an answer.
Acceptable excuses are defined by the predicate~$E$, which describes what is wrong with an input term, for example not having enough type annotations.

\begin{figure}
  \small
  \bgroup
  \renewcommand{\arraystretch}{1.5}
  \begin{tabular}{ r l }
    $\boxed{\Gamma |- \isTerm{t} \syn A}$ & A raw term $t$ synthesises a type $A$ under $\Gamma$ \\
    $\boxed{\Gamma |- \isTerm{t} \chk A}$ &A raw term $t$ checks against a type $A$ under $\Gamma$
  \end{tabular}
  \egroup
  \centering
  \begin{mathpar}
    \inferrule{(\isTerm{x} : A) \in \Gamma}{\Gamma \vdash \isTerm{x} \syn A}\,\SynRule{Var}
    \and
    \inferrule{\Gamma \vdash \isTerm{t} \chk A}{\Gamma \vdash \isTerm{(t \annotate A)}\syn A}\,\SynRule{Anno}
    \and
    \inferrule{\Gamma \vdash \isTerm{t} \syn A \\ A = B}{\Gamma \vdash \isTerm{t} \chk B}\,\ChkRule{Sub}
    \and
    \inferrule{\Gamma, \isTerm{x} : A \vdash \isTerm{t} \chk B}{\Gamma \vdash \isTerm{\lam{x}t} \chk A \bto B}\,\ChkRule{Abs}
    \and
    \inferrule{\Gamma \vdash \isTerm{t} \syn A \bto B \\ \Gamma \vdash \isTerm{u} \chk A}{\Gamma \vdash \isTerm{t\;u} \syn B}\,\SynRule{App}
  \end{mathpar}
  \caption{Bidirectional typing derivations for simply typed $\lambda$-calculus}
  \label{fig:STLC-bidirectional-typing-derivations}
\end{figure}

Now our goal is to use \cref{def:STLC-type-synthesiser} as a specification and implement it using a \emph{bidirectional} type synthesiser, which attempts to produce \emph{bidirectional} typing derivations defined in \cref{fig:STLC-bidirectional-typing-derivations}.
Here we briefly recap the basic ideas of bidirectional type synthesis.
While the problem of type synthesis is not decidable in general, for certain kinds of terms it is still possible to synthesise their types --- for example, the $\SynRule{Var}$ rule says that the type of a variable can be looked up in the context.
For other kinds of terms, we can switch to the simpler problem of type \emph{checking}, where the expected type of a term is given as input so that there is more information to work with.
When a bidirectional type synthesiser traverses a raw term, it switches between the two \emph{modes}, synthesising wherever possible, and checking otherwise.
If a bidirectional type system is designed well (i.e.~\emph{mode-correct}, which will be defined in \cref{sec:type-synthesis}), then there will be a flow of type information in each rule that allows us to determine unknown types (e.g.~types to be checked) from known ones (e.g.~types previously synthesised), and suggests how bidirectional type synthesis is performed --- for example, the $\SynRule{App}$ rule says that to synthesise the type of an application $\isTerm{t\;u}$, we first synthesise the type of~$\isTerm{t}$, which should be an implication $A \bto B$, from which we can extract the expected type of~$\isTerm{u}$, namely~$A$, and perform checking; then the type of the whole application, namely~$B$, can also be extracted from the previously synthesised type $A \bto B$.

%The $\ChkRule{Abs}$ rule checks whether the type of an abstraction $\isTerm{\lam{x}t}$ is $A \bto B$ by checking whether $\isTerm{t}$~has type~$B$ assuming $\isTerm{x}$~has type~$A$ in the context, where both types $A$~and~$B$ can be determined from the type $A \bto B$ given initially.
%For an annotated term $t \bbcolon A$, the $\SynRule{Anno}$ rule simply synthesises~$A$ as its type, provided that $t$~can be successfully checked to have type~$A$.
%Finally, the subsumption rule $\ChkRule{Sub}$ says that a term synthesising a type~$A$ can be checked to have type~$B$ if $A = B$ --- in general this type equality can be replaced by some subtyping relation.

\begin{figure}
  \centering
  \small
  \judgbox{|- \isTerm{t}^\dir{d}}{A raw term~$\isTerm{t}$ is in mode~$d$}
  \begin{mathpar}
    \inferrule{x \in V}{V |- \isTerm{x}^{\syn}}\,\SynRule{Var}
    \and
    \inferrule{V |- \isTerm{t}^{\chk}}{V |- (\isTerm{t \annotate A})^{\syn}}\,\SynRule{Anno}
    \and
    \inferrule{V |- \isTerm{t}^{\syn}}{V |- \isTerm{t}^{\chk}}\,\ChkRule{Sub}
    \\
    \inferrule{V, x |- \isTerm{t}^{\chk}}
    {V |- (\isTerm{\lam{x}t})^{\chk}}\,\ChkRule{Abs}
    \and
    \inferrule{V |- \isTerm{t}^{\syn} \quad V |- \isTerm{u}^{\chk}}
    {V |- (\isTerm{t\;u})^{\syn}}\,\SynRule{App}
  \end{mathpar}
  \caption{Mode derivations for simply typed $\lambda$-calculus}
  \label{fig:STLC-mode-derivations}
\end{figure}

While it is possible for a bidirectional type synthesiser to do its job in one go, which can be thought of as adding both mode and typing information to a raw term and arriving at a bidirectional typing derivation, it is beneficial to have a preprocessing step which adds only mode information, based on which the synthesiser then continues to add typing information.
More precisely, the preprocessing step attempts to produce \emph{mode derivations} as defined in \cref{fig:STLC-mode-derivations}, where the rules are exactly the mode part of the bidirectional typing rules~(\cref{fig:STLC-bidirectional-typing-derivations}).

\begin{definition}
\label{def:STLC-mode-preprocessor}
A \emph{mode preprocessor} decides for any raw term~$V \vdash \isTerm{t}$ and mode~$\dir{d}$ whether $V \vdash \isTerm{t}^\dir{d}$.
\end{definition}

One (smaller) benefit of mode preprocessing is that it helps to simplify the synthesiser, whose computation can be partly directed by a mode derivation.
More importantly, whether there is a mode derivation for a term is actually very useful information to the user, because it corresponds to whether the term has enough type annotations:
Observe that the $\SynRule{Anno}$ and $\ChkRule{Sub}$ rules allow us to switch between the synthesising and checking modes;
the switch from synthesising to checking is free, whereas the opposite direction requires a type annotation.
That is, any term in synthesising mode is also in checking mode, but not necessarily vice versa.
A type annotation is required wherever a term that can only be in checking mode is required to be in synthesising mode, and a term does not have a mode derivation if and only if type annotations are missing in such places.
(We will treat all these more rigorously in \cref{sec:pre-synthesis}.)
For example, an abstraction is strictly in checking mode, but the left sub-term of an application has to be synthesising, so the term $\isTerm{(\lam{x}t)\;u}$ does not have a mode derivation unless we annotate the abstraction.

Perhaps most importantly, mode derivations enable us to give bidirectional type synthesisers a tight definition: if we restrict the domain of a synthesiser to terms in synthesising mode (i.e.~having enough type annotations for performing synthesis), then it is possible for the synthesiser to \emph{decide} whether there is a suitable typing derivation.

\begin{definition}
\label{def:STLC-bidirectional-type-synthesiser}
A \emph{bidirectional type synthesiser} decides for any context~$\Gamma$ and synthesising term $|\Gamma| \vdash \isTerm{t}^{\syn}$ whether $\Gamma \vdash \isTerm{t} \syn A$ for some type~$A$.
\end{definition}

Now we can get back to our goal of implementing a type synthesiser~(\cref{def:STLC-type-synthesiser}).

\begin{theorem}
A type synthesiser that uses `not in synthesising mode' as its excuse can be constructed from a mode preprocessor and a bidirectional type synthesiser.
\end{theorem}

The construction is straightforward:
Run the mode preprocessor on the input term~$|\Gamma| \vdash t$.
If there is no synthesising mode derivation, report that $t$~is not in synthesising mode (the third outcome).
Otherwise $|\Gamma| \vdash t^{\syn}$, and we can run the bidirectional type synthesiser.
If it finds a derivation of $\Gamma \vdash t \syn A$ for some type~$A$, return a derivation of $\Gamma \vdash t : A$ (the first outcome), which is possible because the bidirectional typing~(\cref{fig:STLC-bidirectional-typing-derivations}) is \emph{sound} with respect to the original typing~(\cref{fig:STLC-typing-derivations}).

\begin{lemma}[Soundness]
If\/ $\Gamma \vdash t :^\dir{d} A$, then $\Gamma \vdash t : A$.
\end{lemma}

If there is no derivation of $\Gamma \vdash t \syn A$ for any type~$A$, report that there is no derivation of $\Gamma \vdash t : A$ for any~$A$ either (the second outcome), because the bidirectional typing is \emph{complete} with respect to the original typing.

\begin{lemma}[Completeness]
If\/ $|\Gamma| \vdash t^\dir{d}$ and\/ $\Gamma \vdash t : A$, then $\Gamma \vdash t :^\dir{d} A$.
\end{lemma}

We will construct a mode preprocessor~(\cref{sec:mode-preprocessing}) and a bidirectional type synthesiser~(\cref{sec:type-synthesis}) and prove both lemmas~(\cref{sec:soundness-and-completeness}) for all syntax-directed bidirectional simple type systems.
To quantify over all such systems, we need their generic definitions, which we formulate in \cref{sec:defs}.

\LT{We avoid using the term `function type' and its conventional notation $\to$ at the object level on purpose, as it may be confused with function types in our type-theoretic meta-language.}

%!TEX root = BiSig.tex

\section{Simply Typed Languages and Bidirectional Type Systems}\label{sec:defs}
This section provides generic definitions
%\footnote{%
%Here's the 'small print': Our definitions cover typical examples and set the scope for bidirectional simple type systems discussed, but do not offer comprehensive coverage of all possibilities.}
of simple types, simply typed languages, and bidirectional type systems, using simply typed $\lambda$-calculus in \cref{sec:key-ideas} as our running example.

Our definitions are formulated in two steps:
\begin{enumerate*}
  \item first we introduce a notion of arity and a notion of signature which includes a set of operation symbols and an assignment of arities to symbols;
\item then, given a signature, we define sets of raw terms and derivations by primitive rules such as $\Rule{Var}$ and a rule schema for constructs $\tmOp_o$ indexed by an operational symbol~$o$.
\end{enumerate*}
All these definitions are inductive as usual but include a \emph{rule schema} giving rise to as many rules as there are operation symbols in a signature.
Upon moving from simple types to bidirectional typing, the notion of arity, initially as the number of arguments of an operation, is enriched to incorporate an extension context for variable binding and the mode for the direction of type information flow.

\subsection{Signatures and Simple Types} \label{subsec:simple-types}

For simple types, the only datum needed for specifying a type construct is its number of arguments:
\begin{definition} \label{def:simple-signature}
  \begin{figure}
%    \begin{minipage}[b]{.6\textwidth}
      \centering
      \small
      \judgbox{\Xi|-_{\Sigma} A}{$A$ is a well-formed type with type variables in $\Xi$}
      \begin{mathpar}
        \inferrule{X_i \in \Xi}{\Xi |-_\Sigma X_i} \and
        \inferrule{\Xi |-_{\Sigma} A_1 \\ \cdots \\ \Xi |-_{\Sigma} A_n}{\Xi |-_{\Sigma} \tyOp_i(A_1, \ldots, A_n)}\;\text{where $n = \arity(i)$}
      \end{mathpar}
      \caption{Simple types}
      \label{fig:simple-type}
%    \end{minipage}
%    \begin{minipage}[b]{.35\textwidth}
%      \centering
%      \small
%      \judgbox{\Xi |-_{\Sigma} \Gamma}{}
%      \begin{mathpar}
%        \inferrule{ }{\Xi |-_{\Sigma} \cdot }\and
%        \inferrule{\Xi |-_{\Sigma} A \\ \Xi|-_{\Sigma} \Gamma}{\Xi |-_{\Sigma} \Gamma, x : A}
%      \end{mathpar}
%      \caption{Contexts}
%    \label{fig:simple-context}
%    \end{minipage}
  \end{figure}
  A \emph{signature} $\Sigma$ for simple types consists of a set\footnote{%
    Even though our theory is developed in Martin-L\"of type theory, the term `set' is used instead of `type' to avoid the obvious confusion. 
    Also, as we assume \AxiomK, all types are legitimate sets in the sense of homotopy type theory.
  }
  $I$ with a decidable equality and an \emph{arity} function $\arity\colon I \to \mathbb{N}$.
  For a signature $\Sigma$, a \emph{type} $A : \Type_{\Sigma}(\Xi)$ over a variable set $\Xi$ is either
  \begin{enumerate}
    \item a variable in $\Xi$ or
    \item $\tyOp_{i}(A_1, \ldots, A_n)$ for some $i:I$ with $\arity(i) = n$ and types $A_1,\ldots, A_n$.
  \end{enumerate}
  A \emph{context} $\Gamma \colon \Cxt_{\Sigma}(\Xi)$ over $\Xi$ is formed by $\cdot$ for the empty context and $\Gamma, A$ for an additional type $A$.
\end{definition}

\begin{example} \label{ex:type-signature-for-function-type}
  Simply typed $\lambda$-calculus includes function types $A \bto B$ and typically a base type~$\mathtt{b}$ to ensure that the set of all types without type variables is non-empty.
  The type signature $\Sigma_{\bto}$ used to define types in simply typed $\lambda$-calculus consists of a binary operation $\mathsf{fun}$ and a nullary operation $\mathsf{b}$ where $\arity(\mathsf{fun}) = 2$ and $\arity(\mathsf{b}) = 0$.
  Then, all types in simply typed $\lambda$-calculus can be given as $\Sigma_{\bto}$-types over the empty set with $A \bto B$ introduced as $\tyOp_{\mathsf{fun}}(A, B)$ and $\mathtt{b}$ as $\tyOp_{\mathsf{b}}$. 
\end{example}

The \emph{substitution} for a function $\rho\colon \Xi \to \Type_{\Sigma}(\Xi')$, denoted by $\rho\colon \Sub{\Xi}{\Xi'}$, is a map which sends a type $A$ of $\Type_{\Sigma}(\Xi)$ to $\simsub{A}{\rho}$ of $\Type_{\Sigma}(\Xi')$ and is defined as usual.
By abuse of notation, the substitution $\simsub{\Gamma}{\rho}$ of a context $\Gamma$ is defined by applying substitution to each $A$ in $\Gamma$.

\subsection{Binding Signatures and Simply Typed Languages} \label{subsec:binding-sig}

A simply typed language consists of
\begin{enumerate*}
  \item a set of raw terms $\isTerm{t}$ indexed by a set $V$ of untyped variables where each construct is allowed to bind some variables like $\Rule{Abs}$ and to take multiple arguments like $\Rule{App}$;
  \item a set of typing derivations indexed by a list $\Gamma$ of typed variables, a type~$A$, and a raw term $\isTerm{t}$ to set their type constraints. 
\end{enumerate*}
Therefore, to specify a term construct, we enrich the notion of arity with some sort for typing and extension context for variable binding:
\begin{definition}\label{def:binding-arity}
  A \emph{binding arity} with a sort $T$ is an inhabitant of $\left(T^* \times T\right)^* \times T$ where $T^*$ is the set of lists over $T$.
  In a binding arity $(((\Delta_1, A_1), \ldots, (\Delta_n, A_n)), A)$, every $\Delta_i$ and $A_i$ refers to the \emph{extension context} and the sort of the $i$-th argument, respectively, and $A$ the target sort.
  For brevity, a binding arity is denoted by $\bargs \to A$ where $[\Delta_i]$ is omitted if $\Delta_i$ is empty.
\end{definition}

For example, consider the $\Rule{Abs}$ rule (\Cref{fig:STLC-typing-derivations}).
Its binding arity has the sort $\Type_{\Sigma_{\bto}}\{A, B\}$ and is
\[
  {[A]B} \to {(A \bto B)}.
\]
This means that the $\Rule{Abs}$ rule for derivations of\/ $\Gamma |- \lam{x}t : A \bto B$ contains:
\begin{enumerate*}
  \item a derivation of\/ $\Gamma, x : A |- t : B$ as an argument of type $B$ in a \emph{context extended} by variable $x$ of type $A$;
  \item the type $A\bto B$ for itself.
\end{enumerate*}
In a similar vein, the arity of the $\Rule{App}$ rule is denoted as
\[
  {(A \bto B), A} \to {B}
\]
saying that any derivation of\/ $\Gamma \vdash t\;u : B$ accepts derivations $\Gamma \vdash t : A \bto B$ and $\Gamma \vdash u : A$ as its arguments.
These arguments have types $A \bto B$ and $A$ respectively and empty extension contexts.

Next, akin to a signature, a binding signature $\Omega$ consists of a set of operation symbols along with their respective binding arities.
This also includes the set $\Xi$ of type variables that might be present in the binding arity, each operation is assigned the sort $\Type_{\Sigma}(\Xi)$:
\begin{definition}\label{def:binding-signature}
  For a type signature $\Sigma$, a \emph{binding signature} $\Omega$ consists of a set $O$ and a function
  \[
    \arity(o) \colon O \to \sm{\Xi : \UU} \left(\Cxt_\Sigma(\Xi) \times \Type_\Sigma(\Xi)\right)^* \times \Type_\Sigma(\Xi).
  \]
\end{definition}
That is, each inhabitant $o: O$  is associated with a set $\Xi$ of type variables and a binding arity $\arity(o)$ with the sort $\Type_{\Sigma}(\Xi)$, denoted by $\bop$.
By a \emph{simply typed language} $(\Sigma, \Omega)$, we mean a pair of a type signature $\Sigma$ and a binding signature $\Omega$.

The inclusion of a variable set $\Xi$ for each operation, called its \emph{local context}, plays a crucial role in type synthesis.
For a typing rule like $\ChkRule{Abs}$, we will need to substitute \emph{concrete types}, i.e.\ types without any type variables, for variables $A, B$.
But, we must first identify for which type variables to substitute.
As such, this information forms part of the arity of an operation, and typing derivations, defined subsequently, will include an instantiation to assign concrete types to variables.

\begin{definition}
\begin{figure}
  \centering
  \small
  \judgbox{V |-_{\Sigma, \Omega} \isTerm{t}}{$\isTerm{t}$ is a raw term for a simply typed language $(\Sigma, \Omega)$ with free variables in~$V$}
  \begin{mathpar}
    \inferrule{\isTerm{x} \in V}{V |-_{\Sigma, \Omega} \isTerm{x}}\,\Rule{Var}
    \and
    \inferrule{\cdot |-_{\Sigma} A \\ V |-_{\Sigma, \Omega}\isTerm{t}}{V |-_{\Sigma, \Omega} \isTerm{t \annotate A}}\,\Rule{Anno}
    \\
    \inferrule{V, \vars{x}_\isTerm{1} |-_{\Sigma, \Omega} \isTerm{t_1} \\ \cdots \\ V, \vars{x}_\isTerm{n} |-_{\Sigma, \Omega} \isTerm{t_n} } {V |-_{\Sigma, \Omega} \tmOpts }\,\Rule{Op} \and
    \text{for $\bop$ in $\Omega$}
  \end{mathpar}
  \caption{Raw terms}
  \label{fig:raw-terms}
\end{figure}
  For a simply typed language $(\Sigma, \Omega)$, the set of \emph{raw terms} indexed by a context~$V$ of free variables consists of
  \begin{enumerate*}
    \item variables in $V$,
    \item annotations $\isTerm{t \annotate A}$ for some raw term $t$ in $V$ and a type $A$, and
    \item a construct $\tmOpts$ for some $\bop$ in $O$, where $\vars{x}_{\isTerm{i}}$'s are lists of variables whose length is equal to the length of~$\Delta_i$ and $t_i$'s are raw terms in the variable list $V, \vars{x}_i$
  \end{enumerate*}
  corresponding to rules $\Rule{Var}$, $\Rule{Anno}$, and $\Rule{Op}$ in \Cref{fig:raw-terms} respectively.
\end{definition}


The definition of typing derivations is a bit more involved.
We need some information to compare types on the object level during type synthesis and substitute those type variables in a typing derivation $\Gamma \vdash \tmOpts : A$ for an operation $o$ in $\Omega$ at some point.
Here we choose to include a substitution $\rho$ from the local context $\Xi$ to $\emptyset$ as part of its typing derivation explicitly:
\begin{definition}\label{def:typing-derivations}
  \begin{figure}
    \centering
    \small
    \judgbox{\Gamma |-_{\Sigma, \Omega} \isTerm{t} : A}{A raw term $\isTerm{t}$ has a concrete type $A$ under $\Gamma$ for a simply typed language $(\Sigma, \Omega)$}
    \begin{mathpar}
      \inferrule{(\isTerm{x} : A) \in \Gamma}{\Gamma |-_{\Sigma, \Omega} \isTerm{x} : A}\,\Rule{Var}
      \and
      \inferrule{\Gamma |-_{\Sigma, \Omega} \isTerm{t} : A}{\Gamma |-_{\Sigma, \Omega} (\isTerm{t \annotate A}) : A}\,\Rule{Anno}
      \and
      \inferrule{\rho : \Sub{\Xi}{\emptyset} \\  \Gamma, \vec{\isTerm{x}}_\isTerm{1} : \simsub{\Delta_{1}}{\rho} |-_{\Sigma, \Omega} \isTerm{t_1} : \simsub{A_{1}}{\rho} \quad\cdots\quad \Gamma, \vec{\isTerm{x}}_\isTerm{n} : \simsub{\Delta_{n}}{\rho} |-_{\Sigma, \Omega} \isTerm{t_n} : \simsub{A_{n}}{\rho}}
      {\Gamma |-_{\Sigma, \Omega} \tmOpts : \simsub{A_0}{\rho}}\,\Rule{Op}
      \\
    \text{for $\bop$ in $\Omega$}
    \end{mathpar}
    \caption{Typing derivations}
    \label{fig:extrinsic-typing}
  \end{figure}
  For a simply typed language $(\Sigma, \Omega)$ the set of \emph{typing derivations} of $\Gamma \vdash \isTerm{t} : A$, indexed by a context $\Gamma$, a raw term with free variables in $\erase{\Gamma}$, and a type $A$, consists of 
  \begin{enumerate}
    \item a derivation of $\Gamma |-_{\Sigma, \Omega} x : A$ if $x : A$ is in $\Gamma$,
    \item a derivation of $\Gamma |-_{\Sigma, \Omega} (t \annotate A) : A$ if $\Gamma \vdash_{\Sigma, \Omega} \isTerm{t} : A$ has a derivation, and
    \item a derivation of $\Gamma |-_{\Sigma, \Omega} \tmOpts : \simsub{A_0}{\rho}$ for $\bop$ if there is a function $\rho\colon \Xi \to \Type_{\Sigma}(\emptyset)$ and a derivation of $\Gamma, \vars{x}_{\isTerm{i}} : \Delta_i |-_{\Sigma, \Omega} \isTerm{t_i} : \simsub{A_i}{\rho}$ for each $i$,
  \end{enumerate}
  corresponding to rules $\Rule{Var}$, $\Rule{Anno}$, and $\Rule{Op}$ in \Cref{fig:extrinsic-typing} respectively.
\end{definition}

Raw terms (\Cref{fig:STLC-raw-terms}) and typing derivations (\Cref{fig:STLC-typing-derivations}) for simply typed $\lambda$-calculus can be specified by the type signature $\Sigma_{\bto}$ (\Cref{ex:type-signature-for-function-type}) and the binding signature consisting of 
 \[
   \aritysymbol{\mathsf{app}}{A, B}{(A \bto B), A}{B}
   \quad\text{and}\quad
   \aritysymbol{\mathsf{abs}}{A , B}{[A]B}{(A \bto B)}.
 \]
Rules $\Rule{Abs}$ and $\Rule{App}$ in simply typed $\lambda$-calculus are subsumed by the $\Rule{Op}$ rule schema, as applications $t\;u$ and abstractions $\lam{x}t$ can be introduced uniformly as $\tmOp_{\mathsf{app}}(t, u)$ and $\tmOp_{\mathsf{abs}}(x.t)$, respectively.

\subsection{Bidirectional Binding Signatures and Bidirectional Type Systems} \label{subsec:bidirectional-system}
For a bidirectional type system, typing judgements appear in two forms: $\Gamma |- t \syn A$ and $\Gamma |- t \chk A$, but these two typing judgements can be considered as a single typing judgement $\Gamma |- t :^\dir{d} A$, additionally indexed by a \emph{mode} $d : \Mode$---which can either be $\syn$ or $\chk$.
Therefore, to define a bidirectional type system, we further enrich the concept of binding arity by incorporating modes:

\begin{definition} \label{def:bidirectional-binding-signature}
  A \emph{bidirectional binding arity} with a sort $T$ is an inhabitant of
  \[
    \left(T^* \times T \times \Mode \right)^* \times T \times \Mode.
  \]
  For clarity, an arity is denoted by $\biarity$.
\end{definition}
When compared to a binding arity, a bidirectional binding arity provides further details, specifying the mode for each of its arguments and for the conclusion.
Take the $\ChkRule{Abs}$ rule (\Cref{fig:STLC-bidirectional-typing-derivations}) for $\lam{x}t$ as an example.
It has the arity ${[A]B^{\chk}}\to{(A \bto B)^{\chk}}$, indicating additionally that both $\lam{x}t$ and its argument $t$ are checking.
Likewise, the $\SynRule{App}$ rule has the arity ${(A \bto B)^{\syn}, A^{\chk}} \to {B^{\syn}}$.

\begin{definition}
  For a type signature $\Sigma$, a \emph{bidirectional binding signature} $\Omega$ is a set $O$ with
  \[
    \mathit{ar}\colon O \to \sum_{\Xi : \UU} \left(\Cxt_{\Sigma}(\Xi) \times \Type_{\Sigma}(\Xi) \times {\Mode}\right)^* \times \Type_{\Sigma}(\Xi) \times {\Mode}.
  \]
\end{definition}
We write $\biop$ for an operation $o$ with a variable set $\Xi$ and its bidirectional binding arity with sort $\Type_{\Sigma}(\Xi)$.
We call it \emph{checking} if $d$ is ${\chk}$ or \emph{synthesis} if $d$ is ${\syn}$; similarly its $i$-th argument is checking if $d_i$ is $\chk$ and synthesis if $d_i$ is $\syn$.
A bidirectional type system $(\Sigma, \Omega)$ refers to a pair of a type signature $\Sigma$ and a bidirectional binding signature $\Omega$.
Every bidirectional binding signature $\Omega$ gives rise to a binding signature $\erase{\Omega}$ if we erase modes from $\Omega$, called the \emph{(mode) erasure} of $\Omega$.
Hence a bidirectional type system $(\Sigma, \Omega)$ also specifies a simply typed language $(\Sigma, \erase{\Omega})$, including raw terms and typing derivations.

\begin{definition}\label{def:bidirectional-typing-derivations}\label{def:mode-derivations}
  \begin{figure}
    \centering
    \small
    \judgbox{\Gamma |-_{\Sigma, \Omega} \isTerm{t} :^\dir{d} A}{$\isTerm{t}$ has a concrete type $A$ in mode $\dir{d}$ under $\Gamma$ for a bidirectional type system $(\Sigma, \Omega)$} 
    \begin{mathpar}
      \inferrule{(x : A) \in \Gamma}{\Gamma |-_{\Sigma, \Omega} \isTerm{x} :^{\syn} A}\,\SynRule{Var}
      \and
      \inferrule{\Gamma |-_{\Sigma, \Omega} \isTerm{t} :^{\chk} A}{\Gamma |-_{\Sigma, \Omega} (\isTerm{t \annotate A}) :^{\syn}  A}\,\SynRule{Anno}
      \and
      \inferrule{\Gamma |-_{\Sigma, \Omega} \isTerm{t} :^{{\syn}} B \\ B = A}{\Gamma |-_{\Sigma, \Omega} \isTerm{t} :^{\chk} A}\,\ChkRule{Sub}
      \\
      \inferrule{\rho\colon \Sub{\Xi}{\emptyset} \\ \Gamma, \simsub{\vars{x}_\isTerm{1} : \Delta_{1}}{\rho} |-_{\Sigma, \Omega} \isTerm{t_{1}} :^\dir{d_1} \simsub{A_1}{\rho} \\
        \cdots \\
      \Gamma, \simsub{\vars{x}_\isTerm{n} : \Delta_{n}}{\rho} |-_{\Sigma, \Omega} \isTerm{t_n} :^{\dir{d_n}} \simsub{A_{n}}{\rho}}
      {\Gamma |-_{\Sigma, \Omega} \tmOpts :^{\dir{d}} \simsub{A_0}{\rho}} \,\mathsf{Op}
      \\ \text{for $\biop$ in $\Omega$}
    \end{mathpar}
    \caption{Bidirectional typing derivations}
    \label{fig:bidirectional-typing-derivations}
  \end{figure}
  \begin{figure}
    \centering
    \small
    \judgbox{V |-_{\Sigma, \Omega} \isTerm{t}^\dir{d}}{$\isTerm{t}$ is in mode $d$ with free variables in $V$ for a bidirectional type system $(\Sigma, \Omega)$}
    \begin{mathpar}
      \inferrule{x \in V}{V |-_{\Sigma, \Omega} \isTerm{x}^{\syn}}\,\SynRule{Var}
      \and
      \inferrule{\cdot |-_{\Sigma} A \\ V |-_{\Sigma, \Omega}\isTerm{t}^{\chk}}{V |-_{\Sigma, \Omega} (\isTerm{t \annotate A})^{\syn}}\,\SynRule{Anno}
      \and
      \inferrule{V |-_{\Sigma, \Omega} \isTerm{t}^{\syn}}{V |-_{\Sigma, \Omega} \isTerm{t}^{\chk}}\,\ChkRule{Sub}
      \and
      \inferrule{V, \vars{x}_1 |-_{\Sigma, \Omega} \isTerm{t_1}^\dir{d_1} \\ \cdots \\ V, \vars{x}_{n} |-_{\Sigma, \Omega} \isTerm{t_n}^\dir{d_n}}
      {V |-_{\Sigma, \Omega} \tmOpts^\dir{d}}\,\Rule{Op}
      \and \text{for $\biop$}
    \end{mathpar}
    \caption{Mode derivations}
    \label{fig:mode-derivations}
  \end{figure}
  For a bidirectional type system $(\Sigma, \Omega)$,
  \begin{itemize}
    \item the set of \emph{bidirectional typing derivations} of $\Gamma \vdash_{\Sigma, \Omega} t :^\dir{d} A$, indexed by a context $\Gamma$ of typed variables, a raw term $\isTerm{t}$ under $\erase{\Gamma}$, a mode $\dir{d}$, and a type $A$, is defined in \Cref{fig:bidirectional-typing-derivations} and particularly
          \[
            \Gamma |-_{\Sigma, \Omega} \tmOpts :^{\dir{d}} \simsub{A_0}{\rho}
          \]
          has a derivation for $\biop$ in $\Omega$ if there is a function $\rho\colon \Xi \to \Type_{\Sigma}(\emptyset)$ and a derivation of $\Gamma, \vars{x}_{\isTerm{i}} : \Delta_i \vdash_{\Sigma, \Omega} \isTerm{t_i} :^{\dir{d_i}} \simsub{A_i}{\rho}$ for each $i$;
    \item the set of \emph{mode derivations} of $V |-_{\Sigma, \Omega} t^\dir{d}$, indexed by a list $V$ of variables, a raw term $\isTerm{t}$ under $V$, and a mode $d$, is defined in \Cref{fig:mode-derivations}.
  \end{itemize}
  {\small$\boxed{\Gamma \vdash_{\Sigma, \Omega} \isTerm{t} \syn A}$} and {\small$\boxed{\Gamma \vdash_{\Sigma, \Omega} \isTerm{t} \chk A}$} are shorthands for ${\Gamma \vdash_{\Sigma, \Omega} \isTerm{t} :^{\syn} A}$ and ${\Gamma \vdash_{\Sigma, \Omega} \isTerm{t} :^{\chk} A}$ respectively.
  A typing rule is \emph{checking} if its conclusion mode is $\chk$ and \emph{synthesis} if the conclusion mode is $\syn$.
\end{definition}

\begin{example}\label{ex:signature-simply-typed-lambda}
Having established generic definitions, we can now specify the simply typed $\lambda$-calculus and its bidirectional type system---including raw terms, (bidirectional) typing derivations, and mode derivations---using just a pair of signatures $\Sigma_{\bto}$ (\Cref{ex:type-signature-for-function-type}) and $\Omega^{\Leftrightarrow}_\Lambda$ which consists of 
\[
  \aritysymbol{\mathsf{abs}}{A , B}{[A]B^{\chk}}{(A \bto B)^{\chk}}
  \quad\text{and}\quad
  \aritysymbol{\mathsf{app}}{A, B}{(A \bto B)^{\syn}, A^{\chk}}{B^{\syn}}.
\]
\end{example}
More importantly, we are able to reason about constructions and properties that hold for any bidirectional simple type system $(\Sigma, \Omega)$ once and for all.

%!TEX root = BiSig.tex

\section{Mode Preprocessing and Related Properties}\label{sec:pre-synthesis}

\Josh{Overview of the section}

\subsection{Soundness and Completeness}
\label{sec:soundness-and-completeness}

%\Josh{Throughout this sub-section, we quantify over any bidirectional type system~$(\Sigma, \Omega)$, context $\Gamma : \Cxt_\Sigma(\emptyset)$, raw term $\erase\Gamma \vdash_{\Sigma, \erase\Omega} t$, mode~$d$, and type $A : \Type_\Sigma(\emptyset)$.}

\Josh{One theorem to rule them all}

Erasure of a bidirectional binding signature removes mode information and keeps everything else intact; this can be straightforwardly extended by induction to remove mode information from a bidirectional typing derivation and arrive at an ordinary typing derivation.

\begin{lemma}[Soundness]\label{thm:soundness}
If\/ $\Gamma \vdash_{\Sigma, \Omega} t :^\dir{d} A$, then $\Gamma \vdash_{\Sigma, \erase\Omega} t : A$.
\end{lemma}

\begin{proof}
Induction on the given derivation, mapping every bidirectional typing rule to its mode-less counterpart except $\ChkRule{Sub}$, in which case the induction hypothesis $\Gamma |-_{\Sigma,\erase\Omega} t : B$ suffices due to the premise $B = A$.
\end{proof}

We can also remove typing and retain mode information, arriving at a mode derivation instead.

\begin{proposition}\label{thm:typing-removal}
If\/ $\Gamma \vdash_{\Sigma, \Omega} t :^\dir{d} A$, then $\erase\Gamma \vdash_{\Sigma, \Omega} t^\dir{d}$.
\end{proposition}

\begin{proof}
Induction on the given derivation, mapping every rule to its counterpart.
\end{proof}

Conversely, if we have both mode and typing derivations for the same term, we can combine them and obtain a bidirectional typing derivation.

\begin{lemma}[Completeness]\label{thm:completeness}
If\/ $\erase\Gamma |-_{\Sigma, \Omega} t^\dir{d}$ and\/ $\Gamma |-_{\Sigma, \erase\Omega} t : A$, then $\Gamma |-_{\Sigma, \Omega} t :^\dir{d} A$.
\end{lemma}

\begin{proof}
Induction on the given mode derivation.
For \SynRule{Var}, \SynRule{Anno}, and \Rule{Op}, the outermost rule used in the given typing derivation must be the corresponding typing rule, so by the induction hypotheses we have bidirectional typing derivations for all the sub-terms, to which we can then apply the corresponding bidirectional typing rule.
The \ChkRule{Sub} case is similar but slightly simpler: the induction hypothesis directly gives us a derivation of $\Gamma |-_{\Sigma,\Omega} t \syn A$, to which we apply $\ChkRule{Sub}$.
\end{proof}

In short, soundness and completeness are no more than the separation and combination of mode and typing information carried by the three kinds of derivations while keeping their basic structure, which is directed by the same raw term.
%Formulating the notion of mode derivation helps to complete this clean picture.

\subsection{Generalised Mode Preprocessing}
\label{sec:mode-preprocessing}

The goal of this sub-section is to construct a mode preprocessor, which decides for any raw term $V |-_{\Sigma,\erase\Omega} t$ and mode~$\dir d$ whether $V |-_{\Sigma,\Omega} t^\dir{d}$ or not.
In fact we will do better:
If a mode preprocessor returns a proof that no mode derivation exists, that proof (of a negation) does not provide useful information for the user.
It will be more helpful if a preprocessor can produce an explanation of why no mode derivation exists, and even how to fix the input term to have a mode derivation.
We will construct such a \emph{generalised mode preprocessor}~(\cref{thm:generalised-mode-preprocessing}), which can be weakened to an ordinary mode preprocessor~(\cref{thm:mode-preprocessing}) if the additional explanation is not needed.

\renewcommand{\True}{\mathbf{T}}
\renewcommand{\False}{\mathbf{F}}

\begin{figure}
  \centering
  \small
  \begin{tabular}{ r r l }
    & & is in mode~$d$, \\
    $\smash{\boxed{V |-_{\Sigma, \Omega} \isTerm{t}^{\dir{d}\,g\,s}}}$
    & The raw term~$\isTerm{t}$\hspace{-.6em}
    & misses some type annotation iff $g = \False$, and \\
    & & is in mode~$d$ due to an outermost mode cast iff $s = \False$
  \end{tabular}
  \begin{mathpar}
    \inferrule{x \in V}{V |-_{\Sigma, \Omega} \isTerm{x}^{\syn\,\True\,\True}}\,\SynRule{Var}
    \and
    \inferrule{\cdot |-_{\Sigma} A \\ V |-_{\Sigma, \Omega}\isTerm{t}^{\chk\,g\,s}}{V |-_{\Sigma, \Omega} (\isTerm{t \annotate A})^{\syn\,g\,\True}}\,\SynRule{Anno}
    \and
    \inferrule{V |-_{\Sigma, \Omega}\isTerm{t}^{\chk\,g\,\True}}{V |-_{\Sigma, \Omega} \isTerm{t}^{\syn\,\False\,\False}}\,\SynRule{Missing}
    \and
    \inferrule{V |-_{\Sigma, \Omega} \isTerm{t}^{\syn\,g\,\True}}{V |-_{\Sigma, \Omega} \isTerm{t}^{\chk\,g\,\False}}\,\ChkRule{Sub}
    \and
    \inferrule{V, \vec x_1 |-_{\Sigma, \Omega} \isTerm{t_1}^{\dir{d_1}\,g_1\,s_1} \\ \cdots \\ V, \vec x_n |-_{\Sigma, \Omega} \isTerm{t_n}^{\dir{d_n}\,g_n\,s_n}}
    {V |-_{\Sigma, \Omega} \tmOpts^{\dir{d}\,(\bigwedge_i g_i)\,\True}}\,\Rule{Op}
  \end{mathpar}
  \caption{Generalised mode derivations}
  \label{fig:generalised-mode-derivations}
\end{figure}

Intuitively, a term does not have a mode derivation exactly when there are not enough type annotations, but such negative formulations convey little information.%
\todo{still default in the rest of the paper}
Instead, we can provide more information by pointing out the places in the term that require annotations.
For a bidirectional type system, an annotation is required wherever a term is `strictly' (which we will explain shortly) in checking mode but required to be in synthesising mode, in which case there is no rule for switching from checking to synthesising, and thus there is no way to construct a mode derivation.
We can, however, consider \emph{generalised mode derivations} defined in \cref{fig:generalised-mode-derivations} that allow the use of an additional $\SynRule{Missing}$ rule for such switching, so that a derivation can always be constructed.
Given a generalised mode derivation, if it uses $\SynRule{Missing}$ in some places, then those places are exactly where annotations should be supplied, which is helpful information to the user; if it does not use $\SynRule{Missing}$, then the derivation is \emph{genuine} in the sense that it corresponds directly to an original mode derivation.
This can be succinctly formulated as \cref{thm:Pre?-true} below by encoding genuineness as a boolean~$g$ in the generalised mode judgement, which is set to~$\False$ only by the $\SynRule{Missing}$ rule.
(Ignore the boolean~$s$ for now.)

\begin{lemma}\label{thm:Pre?-true}
If\/ $V |-_{\Sigma,\Omega} t^{\dir{d}\,\True\,s}$, then $V |-_{\Sigma,\Omega} t^\dir{d}$.
\end{lemma}

\begin{proof}
Induction on the given derivation.
The $\SynRule{Missing}$ rule cannot appear because $g = \True$, and the other rules are mapped to their counterparts.
\end{proof}

We also want a lemma that covers the case where $g = \False$.

\begin{lemma}\label{thm:Pre?-false}
If\/ $V |-_{\Sigma,\Omega} t^{\dir{d}\,\False\,s}$, then $V \not|-_{\Sigma,\Omega} t^\dir{d}$.
\end{lemma}

This lemma would be wrong if the `strictness' boolean~$s$ was left out of the rules:
Having both $\ChkRule{Sub}$ and $\SynRule{Missing}$, which we call \emph{mode casts}, it would be possible to switch between the two modes freely, which unfortunately means that we could insert a pair of $\ChkRule{Sub}$ and $\SynRule{Missing}$ anywhere, constructing a non-genuine derivation even when there is in fact a genuine one.
The `strictness' boolean~$s$ can be thought of as disrupting the formation of such pairs of mode casts:
Every rule other than the mode casts sets~$s$ to~$\True$, meaning that a term is \emph{strictly} in the mode assigned by the rule (i.e.~not altered by a mode cast), whereas the mode casts set~$s$ to~$\False$.
Furthermore, the sub-derivation of a mode cast has to be strict, so it is impossible to have consecutive mode casts.
Another way to understand the role of~$s$ is that it makes the $\SynRule{Missing}$ rule precise: an annotation is truly missing only when a term is \emph{strictly} in checking mode but is required to be in synthesising mode.
With strictness coming into play, non-genuine derivations are now `truly non-genuine'.

\begin{proof}[Proof of \cref{thm:Pre?-false}]
Induction on the generalised mode derivation, in each case analysing an arbitrary mode derivation and showing that it cannot exist.
The key case is $\SynRule{Missing}$, where we have a sub-derivation of $V |-_{\Sigma,\Omega} t^{\chk\,g\,\True}$ for some boolean~$g$.
We do not have an induction hypothesis and seem to get stuck, because $g$~is not necessarily~$\False$.
But what matters here is that $t$~is \emph{strictly} in checking mode: if we continue to analyse the sub-derivation, the outermost rule must be $\Rule{Op}$ with $\dir d = {\chk}$, implying that $t$~has to be an operation in checking mode.
Then a case analysis shows that it is impossible to have a (synthesising) mode derivation of $V |-_{\Sigma,\Omega} t^{\syn}$.
%\begin{itemize}
%\item $\SynRule{Anno}$:
%Any mode derivation must also use $\SynRule{Anno}$ as the outermost rule, and then the induction hypothesis suffices.
%\item $\ChkRule{Sub}$:
%We have a sub-derivation of $V |-_{\Sigma,\Omega} t^{\syn\,\False\,\True}$.
%Any mode derivation of $V |-_{\Sigma,\Omega} t^{\chk}$ must use $\ChkRule{Chk}$ as the outermost rule.
%(The $\Rule{Op}$ rule with $\dir d = {\chk}$ is impossible because the sub-derivation shows that $t$~is exactly a synthesising term.)
%We then have a sub-derivation of $V |-_{\Sigma,\Omega} t^{\syn}$, and can use the induction hypothesis.
%\end{itemize}
\end{proof}

Now we are ready to construct a generalised mode preprocessor.

\begin{theorem}[Generalised Mode Preprocessing]\label{thm:generalised-mode-preprocessing}
For any raw term $V |-_{\Sigma,\erase\Omega} t$ and mode~$\dir{d}$, there is a derivation of\/ $V |-_{\Sigma,\Omega} t^{\dir{d}\,g\,s}$ for some booleans $g$~and~$s$.
\end{theorem}

The theorem could be proved directly, but that would mix up two case analyses which respectively inspect the input term and apply mode casts depending on what modes are required.
Instead, we distill the case analysis that deals with mode casts into the following \cref{thm:adjustment}, whose antecedent~(\ref{eq:strict-derivation}) is then established by induction on the input term in the proof of \cref{thm:generalised-mode-preprocessing}.

\begin{lemma}\label{thm:adjustment}
For any raw term $V |-_{\Sigma,\erase\Omega} t$, if
\begin{equation}\label{eq:strict-derivation}
V |-_{\Sigma,\Omega} t^{\dir{d'}\,g'\,\True} \quad\text{for some mode~$\dir{d'}$ and boolean~$g'$}
\end{equation}
then for any mode~$\dir{d}$, there is a derivation of\/ $V |-_{\Sigma,\Omega} t^{\dir{d}\,g\,s}$ for some booleans $g$~and~$s$.
\end{lemma}

\begin{proof}
Case analysis on $\dir{d}$ and $\dir{d'}$, adding an outermost mode cast to change the given derivation to a different mode if $d \neq d'$.
Note that it is permissible to add an outermost mode cast because the antecedent~(\ref{eq:strict-derivation}) requires the given derivation to be strict.
\end{proof}



\begin{proof}[Proof of \cref{thm:generalised-mode-preprocessing}]
By \cref{thm:adjustment}, it suffices to prove statement~(\ref{eq:strict-derivation}) by induction on~$t$.

\begin{itemize}
\item $\Rule{Var}$ is mapped to $\SynRule{Var}$.
\item $\Rule{Anno}$:
Let $t = (t' \bbcolon A)$.
By the induction hypothesis and \cref{thm:adjustment}, there is a derivation of $V |-_{\Sigma,\Omega} {t'}^{\chk\,g\,s}$ for some booleans $g$~and~$s$, to which we apply $\SynRule{Anno}$.
\item $\Rule{Op}$:
Let $t = \tmOpts$.
By the induction hypotheses and \cref{thm:adjustment}, there is a derivation of $V, \vec x_i |-_{\Sigma,\Omega} {t_i}^{\dir{d_i}\,g_i\,s_i}$ for each sub-term~$t_i$; apply $\Rule{Op}$ to all the derivations.
\vspace{-\topsep-\baselineskip}
\end{itemize}
\end{proof}

Finally, having constructed a generalised mode preprocessor, it is easy to derive an ordinary one.

\begin{corollary}[Mode Preprocessing]\label{thm:mode-preprocessing}
  It is decidable for any raw term $V |-_{\Sigma,\erase\Omega} t$ and mode~$\dir{d}$ whether $V |-_{\Sigma,\Omega} \isTerm{t}^\dir{d}$.
\end{corollary}

\begin{proof}
By \cref{thm:generalised-mode-preprocessing}, there is a derivation of $V |-_{\Sigma,\erase\Omega} t^{\dir{d}\,g\,s}$ for some booleans $g$~and~$s$, and then simply check~$g$ and apply either \cref{thm:Pre?-true} or \cref{thm:Pre?-false} to obtain $V |-_{\Sigma,\Omega} t^\dir{d}$ or $V \not|-_{\Sigma,\Omega} t^\dir{d}$.
\end{proof}

%We can also establish the converse of \cref{thm:Pre?-true,thm:Pre?-false}.
%
%\begin{corollary}\label{thm:toPre?-true}
%If\/ $V |-_{\Sigma,\Omega} t^\dir{d}$, then $V |-_{\Sigma,\Omega} t^{\dir{d}\,\True\,s}$ for some boolean~$s$.
%\end{corollary}
%
%\begin{corollary}\label{thm:toPre?-false}
%If\/ $V \not|-_{\Sigma,\Omega} t^\dir{d}$, then $V |-_{\Sigma,\Omega} t^{\dir{d}\,\False\,s}$ for some boolean~$s$.
%\end{corollary}
%
%Whereas it is possible to prove \cref{thm:toPre?-true} directly by induction, converting a mode derivation to a generalised one, the negated antecedent of \cref{thm:toPre?-false} provides little information for constructing a required derivation, and basically we have to construct such a derivation from scratch for a raw term.
%But this is exactly what generalised mode preprocessing does, so we can simply reuse \cref{thm:generalised-mode-preprocessing}.
%And \cref{thm:toPre?-true} can also be proved easily in the same way.
%
%\begin{proof}[Proofs of \cref{thm:toPre?-true,thm:toPre?-false}]
%
%By \cref{thm:generalised-mode-preprocessing}, there is a derivation of $V |-_{\Sigma,\Omega} t^{\dir{d}\,g\,s}$ for some booleans $g$~and~$s$.
%If $V |-_{\Sigma,\Omega} t^\dir{d}$, then $g$~has to be $\True$ since $g = \False$ leads to a contradiction with \cref{thm:Pre?-false}.
%Symmetrically, if $V \not|-_{\Sigma,\Omega} t^\dir{d}$, then $g$~has to be~$\False$ so as not to contradict \cref{thm:Pre?-true}.
%\end{proof}

\subsection{Annotatability}
\label{sec:annotatability}

\begin{figure}
  \centering\small
  \judgbox{\isTerm{t} \sqsupseteq \isTerm{u}}{A raw term $t$ is more annotated than $u$ (for some bidirectional type system $(\Sigma, \Omega)$)}
  \begin{mathpar}
    \inferrule{\isTerm{t} \sqsupseteq \isTerm{u}}
              {(\isTerm{t} \annotate A) \sqsupseteq \isTerm{u}}\;\Rule{More}
    \and
    \inferrule{\vphantom{x : \Identifier}}
              {\isTerm{x} \sqsupseteq \isTerm{x}}
    \and
    \inferrule{\isTerm{t} \sqsupseteq \isTerm{u}}
              {(\isTerm{t} \annotate A) \sqsupseteq (\isTerm{u} \annotate A)}
    \and
    \inferrule{\isTerm{t_1} \sqsupseteq \isTerm{u_1} \quad \cdots \quad \isTerm{t_n} \sqsupseteq{u_n}}
              {\tmOpts \sqsupseteq \tmOpus}
  \end{mathpar}
  
  \caption{Annotation ordering between raw terms}
  \label{fig:annotation-ordering}
\end{figure}

\citet{Dunfield2021} formulated completeness differently from ours (\cref{thm:completeness}) and proposed \emph{annotatability} as a more suitable name.
Here we formulate annotatability in our theory and discuss its relationship with our completeness and generalised mode preprocessing.

\begin{proposition}[Annotatability]\label{thm:annotatability}
If\/ $\Gamma |-_{\Sigma,\erase\Omega} t : A$, then there exists~$t'$ such that\/ $t' \sqsupseteq t$ and $\Gamma |-_{\Sigma,\Omega} t' :^\dir{d} A$ for some~$\dir{d}$.
\end{proposition}

Defined in \cref{fig:annotation-ordering}, the `annotation ordering' $t' \sqsupseteq t$ means that $t'$~has the same or more annotations than~$t$.
In a sense, annotatability is a reasonable form of completeness: if a term is typable in $(\Sigma, \erase\Omega)$, it may not be typable in $(\Sigma, \Omega)$ directly due to some missing annotations, but will be if those annotations are added correctly.
In our theory, \cref{thm:annotatability} can be easily proved from generalised mode preprocessing.

\begin{proof}[Proof of \cref{thm:annotatability}]
By \cref{thm:generalised-mode-preprocessing}, there is a generalised mode derivation of $V |-_{\Sigma,\erase\Omega} t^{\dir{d}\,g\,s}$ for some mode~$\dir{d}$ and booleans $g$~and~$s$.
Perform induction on this generalised mode derivation to construct a bidirectional typing derivation in the same mode (and also the term~$t'$ and a proof of $t' \sqsupseteq t$, which are determined by the derivations and omitted here).
\begin{itemize}
\item $\SynRule{Var}$, $\SynRule{Anno}$, and $\Rule{Op}$:
The outermost rule of the given typing derivation must be the corresponding one; apply the corresponding bidirectional typing rule to the induction hypotheses.
\item $\ChkRule{Sub}$:
Apply $\ChkRule{Sub}$ to the induction hypothesis.
\item $\SynRule{Missing}$:
This is the interesting case where we map a $\SynRule{Missing}$ rule to an $\SynRule{Anno}$ rule and add to the term a type annotation, which comes from the given typing derivation.
\vspace{-\topsep-\baselineskip}
\end{itemize}
\end{proof}

On the other hand, when using a bidirectional type synthesiser to implement type synthesis~(\cref{thm:implementation}), our completeness is much simpler to use than annotatability, which requires the bidirectional type synthesiser to produce more complex evidence when the synthesis fails.
Annotatability also does not help the user to deal with missing annotations like (generalised) mode preprocessing does: although annotatability seems capable of determining where annotations are missing and even filling them in correctly, its antecedent requires a typing derivation, which is what the user is trying to construct and does not have yet.
Therefore we believe that the notion of annotatability is too complex for what it achieves, and our theory offers simpler and more useful alternatives.

%!TEX root = BiSig.tex

\section{Bidirectional Type Synthesis and Checking} \label{sec:type-synthesis}
This section focuses on defining mode-correctness and deriving bidirectional type synthesis for any mode-correct bidirectional type system $(\Sigma, \Omega)$.
We start with \Cref{sec:mode-correctness} by defining mode-correctness and showing the uniqueness of synthesised types.
This uniqueness means that any two synthesised types for the same raw term $t$ under the same context $\Gamma$ have to be equal.
It will be used especially in \Cref{subsec:bidirectional-synthesis-checking} for the proof of the decidability of bidirectional type synthesis and checking.
Then, we conclude this section with the trichotomy on raw terms in \Cref{subsec:trichotomy}.

\subsection{Mode Correctness}\label{sec:mode-correctness}
As \citet{Dunfield2021} outlined, mode-correctness for a bidirectional typing rule means that 
\begin{enumerate*}
\item each `input' type variable in a premise must be an `output' variable in `earlier' premises, or provided by the conclusion if the rule is checking;
\item each `output' type variable in the conclusion should be some `output' variable in a premise if the rule is synthesising.
\end{enumerate*}
Here `input' variables refer to variables in an extension context and in a checking premise.
It is important to note that the order of premises in a bidirectional typing rule also matters, since synthesised type variables are instantiated incrementally during type synthesis.

Consider the rule $\ChkRule{Abs}$ (\Cref{fig:STLC-bidirectional-typing-derivations}) as an example.
This rule is mode-correct, as the type variables $A$ and $B$ in its only premise are already provided by its conclusion $A \bto B$.
Likewise, the rule $\SynRule{App}$ for an application term $\isTerm{t\;u}$ is mode-correct because:
\begin{enumerate*}
\item the type $A \bto B$ of the first argument $t$ is synthesised, thereby ensuring type variables $A$ and $B$ must be known if successfully synthesised;
\item the type of the second argument $u$ is checked against $A$, which has been synthesised earlier;
\item as a result, the type of an application $t\;u$ can be synthesised.
\end{enumerate*}

Now let us define mode-correctness rigorously.
As we have outlined, the condition of mode-correctness for a synthesising rule is different from that of a checking rule, and the argument order also matters.
Defining the condition directly for a rule, and thus in our setting for an operation, can be somewhat intricate.
Instead, we choose to define the conditions for the argument list---more specifically, triples $\biargvec$ of an extension context $\Delta_i$, a type $A_i$, and a mode $\dir{d_i}$---pertaining to an operation, for an operation, and subsequently for a signature.
We also need some auxiliary definitions for the subset of variables of a type and of an extension context, and the set of variables that have been synthesised:
\begin{definition}
  The finite subset\footnote{%
  There are various definitions for finite subsets of a set within Martin-L\"{o}f type theory.
  However, for our purposes, the choice among these definitions is not a matter of concern.}
  of \emph{(free) variables} of a type $A$ is denoted by $\fv(A)$.
  The subset $\fv(\Delta)$ of variables in an extension context~$\Delta$ is defined by\/ $\fv(\cdot) = \emptyset$ and\/ $\fv(\Delta, A) = \fv(\Delta) \cup \fv(A)$.
  For an argument list $\biargs$, define its set of \emph{synthesised type variables} inductively by 
  \begin{align}
    \label{eq:synvar1}\synvar(\cdot)                   & = \emptyset  \\
    \label{eq:synvar2}\synvar(\biargs, \chkbiarg[n+1]) & = \phantom{\fv(A_{n+1}) \cup {}} \synvar(\biargs) \\
    \label{eq:synvar3}\synvar(\biargs, \synbiarg[n+1]) & = \fv(A_{n+1}) \cup           \synvar(\biargs).
  \end{align}
\end{definition}
This subset contains type variables of a synthesising argument and they are exactly those type variables that will be synthesised during type synthesis.

\begin{definition}\label{def:mode-correctness-args}
  The \emph{mode-correctness} $\MCas\left(\biargvec\right)$ for an argument list $\biargs$ with respect to a subset $S$ of $\Xi$ is a (proof-relevant) predicate defined by \LT{is it a prop?}
  \begin{align}
    \label{eq:MC1} \MCas( \cdot ) & = \top \\
    \label{eq:MC3} \MCas\left(\biargvec, \chkbiarg[n]\right)
                                  & = \fv(\Delta_n, A_n) \subseteq \left( S \cup \synvar\left(\biargvec\right)\right) \land \MCas\left(\biargvec\right) \\
    \label{eq:MC2} \MCas\left(\biargvec, \synbiarg[n]\right) 
                                  & = \phantom{, A_n} \fv(\Delta_n) \subseteq \left( S \cup \synvar\left(\biargvec\right)\right) \land  \MCas\left(\biargvec\right)
  \end{align}
  where \eqref{eq:MC1} means an empty list is always mode-correct.
\end{definition}
This definition encapsulates the idea that every `input' type variable, possibly derived from an extension context~$\Delta_n$ or a checking argument~$A_n$, must be an `output' variable from $\synvar(\biargvec)$ or, if the rule is checking, belong to the set $S$ of `input' variables in its conclusion.
This condition must be met for every tail of the argument list as well to ensure that `output' variables accessible at each argument position are from preceding arguments only, hence an inductive definition.
\begin{definition}\label{def:mode-correctness}
  An operation $\biop$ is \emph{mode-correct} if 
  \begin{enumerate}
    \item either $d$ is $\chk$, its argument list is mode-correct with respect to $\fv(A_0)$, and the union $\fv(A_0) \cup \synvar(\biargvec)$ contains every inhabitant of $\Xi$;
    \item or $d$ is $\syn$, its argument list is mode-correct with respect to $\emptyset$, and $\synvar(\biargvec)$ contains every inhabitant of $\Xi$ and, particularly, $\fv(A_0)$.
  \end{enumerate}
  A bidirectional binding signature $\Omega$ is \emph{mode-correct} if its operations are all mode-correct.
\end{definition}
For a checking operation, an `input' variable of an argument could be derived from~$A_0$, as these are known during type checking as an input.
Since every inhabitant of~$\Xi$ can be located in either~$A_0$ or synthesised variables, we can determine a concrete type for each inhabitant of~$\Xi$ during type synthesis.
On the other hand, for a synthesising operation, we do not have any known variables at the onset of type synthesis, so the argument list should be mode-correct with respect to~$\emptyset$.
Also, the set of synthesised variables alone should include every type variable in~$\Xi$ and particularly in~$A_n$.

It is easy to check the bidirectional type system $(\Sigma_{\bto}, \Omega^{\Leftrightarrow}_{\Lambda}$) in \Cref{ex:signature-simply-typed-lambda} for simply typed $\lambda$-calculus is mode-correct according to our definition.

\begin{remark}
  Mode-correctness is fundamentally a condition for bidirectional typing \emph{rules}, not for derivations.
  Thus, this property cannot be formalised without treating rules as some mathematical object, such as the notion of bidirectional binding signature presented in \Cref{sec:defs}.
  This contrasts with the properties in \Cref{sec:pre-synthesis}, which can still be specified for individual systems even in the absence of a generic definition of bidirectional type systems.
\end{remark}

Now, we set out to show the uniqueness of synthesised types for a mode-correct bidirectional type system.
For a specific system, its proof is typically a straightforward induction on the typing derivations.
However, since mode-correctness is inductively defined on the argument list, our proof proceeds by induction on both the typing derivations and the argument list:
\begin{lemma}[Uniqueness of Synthesised Types]\label{thm:unique-syn}
  In a mode-correct bidirectional type system $(\Sigma, \Omega)$, the synthesised types of any two derivations
  \[
    \Gamma |-_{\Sigma, \Omega} \isTerm{t} \syn A
    \quad\text{and}\quad
    \Gamma |-_{\Sigma, \Omega} \isTerm{t} \syn B
  \]
  for the same term $t$ must be equal, i.e.\ $A = B$.
\end{lemma}
\begin{proof}%[Proof of \Cref{thm:unique-syn}]
  We prove the statement by induction on derivations $d_1$ and $d_2$ for $\Gamma |-_{\Sigma, \Omega} \isTerm{t} \syn A$ and $\Gamma |-_{\Sigma, \Omega} \isTerm{t} \syn B$.
  Our system is syntax-directed, so $d_1$ and $d_2$ must be derived from the same rule: 
  \begin{itemize}
    \item $\SynRule{Var}$ follows from that each variable as a raw term refers to the same variable in its context.
    \item $\SynRule{Anno}$ holds trivially, since the synthesised type $A$ is from the term $t \annotate A$ in question.
    \item $\Rule{Op}$: Recall that a derivation of\/ $\Gamma |- \tmOpts \syn A$ contains a substitution $\rho$ from the local context $\Xi$ to concrete types.
      To prove that any two typing derivations has the same synthesised type, it suffices to show that those substitutions $\rho_1$~and~$\rho_2$ of $d_1$~and~$d_2$, respectively, agree on variables in $\synvar(\biargs)$ so that $\simsub{A_0}{\rho_1} = \simsub{A_0}{\rho_2}$.
      We prove it by induction on the argument list:
      \begin{enumerate}
        \item For the empty list, the statement is vacuously true by \Cref{eq:synvar1}.
        \item If $\dir{d_{i+1}}$ is~$\chk$, then the statement holds for \Cref{eq:synvar2} by induction hypothesis.
        \item If $\dir{d_{i+1}}$ is~$\syn$, then $\simsub{\Delta_{i+1}}{\rho_1} = \simsub{\Delta_{i+1}}{\rho_2}$ by \Cref{eq:MC2} and induction hypothesis (of the list).
          Therefore, under the same context $\Gamma, \simsub{\Delta_{i+1}}{\rho_1} = \Gamma, \simsub{\Delta_{i+1}}{\rho_2}$ the term $t_{i+1}$ must have the same synthesised type $\simsub{A_{i+1}}{\rho_1} = \simsub{A_{i+1}}{\rho_1}$ by induction hypothesis (of the typing derivation), so $\rho_1$ and $\rho_2$ agree on $\fv(A_{i+1})$ in addition to $\synvar(\biargs)$, as required for \Cref{eq:synvar3} in the definition of $\synvar$.
      \end{enumerate}
    \vspace{-\topsep-\baselineskip}
  \end{itemize}
\end{proof}

%Uniqueness of the synthesised types is a prevalent property in bidirectional type systems, although the specific proofs can vary depending on the constructs in the system.
%For instance, for derivations of $\Gamma |- t\;u \syn B_i$ for $i = 1, 2$ in simply typed $\lambda$-calculus, the hypothesis is applied to their sub-derivations $\Gamma |- t \syn A_i \bto B_i$ to conclude that $A_1 \bto B_1 = A_2 \bto B_2$ and thus $B_1 = B_2$.
%On the other hand, our proof is based on mode-correctness and need not consider specific sub-derivations.

\subsection{Decidability of Bidirectional Type Synthesis and Checking}\label{subsec:bidirectional-synthesis-checking}

We have arrived at the main technical contribution of this paper.

\begin{theorem}[Decidability of Bidirectional Type Synthesis and Checking] \label{thm:bidirectional-type-synthesis-checking}
  In a mode-correct bidirectional type system $(\Sigma, \Omega)$,
  \begin{enumerate}
    \item if\/ $\erase{\Gamma} |-_{\Sigma, \Omega} \isTerm{t}^{\syn}$, then it is decidable whether $\Gamma |-_{\Sigma, \Omega} \isTerm{t} \syn A$ for some $A$;
    \item if\/ $\erase{\Gamma} |-_{\Sigma, \Omega} \isTerm{t}^{\chk}$, then it is decidable for any~$A$ whether $\Gamma |-_{\Sigma, \Omega} \isTerm{t} \chk A$.
  \end{enumerate}
\end{theorem}

The interesting part of the theorem is the case for the $\Rule{Op}$ rule and it is rather complex.
Here we shall give its insight first instead of jumping into the details.
Recall that a typing derivation for $\tmOpts$ contains a substitution $\rho\colon \Xi \to \Type_{\Sigma}(\emptyset)$.
The goal of type synthesis is exactly to define such a substitution $\rho$, so we have to start with an `accumulating' substitution: a substitution $\rho_0$ that is partially defined on $\fv(A_0)$ if $d$ is $\chk$ or otherwise nowhere.
By mode-correctness, the accumulating substitution~$\rho_i$ will be defined on enough synthesised variables so that type synthesis or checking can be performed on $t_{i}$ with the context $\Gamma, \vars{x}_{i} : \simsub{\Delta_{i}}{\rho_{i}}$ based on its mode derivation $\erase{\Gamma}, \vars{x}_i |-_{\Sigma, \Omega} t_i^{\dir{d_i}}$.
If we visit a synthesising argument $\synbiarg[i+1]$, then we may extend the domain of $\rho_i$ to include the synthesised variables $\fv(A_{i + 1})$ if type synthesis is successful and also that the synthesised type can be \emph{unified with $A_{i+ 1}$} and thereby \emph{extend} $\rho_i$ to $\bar{\rho_i} = \rho_{i+1}$ with the unifier.
Then, if we go through every $t_i$ successfully, we will have a total substitution $\rho_n$ by mode-correctness and a derivation of $\Gamma, \vars{x}_i : \Delta_i |-_{\Sigma, \Omega} t_i :^{\dir{d_i}} \simsub{A}{\rho_n}$ for each sub-term $t_i$.

\begin{remark}
To make the argument above sound, it is necessary to compare types and solve a unification problem.
Hence, we assume that the set~$\Xi$ of type variables has a decidable equality, thereby ensuring that the set $\Type_{\Sigma}(\Xi)$ of types also has a decidable equality.\footnote{%
To simplify our choice, we could simply confine $\Xi$ to any set within the family of sets $\Fin(n)$ of naturals less than~$n$, given that these sets have a decidable equality and the arity of a type construct is finite.
Indeed, in our formalisation, we adopt $\Fin(n)$ as the set of type variables in the definition of $\Type_{\Sigma}$ (see \Cref{sec:formalisation} for details).
For the sake of clarity in presentation, though, we keep using named variables and just assume that $\Xi$ has a decidable equality.}
\end{remark}
We need some auxiliary definitions for the notion of extension to state the unification problem:
\begin{definition}
By an \emph{extension}\/ $\sigma \geq \rho$ of a partial substitution $\rho$ we mean that the domain $\dom(\sigma)$ of $\sigma$ contains the domain of $\rho$ and $\sigma(x) = \rho(x)$ for every\/ $x$ in $\dom(\rho)$.
  By a \emph{minimal extension}\/ $\bar{\rho}$ of $\rho$ satisfying $P$ we mean an extension $\bar{\rho} \geq \rho$ with $P(\bar{\rho})$ such that $\sigma \geq \bar{\rho}$ whenever $\sigma \geq \rho$ and $P(\sigma)$.
\end{definition}
\begin{lemma}\label{lem:unify}
  For any\/ $A$ of\/ $\Type_{\Sigma}(\Xi)$, $B$ of\/ $\Type_{\Sigma}(\emptyset)$, and a partial substitution\/ $\rho \colon \Xi \to \Type_{\Sigma}(\emptyset)$, 
  \begin{enumerate}
    \item either there is a minimal extension\/ $\bar{\rho}$ of\/ $\rho$ such that\/ $\simsub{A}{\bar{\rho}} = B$,
    \item or there is no extension\/ $\sigma$ of\/ $\rho$ such that\/ $\simsub{A}{\sigma} = B$
  \end{enumerate}
\end{lemma}
This lemma can be inferred from the correctness of first-order unification~\citep{McBride2003,McBride2003a}, or be proved directly without unification.
We are now ready for the decidability proof.

\begin{proof}[Proof of {\Cref{thm:bidirectional-type-synthesis-checking}}]
  We prove this statement by induction on the mode derivation\/ $\erase{\Gamma} |-_{\Sigma, \Omega} \isTerm{t}^{\dir{d}}$.
  The two cases \SynRule{Var} and \SynRule{Anno} are straightforward and independent of mode-correctness.
  The case \ChkRule{Sub} invokes the uniqueness of synthesised types to refute the case that $\Gamma |-_{\Sigma, \Omega} \isTerm{t} \syn B$ but $A \neq B$ for a given type $A$.
  The first three cases follow essentially the same reasoning provided by \citet{Wadler2022}, but we still present the reasoning here for the sake of completeness.
  The last case \Rule{Op} is new and has been discussed above.
  For brevity we omit the subscript $(\Sigma, \Omega)$.
  \begin{itemize}
    \item \SynRule{Var}: If $\erase{\Gamma} |- \isTerm{t}^{\syn}$, then $(x : A) \in \Gamma$ and thus $\Gamma |- \isTerm{x} \syn A$.

    \item \SynRule{Anno}: For $\erase{\Gamma} |- (\isTerm{t} \annotate A)^{\syn}$, it is decidable whether $\Gamma |- \isTerm{t} \chk A$ by induction hypothesis.
      \begin{itemize}
        \item If $\Gamma |- \isTerm{t} \chk A$, then $\Gamma |- \isTerm{t \annotate A} \syn A$.
        \item If $\Gamma |/- t \chk A$ but $\Gamma |- \isTerm{t \annotate A} \syn$, then by inversion $\Gamma |- t \chk A$, leading to a contradiction.
      \end{itemize}
      
    \item \ChkRule{Sub}: If $\erase{\Gamma} |-t^{\chk}$, then $\Gamma |- t^{\syn}$ by inversion.
      By induction hypothesis, we have two cases: %it is decidable whether $\Gamma |- t \syn B$ for some $B$:
      \begin{itemize}
        \item If $\Gamma |/- t \syn C$ for any $C$ but $\Gamma |- t \chk A$, then by inversion $\Gamma |- t \syn B$ for some $B = A$, thus a contradiction.
        \item If $\Gamma |- t \syn B$ for some $B$, then by decidable equality on $\Type_{\Sigma}(\Xi)$ either $A = B$ or $A \neq B$: 
          \begin{itemize}
            \item if $A = B$ then we have\/ $\Gamma |- t \chk A$;
            \item if $A \neq B$ but $\Gamma |- t \chk A$, then by inversion $\Gamma |- t \syn A$.
              However, by \Cref{thm:unique-syn}, synthesised types $A$ and $B$ must be equal, so we derive a contradiction.
          \end{itemize}
      \end{itemize}
    \item \Rule{Op}:
      For a mode derivation of $\erase{\Gamma} |- \tmOpts^{\dir{d}}$, we first claim:
      \begin{claim}\label{lem:args-induction}
        For an argument list $\biargs$ and any partial substitution $\rho$ from $\Xi$ to $\emptyset$
        \begin{enumerate}
          \item either there is a minimal extension $\ext{\rho}$ of $\rho$ such that 
            \begin{equation} \label{eq:claim}
              \dom(\ext{\rho}) \supseteq \synvar(\biargs)
              \quad\text{and}\quad
%              \text{the domain of $\ext{\rho}$ contains $\synvar(\biargs)$ and\/} \quad
              \Gamma, \vars{x}_\isTerm{i} : \simsub{\Delta_i}{\ext{\rho}} |- \isTerm{t_i} \colon \simsub{A_i}{\ext{\rho}}^{\dir{d_i}}
              \quad\text{for $i = 1, \ldots, n$};
            \end{equation}
          \item or there is no extension $\sigma$ of $\rho$ such that \eqref{eq:claim} holds.
        \end{enumerate}
      \end{claim}

      Then, we proceed with a case analysis on $\dir{d}$ in the mode derivation:
      \begin{itemize}
        \item $\dir{d}$ is $\syn$: We apply our claim with the partial substitution $\rho_0$ defined nowhere.
          \begin{enumerate}
            \item If there is no $\sigma \geq \rho$ such that \eqref{eq:claim} holds but $\Gamma |- \tmOpts \syn A$ for some $A$, then by inversion we have a substitution $\rho\colon \Sub{\Xi}{\emptyset}$ such that
              \[
                \Gamma, \vars{x}_\isTerm{i} : \simsub{\Delta_i}{\rho} |- \isTerm{t_i} \colon \simsub{A_i}{\rho}^{\dir{d_i}}
              \]
              for every $i$.
              Obviously, $\rho \geq \rho_0$ and ${\Gamma, \vars{x}_\isTerm{i} : \simsub{\Delta_i}{\rho} |- \isTerm{t_i} \colon \simsub{A_i}{\rho}^{\dir{d_i}}}$ for every $i$ and it contradicts the assumption that no such an extension exists.

            \item If there exists a minimal $\ext{\rho} \geq \rho_0$ defined on $\synvar(\biargs)$ such that \eqref{eq:claim} holds, then by mode-correctness $\ext{\rho}$ is total and thus
              \[
                \Gamma |- \tmOpts \syn \simsub{A_0}{\ext{\rho}}.
              \]
          \end{enumerate}

        \item $\dir{d}$ is $\chk$: Let $A$ be a type and apply \Cref{lem:unify} with $\rho_0$ defined nowhere.
          \begin{enumerate}
            \item If there is no $\sigma \geq \rho_0$ such that $\simsub{A_0}{\sigma} = A$ but $\Gamma |- \tmOpts \chk A$, then by inversion there is a substitution $\rho$ such that $A = \simsub{A_0}{\rho}$, thus a contradiction.
            \item If there is a minimal $\ext{\rho} \geq \rho_0$ such that $\simsub{A_0}{\ext{\rho}} = A$, then apply our claim with $\ext{\rho}$:
              \begin{enumerate}
                \item If there is no $\sigma \geq \ext{\rho}$ satisfying \eqref{eq:claim} but $\Gamma |- \tmOpts \chk A$, then by inversion there is $\gamma$ such that $\simsub{A_0}{\gamma} = A$ and also ${\Gamma, \vars{x}_\isTerm{i} : \simsub{\Delta_i}{\gamma} |- \isTerm{t_i} \colon \simsub{A_i}{\gamma}^{\dir{d_i}}}$ for every $i$.
                  Given that $\ext{\rho} \geq \rho$ is minimal such that $\simsub{A_0}{\ext{\rho}} = A$, then $\gamma$ is an extension of $\ext{\rho}$ but by assumption no such an extension satisfying ${\Gamma, \vars{x}_\isTerm{i} : \simsub{\Delta_i}{\gamma} |- \isTerm{t_i} \colon \simsub{A_i}{\gamma}^{\dir{d_i}}}$ exists, thus a contradiction.
                
                \item If there is a minimal $\ext{\ext{\rho}} \geq \ext{\rho}$ s.t.\ \eqref{eq:claim}, then by mode-correctness $\ext{\ext{\rho}}$ is total and
                  \[
                    \Gamma |- \tmOpts \chk \simsub{A_0}{\ext{\ext{\rho}}}
                  \]
                  where $\simsub{A_0}{\ext{\ext{\rho}}} = \simsub{A_0}{\ext{\rho}} = A$ since $\ext{\ext{\rho}}(x) = \ext{\rho}$ for every $x$ in the domain of $\ext{\rho}$.
              \end{enumerate}
          \end{enumerate}
      \end{itemize}
      \begin{claimproof}
        We prove it by induction on the list $\biargs$:
        \begin{enumerate}
          \item For the empty list, $\rho$ is the minimal extension of $\rho$ itself satisfying \eqref{eq:claim} trivially. 
          \item For $\biargvec, \biarg[m+1]$, by induction hypothesis on the list, we have two cases:
            \begin{enumerate}
              \item If there is no $\sigma \geq \rho$ such that \eqref{eq:claim} holds for all $1 \leq i \leq m$ but a minimal $\gamma \geq \rho$ such that~\eqref{eq:claim} holds for all $1 \leq i \leq m + 1$, then we clearly have a contradiction.
              \item There is a minimal $\ext{\rho} \geq \rho$ s.t.\ \eqref{eq:claim} holds for $1 \leq i \leq m$.
                By case analysis on $\dir{d_{m+1}}$:
                \begin{itemize}
                  \item $\dir{d_{m+1}}$ is $\chk$: By mode-correctness, $\simsub{\Delta_{m+1}}{\ext{\rho}}$ and $\simsub{A_{m+1}}{\ext{\rho}}$ are defined.
                    By the induction hypothesis $ \Gamma, \vars{x}_\isTerm{m+1} : \simsub{\Delta_{m+1}}{\ext{\rho}} |- \isTerm{t_{m+1}} \chk \simsub{A_{m+1}}{\ext{\rho}}$ is decidable.
                    Clearly, if $\Gamma, \vars{x}_\isTerm{m+1} : \simsub{\Delta_{m+1}}{\ext{\rho}} |- \isTerm{t_{m+1}} \chk \simsub{A_{m+1}}{\ext{\rho}}$ then the desired statement is proved; otherwise we can also easily derive a contradiction.

                  \item $\dir{d_{m+1}}$ is $\syn$: By mode-correctness, $\simsub{\Delta_{m+1}}{\ext{\rho}}$ is defined.
                    By the induction hypothesis, it is decidable that $\Gamma, \vars{x}_\isTerm{m+1} : \simsub{\Delta_{m+1}}{\ext{\rho}} |- \isTerm{t_{m+1}} \syn A$ for some $A$.
                    \begin{enumerate}
                      \item If $\Gamma, \vars{x}_\isTerm{m+1} : \simsub{\Delta_{m+1}}{\ext{\rho}} |/- \isTerm{t_{m+1}} \syn A$ for any $A$ but there is $\gamma \geq \rho$ s.t.\ \eqref{eq:claim} holds for $1 \leq i \leq m+1$, then $\gamma \geq \ext{\rho}$.
                        Therefore $\simsub{\Delta_{m+1}}{\ext{\rho}} = \simsub{\Delta_{m+1}}{\gamma}$ and we derive a contradiction because $\Gamma, \vars{x}_\isTerm{m+1} : \simsub{\Delta_{m+1}}{\ext{\rho}} |- \isTerm{t_{m+1}} \syn \simsub{A_{m+1}}{\gamma}$.
                      \item If $\Gamma, \vars{x}_\isTerm{m+1} : \simsub{\Delta_{m+1}}{\ext{\rho}} |- \isTerm{t_{m+1}} \syn A$ for some $A$, then by \Cref{lem:unify}: % we have two cases: %we unify $A$ with $A_{m+1}$ extending $\ext{\rho}$.
                        \begin{itemize}
                          \item Suppose no $\sigma \geq \ext{\rho}$ such that $\simsub{A_{m+1}}{\sigma} = A$ but an extension $\gamma \geq \rho$ such that \eqref{eq:claim} holds for $1 \leq i \leq m + 1$. 
                            Then, $\gamma \geq \ext{\rho}$ by the minimality of $\ext{\rho}$ and thus
                            $\Gamma, \vars{x}_\isTerm{m+1} : \simsub{\Delta_{m+1}}{\ext{\rho}} |- \isTerm{t_{m+1}} \syn \simsub{A_{m+1}}{\gamma}$.
                            However, by \Cref{thm:unique-syn}, the synthesised type $\simsub{A_{m+1}}{\gamma}$ must be unique, so $\gamma$ is an extension of $\ext{\rho}$ such that $\simsub{A_{m+1}}{\gamma} = A$, leading to a contradiction.
                          \item If there is a minimal $\ext{\ext{\rho}} \geq \ext{\rho}$ such that $\simsub{A_{m+1}}{\ext{\ext{\rho}}} = A$, then it is not hard to show that $\ext{\ext{\rho}}$ is also the minimal extension of $\rho$ such that \eqref{eq:claim} holds for all $1 \leq i \leq m + 1$.
                        \end{itemize}
                    \end{enumerate}
                \end{itemize}
            \end{enumerate}
        \end{enumerate}
        Therefore, we have proved our claim for any argument list by induction.
      \end{claimproof}
  \end{itemize}
  We now have completed the decidability proof by induction on the mode derivation $\erase{\Gamma} |-_{\Sigma, \Omega} t^{\dir{d}}$.
\end{proof}

The formal counterpart of the above proof in \Agda functions as two top-level programs for type checking and synthesis.
These programs either compute the typing derivation or provide a proof of contradiction.
Each case analysis simply branches depending on the outcomes of bidirectional type synthesis and checking for each sub-term, as well as the unification process.
If a contradiction proof is not of interest for implementation, these programs can be simplified by discarding the cases that yield such contradiction proofs.
Alternatively, we could consider generalising typing derivations instead, like our generalised mode derivations (\Cref{fig:generalised-mode-derivations}).
This could accommodate ill-typed cases and reformulate contradiction proofs positively to deliver more informative error messages.
This would assist programmers in resolving issues with ill-typed terms, rather than simply returning a blatant `no'.

\subsection{Trichotomy on Raw Terms by Type Synthesis} \label{subsec:trichotomy}

Combining the bidirectional type synthesiser with the mode decorator, soundness, and completeness from \cref{sec:pre-synthesis}, we get a type synthesiser parameterised by $(\Sigma, \Omega)$, generalising \cref{thm:implementation}.

\begin{corollary}[Trichotomy on Raw Terms]\label{cor:trichotomy}
  For any mode-correct bidirectional type system $(\Sigma, \Omega)$, 
  exactly one of the following holds:
  \LT{Why downplay the pre-condition?}
  \begin{enumerate}
    \item (\/$\erase{\Gamma} |-_{\Sigma, \Omega} \isTerm{t}^{\syn}$ and)\/ $\Gamma |-_{\Sigma, \Omega} \isTerm{t} : A$ for some type~$A$.
    \item (\/$\erase{\Gamma} |-_{\Sigma, \Omega} \isTerm{t}^{\syn}$ but)\/ $\Gamma |/-_{\Sigma, \Omega} \isTerm{t} : A$ for any type~$A$.
    \item \phantom{(\/$\erase{\Gamma} |-_{\Sigma, \Omega} \isTerm{t}^{\syn}$}\llap{$\erase{\Gamma} |/-_{\Sigma, \Omega} \isTerm{t}^{\syn}$\kern.5pt}.
  \end{enumerate}
\end{corollary}

%!TEX root = BiSig.tex

\documentclass[BiSig.tex]{subfiles}

%% ODER: format ==         = "\mathrel{==}"
%% ODER: format /=         = "\neq "
%
%
\makeatletter
\@ifundefined{lhs2tex.lhs2tex.sty.read}%
  {\@namedef{lhs2tex.lhs2tex.sty.read}{}%
   \newcommand\SkipToFmtEnd{}%
   \newcommand\EndFmtInput{}%
   \long\def\SkipToFmtEnd#1\EndFmtInput{}%
  }\SkipToFmtEnd

\newcommand\ReadOnlyOnce[1]{\@ifundefined{#1}{\@namedef{#1}{}}\SkipToFmtEnd}
\usepackage{amstext}
\usepackage{amssymb}
\usepackage{stmaryrd}
\DeclareFontFamily{OT1}{cmtex}{}
\DeclareFontShape{OT1}{cmtex}{m}{n}
  {<5><6><7><8>cmtex8
   <9>cmtex9
   <10><10.95><12><14.4><17.28><20.74><24.88>cmtex10}{}
\DeclareFontShape{OT1}{cmtex}{m}{it}
  {<-> ssub * cmtt/m/it}{}
\newcommand{\texfamily}{\fontfamily{cmtex}\selectfont}
\DeclareFontShape{OT1}{cmtt}{bx}{n}
  {<5><6><7><8>cmtt8
   <9>cmbtt9
   <10><10.95><12><14.4><17.28><20.74><24.88>cmbtt10}{}
\DeclareFontShape{OT1}{cmtex}{bx}{n}
  {<-> ssub * cmtt/bx/n}{}
\newcommand{\tex}[1]{\text{\texfamily#1}}	% NEU

\newcommand{\Sp}{\hskip.33334em\relax}


\newcommand{\Conid}[1]{\mathit{#1}}
\newcommand{\Varid}[1]{\mathit{#1}}
\newcommand{\anonymous}{\kern0.06em \vbox{\hrule\@width.5em}}
\newcommand{\plus}{\mathbin{+\!\!\!+}}
\newcommand{\bind}{\mathbin{>\!\!\!>\mkern-6.7mu=}}
\newcommand{\rbind}{\mathbin{=\mkern-6.7mu<\!\!\!<}}% suggested by Neil Mitchell
\newcommand{\sequ}{\mathbin{>\!\!\!>}}
\renewcommand{\leq}{\leqslant}
\renewcommand{\geq}{\geqslant}
\usepackage{polytable}

%mathindent has to be defined
\@ifundefined{mathindent}%
  {\newdimen\mathindent\mathindent\leftmargini}%
  {}%

\def\resethooks{%
  \global\let\SaveRestoreHook\empty
  \global\let\ColumnHook\empty}
\newcommand*{\savecolumns}[1][default]%
  {\g@addto@macro\SaveRestoreHook{\savecolumns[#1]}}
\newcommand*{\restorecolumns}[1][default]%
  {\g@addto@macro\SaveRestoreHook{\restorecolumns[#1]}}
\newcommand*{\aligncolumn}[2]%
  {\g@addto@macro\ColumnHook{\column{#1}{#2}}}

\resethooks

\newcommand{\onelinecommentchars}{\quad-{}- }
\newcommand{\commentbeginchars}{\enskip\{-}
\newcommand{\commentendchars}{-\}\enskip}

\newcommand{\visiblecomments}{%
  \let\onelinecomment=\onelinecommentchars
  \let\commentbegin=\commentbeginchars
  \let\commentend=\commentendchars}

\newcommand{\invisiblecomments}{%
  \let\onelinecomment=\empty
  \let\commentbegin=\empty
  \let\commentend=\empty}

\visiblecomments

\newlength{\blanklineskip}
\setlength{\blanklineskip}{0.66084ex}

\newcommand{\hsindent}[1]{\quad}% default is fixed indentation
\let\hspre\empty
\let\hspost\empty
\newcommand{\NB}{\textbf{NB}}
\newcommand{\Todo}[1]{$\langle$\textbf{To do:}~#1$\rangle$}

\EndFmtInput
\makeatother
%
%
%
%
%
%
% This package provides two environments suitable to take the place
% of hscode, called "plainhscode" and "arrayhscode". 
%
% The plain environment surrounds each code block by vertical space,
% and it uses \abovedisplayskip and \belowdisplayskip to get spacing
% similar to formulas. Note that if these dimensions are changed,
% the spacing around displayed math formulas changes as well.
% All code is indented using \leftskip.
%
% Changed 19.08.2004 to reflect changes in colorcode. Should work with
% CodeGroup.sty.
%
\ReadOnlyOnce{polycode.fmt}%
\makeatletter

\newcommand{\hsnewpar}[1]%
  {{\parskip=0pt\parindent=0pt\par\vskip #1\noindent}}

% can be used, for instance, to redefine the code size, by setting the
% command to \small or something alike
\newcommand{\hscodestyle}{}

% The command \sethscode can be used to switch the code formatting
% behaviour by mapping the hscode environment in the subst directive
% to a new LaTeX environment.

\newcommand{\sethscode}[1]%
  {\expandafter\let\expandafter\hscode\csname #1\endcsname
   \expandafter\let\expandafter\endhscode\csname end#1\endcsname}

% "compatibility" mode restores the non-polycode.fmt layout.

\newenvironment{compathscode}%
  {\par\noindent
   \advance\leftskip\mathindent
   \hscodestyle
   \let\\=\@normalcr
   \let\hspre\(\let\hspost\)%
   \pboxed}%
  {\endpboxed\)%
   \par\noindent
   \ignorespacesafterend}

\newcommand{\compaths}{\sethscode{compathscode}}

% "plain" mode is the proposed default.
% It should now work with \centering.
% This required some changes. The old version
% is still available for reference as oldplainhscode.

\newenvironment{plainhscode}%
  {\hsnewpar\abovedisplayskip
   \advance\leftskip\mathindent
   \hscodestyle
   \let\hspre\(\let\hspost\)%
   \pboxed}%
  {\endpboxed%
   \hsnewpar\belowdisplayskip
   \ignorespacesafterend}

\newenvironment{oldplainhscode}%
  {\hsnewpar\abovedisplayskip
   \advance\leftskip\mathindent
   \hscodestyle
   \let\\=\@normalcr
   \(\pboxed}%
  {\endpboxed\)%
   \hsnewpar\belowdisplayskip
   \ignorespacesafterend}

% Here, we make plainhscode the default environment.

\newcommand{\plainhs}{\sethscode{plainhscode}}
\newcommand{\oldplainhs}{\sethscode{oldplainhscode}}
\plainhs

% The arrayhscode is like plain, but makes use of polytable's
% parray environment which disallows page breaks in code blocks.

\newenvironment{arrayhscode}%
  {\hsnewpar\abovedisplayskip
   \advance\leftskip\mathindent
   \hscodestyle
   \let\\=\@normalcr
   \(\parray}%
  {\endparray\)%
   \hsnewpar\belowdisplayskip
   \ignorespacesafterend}

\newcommand{\arrayhs}{\sethscode{arrayhscode}}

% The mathhscode environment also makes use of polytable's parray 
% environment. It is supposed to be used only inside math mode 
% (I used it to typeset the type rules in my thesis).

\newenvironment{mathhscode}%
  {\parray}{\endparray}

\newcommand{\mathhs}{\sethscode{mathhscode}}

% texths is similar to mathhs, but works in text mode.

\newenvironment{texthscode}%
  {\(\parray}{\endparray\)}

\newcommand{\texths}{\sethscode{texthscode}}

% The framed environment places code in a framed box.

\def\codeframewidth{\arrayrulewidth}
\RequirePackage{calc}

\newenvironment{framedhscode}%
  {\parskip=\abovedisplayskip\par\noindent
   \hscodestyle
   \arrayrulewidth=\codeframewidth
   \tabular{@{}|p{\linewidth-2\arraycolsep-2\arrayrulewidth-2pt}|@{}}%
   \hline\framedhslinecorrect\\{-1.5ex}%
   \let\endoflinesave=\\
   \let\\=\@normalcr
   \(\pboxed}%
  {\endpboxed\)%
   \framedhslinecorrect\endoflinesave{.5ex}\hline
   \endtabular
   \parskip=\belowdisplayskip\par\noindent
   \ignorespacesafterend}

\newcommand{\framedhslinecorrect}[2]%
  {#1[#2]}

\newcommand{\framedhs}{\sethscode{framedhscode}}

% The inlinehscode environment is an experimental environment
% that can be used to typeset displayed code inline.

\newenvironment{inlinehscode}%
  {\(\def\column##1##2{}%
   \let\>\undefined\let\<\undefined\let\\\undefined
   \newcommand\>[1][]{}\newcommand\<[1][]{}\newcommand\\[1][]{}%
   \def\fromto##1##2##3{##3}%
   \def\nextline{}}{\) }%

\newcommand{\inlinehs}{\sethscode{inlinehscode}}

% The joincode environment is a separate environment that
% can be used to surround and thereby connect multiple code
% blocks.

\newenvironment{joincode}%
  {\let\orighscode=\hscode
   \let\origendhscode=\endhscode
   \def\endhscode{\def\hscode{\endgroup\def\@currenvir{hscode}\\}\begingroup}
   %\let\SaveRestoreHook=\empty
   %\let\ColumnHook=\empty
   %\let\resethooks=\empty
   \orighscode\def\hscode{\endgroup\def\@currenvir{hscode}}}%
  {\origendhscode
   \global\let\hscode=\orighscode
   \global\let\endhscode=\origendhscode}%

\makeatother
\EndFmtInput
%
%
\ReadOnlyOnce{agda.fmt}%


\RequirePackage[T1]{fontenc}
\RequirePackage[utf8]{inputenc}
\RequirePackage{amsfonts}

\providecommand\mathbbm{\mathbb}

% TODO: Define more of these ...
\DeclareUnicodeCharacter{737}{\textsuperscript{l}}
\DeclareUnicodeCharacter{8718}{\ensuremath{\blacksquare}}
\DeclareUnicodeCharacter{8759}{::}
\DeclareUnicodeCharacter{9669}{\ensuremath{\triangleleft}}
\DeclareUnicodeCharacter{8799}{\ensuremath{\stackrel{\scriptscriptstyle ?}{=}}}
\DeclareUnicodeCharacter{10214}{\ensuremath{\llbracket}}
\DeclareUnicodeCharacter{10215}{\ensuremath{\rrbracket}}
\DeclareUnicodeCharacter{27E6}{\ensuremath{\llbracket}}
\DeclareUnicodeCharacter{27E7}{\ensuremath{\rrbracket}}
\DeclareUnicodeCharacter{2200}{\ensuremath{\forall}}

\DeclareUnicodeCharacter{2294}{\ensuremath{\sqcup}}
\DeclareUnicodeCharacter{2080}{\ensuremath{_0}}
\DeclareUnicodeCharacter{2081}{\ensuremath{_1}}
\DeclareUnicodeCharacter{2082}{\ensuremath{_2}}
\DeclareUnicodeCharacter{2083}{\ensuremath{_3}}
\DeclareUnicodeCharacter{2084}{\ensuremath{_4}}

\DeclareUnicodeCharacter{2115}{\ensuremath{\mathbb{N}}}
\DeclareUnicodeCharacter{2236}{:}
\DeclareUnicodeCharacter{2237}{\ensuremath{\mathrel{::}}}
\DeclareUnicodeCharacter{03A3}{\ensuremath{\Sigma}}
\DeclareUnicodeCharacter{039B}{\ensuremath{\Lambda}}
\DeclareUnicodeCharacter{039E}{\ensuremath{\Xi}}

\DeclareUnicodeCharacter{03B9}{\ensuremath{\iota}}
\DeclareUnicodeCharacter{03BB}{\ensuremath{\lambda}}
\DeclareUnicodeCharacter{03C0}{\ensuremath{\pi}}
\DeclareUnicodeCharacter{03C3}{\ensuremath{\sigma}}
\DeclareUnicodeCharacter{03C9}{\ensuremath{\omega}}

\DeclareUnicodeCharacter{2032}{\ensuremath{\prime}}
\DeclareUnicodeCharacter{2113}{\ensuremath{\ell}}
\DeclareUnicodeCharacter{2207}{\ensuremath{\nabla}}
\DeclareUnicodeCharacter{220B}{\ensuremath{\ni}}
\DeclareUnicodeCharacter{2264}{\ensuremath{\leq}}
\DeclareUnicodeCharacter{21D2}{\ensuremath{\Rightarrow}}
\DeclareUnicodeCharacter{22A2}{\ensuremath{\vdash}}
\DeclareUnicodeCharacter{22A4}{\ensuremath{\top}}
\DeclareUnicodeCharacter{22A5}{\ensuremath{\bot}}

\DeclareUnicodeCharacter{1D57}{\ensuremath{^t}}

% TODO: This is in general not a good idea.
\providecommand\textepsilon{$\epsilon$}
\providecommand\textmu{$\mu$}


%Actually, varsyms should not occur in Agda output.

% TODO: Make this configurable. IMHO, italics doesn't work well
% for Agda code.

\renewcommand\Varid[1]{\mathord{\textsf{#1}}}
\let\Conid\Varid
\newcommand\Keyword[1]{\textsf{\textbf{#1}}}

\EndFmtInput



\begin{document}

\section{Formalisation} \label{sec:formalisation}
\begin{enumerate}
  \item (Bidirectional) binding signature, functor, and terms (extrinsic typing, raw terms, raw terms in some mode) --- 2.5 pp
  \item Compare the induction principle and the \Agda proof of Soundness --- 2 pp
  \item type synthesis and checking -- 0.5 p
  \item Examples including STLC (PCF), application in spine form --- 1p

\end{enumerate}

\begin{figure}
  \small
  \begin{hscode}\SaveRestoreHook
\column{B}{@{}>{\hspre}l<{\hspost}@{}}%
\column{3}{@{}>{\hspre}l<{\hspost}@{}}%
\column{5}{@{}>{\hspre}l<{\hspost}@{}}%
\column{7}{@{}>{\hspre}l<{\hspost}@{}}%
\column{14}{@{}>{\hspre}l<{\hspost}@{}}%
\column{16}{@{}>{\hspre}l<{\hspost}@{}}%
\column{19}{@{}>{\hspre}l<{\hspost}@{}}%
\column{24}{@{}>{\hspre}l<{\hspost}@{}}%
\column{E}{@{}>{\hspre}l<{\hspost}@{}}%
\>[3]{}\Keyword{record}\;\Conid{ArgD}\;(\Conid{Ξ}\;\mathbin{:}\;\Conid{ℕ})\;\mathbin{:}\;\Conid{Set}\;\Keyword{where}{}\<[E]%
\\
\>[3]{}\hsindent{2}{}\<[5]%
\>[5]{}\Keyword{field}{}\<[E]%
\\
\>[5]{}\hsindent{2}{}\<[7]%
\>[7]{}\Varid{cxt}\;{}\<[14]%
\>[14]{}\mathbin{:}\;\Conid{Cxt}\;\Conid{Ξ}{}\<[E]%
\\
\>[5]{}\hsindent{2}{}\<[7]%
\>[7]{}\Varid{mode}\;{}\<[14]%
\>[14]{}\mathbin{:}\;\Conid{Mode}{}\<[E]%
\\
\>[5]{}\hsindent{2}{}\<[7]%
\>[7]{}\Varid{type}\;{}\<[14]%
\>[14]{}\mathbin{:}\;\Conid{TExp}\;\Conid{Ξ}{}\<[E]%
\\[\blanklineskip]%
\>[3]{}\Conid{ArgsD}\;\mathbin{:}\;\Conid{ℕ}\;\Varid{→}\;\Conid{Set}{}\<[E]%
\\
\>[3]{}\Conid{ArgsD}\;\Conid{Ξ}\;\mathrel{=}\;\Conid{List}\;(\Conid{ArgD}\;\Conid{Ξ}){}\<[E]%
\\[\blanklineskip]%
\>[3]{}\Keyword{record}\;\Conid{ConD}\;\mathbin{:}\;\Conid{Set}\;\Keyword{where}{}\<[E]%
\\
\>[3]{}\hsindent{2}{}\<[5]%
\>[5]{}\Keyword{constructor}\;\Varid{ι}{}\<[E]%
\\
\>[3]{}\hsindent{2}{}\<[5]%
\>[5]{}\Keyword{field}{}\<[E]%
\\
\>[5]{}\hsindent{2}{}\<[7]%
\>[7]{}\{\mskip1.5mu \Varid{vars}\mskip1.5mu\}\;{}\<[16]%
\>[16]{}\mathbin{:}\;\Conid{ℕ}{}\<[E]%
\\
\>[5]{}\hsindent{2}{}\<[7]%
\>[7]{}\Varid{mode}\;{}\<[16]%
\>[16]{}\mathbin{:}\;\Conid{Mode}{}\<[E]%
\\
\>[5]{}\hsindent{2}{}\<[7]%
\>[7]{}\Varid{type}\;{}\<[16]%
\>[16]{}\mathbin{:}\;\Conid{TExp}\;{}\<[24]%
\>[24]{}\Varid{vars}{}\<[E]%
\\
\>[5]{}\hsindent{2}{}\<[7]%
\>[7]{}\Varid{args}\;{}\<[16]%
\>[16]{}\mathbin{:}\;\Conid{ArgsD}\;\Varid{vars}{}\<[E]%
\\[\blanklineskip]%
\>[3]{}\Keyword{record}\;\Conid{Desc}\;\mathbin{:}\;\Conid{Set₁}\;\Keyword{where}{}\<[E]%
\\
\>[3]{}\hsindent{2}{}\<[5]%
\>[5]{}\Keyword{constructor}\;\Varid{desc}{}\<[E]%
\\
\>[3]{}\hsindent{2}{}\<[5]%
\>[5]{}\Keyword{field}{}\<[E]%
\\
\>[5]{}\hsindent{2}{}\<[7]%
\>[7]{}\Conid{Op}\;{}\<[19]%
\>[19]{}\mathbin{:}\;\Conid{Set}{}\<[E]%
\\
\>[5]{}\hsindent{2}{}\<[7]%
\>[7]{}\{\kern-.9pt\vrule width .75pt height 7.125pt depth 1.975pt\kern-1.5pt\;\Varid{decOp}\;\kern-1.5pt\vrule width .75pt height 7.125pt depth 1.975pt\kern-.9pt\}\;{}\<[19]%
\>[19]{}\mathbin{:}\;\Conid{DecEq}\;\Conid{Op}{}\<[E]%
\\
\>[5]{}\hsindent{2}{}\<[7]%
\>[7]{}\Varid{rules}\;{}\<[19]%
\>[19]{}\mathbin{:}\;\Conid{Op}\;\Varid{→}\;\Conid{ConD}{}\<[E]%
\ColumnHook
\end{hscode}\resethooks
  \caption{Definition of Bidirectional Binding Signature}  
\end{figure}

\end{document}

%!TEX root = BiSig.tex

\section{Concluding Remarks} \label{sec:future}

\Josh{Getting rid of the assumption of being syntax-directed (the triangular picture) --- 0.5p}
\Josh{Ornaments -- 0.25p}
\Josh{NDGP -- 0.25p (optional?), \citep{Ko2022} }

\LT{polymorphic algebraic theories, dependent signatures --- 0.5p}

\paragraph{Beyond simple types}
While algebraic approaches to polymorphic types have been developed~\citep{Fiore2013,Hamana2011}, these approaches do not take subtyping into account.
Subtyping is essential for formulating important concepts such as \emph{principal types} in type synthesis.
On the other hand, in the realm of dependent types, \citeauthor{Cartmell1986}'s generalized algebraic theories~\citeyearpar{Cartmell1986} can handle a wide variety of dependent type theories.
\citet{Bezem2021} investigate the notion of presentation (extending the notion of signature) in the context of generalized algebraic theories.
Nonetheless, type synthesis for dependent types requires normalisation or some form of conversion to check type equality.
Normalisation in its generic form still remains out of reach and recent advances in this topic are discussed in the doctoral thesis by~\citet{Valliappan2023}.

\todo[inline]{We leave the problem of presenting bidirectional typing for more general and advanced language features as future work (such as those related to polymorphic types~\citep{Pierce2000,Peyton-Jones2007,Dunfield2013,Xie2018}), which requires advancement in language formalisation~(\cref{sec:language-formalisation}) that is orthogonal to our work.}

%\subsubsection{Theories of abstract syntax with variable binding}
%\label{sec:theory-of-syntax}
%
%The aforementioned frameworks except \citeauthor{Gheri2020}'s are at least inspired by \varcitet{Fiore1999}{'s} initial semantics for abstract syntax with variable binding using category theory.
%The main idea is that the set of (untyped) abstract syntax trees for a language consists of
%\begin{enumerate*}
%  \item a family of sets $\Term_{\Gamma}$ of well-scoped terms under a context~$\Gamma$ with
%  \item variable renaming for a function $\sigma\colon \Gamma \to \Delta$ between variables acting as a functorial map from $\Term_{\Gamma}$ to $\Term_{\Delta}$, i.e.\ a presheaf $\Term\colon \mathbb{F} \to \mathsf{Set}$, and
%  \item an initial algebra $[\mathsf{v}, \mathsf{op}]$ on~$\Term$ given by the variable rule as a map $\mathsf{v}$ from the presheaf~$V$ of variables (i.e.\ the embedding $V\colon \mathbb{F} \hookrightarrow \mathsf{Set}$) to $\Term$ and other constructs as $\mathsf{op}\colon \mathbb{\Sigma}\Term \to \Term$ where $\mathbb{F}$ is the category of contexts, the functor $\mathbb{\Sigma}\colon \mathsf{Set}^\mathbb{F} \to \mathsf{Set}^\mathbb{F}$ encodes the arities of constructs, and the initiality amounts to structural recursion, i.e.\ \emph{term traversal}.
%\end{enumerate*}
%To put it succinctly, it is the free $\mathbb{\Sigma}$-algebra over the presheaf~$V$ of variables.
%
%Fortunately, constructing the initial algebra of terms in type theory boils down to defining an inductive type with a few constructors that align with the variable rule and a rule schema for language constructs specified by a signature~\citep{Fiore2022}.

%Substitution is also modelled categorically, but it does not play a role in this paper.%
%\todo{Reveal a bit about this paper?}

%\begin{remark} \label{re:type-signature}
%Most of existing theories treat types independently of terms, thereby excluding them from signatures.
%To the best of our knowledge, the only exception to this approach is found in the work of~\citet{Arkor2020}, which incorporates signatures for both terms and types.
%Interestingly, this inclusion is also critical for type synthesis for comparing a concrete type $N \bto N$ with an abstract type $A \bto B$, where $A$ and $B$ are type variables in a typing rule.
%\end{remark}


\begin{acks}
We thank Nathanael Arkor for the useful conversation and to thank the Programming Languages and Formal Methods Laboratory at Academia Sinica for the opportunity of sharing earlier ideas.

The work is supported by the Ministry of Science and Technology of Taiwan under grant MOST 109-2222-E-001-002-MY3.
\end{acks}

\bibliographystyle{ACM-Reference-Format}

\IfFileExists{reference.bib}{%
  \bibliography{reference}
}{\bibliography{ref}}

\end{document}
