\documentclass[acmsmall,screen]{acmart}

\ifPDFTeX
\usepackage[T1]{fontenc}
\usepackage[utf8]{inputenc}
\usepackage[british]{babel}
\else
\usepackage{polyglossia}
\setdefaultlanguage{british}
\fi

\usepackage{bm}
\usepackage{mathtools}
\usepackage{mathpartir}
\input{mathlig}
%\usepackage{xfrac}
\usepackage{xspace}
%\usepackage{hyperref}
\usepackage[capitalise,noabbrev]{cleveref}

%\usepackage[inline]{enumitem} % for environment enumerate*
%\setlist[enumerate]{mode=unboxed}

\setlength {\marginparwidth }{2cm}
\usepackage[color=yellow,textsize=small]{todonotes}

\usepackage{microtype}

\citestyle{acmauthoryear}
\AtEndPreamble{%
\theoremstyle{acmdefinition}
\newtheorem{remark}[theorem]{Remark}}

\input{type-notation} %% for type theory notation copied from HoTT Book
\newcommand{\Agda}{\textsc{Agda}\xspace}

\newcommand{\arity}{\mathit{ar}}

\newcommand{\xto}[1]{\xrightarrow{#1}}
  
\newcommand{\fv}{\mathit{fv}}

\newcommand{\tmOp}{\mathsf{op}}
\newcommand{\tyOp}{\mathsf{op}}

\mathchardef\mhyphen="2D % hyphen in mathmode

\newcommand{\Type}{\mathsf{Ty}}
\newcommand{\Term}{\mathsf{Tm}}
\newcommand{\Cxt}{\mathsf{Cxt}}
\newcommand{\Mode}{\mathsf{Mode}}
\newcommand{\var}{\mathsf{var}}
\newcommand{\simsub}[2]{{#1}\!\left<{#2}\right>}
\newcommand{\bto}{\mathbin{\bm{\supset}}}
\newcommand{\btimes}{\mathbin{\bm{\wedge}}}

\newcommand{\isTerm}[1]{{\textcolor{blue}{#1}}}
\newcommand{\isType}[1]{{\textcolor{red}{#1}}}
\newcommand{\isCxt}[1]{{\textcolor{orange}{#1}}}
\newcommand{\isDir}[1]{{\mathrel{\textcolor{purple}{#1}}}}

\newcommand{\Identifier}{\mathsf{Id}}
\newcommand{\annote}{\mathrel{\boldsymbol{:}}}
\newcommand{\abs}{\mathsf{abs}}
\newcommand{\app}{\mathsf{app}}

\newcommand{\Sub}[2]{\mathsf{Sub}_{\Sigma}(#1,#2)}
\newcommand{\Ren}[2]{\mathsf{Ren}(#1, #2)}
\newcommand{\chk}{\Leftarrow}
\newcommand{\syn}{\Rightarrow}


\newcommand{\LT}[1]{\todo[author=LT,inline,color=green!40,caption={}]{{#1}}}
\newcommand{\Josh}[1]{\todo[author=Josh,inline,color=green!40,caption={}]{{#1}}}

\begin{document}

\author{Liang-Ting Chen}
\email{liang.ting.chen.tw@gmail.com}
\orcid{0000-0002-3250-1331}
\author{Hsiang-Shang Ko}
\orcid{0000-0002-2439-1048}
\email{joshko@iis.sinica.edu.tw}

\affiliation{%
  \institution{Academia Sinica}
  \streetaddress{128 Academia Road}
  \city{Taipei}
  \country{Taiwan}
  \postcode{115}
}

\title{A Theory of Bidirectional Type Synthesis for Simple Types}

\begin{abstract}
  Type checking and inference serve as the passage from parsed abstract syntax trees to well-typed programs.
  While progress has been made in developing techniques for typechecking on a case-by-case basis, the lack of a general theory poses a difficulty of implementing a robust `type checker generator' for diverse languages.

  Motivated by this issue, we develop a \emph{simple} but \emph{general} and \emph{constructive} theory of bidirectional type synthesis where bidirectional type synthesis refers to the mutual use of type checking and inference based on bidirectional typing rules for synthesising the type of a program with type annotations.
  First, our theory is simple but general as it deals with any simple type theory and its bidirectional type system that can be specified by a signature, analogous to the role of grammar in parsing.
  Second, our theory is constructive because it is not only formalised in a proof assistant based on constructive type theory but also formulated positively in order to compute.
  Specifically, a type synthesiser for a simple type theory specified by some mode-correct signature is the proof of logical decidability of synthesis, analogous to a parser generator.
  The notion of bidirectionalisation is introduced to refine annotability in bidirectional typing.
  Type synthesis and bidirectionalisation lead to the trichotomy of raw terms which models a practical type synthesiser that throws an exception when its input does not have enough annotations potentially with a proof to locate missing annotations.
  Hence, by formalizing our general theory of bidirectional type synthesis constructively, we provide a provably correct `type checker generator' for simple type theories, setting the stage for further advances towards a more general theory of type synthesis.
\end{abstract}

\begin{CCSXML}
<ccs2012>
   <concept>
       <concept_id>10011007.10011006.10011041</concept_id>
       <concept_desc>Software and its engineering~Compilers</concept_desc>
       <concept_significance>500</concept_significance>
       </concept>
   <concept>
       <concept_id>10011007.10011006.10011039.10011040</concept_id>
       <concept_desc>Software and its engineering~Syntax</concept_desc>
       <concept_significance>500</concept_significance>
       </concept>
   <concept>
       <concept_id>10011007.10011074.10011099.10011692</concept_id>
       <concept_desc>Software and its engineering~Formal software verification</concept_desc>
       <concept_significance>300</concept_significance>
       </concept>
   <concept>
       <concept_id>10003752.10003790.10011740</concept_id>
       <concept_desc>Theory of computation~Type theory</concept_desc>
       <concept_significance>500</concept_significance>
       </concept>
 </ccs2012>
\end{CCSXML}

\ccsdesc[500]{Software and its engineering~Compilers}
\ccsdesc[500]{Software and its engineering~Syntax}
\ccsdesc[500]{Theory of computation~Type theory}
\ccsdesc[300]{Software and its engineering~Formal software verification}

\keywords{\Agda, abstract syntax, bidirectional typing, type synthesis, type checking, type inference, language formalisation, datatype-generic programming}

\maketitle

%!TEX root = BiSig.tex

\section{Introduction}\label{sec:intro}

Type checking and inference serve as the transition from parsed abstract syntax trees, referred to as \emph{raw terms}, to well-typed programs, known as \emph{well-typed terms}.
Type inference algorithms were developed for determining the type of any unannotated raw term, but it was discovered that full parametric polymorphism makes the type inference problem undecidable~\citep{Wells1999} as well as dependent types~\citep{Dowek1993}.
Recognising the limitations of type inference, the concept of bidirectional type synthesis based on bidirectional typing emerged as an alternative approach, offering decidable algorithms for determining the types of \emph{suitably annotated} programs in programming languages with polymorphic types, dependent types, gradual types, and more.

The core idea of bidirectional type synthesis is simple.
First we start with bidirectional typing which extends typing judgements with two modes that govern the flow of types, hence its name.
That is, typing judgements are classified into two forms:
  \[
    (\text{synthesising})\quad \Gamma \vdash \isTerm{t} \syn A
    \quad\text{and}\quad
    (\text{checking})\quad \Gamma \vdash \isTerm{t} \chk A
  \]
where the type of a term can be synthesised as the output from the information in its context and the term itself as inputs; it can be checked against a type additionally as an input, depending on \emph{the mode of a raw term}.
Then, an algorithm of bidirectional type synthesis can be `read off' based on how type information flows, if the output can be determined by inputs without guessing by design.
  For example, the type of a raw term $\isTerm{t}$ may be synthesised by looking up the variable in its context if $\isTerm{t}$ is a variable or by using a type annotation $A$ provided in the term if $\isTerm{t}$ is of the form $t \annotate A$ after checking $\isTerm{t}$ against the given type~$A$ first, ensuring the validity of the provided annotation.

Type annotations may be seen burdensome at first, but in fact they are useful for programmers to understand the purpose of a program.
This claim was put forth by~\citet{Pierce2000} and supported by software engineering practices, since type annotations are often included as part of top-level definitions and of the documentation even for languages such as variants of Hindley--Damas--Milner type system where (complete) type inference is possible.
As a result, the focus of type-checking technology has shifted from type inference to bidirectional type synthesis, considering both theoretical limitations and practical considerations.

  A recent survey paper by~\citet{Dunfield2021} summarised the design principles of bidirectional typing and its coverage of applications so far.
  Bidirectional type synthesis does not require unification---the traditional technique used in type inference.
Also argued by~\citet{Pierce2000}, missing annotations in bidirectional type synthesis can be recovered using the local information only, so introducing long-distance unification constraints to bidirectional type synthesis actually breaks its premise of being local.
 We may rightly claim that bidirectional type synthesis is the only approach that works in theory and in practice for a variety of languages.

Unlike parsing, however, which has theory and practice that work for a wide class of languages, bidirectional typing has only been developed on a case-by-case basis, lacking a general theory.
Although it is rather easy to `read off' an algorithm of bidirectional type synthesis, a theory that provides explicit specifications with rigorously proven properties is still absent.
By now, we can only offer informal design principles and considerations for individual systems~\citep{Dunfield2021}.
Likewise, there is no off-the-shelf `type-checker generator', and each type checker based on bidirectional typing has to be implemented independently.

To address the aforementioned issue, this paper aims to develop a theory of bidirectional type synthesis and a type-checker generator.
While competing ways exist for designing typing rules, such as those related to polymorphic types by~\citet{Pierce2000,Peyton-Jones2007,Dunfield2013,Xie2018} to name a few, the derivation of an algorithm is relatively standard once a set of bidirectional typing rules are in place.
Therefore, the goal of our theory is not to formulate bidirectional typing for different language features, but rather to define the `grammar' of bidirectional typing rules, referred to as the \emph{bidirectional binding signature} in this paper, along with its essential criteria including \emph{mode-correctness}, \emph{completeness}, and \emph{annotatability} in bidirectional typing that are adequate for deriving a type synthesis algorithm as outlined informally \emph{op.\ cit}.
To illustrate the core idea, we confine our discussion to simple type theories and their bidirectional type systems that have syntax-directed typing rules for simplicity, while more advanced type theories are left for future work.

\subsection{Related work}
As we will mechanise our theory not only for all simple type theories that can be specified by some signature but also with perspectives and techniques different from existing approaches in bidirectional typing, it is timely to take a review of the related work upon which this work is built.

\paragraph{Language formalization and its frameworks}
Formalising the metatheory of a programming language was initiated by the \PoplMark challenge~\citep{Aydemir2005} and this objective has been manifested in a textbook by~\citet{Wadler2022} where programming language concepts are all discussed formally in the proof assistant~\Agda, including bidirectional type synthesis for~\PCF.

It was soon observed that rudimentary operations and meta-properties for variable binding and substitution constitute the bulk of language formalisation, so it is rather unwise to tackle language formalisation on a case-by-case basis.
Hence, there are language-formalisation frameworks by~\citet{Ahrens2018,Fiore2022,Gheri2020,Ahrens2022,Allais2021} to establish substitution, term traversal, and their meta-properties once and for all supported abstract syntax with variable binding.
One of the core ideas of these frameworks is the concept of \emph{descriptions} (coined by~\citet{Chapman2010} for datatype-generic programming) or \emph{binding signatures} (coined by~\citet{Aczel1978} in line with the term in universal algebra) for specifying the set of typing rules of a language.
It is noteworthy that these frameworks only work for simple types at most and none of them can express polymorphic types nor dependent types.

\begin{remark}\label{re:type-synthesis-as-decidability-proof}
It is interesting to note that the algorithm of bidirectional type synthesis by~\citeauthor{Wadler2022} is formulated very differently from the literature:
\begin{enumerate*}
  \item the former algorithm is presented as a (logical) decidability proof of whether a bidirectionally decorated raw term can be bidirectionally typed or equivalently a dependently typed program that takes a raw term $\isTerm{t}$ and returns a typing derivation for $\isTerm{t}$ if there is one or the evidence of a contradiction otherwise;

  \item an algorithm in the latter form is presented as \emph{algorithmic system} relations such as $\Gamma |- \isTerm{t} \syn A \mapsto \isTerm{t}'$, which is read that type annotations can be added to $\isTerm{t}$ in the external language to yield $\isTerm{t}'$ of type $A$ in the internal language, and accompanied with the \emph{soundness} and \emph{completeness} statements.
\end{enumerate*}
\end{remark}


\paragraph{Theories of abstract syntax with variable binding}
\cite{Fiore1999,Hirschowitz2010,Ahrens2018,Fiore2022,Ahrens2021,Arkor2020,Hirschowitz2022}
\cite{Fiore2013,Hamana2011,Hamana2022}


\paragraph{Type checker generation}
\cite{Gast2004,Grewe2015,Pacak2020,Cimini2020}

\subsection{Plan and Contributions of this paper}

Moreover, given this opportunity of re-examining elements of bidirectional typing, we begin our theory from a general idea of type synthesis independent of bidirectional type synthesis and spot subtleties that are rarely, if any, discussed in the literature on bidirectional typing. 

Our approach involves constructing the theory within Martin-L\"of type theory and formalising it using the \Agda proof assistant.
This formalisation provides a generic type synthesis algorithm that is applicable to all calculi employing bidirectional typing for simple types, hence essentially a verified type-checker generator.
By rigorously defining its core principles, we aim to lay the foundation for further advancements in bidirectional typing.

%  Our work addresses the problem with existing methodologies, setting the stage for further advances towards a more general theory of type synthesis.

  %By formalizing our theory of bidirectional type synthesise in \Agda, we provide a framework capable of handling bidirectional type systems of simple types uniformly.
  %Our work addresses the problem with existing methodologies, 

Points to elaborate:
\begin{enumerate}
  \item The existing work of bidirectional typing does not take account of practical aspects such as missing annotation.
    Lightly or rarely discussed.
  \item A meta-problem: existing work is based on classical logic (or left unspecified at least).
    Soundness of completeness of a type checking algorithm in a less straightforward way.
\end{enumerate}
Assumption to justify:
\begin{enumerate}
  \item Use application in spin form to justify our design choice: type expression of arbitrary rank and the number of constructs in a calculus.
  \item Argue that syntax-directed form is essential and practical.
  \item Pfenning's recipe is only of design guide not a technical requirement (there are many exceptions).
\end{enumerate}

Our contributions:
\begin{enumerate}
  \item Generic: Type checking for simple type theories;
  \item Computational formulation: Bidirectionalisation over annotatability; Decidability over soundness and completeness;
  \item Verified and Constructive: Fully formalised in Martin-Löf type theory (with Axiom K).
    LEM is not assumed, so a decidable statement has both computational and non-trivial logical readings.
\end{enumerate}

Plan of this paper:
\begin{enumerate}
  \item Introduction (\Cref{sec:intro}) 
    \LT{1+3 pp}
  \item Key ideas (\Cref{sec:key-ideas})
    \Josh{2.5 pp}
  \item Definitions for bidirectional type systems (bidirectional binding signature, bidirectional type systems, signature erasure and mode annotation) (\Cref{sec:defs})
    \LT{4 pp}
  \item Soundness, completeness, bidirectionalisation, and annotatability (\Cref{sec:pre-synthesis})
    \Josh{Erasure from bidirectional typing derivation to raw terms with a mode}
    \Josh{3 pp}
  \item Bidirectional type inference (\Cref{sec:type-synthesis})
    \LT{4 pp}
  \item Formalisation and further examples (\Cref{sec:formalisation})
    \LT{5.5pp}
  \item Concluding remarks (\Cref{sec:future})
    \LT{1 p} \Josh{1p}
\end{enumerate}
  \LT{better title?}





%!TEX root = BiSig.tex

\section{The Key Ideas} \label{sec:key-ideas}
\begin{figure}
  \centering
  \small
  \begin{mathpar}
    \boxed{\Gamma \vdash \isTerm{t} \syn A}
    \quad\text{and}\quad \boxed{\Gamma \vdash \isTerm{t} \chk A}
    \\
    \inferrule{(x : A) \in \Gamma}{\Gamma \vdash \isTerm{x} \syn A}\,\SynRule{Var}
    \and
    \inferrule{\Gamma \vdash \isTerm{t} \chk B \\ A = B}{\Gamma \vdash \isTerm{t : B}\syn A}\,\SynRule{Anno}
    \and
    \inferrule{\Gamma \vdash \isTerm{t} \syn B \\ A = B}{\Gamma \vdash \isTerm{t} \chk A}\,\ChkRule{Sub}
    \and
    \inferrule{\Gamma \vdash \isTerm{t} \syn A \bto B \\ \Gamma \vdash \isTerm{u} \chk A}{\Gamma \vdash \isTerm{t\;u} \syn B}\,\SynRule{App}
    \and
    \inferrule{\Gamma, \isTerm{x} : A \vdash \isTerm{t} \chk B}{\Gamma \vdash \isTerm{\lam{x}t} \chk A \bto B}\,\ChkRule{Abs}
  \end{mathpar}
  \caption{Bidirectional simply typed $\lambda$-calculus $\Lambda^{\Leftrightarrow}_{\bto}$}
  \label{fig:bi-stlc}
\end{figure}
\begin{remark}
  We avoid using the term `function type' and its conventional notation $\to$ at the object level on purpose, as it may be confused with function types in our type-theoretic meta-language.
\end{remark}



%!TEX root = BiSig.tex

\section{Simple Type Systems and Bidirectional Type Systems}\label{sec:defs}
This section provides generic definitions
%\footnote{%
%Here's the 'small print': Our definitions cover typical examples and set the scope for bidirectional simple type systems discussed, but do not offer comprehensive coverage of all possibilities.}
of simple types, simple type systems, and bidirectional type systems using simply typed $\lambda$-calculus in \Cref{sec:key-ideas} as our running example.

Our definitions are defined in two steps:
\begin{enumerate*}
\item we first introduce a notion of signature which includes a set of operation symbols and an assignment of arities to each operation;
\item then, we define abstract syntax trees inductively with primitive constructs such as $\Rule{Var}$ and constructs specified by a given signature.
\end{enumerate*}
Upon moving from simple types to bidirectionally typed terms, the notion of arity, initially as the number of arguments of an operation, is enriched to incorporate an extension context for variable binding and the direction of type information flow.

\subsection{Signatures and Simple Types} \label{subsec:simple-types}
Simple types of a language is understood intuitively as a collection of types whose formation judgement is not indexed by anything but a fixed set of type variables and the only datum needed for specifying a type construct is its number of arguments: 

\begin{definition} \label{def:simple-signature}
%  \begin{figure}
%%    \begin{minipage}[b]{.6\textwidth}
%      \centering
%      \small
%      \judgbox{\Xi|-_{\Sigma} A}{$A$ is a well-formed type with type variables in $\Xi$}
%      \begin{mathpar}
%        \inferrule{\Xi \ni X_i}{\Xi |-_\Sigma X_i} \and
%        \inferrule{\Xi |-_{\Sigma} A_1 \\ \cdots \\ \Xi |-_{\Sigma} A_n}{\Xi |-_{\Sigma} \tyOp_i(A_1, \ldots, A_n)}\;\text{where $n = \arity(i)$}
%      \end{mathpar}
%      \caption{Simple types}
%      \label{fig:simple-type}
%%    \end{minipage}
%%    \begin{minipage}[b]{.35\textwidth}
%%      \centering
%%      \small
%%      \judgbox{\Xi |-_{\Sigma} \Gamma}{}
%%      \begin{mathpar}
%%        \inferrule{ }{\Xi |-_{\Sigma} \cdot }\and
%%        \inferrule{\Xi |-_{\Sigma} A \\ \Xi|-_{\Sigma} \Gamma}{\Xi |-_{\Sigma} \Gamma, x : A}
%%      \end{mathpar}
%%      \caption{Contexts}
%%    \label{fig:simple-context}
%%    \end{minipage}
%  \end{figure}
  A \emph{signature} $\Sigma$ for simple types consists of a set\footnote{%
    Even though our theory is developed in Martin-L\"of type theory, the term `set' is used instead of `type' to avoid the obvious confusion. 
    As we work in Martin-L\"of type theory with \AxiomK, all types are legitimately sets in the sense of homotopy type theory where all inhabitants of its identity type are proved equal.
  }
  $I$ with a decidable equality and an \emph{arity} function $\arity\colon I \to \mathbb{N}$.
  For a signature $\Sigma$, a \emph{type} $A : \Type_{\Sigma}(\Xi)$ over a variable set $\Xi$ is either
  \begin{enumerate}
    \item a variable in $\Xi$ or
    \item $\tyOp_{i}(A_1, \ldots, A_n)$ for some $i:I$ with $\arity(i) = n$ and types $A_1,\ldots, A_n$.
  \end{enumerate}
  %as defined in \Cref{fig:simple-type}.
  A \emph{context} $\Gamma \colon \Cxt_{\Sigma}(\Xi)$ over $\Xi$ is a list $\Gamma$ of types over $\Xi$. 
\end{definition}

\begin{example} \label{ex:type-signature-for-function-type}
  Simply typed $\lambda$-calculus includes function types $A \bto B$ and typically a base type~$\mathtt{b}$ to ensure that the set of all types without type variables is non-empty.
  The type signature $\Sigma_{\bto}$ used to define types in simply typed $\lambda$-calculus consists of a binary operation $\mathsf{fun}$ and a nullary operation $\mathsf{b}$ where $\arity(\mathsf{fun}) = 2$ and $\arity(\mathsf{b}) = 0$.
  Then, all types in simply typed $\lambda$-calculus can be given as $\Sigma_{\bto}$-types over the empty set with $A \bto B$ introduced as $\tyOp_{\mathsf{fun}}(A, B)$ and $\mathtt{b}$ as $\tyOp_{\mathsf{b}}$. 
\end{example}


As \Cref{def:simple-signature} includes type variables, we introduce the typical meta-operations: substitution.
\begin{definition}
  The \emph{(simultaneous) type substitution} for a function $\rho\colon\Xi \to \Type_{\Sigma}(\Xi')$ is a map which sends $A : \Type_{\Sigma}(\Xi)$ to $\simsub{A}{\rho} : \Type_{\Sigma}(\Xi')$ and is defined as usual.
  By abuse of notation, the substitution $\simsub{\Gamma}{\rho}$ of a context $\Gamma$ is defined by applying substitution $\simsub{A}{\rho}$ to each type $A$ in the context.
\end{definition}

We may confine arbitrary variable set $\Xi$ to the family of sets $\Fin(n)$ of natural numbers less than $n$ as sets of type variables, following the inductive definition by \citet{Dybjer1994}, for the reasons below.
In bidirectional type synthesis, it is necessary to have a decidable equality on $\Type_{\Sigma}(\Xi)$, as we need to compare the synthesized type $B$ with a given type $A$ for the $\Rule{Sub}$ rule.
If $\Xi$ such as $\Fin(n)$ has a decidable equality, the decidable equality on $\Type_{\Sigma}(\Xi)$ can be given.
In addition, the arity of a construct is always finite, so it suffices to use natural numbers to denote type variables in a type.
Therefore, we formally restrict the set $\Xi$ of type variables to the family of $\Fin(n)$, but for illustrative purposes we still use named variables for presentation.
\subsection{Binding Signatures and Simple Type Systems} \label{subsec:binding-sig}

A simple type system consists of
\begin{enumerate*}
  \item a set of raw terms $\isTerm{t}$ indexed by a set $V$ of untyped variables where each construct is allowed to bind some variables like $\Rule{Abs}$ and to take multiple arguments like $\Rule{App}$;
  \item a set of typing derivations indexed by a list $\Gamma$ of typed variables, a type~$A$, and a raw term $\isTerm{t}$ to set their type constraints. 
\end{enumerate*}
Therefore, to specify a construct like $\Rule{Abs}$ and $\Rule{App}$, we need to enrich the notion of signature with a type and an extension context for variable binding for each argument and a type for the operation:
\begin{definition}\label{def:binding-signature}
  For a type signature $\Sigma$, a \emph{binding signature} $\Omega$ consists of a set $O$ and a function
  \[
    \mathit{ar}\colon O \to \sm{\Xi : \N} \left(\Cxt_\Sigma(\Xi) \times \Type_\Sigma(\Xi)\right)^* \times \Type_\Sigma(\Xi).
  \]
\end{definition}
Each inhabitant $o: O$, meant to represent a term construct $\tmOp_o$ in a simple type system, has a triple $\left(\Xi, \left[\left(\Delta_1; A_1\right), \ldots, \left(\Delta_{n}; A_{n}\right) \right], A_0\right)$ as its arity $\arity_o$ where
\begin{enumerate}
  \item $\Xi$ is the number of type variables that are allowed to appear in a typing rule, 
  \item $\left[\left(\Delta_1; A_{1}\right), \ldots, \left(\Delta_{n}; A_{n}\right) \right]$ is a list of pairs for specifying arguments where
    \begin{enumerate}
      \item $\Delta_i : \Cxt_{\Sigma}(\Xi)$ is the \emph{extension context} consisting of types for bound variables and
      \item $A_i : \Term_{\Sigma}(\Xi)$ is for the type of the $i$-th argument (i.e.\ premise) of $\tmOp_{o}$;
    \end{enumerate}
  \item $A_0 : \Type_\Sigma(\Xi)$ is for the type of the construct $\tmOp_o$ in the conclusion.
\end{enumerate}
For brevity, we write $\aritysymbol{o}{\Xi}{[\Delta_1]A_{1}, \ldots, \left[\Delta_{n}\right] A_{n}}{A_0}$ to indicate an operation with its arity and omit $[\Delta]$ if $\Delta$ is empty. 
For example, the $\Rule{Abs}$ rule in \Cref{fig:STLC-typing-derivations} can be specified by the operation 
\[
  \aritysymbol{\mathsf{abs}}{(A, B)}{[A]B}{(A \bto B)}
\]
indicating that in the $\Rule{Abs}$ rule for a raw term $\lam{x}t$
\begin{enumerate*}
  \item there are \emph{two} type variables named $A$ and $B$;
  \item an argument (i.e.\ a typing derivation of $\Gamma, x : A |- t : B$) of type $B$ in a context extended by a variable $x$ of type $A$;
  \item the type $A\bto B$ for the conclusion.
\end{enumerate*}
Similarly, the $\Rule{App}$ rule for $t\;u$ can be specified by $\aritysymbol{\mathsf{app}}{(A, B)}{(A \bto B), A}{B}$ which takes two arguments (i.e.\ typing derivations for $t$ and $u$) under the same context so that their extension contexts are both empty.

Also for brevity, a simple type system $(\Sigma, \Omega)$ refers to the simple type system specified by some type signature $\Sigma$ and a binding signature $\Omega$. 

\begin{definition}
\begin{figure}
  \centering
  \small
  \judgbox{V |-_{\Sigma, \Omega} \isTerm{t}}{Given a list V of variables, $\isTerm{t}$ is a raw term for the signature $(\Sigma, \Omega)$ with free variables in $V$}
  \begin{mathpar}
    \inferrule{\isTerm{x} \in V}{V |-_{\Sigma, \Omega} \isTerm{x}}\,\Rule{Var}
    \and
    \inferrule{\cdot |-_{\Sigma} A \\ V |-_{\Sigma, \Omega}\isTerm{t}}{V |-_{\Sigma, \Omega} \isTerm{t \annotate A}}\,\Rule{Anno}
    \\
    \inferrule{V, \vars{x}_\isTerm{1} |-_{\Sigma, \Omega} \isTerm{t_1} \\ \cdots \\ V, \vars{x}_\isTerm{n} |-_{\Sigma, \Omega} \isTerm{t_n} } {V |-_{\Sigma, \Omega} \tmOpts }\,\Rule{Op} \and
    \text{for $\aritysymbol{o}{\Xi}{[\Delta_1]A_{1}, \ldots, [\Delta_{n}] A_{n}}{A_0}$ in $\Omega$}
  \end{mathpar}
  \caption{Raw terms}
  \label{fig:raw-terms}
\end{figure}
  For a simple type system $(\Sigma, \Omega)$, the set of \emph{raw terms} indexed by a context~$V$ of free variables consists of
  \begin{enumerate}
    \item variables in $V$,
    \item annotations $\isTerm{t \annotate A}$ for some raw term $t$ in $V$ and a type $A$, and
    \item a construct $\tmOpts$ for some $\aritysymbol{o}{\Xi}{[\Delta_1]A_{1}, \ldots, [\Delta_{n}] A_{n}}{A_0}$ in $O$, where $\vars{x}_{\isTerm{i}}$'s are lists of variables whose length is equal to the length of~$\Delta_i$ and $t_i$'s are raw terms in the variable list $V, \vars{x}_i$
  \end{enumerate}
  corresponding to rules $\Rule{Var}$, $\Rule{Anno}$, and $\Rule{Op}$ in \Cref{fig:raw-terms} respectively.
\end{definition}


The definition of typing derivations is a bit more involved.
We need some information to compare types on the object level during type synthesis and substitute those type variables in a typing derivation $\Gamma \vdash \tmOpts : A$ for an operation $o$ in $\Omega$ at some point.
Hence we choose to include a substitution as part of its typing derivation explicitly:
\begin{definition}\label{def:typing-derivations}
  \begin{figure}
    \centering
    \small
    \judgbox{\Gamma |-_{\Sigma, \Omega} \isTerm{t} : A}{A raw term $\isTerm{t}$ has a type $A$ without type variables under $\Gamma$ for the signature $(\Sigma, \Omega)$}
    \begin{mathpar}
      \inferrule{(\isTerm{x} : A) \in \Gamma}{\Gamma |-_{\Sigma, \Omega} \isTerm{x} : A}\,\Rule{Var}
      \and
      \inferrule{\Gamma |-_{\Sigma, \Omega} \isTerm{t} : A}{\Gamma |-_{\Sigma, \Omega} (\isTerm{t \annotate A}) : A}\,\Rule{Anno}
      \and
      \inferrule{\rho : \Sub{\Xi}{\emptyset} \\  \Gamma, \vec{\isTerm{x}}_\isTerm{1} : \simsub{\Delta_{1}}{\rho} |-_{\Sigma, \Omega} \isTerm{t_1} : \simsub{A_{1}}{\rho} \quad\cdots\quad \Gamma, \vec{\isTerm{x}}_\isTerm{n} : \simsub{\Delta_{n}}{\rho} |-_{\Sigma, \Omega} \isTerm{t_n} : \simsub{A_{n}}{\rho}}
      {\Gamma |-_{\Sigma, \Omega} \tmOpts : \simsub{A_0}{\rho}}\,\Rule{Op}
      \\
    \text{for $\aritysymbol{o}{\Xi}{[\Delta_1]A_{1}, \ldots, [\Delta_{n}] A_{n}}{A_0}$ in $\Omega$}
    \end{mathpar}
    \caption{Typing derivations}
    \label{fig:extrinsic-typing}
  \end{figure}
  For a simple type system $(\Sigma, \Omega)$, the set of \emph{typing derivations} of $\Gamma \vdash \isTerm{t} : A$ indexed by a context $\Gamma$, a raw term with free variables in $\erase{\Gamma}$, and a type $A$ consists of 
  \begin{enumerate}
    \item a derivation of $\Gamma |-_{\Sigma, \Omega} x : A$ if $x : A$ is in $\Gamma$,
    \item a derivation of $\Gamma |-_{\Sigma, \Omega} (t \annotate A) : A$ if $\Gamma \vdash_{\Sigma, \Omega} \isTerm{t} : A$ has a derivation, and
    \item a derivation of $\Gamma |-_{\Sigma, \Omega} \tmOpts : \simsub{A_0}{\rho}$ for $\aritysymbol{o}{\Xi}{[\Delta_1]A_1, \ldots, [\Delta_{n}]A_{n}}{A_0}$ if there is a function $\rho\colon \Xi \to \Type_{\Sigma}(\emptyset)$ and a derivation of $\Gamma, \vars{x}_{\isTerm{i}} : \Delta_i \vdash_{\Sigma, \Omega} \isTerm{t_i} : \simsub{A_i}{\rho}$ for each $i$,
  \end{enumerate}
  corresponding to rules $\Rule{Var}$, $\Rule{Anno}$, and $\Rule{Op}$ in \Cref{fig:extrinsic-typing} respectively.
\end{definition}
Note that our simple type system is always syntax-directed, since each rule in \Cref{fig:raw-terms} corresponds exactly to one rule in \Cref{fig:extrinsic-typing}. 
The $\Rule{Op}$ rule in both figures is a rule schema and gives rise to as many rules as there are operations $o : O$.

\begin{example}
  Raw terms (\Cref{fig:STLC-raw-terms}) and typing derivations (\Cref{fig:STLC-typing-derivations}) for simply typed $\lambda$-calculus can be specified by the type signature $\Sigma_{\bto}$ (\Cref{ex:type-signature-for-function-type}) and the binding signature consisting of 
 \[
   \aritysymbol{\mathsf{app}}{(A, B)}{(A \bto B), A}{B}
   \quad\text{and}\quad
   \aritysymbol{\mathsf{abs}}{(A , B)}{[A]B}{(A \bto B)}.
 \]
 Rules $\Rule{Abs}$ and $\Rule{App}$ in simply typed $\lambda$-calculus are subsumed by the $\Rule{Op}$ rule schema, as an application $t\;u$ and an abstraction $\lam{x}t$ can be introduced as $\tmOp_{\mathsf{app}}(t, u)$ and $\tmOp_{\mathsf{abs}}(x.t)$, respectively.
\end{example}

\subsection{Bidirectional Binding Signatures and Bidirectional Type Systems} \label{subsec:bidirectional-system}
Finally, for a bidirectional type system, we have raw terms and typing derivations like a simple type system, but its typing judgements come in two forms $\Gamma |- t \syn A$ and $\Gamma |- t \chk A$.
Observe that these two typing judgements are just one typing judgement $\Gamma |- t :^\dir{d} A$ indexed additionally by a \emph{mode} $d : \Mode$---which is either $\syn$ or $\chk$---so that $\Gamma |- t \syn A$ and $\Gamma |- t \chk A$ are just abbreviations of $\Gamma |- t :^{\syn} A$ and $\Gamma |- t :^{\chk} A$ respectively.
Hence, to specify a bidirectional type system, we need further enrich the notion of binding signature with modes as follows.

\begin{definition} \label{def:bidirectional-binding-signature}
  For a type signature $\Sigma$, a \emph{bidirectional binding signature} $\Omega$ is a set $O$ with a function
  \[
    \mathit{ar}\colon O \to \sum_{\Xi : \N} \left(\Cxt_{\Sigma}(\Xi) \times \Type_{\Sigma}(\Xi) \times {\Mode}\right)^* \times \Type_{\Sigma}(\Xi) \times {\Mode}.
  \]
\end{definition}
Each inhabitant $o:O$ is similarly meant to represent a term construct $\tmOp_o$ in a bidirectional type system but the arity of $o : O$ becomes a $4$-tuple $\left(\Xi, \left[\left(\Delta_1; A_1; d_1\right), \ldots, \left(\Delta_{n}; A_{n}; d_n\right) \right], A_0, d\right)$
including additionally $d$ and $d_i$'s for the modes of the construct $\tmOp_o$ and its arguments respectively.

For brevity, we write $\aritysymbol{o}{\Xi}{[\Delta_1]A_{1}^{\dir{d_1}}, \ldots, [\Delta_{n}] A^{\dir{d_n}}_{n}}{A_0^{\dir{d}}}$ to indicate an operation with its arity. 
For example, the $\ChkRule{Abs}$ rule in \Cref{fig:STLC-bidirectional-typing-derivations} can be specified by $\aritysymbol{\mathsf{abs}}{A , B}{[A]B^{\chk}}{(A \bto B)^{\chk}}$ indicating additionally that $\lam{x}t$ and its argument $t$ are in checking mode.
Similarly, the $\SynRule{App}$ rule can be specified by $\aritysymbol{\mathsf{app}}{A, B}{(A \bto B)^{\syn}, A^{\chk}}{B^{\syn}}$.

We also say a bidirectional type system $(\Sigma, \Omega)$ referring to the bidirectional type system specified by some type signature $\Sigma$ and a bidirectional binding signature $\Omega$.

\begin{remark}
  Every bidirectional binding signature $\Omega$ gives rise to a binding signature $\erase{\Omega}$ if we erase modes from $\Omega$, so a bidirectional binding signature $\Omega$ also specifies a simple type system, including its raw terms and typing derivations.
  We call $\erase{\Omega}$ the \emph{(mode) erasure} of $\Omega$.
\end{remark}


We generalise bidirectional typing rules (\Cref{fig:STLC-bidirectional-typing-derivations}) as well as mode derivations (\Cref{fig:STLC-mode-derivations}) for simply typed $\lambda$-calculus to derivations for a bidirectional type system $(\Sigma, \Omega)$ where $\Omega$ is some bidirectional binding signature, similar to \Cref{def:typing-derivations} but with an additional $\ChkRule{Sub}$ rule:
\begin{definition}\label{def:bidirectional-typing-derivations}\label{def:mode-derivations}
  \begin{figure}
    \centering
    \small
    \judgbox{\Gamma |-_{\Sigma, \Omega} \isTerm{t} :^\dir{d} A}{A raw term $\isTerm{t}$ is of type $A$ without type variables in mode $\dir{d}$ under $\Gamma$}
    \begin{mathpar}
      \inferrule{(x : A) \in \Gamma}{\Gamma |-_{\Sigma, \Omega} \isTerm{x} :^{\syn} A}\,\SynRule{Var}
      \and
      \inferrule{\Gamma |-_{\Sigma, \Omega} \isTerm{t} :^{\chk} A}{\Gamma |-_{\Sigma, \Omega} (\isTerm{t \annotate A}) :^{\syn}  A}\,\SynRule{Anno}
      \and
      \inferrule{\Gamma |-_{\Sigma, \Omega} \isTerm{t} :^{{\syn}} B \\ B = A}{\Gamma |-_{\Sigma, \Omega} \isTerm{t} :^{\chk} A}\,\ChkRule{Sub}
      \\
      \inferrule{\rho\colon \Sub{\Xi}{\emptyset} \\ \Gamma, \simsub{\vars{x}_\isTerm{1} : \Delta_{1}}{\rho} |-_{\Sigma, \Omega} \isTerm{t_{1}} :^\dir{d_1} \simsub{A_1}{\rho} \\
        \cdots \\
      \Gamma, \simsub{\vars{x}_\isTerm{n} : \Delta_{n}}{\rho} |-_{\Sigma, \Omega} \isTerm{t_n} :^{\dir{d_n}} \simsub{A_{n}}{\rho}}
      {\Gamma |-_{\Sigma, \Omega} \tmOpts :^{\dir{d}} \simsub{A_0}{\rho}} \,\mathsf{Op}
      \\ \text{for $\aritysymbol{o}{\Xi}{[\Delta_1]A_{1}^{\dir{d_1}}, \ldots, [\Delta_{n}] A_{n}^{\dir{d_n}}}{A_0^{\dir{d}}}$ in $\Omega$}
    \end{mathpar}
    Abbreviations: $\boxed{\Gamma \vdash_{\Sigma, \Omega} \isTerm{t} \syn A} \defeq \boxed{\Gamma \vdash_{\Sigma, \Omega} \isTerm{t} :^{\syn} A}$ and $\boxed{\Gamma \vdash_{\Sigma, \Omega} \isTerm{t} \chk A} \defeq \boxed{\Gamma \vdash_{\Sigma, \Omega} \isTerm{t} :^{\chk} A}$
    \caption{Bidirectional typing derivations}
    \label{fig:bidirectional-typing-derivations}
  \end{figure}
  \begin{figure}
    \centering
    \small
    \judgbox{V |-_{\Sigma, \Omega} \isTerm{t}^\dir{d}}{A raw term $\isTerm{t}$ is in mode $d$ with free variables in $V$}
    \begin{mathpar}
      \inferrule{x \in V}{V |-_{\Sigma, \Omega} \isTerm{x}^{\syn}}\,\SynRule{Var}
      \and
      \inferrule{\cdot |-_{\Sigma} A \\ V |-_{\Sigma, \Omega}\isTerm{t}^{\chk}}{V |-_{\Sigma, \Omega} (\isTerm{t \annotate A})^{\syn}}\,\SynRule{Anno}
      \and
      \inferrule{V |-_{\Sigma, \Omega} \isTerm{t}^{\syn}}{V |-_{\Sigma, \Omega} \isTerm{t}^{\chk}}\,\ChkRule{Sub}
      \and
      \inferrule{V, \vars{x}_1 |-_{\Sigma, \Omega} \isTerm{t_1}^\dir{d_1} \\ \cdots \\ V, \vars{x}_{n} |-_{\Sigma, \Omega} \isTerm{t_n}^\dir{d_n}}
      {V |-_{\Sigma, \Omega} \tmOpts^\dir{d}}\,\Rule{Op}
      \and \text{for $\aritysymbol{o}{\Xi}{[\Delta_1]A_{1}^{\dir{d_1}}, \ldots, [\Delta_{n}] A^{\dir{d_n}}_{n}}{A_0^{\dir{d}}}$}
    \end{mathpar}
    \caption{Mode derivations}
    \label{fig:mode-derivations}
  \end{figure}
  For a bidirectional type system $(\Sigma, \Omega)$,
  \begin{itemize}
    \item the set of \emph{bidirectional typing derivations} of $\Gamma \vdash_{\Sigma, \Omega} t :^\dir{d} A$ indexed by a context $\Gamma$ of typed variables, a raw term $\isTerm{t}$ under $\erase{\Gamma}$, a mode $\dir{d}$, and a type $A$ is defined in \Cref{fig:bidirectional-typing-derivations} and 
      \begin{itemize}
        \item in particular, a derivation of
          \[
            \Gamma |-_{\Sigma, \Omega} \tmOpts :^{\dir{d}} \simsub{A_0}{\rho}
          \]
          for $\aritysymbol{o}{\Xi}{[\Delta_1]A_{1}^{\dir{d_1}}, \ldots, [\Delta_{n}] A_{n}^{\dir{d_n}}}{A_0^{\dir{d}}}$ in $\Omega$ if there is a function $\rho\colon \Xi \to \Type_{\Sigma}(\emptyset)$ and a derivation of $\Gamma, \vars{x}_{\isTerm{i}} : \Delta_i \vdash_{\Sigma, \Omega} \isTerm{t_i} :^{\dir{d_i}} \simsub{A_i}{\rho}$ for each $i$;
      \end{itemize}
    \item the set of \emph{mode derivations} of $V |-_{\Sigma, \Omega} t^\dir{d}$ indexed by a list $V$ of variables, a raw term $\isTerm{t}$ under $V$, and a mode $d$ is defined in \Cref{fig:mode-derivations}.
  \end{itemize}
\end{definition}

\begin{example}
  Bidirectional typing derivations and mode derivations for simply typed $\lambda$-calculus can be specified by the type signature $\Sigma_{\bto}$ (\Cref{ex:type-signature-for-function-type}) and the bidirectional binding signature consisting of $\aritysymbol{\mathsf{abs}}{A , B}{[A]B^{\chk}}{(A \bto B)^{\chk}}$ and $\aritysymbol{\mathsf{app}}{A, B}{(A \bto B)^{\syn}, A^{\chk}}{B^{\syn}}$.
\end{example}

After developing these generic definitions, we are now able to reason about constructions and properties that hold for any bidirectional simple type system $(\Sigma, \Omega)$.

%!TEX root = BiSig.tex

\section{Soundness, Completeness, and Bidirectionalisation} \label{sec:annotatability}

\subsection{Soundness and Completeness}
\begin{theorem}[Soundness]\label{thm:term-soundness}
    
\end{theorem}

\begin{theorem}[Completeness]\label{thm:term-completeness}
    
\end{theorem}

\subsection{Bidirectionalisation}


\begin{proposition}[Decidability for Bidirectionalisation] \label{thm:bidirectionalisation}
  It is decidable whether a raw term~$t$ is in mode~$d$ for any $t$ and $d$.
\end{proposition}

\subsection{Annotatability is not Completeness}

\begin{figure}
  \centering
  \small
  \bgroup
  \renewcommand{\arraystretch}{1.5}
  \begin{tabular}{ r l }
    $\boxed{|-_{\Sigma, \Omega} \isTerm{t}\;\dir{d}}$ & $t$ is a raw term possibly with missing annotation or redundant casts in mode $d$ \\
    $\boxed{|-_{\Sigma, \Omega} \isTerm{t}\;\dir{d}}$ & $t$ is a raw term without missing annotation \\
    $\boxed{|-_{\Sigma, \Omega} \isTerm{t}\;\dir{d}}$ & $t$ is a raw term without redundant casts \\
  \end{tabular}
  \egroup
  \LT{Design a notation for \textsf{Pre?} so that a raw term without missing annotation or redundant casts is exactly the intersection of these two cases.}
  \begin{mathpar}
    \inferrule{|-_{\Sigma, \Omega}\isTerm{t}^{\chk}}{|-_{\Sigma, \Omega} \isTerm{t}^{\syn}}\,\SynRule{Anno?}
    \and
    \inferrule{x : \Identifier}{|-_{\Sigma, \Omega} \isTerm{x}^{\syn}}
    \and
    \inferrule{|-_{\Sigma, \Omega}\isTerm{t}^{\chk}}{|-_{\Sigma, \Omega} (\isTerm{t \annotate A})^{\syn}}
    \and
    \inferrule{|-_{\Sigma, \Omega} \isTerm{t}^{\syn}}{|-_{\Sigma, \Omega} \isTerm{t}^{\chk}}
    \and
    \inferrule{|-_{\Sigma, \Omega} \isTerm{t_1}^\dir{d_1} \quad \cdots \quad |-_{\Sigma, \Omega} \isTerm{t_n}^\dir{d_n}}
    {|-_{\Sigma, \Omega} \isTerm{\tmOp_o(\vec{x}_1.\, t_1; \ldots;\vec{x}_n.\, t_n)}^\dir{d}}
  \end{mathpar}
  \caption{Preprocessed raw terms, II}
  \label{fig:raw-with-missing-annotations}
\end{figure}

\begin{figure}
  \centering\small
  \judgbox{\isTerm{t} \sqsupseteq \isTerm{u}}{A raw term $t$ is more annotated than $u$ (for some bidirectional type system $(\Sigma, \Omega)$)}
  \begin{mathpar}
    \inferrule{\isTerm{t} \sqsupseteq \isTerm{u}}{(\isTerm{t} : A) \sqsupseteq \isTerm{u}}\;\Rule{More}
    \and
    \inferrule{\vphantom{x : \Identifier}}{\isTerm{x} \sqsupseteq \isTerm{x}}
    \and
    \inferrule{\isTerm{t} \sqsupseteq \isTerm{u}}{(\isTerm{t} \annotate A) \sqsupseteq (\isTerm{u} \annotate A)}
    \and
    \inferrule{\isTerm{t_1} \sqsupseteq \isTerm{u_1} \quad \cdots \quad \isTerm{t_n} \sqsupseteq{u_n}}
    {\isTerm{\tmOp_o(\vec{x}_1. t_1; \ldots; \vec{x}_n. t_n)} \sqsupseteq \isTerm{\tmOp_o(\vec{x}_1.u_1; \ldots; \vec{x}_n.u_n)}}
  \end{mathpar}
  
  \caption{Annotation ordering between raw terms}
  \label{fig:annotation-order}
\end{figure}


%! TEX root = BiSig.tex

\section{Bidirectional Type Synthesis and Checking} \label{sec:type-synthesis}

\subsection{Mode Correctness}

\begin{theorem}[Uniqueness of Synthesised Types]\label{thm:unique-syn}
  Let $(\Sigma, \Omega)$ be a bidirectional type system where $\Omega$ is mode-correct.
  Then, for every raw term $t$ in synthesising mode and two derivations of 
  \[
    \Gamma \vdash_{\Sigma, \Omega} t \Rightarrow A
    \quad\text{and}\quad
    \Gamma \vdash_{\Sigma, \Omega} t \Rightarrow B
  \]
  for some types $A$ and $B$ respectively, the synthesised types must be equal, i.e.\ $A = B$.
\end{theorem}
 

\subsection{Trichotomy of Raw Terms by Type Synthesis}

\begin{theorem}[Decidability of Bidirectional Typing Synthesis] \label{thm:bidirectional-type-synthesis}
  For every type signature $\Sigma$ and bidirectional binding signature $\Omega$ for $\Sigma$ that is mode-correct, the following two statements hold.

  \begin{enumerate}
    \item For every context $\Gamma$ and raw term $\isTerm{t}$ in synthesising mode, it is decidable that there is a type~$A$ and a derivation of
      \[
        \Gamma \vdash_{\Sigma, \Omega} \isTerm{t} \Rightarrow A.
      \]
    \item For every context $\Gamma$, raw term $\isTerm{t}$ in checking mode, and type $A$, it is decidable that there is a derivation of
      \[
        \Gamma \vdash_{\Sigma, \Omega} \isTerm{t} \Leftarrow A.
      \]
  \end{enumerate}
\end{theorem}

\begin{corollary}[Trichotomy of Raw Terms]
  Let $\Sigma$ be a type signature and $\Omega$ a bidirectional binding signature.
  Then, for every context $\Gamma$ and raw term $t$, exactly one of the following statements holds:
  \begin{enumerate}
    \item $t$ is synthesising and there exists a typing derivation of\, $\Gamma \vdash_{\Sigma, \Omega} t : A$ for some type $A$.
    \item $t$ is synthesising but there is no typing derivation of\, $\Gamma \vdash_{\Sigma, \Omega} t : A$ for any type $A$.
    \item $t$ is not synthesising.
  \end{enumerate}
\end{corollary}
\begin{proof}
  Combine  \Cref{thm:term-soundness,thm:term-completeness,thm:bidirectionalisation} with \Cref{thm:bidirectional-type-synthesis}.
  
\end{proof}

\subsection{Decidability of Bidirectional Type Synthesis}

\begin{proof}[Proof of {\Cref{thm:bidirectional-type-synthesis}}]
  We prove this statement by induction on $t$.

  \begin{description}
    \item[(var)] If $x : A$ is in the context $\Gamma$, then there exists a type $A$ such that $\Gamma \vdash_{\Sigma, \Omega} x \Rightarrow A$.
      Otherwise, if $\Gamma \vdash_{\Sigma, \Omega} x \Rightarrow A$ is derivable, then by inversion $(x : A)$ must be in $\Gamma$, a contradiction.
    \item[(anno)]
    \item[(sub)]
%    \item For every mode-correct operation $o \colon \Xi \rhd [\Delta_1]A_{1}^{d_1}, \ldots, [\Delta_{n}] A^{d_n}_{n} \to A^{d}$, a context $\Gamma$, and a raw term
%      \[
%        \isTerm{\tmOp_{o}(\vec{x}_1.t_1,\ldots, \vec{x}_n.t_n)}, 
%      \]
%      with a mode $d$, it is decidable that there exists a substitution $\rho$ from $\Xi$ to $\emptyset$ and a typing derivation of
%      \[
%        \Gamma \vdash_{\Sigma, \Omega} \isTerm{\tmOp_{o}(\vec{x}_1.t_1,\ldots, \vec{x}_n.t_n)} : \simsub{A}{\rho}^\isDir{d}.
%      \]
    \item For every set $\Xi$ of type variables, list of
      \[
        \left(\Delta_i, A_i, d_i\right) : \Cxt_{\Sigma}(\Xi) \times \Type_{\Sigma}(\Xi) \times \Mode, 
      \]
      for $i = 1, \ldots, n$, \emph{partial} substitution $\rho$ from $\Xi$ to $\emptyset$, context $\Gamma$, and raw terms $t_i$'s in mode $d_i$,
      \begin{enumerate}
        \item either there is a minimal extension $\ext{\rho}$ of $\rho$ such that all     of following judgements are derivable
          \[
            \Gamma, \simsub{\Delta_i}{\ext{\rho}} \vdash_{\Sigma, \Omega} t_i \colon \simsub{A_i}{\ext{\rho}}^{d_i}
          \]

        \item or there is no extension $\sigma$ such that all $\Gamma, \simsub{\Delta_i}{\sigma} \vdash_{\Sigma, \Omega} t_i \colon \simsub{A_i}{\sigma}^{d_i}$ has a typing derivation. 
      \end{enumerate}
  \end{description}
\end{proof}

%!TEX root = BiSig.tex

\section{Formalisation} \label{sec:formalisation}
\LT{how to type Agda code without lhs2tex?}
\begin{enumerate}
  \item (Bidirectional) binding signature, functor, and terms (extrinsic typing, raw terms, raw terms in some mode) --- 3 pp
  \item Compare the induction principle and the \Agda proof of Soundness --- 2 pp
  \item type synthesis and checking -- 0.5 p
  \item Examples including STLC (PCF), application in spine form --- 1p
\end{enumerate}

\LT{7 pages in total}

%!TEX root = BiSig.tex

\section{Concluding Remarks} \label{sec:future}

\Josh{Getting rid of the assumption of being syntax-directed (the triangular picture) --- 0.5p}
\Josh{Ornaments -- 0.25p}
\Josh{NDGP -- 0.25p (optional?), \citep{Ko2022} }

\LT{polymorphic algebraic theories, dependent signatures --- 0.5p}

\paragraph{Beyond simple types}
While algebraic approaches to polymorphic types have been developed~\citep{Fiore2013,Hamana2011}, these approaches do not take subtyping into account.
Subtyping is essential for formulating important concepts such as \emph{principal types} in type synthesis.
On the other hand, in the realm of dependent types, \citeauthor{Cartmell1986}'s generalized algebraic theories~\citeyearpar{Cartmell1986} can handle a wide variety of dependent type theories.
\citet{Bezem2021} investigate the notion of presentation (extending the notion of signature) in the context of generalized algebraic theories.
Nonetheless, type synthesis for dependent types requires normalisation or some form of conversion to check type equality.
Normalisation in its generic form still remains out of reach and recent advances in this topic are discussed in the doctoral thesis by~\citet{Valliappan2023}.


\begin{acks}
We thank Nathanael Arkor for the useful conversation and to thank the Programming Languages and Formal Methods Laboratory at Academia Sinica for the opportunity of sharing earlier ideas.

The work is supported by the Ministry of Science and Technology of Taiwan under grant MOST 109-2222-E-001-002-MY3.
\end{acks}

\bibliographystyle{ACM-Reference-Format}

\IfFileExists{reference.bib}{%
  \bibliography{reference}
}{\bibliography{ref}}

\end{document}
