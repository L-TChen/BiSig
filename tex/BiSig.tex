\documentclass[acmsmall,screen]{acmart}
\usepackage[T1]{fontenc}
\usepackage[utf8]{inputenc}
\usepackage[british]{babel}

\usepackage{bm,braket,mathtools,mathpartir,bussproofs}
\EnableBpAbbreviations
\usepackage[all,cmtip]{xy}

\usepackage{xfrac,xspace,xcolor,subcaption}
\usepackage{hyperref}
\usepackage[capitalise,noabbrev]{cleveref}
\usepackage[inline]{enumitem} % for environment enumerate*
\setlist[enumerate]{mode=unboxed}

\setlength {\marginparwidth }{2cm}
\usepackage[color=yellow,textsize=small]{todonotes}
\newcommand{\LT}[1]{\todo[author=LT,inline,color=green!40,caption={}]{{#1}}}
\newcommand{\Josh}[1]{\todo[author=Josh,inline,color=green!40,caption={}]{{#1}}}

\usepackage{subfiles}

\usepackage{microtype}

\citestyle{acmauthoryear}

\AtEndPreamble{%
\theoremstyle{acmdefinition}
\newtheorem{remark}[theorem]{Remark}}

\input{type-notation} %% for type theory notation copied from HoTT Book
\newcommand{\Agda}{\textsc{Agda}\xspace}

\newcommand{\arity}{\mathit{ar}}

\newcommand{\xto}[1]{\xrightarrow{#1}}
  
\newcommand{\fv}{\mathit{fv}}

\newcommand{\tmOp}{\mathsf{op}}
\newcommand{\tyOp}{\mathsf{op}}

\mathchardef\mhyphen="2D % hyphen in mathmode

\newcommand{\Type}{\mathsf{Ty}}
\newcommand{\Term}{\mathsf{Tm}}
\newcommand{\Cxt}{\mathsf{Cxt}}
\newcommand{\Mode}{\mathsf{Mode}}
\newcommand{\var}{\mathsf{var}}
\newcommand{\simsub}[2]{{#1}\!\left<{#2}\right>}
\newcommand{\bto}{\mathbin{\bm{\supset}}}
\newcommand{\btimes}{\mathbin{\bm{\wedge}}}

\newcommand{\isTerm}[1]{{\textcolor{blue}{#1}}}
\newcommand{\isType}[1]{{\textcolor{red}{#1}}}
\newcommand{\isCxt}[1]{{\textcolor{orange}{#1}}}
\newcommand{\isDir}[1]{{\mathrel{\textcolor{purple}{#1}}}}

\newcommand{\Identifier}{\mathsf{Id}}
\newcommand{\annote}{\mathrel{\boldsymbol{:}}}
\newcommand{\abs}{\mathsf{abs}}
\newcommand{\app}{\mathsf{app}}

\newcommand{\Sub}[2]{\mathsf{Sub}_{\Sigma}(#1,#2)}
\newcommand{\Ren}[2]{\mathsf{Ren}(#1, #2)}
\newcommand{\chk}{\Leftarrow}
\newcommand{\syn}{\Rightarrow}


\begin{document}

\author{Liang-Ting Chen}
\email{liang.ting.chen.tw@gmail.com}
\orcid{0000-0002-3250-1331}
\author{Hsiang-Shang Ko}
\orcid{0000-0002-2439-1048}
\email{joshko@iis.sinica.edu.tw}

\affiliation{%
  \institution{Academia Sinica}
  \streetaddress{128 Academia Road}
  \city{Taipei}
  \country{Taiwan}
  \postcode{115}
}

\title{A Practical Metatheory of Bidirectional Type Synthesis for Simple Types}

\begin{abstract}
  This paper proposes a formal theory of bidirectional typing.
\end{abstract}

\maketitle

\section{Introduction}\label{sec:intro}

\begin{enumerate}
  \item Type checking as a transition from (parsed) abstract syntax trees to well-typed terms.
  \item Bidirectional typing is the only solution that works that scales well.
  \item Not yet a general theory but by now a design principle.
    No `type checker generator' as each type checker is still designed and implemented on a case-by-case basis, but can be derived from a syntax-directed type theory `easily'.
  \item The existing work of bidirectional typing does not take account of missing annotations carefully enough. 
  \item A meta-problem: existing work is based on (or at least left unspecified) classical logic.
    One has to state soundness of completeness of their type checking algorithm in a less straightforward way.
  \item A review of what has been done towards this line of research.
  \item Finally, our responses (contributions of this paper).
\end{enumerate}

\subsection{Related work}

\citep{Xie2018}
\subsubsection{Language formalization and its frameworks}
\cite{Wadler2022}
PoplMark Challenge~\citep{Aydemir2005}

\cite{Cimini2020,Cimini2022}
\citep{Fiore2022}
\cite{Allais2021,Gheri2020,Ahrens2022}

\subsubsection{Algebraic theory of type theories}
\cite{Fiore1999,Fiore2022,Ahrens2021,Arkor2020,Hamana2011}


\subsubsection{Type checker generation}
\cite{Gast2004,Grewe2015,Pacak2020,Cimini2020}

\subsubsection{Bidirectional typing}

\cite{Pierce2000}
\cite{Dunfield2021}


\subsection{Plan and Contributions of this paper}

Our assumptions and contributions:
\begin{enumerate}
  \item Assumption: Syntax-directed typing rules
  \item Practical: bidirectionalisation over annotatability, decidability over soundness and completeness.
  \item Generic: Type checking once and for all bidirectional type systems for simple types.
  \item Correct and Constructive: Fully formalised in Martin-Löf type theory (with Axiom K).
    LEM is not assumed, so a decidable statement has both computational and non-trivial logical readings.
\end{enumerate}

Plan of this paper:
\begin{enumerate}
  \item Key ideas (\Cref{sec:key-ideas})
  \item Definitions for simple type theories and for bidirectional type systems (signature erasure and mode annotation) (\Cref{sec:defs})
  \item Soundness, completeness, bidirectionalisation, and annotatability (\Cref{sec:annotatability})
    \Josh{Erasure from bidirectional typing derivation to raw terms with a mode}
  \item Bidirectional type inference (\Cref{sec:type-synthesis})
  \item Formalisation with further examples (\Cref{sec:formalisation})
  \item Future work (\Cref{sec:future})
\end{enumerate}
  \LT{better title?}


\section{Core Ideas with Bidirectional Simply Typed \texorpdfstring{$\lambda$}{lambda}-Calculus} \label{sec:key-ideas}
\begin{figure}
  \centering
  \small
  \begin{mathpar}
    \boxed{\Gamma \vdash \isTerm{t} \Rightarrow A}
    \quad\text{and}\quad \boxed{\Gamma \vdash \isTerm{t} \Leftarrow A}
    \\
    \inferrule{(x : A) \in \Gamma}{\Gamma \vdash \isTerm{x} \Rightarrow A}\;(\text{var}^\Rightarrow)
    \and
    \inferrule{\Gamma \vdash \isTerm{t} \Leftarrow B \\ A = B}{\Gamma \vdash \isTerm{t : B}\Rightarrow A}\;(\text{anno})
    \and
    \inferrule{\Gamma \vdash \isTerm{t} \Rightarrow B \\ A = B}{\Gamma \vdash \isTerm{t\uparrow} \Leftarrow A}\;(\text{sub})
    \and
    \inferrule{\Gamma \vdash \isTerm{t} \Rightarrow A \bto B \\ \Gamma \vdash \isTerm{u} \Leftarrow A}{\Gamma \vdash \isTerm{t\;u} \Rightarrow B}\;(\text{app})
    \and
    \inferrule{\Gamma, \isTerm{x} : A \vdash \isTerm{t} \Leftarrow B}{\Gamma \vdash \isTerm{\lam{x}t} \Leftarrow A \bto B}\;(\text{abs})
  \end{mathpar}
  \caption{The extrinsic style of bidirectional simply typed $\lambda$-calculus with type variables}
  \label{fig:bi-stlc}
\end{figure}
\begin{remark}
  We avoid using the term `function type' and its conventional notation $\to$ at the object level on purpose, as it may be confused with function types in our type-theoretic meta-language.
\end{remark}

\section{Definitions for Simple Type Theories and Bidirectional Type Systems}\label{sec:defs}

\subsection{Simple Types}
\begin{definition}
  A \emph{(simple type) signature} $\Sigma$ consists of a set $I$ and an \emph{arity} function $\arity\colon I \to \mathbb{N}$.
  An inhabitant $i : I$ is meant to represent the \emph{$i$-th operation} $\tyOp_i$ and its \emph{number} $ \arity(i)$ of arguments.

  The judgements of \emph{$\Sigma$-types} and \emph{$\Sigma$-contexts} over a variable set $\Theta$ are defined inductively by \Cref{fig:simple-type,fig:simple-context} respectively.
We write $A : \Type_{\Sigma}(\Theta)$ for $\Theta \vdash_{\Sigma} A$ and $\Gamma \colon \Cxt_{\Sigma}(\Theta)$ for $\Theta \vdash_{\Sigma} \Gamma$ to disambiguate $\Sigma$-types and $\Sigma$-contexts in the case of any confusion.
\end{definition}

\begin{figure}
  \begin{minipage}[b]{.55\textwidth}
    \centering
    \small
    \begin{mathpar}
      \boxed{\Theta\vdash_{\Sigma} A} \\
      \inferrule{\Theta \ni X_i}{\Theta \vdash_\Sigma X_i} \and
      \inferrule{\Theta \vdash_{\Sigma} A_1 \\ \cdots \\ \Theta \vdash_{\Sigma} A_n}{\Theta \vdash_{\Sigma} \tyOp_i(A_1, \ldots, A_n)}
    \end{mathpar}
    where $n = \arity_i$
    \caption{Type formation}
    \label{fig:simple-type}
  \end{minipage}
  \begin{minipage}[b]{.4\textwidth}
    \centering
    \small
    \begin{mathpar}
      \boxed{\Theta \vdash_{\Sigma} \Gamma} \\
      \inferrule{ }{\Theta \vdash_{\Sigma} \cdot }\and
      \inferrule{\Theta \vdash_{\Sigma} A \\ \Theta\vdash_{\Sigma} \Gamma}{\Theta \vdash_{\Sigma} \Gamma, A}
    \end{mathpar}
    \caption{Context formation}
  \label{fig:simple-context}
  \end{minipage}
\end{figure}

\begin{example}\label{ex:implication}
  Simply typed $\lambda$-calculus typically includes a binary implication type denoted by $\bto$ and some base types to ensure that the set of all types is non-empty.
  The type signature $\Sigma_{\bto}$ of simply typed $\lambda$-calculus consists of a binary operation $\bto$ and a base type (nullary operation) $N$.
  In the case of simply typed $\lambda$-calculus with binary products, we can extend $\Sigma_{\bto}$ by adding a binary operation $\btimes$ to represent the binary product type.
  This extended signature is denoted as $\Sigma_{\bto, \btimes}$.
\end{example}

\subsection{Binding Signatures} \label{subsec:binding-sig}
\begin{definition}\label{def:binding-signature}
  For a type signature $\Sigma$, a \emph{binding signature} $\Omega$ consists of a set $O$ and a function
  \[
    \mathit{ar}\colon O \to \sum_{\Xi : \N} \left(\Cxt_\Sigma(\Xi) \times \Type_\Sigma(\Xi)\right)^* \times \Type_\Sigma(\Xi).
  \]
  Each inhabitant $o: O$ is meant to represent a term construct $\tmOp_o$ in a simple type theory with a triple $\left(\Xi, \left[\left(\Delta_1; A_1\right), \ldots, \left(\Delta_{n}; A_{n}\right) \right], A\right)$
  instead of a number as its arity $\arity_o$ where
  \begin{enumerate}
    \item $\Xi$ is the number of type variables, 
    \item $A : \Type_\Sigma(\Xi)$ is the target type of $\tmOp_o$, and
    \item $\left[\left(\Delta_1; A_{1}\right), \ldots, \left(\Delta_{n}; A_{n}\right) \right]$ is a list of pairs where
    \item the $i$-th pair $(\Delta_i, A_i) :\Cxt_{\Sigma}(\Xi) \times \Term_{\Sigma}(\Xi)$ are types of the binding variables and the type of the $i$-th argument of $\tmOp_{o}$.
  \end{enumerate}
  For brevity, we write $o \colon \Xi \rhd (\Delta_1)A_{1}, \ldots, \left[\Delta_{n}\right] A_{n} \to A$ to indicate an operation $o$ with its arity. 
\end{definition}
\begin{remark}[Terminology on binding signature]
  \cite{Aczel1978,Fiore2010}
\end{remark}


\begin{example} \label{ex:STLC-sig}
  Given the type signature $\Sigma_{\bto}$ for the implication type, the term signature $\Lambda_{\bto}$ for simply typed $\lambda$-calculus can be described by operations
  \begin{align*}
    \mathsf{app}\colon A, B \rhd (A \bto B), A \to B && \mathsf{abs}\colon A , B \rhd [A]B \to (A \bto B)
  \end{align*}
    or, verbosely, the signature $\Lambda_{\bto}$ consists of the type $O_\Lambda = \{\app, \abs\}$ with arities
  \begin{align*}
    \arity(\app) = (\{A, B\}, [(\cdot; A \bto B), (\cdot; A)], B)
    && 
    \arity(\abs) = (\{A, B\}, [(\cdot, A; B)], A \bto B)
  \end{align*}
  For the type signature $\Sigma_{\bto, \btimes}$, the term signature $\Omega_{\bto, \btimes}$ for simply typed $\lambda$-calculus with finite products has three additional operations
to $\Omega_{\bto}$:
  \begin{align*}
    \mathsf{pair}\colon A, B \rhd A, B \to A \btimes B
    && \fst \colon A, B \rhd A \btimes B \to A
    && \snd \colon A, B \rhd A \btimes B \to B
  \end{align*}
\end{example}

First, it is noteworthy that the set of term constructs in a type theory need not be finite.
For instance, a type theory may adopt a spine application which takes indefinitely many arguments---for each number $n$ of arguments an $(n+1)$-ary application construct can be introduced, so even if each operation has a definite number of arguments by definition a spine application is still expressible.

More importantly, the inclusion of the set of variables used in an operation is a salient feature of our binding signatures.
Instead of treating application as a family of constructs $\app_{A, B}$ indexed by all types $A$ and $B$, as done by \citet{Fiore2022}, we are able to identify them as a single construct $\app$.
This is not only for brevity but also necessary to compare the type equality during type synthesis and checking.

%\subsection{Intrinsically Typed Terms}
%
%Intrinsically typed terms for a type signature $\Sigma$ and a term signature $\Omega$ are nothing more than derivations of the intrinsic typing judgement $\Gamma \vdash_{\Sigma, \Omega} A$ constructed with only the rule $(\var)$ and the rule scheme $(\tmOp)$ displayed in \Cref{fig:intrinsic-typing}.
%
%The substitution $\rho\colon \Sub{\Xi}{\emptyset}$ is used to instantiate variables in the local context $\Xi$ with concrete types.
%Accordingly, $\rho$ has to be applied to types that appear in the arity to construct a term by $\tmOp$ to ensure all types are well-formed without any use of type variables.
%
%\begin{figure}
%  \centering
%  \small
%  \begin{mathpar}
%    \boxed{\Gamma \vdash_{\Sigma, \Omega} A} \\
%    \inferrule{A \in \Gamma}{\Gamma \vdash_{\Sigma, \Omega} A}\;(\var)
%    \and
%    \inferrule{\rho\colon \Sub{\Xi}{\emptyset}  \\ \Gamma, \simsub{\Delta_{1}}{\rho} \vdash_{\Sigma, \Omega} \simsub{A_{1}}{\rho} \quad\cdots\quad \Gamma, \simsub{\Delta_{n}}{\rho} \vdash_{\Sigma, \Omega} \simsub{A_{n}}{\rho}}
%    {\Gamma \vdash_{\Sigma, \Omega} \simsub{A}{\rho}}\;(\tmOp)
%    \and \text{for $o\colon \Xi \rhd [\Delta_1]A_1, \ldots, [\Delta_{n}]A_{n} \to A$ in $\Omega$}
%  \end{mathpar}
%  \caption{Intrinsic typing rules for a simple type theory $(\Sigma, \Omega)$}
%  \label{fig:intrinsic-typing}
%\end{figure}

\subsection{Simple Type Theory of a Signature}

\begin{figure}
  \centering
  \small
  \begin{mathpar}
    \boxed{\vdash_{\Sigma, \Omega} \isTerm{t}}
    \\
    \inferrule{x : \Identifier}{\vdash_{\Sigma, \Omega} \isTerm{x}}\;(\text{var})
    \and
    \inferrule{\cdot \vdash_{\Sigma} A \\ \vdash_{\Sigma, \Omega}\isTerm{t}}{\vdash_{\Sigma, \Omega} \isTerm{t \annote A}}\;(\text{anno})
    \\
    \inferrule{\vdash_{\Sigma, \Omega} \isTerm{t_1} \quad \cdots \quad \vdash_{\Sigma, \Omega} \isTerm{t_n}}
    {\vdash_{\Sigma, \Omega} \isTerm{\tmOp_o(t_1, \ldots, t_n)}}\;(\text{op}) 
    \and \text{for $o \colon \Xi \rhd [\Delta_1]A_{1}, \ldots, [\Delta_{n}] A_{n} \to A$ in $\Omega$}
  \end{mathpar}
  \caption{Annotatable raw terms for $(\Sigma, \Omega)$}
\end{figure}

\begin{figure}
  \centering
  \small
  \begin{mathpar}
    \boxed{\Gamma \vdash_{\Sigma, \Omega} \isTerm{t} : A} \quad \text{where $\vdash_{\Sigma, \Omega} t$} \\
    \inferrule{(x : A) \in \Gamma}{\Gamma \vdash_{\Sigma, \Omega} \isTerm{x} : A}\;(\var)
    \and
    \inferrule{\Gamma \vdash \isTerm{t} : A}{\Gamma \vdash (\isTerm{t \annote A}) : A}\;(\text{anno})
    \and
    \inferrule{\rho : \Sub{\Xi}{\emptyset} \\ \Gamma, \isTerm{\vec{x}_1} : \simsub{\Delta_{1}}{\rho} \vdash_{\Sigma, \Omega} \isTerm{t_1} : \simsub{A_{1}}{\rho} \quad\cdots\quad \Gamma, \isTerm{\vec{x}_n} : \simsub{\Delta_{n}}{\rho} \vdash_{\Sigma, \Omega} \isTerm{t_n} : \simsub{A_{n}}{\rho}}
    {\Gamma \vdash_{\Sigma, \Omega} \isTerm{\tmOp_o(\vec{x}_1.\,t_1; \ldots; \vec{x}_n.\,t_n)} : \simsub{A}{\rho}}\;(\tmOp)
    \and \text{for $o\colon \Xi \rhd [\Delta_1]A_1, \ldots, [\Delta_{n}]A_{n} \to A$ in $\Omega$}
  \end{mathpar}
  \caption{Typing rules for a simple type theory $(\Sigma, \Omega)$ with annotation}
  \label{fig:extrinsic-typing}
\end{figure}

\subsection{Bidirectional Binding Signatures and their Bidirectional Type Systems}

\begin{definition}
  For a type signature $\Sigma$, a \emph{bidirectional binding signature} $\Omega$ is a set $O$ with a function
  \[
    \mathit{ar}\colon O \to \sum_{\Xi : \N} \left(\Cxt_{\Sigma}(\Xi) \times \Type_{\Sigma}(\Xi) \times \isDir{\Mode}\right)^* \times \Type_{\Sigma}(\Xi) \times \isDir{\Mode}.
  \]
  where $\Mode$ consists of two inhabitants $\chk$ for checking and $\syn$ for synthesis.
  Bidirectional binding signatures are just binding signatures (\Cref{def:binding-signature}) augmented with a mode for each argument and its target of a construct $\tmOp_o$ in a bidirectional system, i.e.
  an arity is a $4$-tuple
  \[
    \left(\Xi, \left[\left(\Delta_1, A_1, d_1\right), \ldots, \left(\Delta_{n}, A_{n}, d_n\right) \right], A, d\right)
  \]
  where $d$ and $d_i$'s indicate the modes of a construct and its arguments respectively.

  For brevity, we write $o \colon \Xi \rhd [\Delta_1]A_{1}^{d_1}, \ldots, [\Delta_{n}] A^{d_n}_{n} \to A^{d}$ to indicate an operation $o$ with its arity. 
\end{definition}

\begin{example}
  The term signature $\Lambda_{\bto}^{\leftrightarrows}$ for bidirectional simply typed $\lambda$-calculus introduced in \Cref{subsec:binding-sig} can be specified by operations 
  \begin{align*}
    \mathsf{app}\colon A, B \rhd (A \bto B)^{\Rightarrow}, A^{\Leftarrow} \to B^{\Rightarrow} &&
    \mathsf{abs}\colon A , B \rhd [A]B^{\Leftarrow} \to (A \bto B)^{\Leftarrow}
  \end{align*}
  extending $\Lambda_{\bto}$ (\Cref{ex:STLC-sig}) with the mode information.
\end{example}

\LT{deriving bidirectional binding signature from annotation}

\LT{Signature Erasure}
\begin{theorem}[Signature annotation]
  
\end{theorem}


\begin{figure}
  \centering
  \small
  \begin{mathpar}
    \boxed{\vdash_{\Sigma, \Omega} \isTerm{t}^\isDir{d}}
    \\
    \inferrule{x : \Identifier}{\vdash_{\Sigma, \Omega} \isTerm{x}^\isDir{\syn}}\;(\text{var})
    \and
    \inferrule{\cdot \vdash_{\Sigma} A \\ \vdash_{\Sigma, \Omega}\isTerm{t}^\isDir{\chk}}{\vdash_{\Sigma, \Omega} \isTerm{t \annote A}^\isDir{\syn}}\;(\text{anno})
    \and
    \inferrule{\vdash_{\Sigma, \Omega} \isTerm{t}^\isDir{\syn}}{\vdash_{\Sigma, \Omega} {\isTerm{t\mathord{\uparrow}}}^\isDir{\chk}}\;(\text{sub})
  \end{mathpar}
  \begin{mathpar}
    \inferrule{\vdash_{\Sigma, \Omega} \isTerm{t_1}^\isDir{d_1} \quad \cdots \quad \vdash_{\Sigma, \Omega} \isTerm{t_n}^\isDir{d_n}}
    {\vdash_{\Sigma, \Omega} \isTerm{\tmOp_o(\vec{x}_1.\, t_1; \ldots;\vec{x}_n.\, t_n)}^\isDir{d}}\;(\text{op})
    \and \text{for $o \colon \Xi \rhd [\Delta_1]A_{1}^{d_1}, \ldots, [\Delta_{n}] A^{d_n}_{n} \to A^{d}$ in $\Omega$}
  \end{mathpar}
  \caption{Raw terms with a mode $d$ for a bidirectional type system $(\Sigma, \Omega)$}
\end{figure}

\begin{figure}
  \centering
  \small
  \begin{mathpar}
    \boxed{\Gamma \vdash_{\Sigma, \Omega} \isTerm{t} : A^\isDir{d}}
    \quad \text{where $\vdash_{\Sigma, \Omega} \isTerm{t}^\isDir{d}$} 
    \\
    \inferrule{(x : A) \in \Gamma}{\Gamma \vdash_{\Sigma, \Omega} \isTerm{x} : A^\syn}\;(\var)
    \and
    \inferrule{\Gamma \vdash_{\Sigma, \Omega} \isTerm{t} : B^\chk \\ A = B}{\mid \Gamma \vdash_{\Sigma, \Omega} (\isTerm{t \annote B}):  A^\syn}\;(\text{anno})
    \and
    \inferrule{\Gamma \vdash_{\Sigma, \Omega} \isTerm{t} : B^{\syn} \\ A = B}{\Gamma \vdash_{\Sigma, \Omega} \isTerm{t\uparrow} : A^\chk}\;(\text{sub})
    \\
    \inferrule{\rho\colon \Sub{\Xi}{\emptyset} \\ \Gamma, \simsub{\isTerm{\vec{x}_{1}} : \Delta_{1}}{\rho} \vdash_{\Sigma, \Omega} \isTerm{t_{1}} : \simsub{A_{1}}{\rho}^{\isDir{d_1}} \\
      \cdots \\
    \Gamma, \simsub{\isTerm{\vec{x}_{n}} : \Delta_{n}}{\rho} \vdash_{\Sigma, \Omega} \isTerm{t_n} : \simsub{A_{n}}{\rho}^{\isDir{d_n}}}
  {\Gamma \vdash_{\Sigma, \Omega} \isTerm{\tmOp_o(\vec{x}_1 .\, t_{1}; \ldots; \vec{x}_{n}.\, t_{n})} : \simsub{A}{\rho}^{\isDir{d}}} \; (\tmOp)
    \and \text{for $o \colon \Xi \rhd [\Delta_1]A_{1}^{d_1}, \ldots, [\Delta_{n}] A^{d_n}_{n} \to A^{d}$ in $\Omega$}
    \\
    \text{Abbreviations: $\boxed{\Gamma \vdash_{\Sigma, \Omega} \isTerm{t} \Rightarrow A} \defeq \boxed{\Gamma \vdash_{\Sigma, \Omega} \isTerm{t} : A^\syn}$ and $\boxed{\Gamma \vdash_{\Sigma, \Omega} \isTerm{t} \Leftarrow A} \defeq \boxed{\Gamma \vdash_{\Sigma, \Omega} \isTerm{t} : A^\chk}$}
  \end{mathpar}
  \caption{Typing rules for a bidirectional system $(\Sigma, \Omega)$}
\end{figure}

\begin{example}
  
\end{example}


\section{Soundness, Completeness, and Bidirectionalisation} \label{sec:annotatability}

\subsection{Soundness and Completeness}

\subsection{Bidirectionalisation}
\begin{theorem}[Decidability for Bidirectionalisation] \label{thm:bidirectionalisation}
  For every raw term $t$ and a mode $d$, it is decidable that $t$ has a mode $d$, i.e.\ $\vdash_{\Sigma, \Omega} \isTerm{t}^\isDir{d}$ is derivable.
\end{theorem}

\subsection{Completeness versus Annotatability}


\section{Bidirectional Type Synthesis and Checking} \label{sec:type-synthesis}

\subsection{Mode Correctness}

\begin{theorem}[Uniqueness of Synthesised Types]\label{thm:unique-syn}
  Let $(\Sigma, \Omega)$ be a bidirectional type system where $\Omega$ is mode-correct.
  Then, for every raw $t$ with a synthesis mode and two derivations of 
  \[
    \Gamma \vdash_{\Sigma, \Omega} t \Rightarrow A
    \quad\text{and}\quad
    \Gamma \vdash_{\Sigma, \Omega} t \Rightarrow B
  \]
  for some types $A$ and $B$ respectively, the synthesised types must be equal, i.e.\ $A = B$.
\end{theorem}
 

\subsection{Trichotomy of Annotatable Raw Terms}

\begin{theorem}[Decidability of Bidirectional Typing Synthesis] \label{thm:bidirectional-type-synthesis}
  For every type signature $\Sigma$ and bidirectional binding signature $\Omega$ for $\Sigma$ that is mode-correct, the following two statements hold.

  \begin{enumerate}
    \item For every context $\Gamma$ and raw term $\isTerm{t}$ with a synthesis mode, it is decidable that there exists a type $A$ and a derivation of
      \[
        \Gamma \vdash_{\Sigma, \Omega} \isTerm{t} \Rightarrow A.
      \]
    \item For every context $\Gamma$, raw term $\isTerm{t}$ with a checking mode, and type $A$, it is decidable that there exists a derivation of
      \[
        \Gamma \vdash_{\Sigma, \Omega} \isTerm{t} \Leftarrow A.
      \]
  \end{enumerate}
\end{theorem}

\begin{corollary}[Trichotomy of Annotatable Raw Terms]
  \LT{trichotomy over decidability?}
  Let $\Sigma$ be a type signature and $\Omega$ a bidirectional binding signature.
  Then, for every context $\Gamma$ and raw term $t$, exactly one of the following statements holds:
  \begin{enumerate}
    \item there exists a typing derivation of $\Gamma \vdash_{\Sigma, \Omega} t : A$ for some type $A$.
    \item there is no typing derivation of $\Gamma \vdash_{\Sigma, \Omega} t : A$ for any type $A$.
    \item $t$ cannot have a synthesis mode.
  \end{enumerate}
\end{corollary}
\begin{proof}
  Combine \Cref{thm:bidirectionalisation} and \Cref{thm:bidirectional-type-synthesis}.
  
\end{proof}



\subsection{Decidability of Bidirectional Type Synthesis}


\begin{proof}[Proof of {\Cref{thm:bidirectional-type-synthesis}}]
  By induction on raw terms $t$ with a mode and for the (op) rule on the list of arguments.
Moreover, if the type $A$ of $t$ is synthesised and checked against $B$ but $A \neq B$, then \Cref{thm:unique-syn} is applied to derive a contradiction.

  \begin{enumerate}
    \item 
        \LT{explain all cases but operation here}
%    \item For every mode-correct operation $o \colon \Xi \rhd [\Delta_1]A_{1}^{d_1}, \ldots, [\Delta_{n}] A^{d_n}_{n} \to A^{d}$, a context $\Gamma$, and a raw term
%      \[
%        \isTerm{\tmOp_{o}(\vec{x}_1.t_1,\ldots, \vec{x}_n.t_n)}, 
%      \]
%      with a mode $d$, it is decidable that there exists a substitution $\rho$ from $\Xi$ to $\emptyset$ and a typing derivation of
%      \[
%        \Gamma \vdash_{\Sigma, \Omega} \isTerm{\tmOp_{o}(\vec{x}_1.t_1,\ldots, \vec{x}_n.t_n)} : \simsub{A}{\rho}^\isDir{d}.
%      \]
    \item For every set $\Xi$ of type variables, list of
      \[
        \left(\Delta_i, A_i, d_i\right) : \Cxt_{\Sigma}(\Xi) \times \Type_{\Sigma}(\Xi) \times \Mode, 
      \]
      \emph{partial} substitution $\rho$ from $\Xi$ to $\emptyset$, context $\Gamma$, and raw terms $t_i$'s with a mode $d_i$,
      \begin{enumerate}
        \item either there is a minimal extension $\ext{\rho}$ of $\rho$ such that all     of following judgements are derivable
          \[
            \Gamma, \simsub{\Delta_i}{\ext{\rho}} \vdash_{\Sigma, \Omega} t_i \colon \simsub{A_i}{\ext{\rho}}^{d_i}
          \]

        \item or there is no extension $\ext{\rho}$ such that all $\Gamma, \simsub{\Delta_i}{\ext{\rho}} \vdash_{\Sigma, \Omega} t_i \colon \simsub{A_i}{\ext{\rho}}^{d_i}$ has a typing derivation. 
      \end{enumerate}
  \end{enumerate}
\end{proof}

\section{Formalisation and More Examples} \label{sec:formalisation}
\LT{how to type Agda code without lhs2tex?}

\begin{enumerate}
  \item Bidirectional binding signature, functor, and terms (extrinsic typing, raw terms, raw terms with a mode, i.e.\ pre-typed terms).
  \item type synthesis and checking.
  \item examples. 
\end{enumerate}



\section{Future Work} \label{sec:future}

\Josh{Getting rid of the assumption of being syntax-directed (the triangular picture)}

\begin{acks}
We thank Nathanael Arkor for the useful conversation and to thank the Programming Languages and Formal Methods Laboratory at Academia Sinica for the opportunity of sharing earlier ideas.

The work is supported by the Ministry of Science and Technology of Taiwan under grant MOST 109-2222-E-001-002-MY3.
\end{acks}

\bibliographystyle{ACM-Reference-Format}

\IfFileExists{reference.bib}{%
  \bibliography{reference}
}{\bibliography{ref}}

\end{document}
