\documentclass[runningheads,orivec,envcountsect,envcountsame]{llncs}

\usepackage[T1]{fontenc}
\usepackage[utf8]{inputenc}
\usepackage[british]{babel}
\usepackage[numbers,sort&compress]{natbib}

\usepackage{soul}
\usepackage{subfiles}
\usepackage{xurl} % allow line breaks anywhere in a URL string
\usepackage[colorlinks,urlcolor=blue,citecolor=blue]{hyperref}

\usepackage{bbold}
\newcommand{\hmmax}{0}
\newcommand{\bmmax}{0}
\usepackage{bm}
\usepackage{mathtools}
\usepackage{stmaryrd}
\usepackage{amssymb}
\input{mathlig}
\usepackage{mathpartir}
\usepackage{annotate-equations}

\usepackage{xspace}
\usepackage[capitalise,noabbrev]{cleveref}

% Used for typesetting Agda code
\IfFileExists{./agda-lhs2tex.sty}{%
  \usepackage{agda-lhs2tex}
  \newcommand{\cons}[1]{\mathbf{##1}}
  \newcommand{\iden}{\mathit}
}{}

\usepackage{xifthen}
\newcommand{\varcitet}[3][]{\citeauthor{#2}#3~\ifthenelse{\isempty{#1}}{\cite{#2}}{\cite[#1]{#2}}}

\usepackage[inline]{enumitem} % for environment enumerate*
\newlist{inlineenum}{enumerate*}{1}
\setlist[inlineenum]{label=(\roman*)}

\usepackage{microtype}
\microtypesetup{babel}

\SetSymbolFont{stmry}{bold}{U}{stmry}{m}{n}

\newcommand{\codefigure}{\small\setlength{\parindent}{0em}\setlength{\mathindent}{0em}\setlength{\abovedisplayskip}{0ex}\setlength{\belowdisplayskip}{0ex}}

\input{type-notation} %% for type theory notation copied from HoTT Book
\newcommand{\Agda}{\textsc{Agda}\xspace}

\newcommand{\arity}{\mathit{ar}}

\newcommand{\xto}[1]{\xrightarrow{#1}}
  
\newcommand{\fv}{\mathit{fv}}

\newcommand{\tmOp}{\mathsf{op}}
\newcommand{\tyOp}{\mathsf{op}}

\mathchardef\mhyphen="2D % hyphen in mathmode

\newcommand{\Type}{\mathsf{Ty}}
\newcommand{\Term}{\mathsf{Tm}}
\newcommand{\Cxt}{\mathsf{Cxt}}
\newcommand{\Mode}{\mathsf{Mode}}
\newcommand{\var}{\mathsf{var}}
\newcommand{\simsub}[2]{{#1}\!\left<{#2}\right>}
\newcommand{\bto}{\mathbin{\bm{\supset}}}
\newcommand{\btimes}{\mathbin{\bm{\wedge}}}

\newcommand{\isTerm}[1]{{\textcolor{blue}{#1}}}
\newcommand{\isType}[1]{{\textcolor{red}{#1}}}
\newcommand{\isCxt}[1]{{\textcolor{orange}{#1}}}
\newcommand{\isDir}[1]{{\mathrel{\textcolor{purple}{#1}}}}

\newcommand{\Identifier}{\mathsf{Id}}
\newcommand{\annote}{\mathrel{\boldsymbol{:}}}
\newcommand{\abs}{\mathsf{abs}}
\newcommand{\app}{\mathsf{app}}

\newcommand{\Sub}[2]{\mathsf{Sub}_{\Sigma}(#1,#2)}
\newcommand{\Ren}[2]{\mathsf{Ren}(#1, #2)}
\newcommand{\chk}{\Leftarrow}
\newcommand{\syn}{\Rightarrow}


\numberwithin{table}{section}

\spnewtheorem{defn}[theorem]{Definition}{\bfseries}{\rmfamily}
\crefformat{defn}{Definition~#2#1#3}

\author{Liang-Ting Chen\orcidID{0000-0002-3250-1331} \and Hsiang-Shang Ko\orcidID{0000-0002-2439-1048}}
\institute{Institute of Information Science, Academia Sinica, Taipei, Taiwan}
\title{A Formal Treatment of Bidirectional Typing}
\authorrunning{L.-T.~Chen and H.-S.~Ko}

\begin{document}

\maketitle

\begin{abstract}
There has been much progress in designing bidirectional type systems and associated type synthesis algorithms, but mainly on a case-by-case basis.
%This situation is in stark contrast to parsing, for which there have been general theories and widely applicable tools such as parser generators.
To remedy the situation, this paper develops a \emph{general} and \emph{formal} theory of bidirectional typing for simply typed languages: for every signature that specifies a mode-correct bidirectionally typed language, there exists a \emph{proof-relevant} type synthesiser which, given an input abstract syntax tree, constructs a typing derivation if any, gives its refutation if not, or reports that the input does not have enough type annotations.
Sufficient conditions for deriving a type synthesiser such as soundness, completeness, and mode-correctness are studied universally for all signatures.
We propose a preprocessing step called \emph{mode decoration}, which helps the user to deal with missing type annotations.
The entire theory is formally implemented in \Agda, so we provide a \emph{verified} generator of proof-relevant type synthesisers as a by-product of our formalism.
\end{abstract}

% Due to ESOP's page limit, the appendix is only available in the ArXiv version.

\newif\ifarxiv 
%\arxivtrue

%!TEX root = BiSig.tex

\section{Introduction}\label{sec:intro}

Type inference is an important mechanism for the transition to well-typed programs from untyped abstract syntax trees, which we call \emph{raw terms}.
Here `type inference' refers specifically to algorithms that ascertain the type of any raw term \emph{without type annotations}.
However, full parametric polymorphism leads to undecidability in type inference, as do dependent types~\cite{Wells1999,Dowek1993}.
In light of these limitations, \emph{bidirectional type synthesis} emerged as a viable alternative, providing algorithms for deciding the types of raw terms that meet some syntactic criteria and usually contain type annotations.
\varcitet{Dunfield2021}{ in their survey paper} summarised the design principles of bidirectional type synthesis and its wide coverage of languages with simple types, polymorphic types, dependent types, gradual types, among others.

The basic idea of bidirectional type synthesis is that while the problem of type inference is not decidable in general, for certain kinds of terms it is still possible to infer their types (for example, the type of a variable can be looked up in the context); for other kinds of terms, we can switch to the simpler problem of type checking, where the expected type of a term is also given so that there is more information to work with.
More formally, every judgement in a bidirectional type system is extended with a \emph{mode:}
\begin{inlineenum}
  \item $\Gamma |- \isTerm{t} \syn A$ for \emph{synthesis} and 
  \item $\Gamma |- \isTerm{t} \chk A$ for \emph{checking}.
\end{inlineenum}
The former indicates that the type~$A$ is computed as output, using both the context~$\Gamma$ and the term~$t$ as input, while for the latter, all three of $\Gamma$, $t$, and~$A$ are given as input.
The algorithm of a bidirectional type synthesiser can usually be `read off' from a well-designed bidirectional type system: as the synthesiser traverses a raw term, it switches between synthesis and checking, following the modes assigned to the judgements in the typing rules.
%if a condition called \emph{mode-correctness} is satisfied.
%every synthesised type in a typing rule is determined by previously synthesised types from its premises and its input if checking.
%As \citet{Pierce2000} noted, bidirectional type synthesis propagates type information locally within adjacent nodes of a term, and does not require unification as Damas--Milner type inference does.
%Moreover, introducing long-distance unification constraints would even undermine the essence of locality in bidirectional type synthesis.

%Also, annotations for, say, top-level definitions make the purpose of a program easier to understand, so they are sometimes beneficial and not necessarily a nuisance.
%In light of these considerations, bidirectional type synthesis can be deemed as a type-checking technique that is more fundamental than unification, as it is capable of handling a broad spectrum of programming languages.%
%\todo{Reorganise into two paragraphs, one on old approaches and the other exclusively on bidirectional type synthesis?}

Despite sharing the same basic idea, bidirectional typing has been mostly developed on a case-by-case basis; this situation is in stark contrast to parsing, which has general theories together with widely applicable techniques and practical tools, notably parser generators.
%\footnote{The same could be said for type checking in general, not just for bidirectional type systems.
%While there are type checker generators grounded in unification~\citep{Gast2004,Grewe2015}, it should be noted that unification-based approaches are not suited to more complex type systems.}
While it is straightforward to derive a type synthesis algorithm,
for bidirectional typing, \citeauthor{Dunfield2021} only present informal design principles that the community learned from individual systems, rather than a general treatment providing logical specifications and rigorously proven properties for a class of systems.
Moreover, unlike the plethora of available parser generators, `type-synthesiser generators' rarely exist, so each type synthesiser has to be independently built, not to mention their correctness.

To mitigate the above problem, we develop with the proof assistant \Agda a general and formal treatment of bidirectional typing and provide a \emph{verified} generator of \emph{proof-relevant} type synthesisers taking a language specification and a (scoped-checked) raw term as input.

\subsubsection{Proof-relevance in type synthesis}
\label{sec:PLFA}

Our work adapts proof-relevant (bidirectional) type synthesis illustrated by \citet{Wadler2022}.
Their treatment deviates from the usual formulation: 
a type synthesis algorithm is traditionally presented as \emph{algorithmic rules} such as $\Gamma |- \isTerm{t} \syn A \mapsto \isTerm{t}'$, denoting that annotations can be added to $\isTerm{t}$ in the surface language to produce $\isTerm{t}'$ of type $A$ in the core language.
Such an algorithm is then followed by soundness and completeness assertions such that the algorithm correctly synthesises the type of a raw term and every typable term can be synthesised if sufficiently annotated.
By contrast, \citeauthor{Wadler2022} exploit the computational and logical nature of Martin-L\"of type theory and formulate \emph{algorithmic soundness, completeness, and decidability in one go}.
%(although they do not emphasise the difference of their approach from the traditional one)

Recall that the law of excluded middle $P + \neg P$ does not hold as an axiom for every $P$ in type theory, and we say $P$~is logically \emph{decidable} if the law holds for~$P$.
For example, a proof of \emph{decidable equality}, i.e. $\forall x, y.~(x = y) + (x \neq y)$, decides whether $x$~and~$y$ are equal and accordingly gives an identity proof or a refutation explicitly; such a decidability proof may or may not be possible depending on the domain of $x$~and~$y$, and is non-trivial in general.
In type theory all proofs as programs terminate, so logical decidability implies algorithmic decidability.
Further, suppose a proof of the following statement:
\begin{quote}
  `For a context $\Gamma$ and a raw term $t$, either a typing derivation of\, $\Gamma |- t : A$ exists for some type~$A$ or any derivation of\, $\Gamma |- t : A$ for some type~$A$ leads to a contradiction'
\end{quote}
or rephrased succinctly as 
\begin{quote}
  `It is \underline{\emph{decidable}} for any $\Gamma$~and~$t$ whether $\Gamma |- t : A$ is derivable for some~$A$'.
\end{quote}
Computationally, the proof yields either a typing derivation for the given raw term~$t$ or a negation proof that such a derivation is impossible where the former case is algorithmic soundness and the latter algorithmic completeness in contrapositive form.
Hence, both algorithmic soundness and completeness in original form are implied.
Hence, the formal proof of the statement as a program is a verified \emph{proof-relevant} type synthesiser, because it does not only return yes/no but also constructs a proof. 
%For a type synthesiser being proof-relevant, all of its intermediate programs have to be proof-relevant, including bidirectional type synthesiser and its mode decorator.

\Josh{emphasise the scope of our approach}
\subsubsection{Mode decoration and bidirectional type synthesis}
\Josh{A missing step between bidirectional type synthesis and parsing}
As raw terms may not be sufficiently annotated, we propose a preprocessing step called \emph{mode decoration} to determine whether a raw term can be processed by a bidirectional type synthesiser later.
A mode decorator constructs a mode derivation for a raw term or pinpoints missing annotations.
A bidirectional type synthesiser and a type synthesiser differ in their domain---bidirectional type synthesis only works for mode-decorated raw terms instead of all raw terms.
Soundness and completeness of bidirectional typing is reformulated as a one-to-one correspondence between \emph{bidirectional typing derivations} and \emph{typing derivations} with \emph{mode derivations} (of a raw term), which is a more refined and useful formulation than annotatability described by \citet[Section~3.2]{Dunfield2021}.
Our proof-relevant bidirectional type synthesiser is complete with respect to mode-decorated terms.

We also prove that a raw term with missing annotations is exactly a raw term without a mode derivation, so our mode decorator is indeed proof-relevant.

By combining mode-decoration and bidirectional type synthesis, we show a trichotomy on raw terms, which is computationally a type synthesiser that checks if a given raw term is sufficiently annotated and if a sufficiently annotated raw term has an ordinary typing derivation, suggesting that type synthesis should not be viewed as a bisection between typable and untypable raw terms but a trichotomy in general.

\subsubsection{Mode-correctness and general definitions of languages}
\label{sec:language-formalisation}
The most essential characteristics of bidirectional typing is \emph{mode-correctness}, since an algorithm can often be `read off' from the definition of a bidirectionally typed language if mode-correct.
It seems that the implications of mode-correctness have only been addressed informally as exhibited in the survey paper~\cite{Dunfield2021}, since mode-correctness is not formally defined as a \emph{property of languages}.
%But, what is a language anyway?

To make the notion of mode-correctness precise, we first give a general definition of bidirectional simple type systems, called \emph{bidirectional binding signature}, extending the sorted version of \varcitet{Aczel1978}{'s binding signature} with bidirectionality.
A general definition of typed languages allows us to define mode-correctness and to investigate its consequences rigorously, including the uniqueness of synthesised types and the decidability of bidirectional type synthesis for all mode-correct signatures.
The latter theorem effectively amounts to a generator of proof-relevant bidirectional type synthesiser for all syntax-directed bidirectional simple type systems that are mode-correct.


To make our exposition accessible we have chosen to work with simply typed languages so that bidirectional type synthesis is established without any other technical assumptions.
%the idea of extending a signature with bidirectionality should be clear enough for further generalisation.
Nevertheless, we briefly discuss a possible extension which includes polymorphically typed languages such as \SystemF with additional assumptions in \cref{sec:future}.

\subsubsection{Contributions and plan of this paper}

%
In short, we develop a general and formal theory of bidirectional type synthesis for simply typed languages, including 
\begin{enumerate}
  \item general definitions for bidirectional type systems and mode-correctness,
  \item the concept of mode decoration and the comparison between our completeness and \varcitet{Dunfield2021}{'s annotatability}, 
    \Josh{generalisation mode decoration?}
  \item rigorously proven consequences of mode-correctness---the uniqueness of synthesised types and the decidability of bidirectional type synthesis, which amounts to
  \item a fully verified generator for proof-relevant type-synthesisers.
\end{enumerate}
Our theory was first developed in \Agda and then translated to the mathematical vernacular, filling the gap between our informal presentation and its verified implementation.

%\begin{enumerate}
%  \item \emph{simple yet general}, working for any syntax-directed bidirectional simple type system that can be specified by a bidirectional binding signature;
%  \item \emph{constructive} based on Martin-L\"of type theory, and compactly formulated in a way that unifies computation and proof, preferring logical decidability over algorithmic soundness, completeness, and decidability; and mode decoration over annotatability.
%\end{enumerate}

This paper is structured as follows.
We first present a concrete overview of our theory using simply typed $\lambda$-calculus in \cref{sec:key-ideas}, prior to developing a general framework for specifying bidirectional type systems in \cref{sec:defs}.
Following this, we discuss mode decoration and related properties in \cref{sec:pre-synthesis}.
The main technical contribution lies in \cref{sec:type-synthesis}, where we introduce mode-correctness and bidirectional type synthesis.
Some examples other than simply typed $\lambda$-calculus are given in \Cref{sec:example}.
We briefly demonstrate the use of our \Agda formalisation as programs in \cref{sec:formalisation} and conclude in \cref{sec:future} with further developments.



\input{S2-outline}
%!TEX root = BiSig.tex

\section{Bidirectionally simply typed languages}\label{sec:defs}
This section provides general definitions
%\footnote{%
%Here's the 'small print': Our definitions cover typical examples and set the scope for bidirectional simple type systems discussed, but do not offer comprehensive coverage of all possibilities.}
of simple types, simply typed languages, and bidirectional type systems, and uses the simply typed $\lambda$-calculus in \cref{sec:key-ideas} as our running example.
These definitions may look dense, especially on first reading.
The reader may choose to skim through this section, in particular the figures, and still get some rough ideas from later sections.

The definitions are formulated in two steps:
\begin{inlineenum}
  \item first we introduce a notion of arity and a notion of signature which includes a set\footnote{%
    Even though our theory is developed in Martin-L\"of type theory, the term `set' is used instead of `type' to avoid the obvious confusion. 
    Indeed, as we assume \AxiomK, all types are legitimately sets in the sense of homotopy type theory~\citep[Definition 3.1.1]{UFP2013}.} of operation symbols and an assignment of arities to symbols;
\item then, given a signature, we define raw terms and typing derivations inductively by primitive rules such as $\Rule{Var}$ and a rule schema for constructs $\tmOp_o$ indexed by an operation symbol~$o$.
\end{inlineenum}
%All these definitions are inductive as usual but include a \emph{rule schema} giving rise to as many rules as there are operation symbols in a signature.
As we move from simple types to bidirectional typing, the notion of arity, initially as the number of arguments of an operation, is enriched to incorporate an extension context for variable binding and the mode for the direction of type information flow.

\subsection{Signatures and simple types} \label{subsec:simple-types}

For simple types, the only datum needed for specifying a type construct is its number of arguments:
\begin{defn} \label{def:simple-signature}
%  \begin{figure}
%      \centering
%      \small
%      \judgbox{\Xi|-_{\Sigma} A}{$A$ is a type with type variables in $\Xi$ for a signature $\Sigma$}
%      \begin{mathpar}
%        \inferrule{X_i \in \Xi}{\Xi |-_\Sigma X_i} \and
%        \inferrule{\Xi |-_{\Sigma} A_1 \\ \cdots \\ \Xi |-_{\Sigma} A_n}{\Xi |-_{\Sigma} \tyOp_i(A_1, \ldots, A_n)}\;\text{where $n = \arity(i)$}
%      \end{mathpar}
%      \caption{Simple types}
%      \label{fig:simple-type}
%  \end{figure}
  A \emph{signature} $\Sigma$ for simple types consists of a set $I$ with a decidable equality and an \emph{arity} function $\arity\colon I \to \mathbb{N}$.
  For a signature $\Sigma$, a \emph{type} $A : \Type_{\Sigma}(\Xi)$ over a variable set $\Xi$ is either
  \begin{enumerate}
    \item a variable in $\Xi$ or
    \item $\tyOp_{i}(A_1, \ldots, A_n)$ for some $i:I$ with $\arity(i) = n$ and types $A_1,\ldots, A_n$.
  \end{enumerate}
\end{defn}

\begin{example} \label{ex:type-signature-for-function-type}
  Function types $A \bto B$ and typically a base type~$\mathtt{b}$ are included in simply typed $\lambda$-calculus and can be specified by the type signature $\Sigma_{\bto}$ consisting of operations $\mathsf{fun}$ and $\mathsf{b}$ where $\arity(\mathsf{fun}) = 2$ and $\arity(\mathsf{b}) = 0$.
  Then, all types in simply typed $\lambda$-calculus can be given as $\Sigma_{\bto}$-types over the empty set with $A \bto B$ introduced as $\tyOp_{\mathsf{fun}}(A, B)$ and $\mathtt{b}$ as $\tyOp_{\mathsf{b}}$. 
\end{example}

\begin{defn}\label{def:substitution}
The \emph{substitution} for a function $\rho\colon \Xi \to \Type_{\Sigma}(\Xi')$, denoted by $\rho\colon \Sub{\Xi}{\Xi'}$, is a map which sends a type $A : \Type_{\Sigma}(\Xi)$ to $\simsub{A}{\rho} : \Type_{\Sigma}(\Xi')$ and is defined as usual.
\end{defn}

\subsection{Binding signatures and simply typed languages} \label{subsec:binding-sig}

A simply typed language specifies
\begin{inlineenum}
  \item a family of sets of raw terms $\isTerm{t}$ indexed by a list~$V$ of variables (that are currently in scope) where each construct is allowed to bind some variables like $\Rule{Abs}$ and to take multiple arguments like $\Rule{App}$;
  \item a family of sets of typing derivations indexed by a typing context~$\Gamma$, a raw term~$\isTerm{t}$, and a type~$A$.
\end{inlineenum}
Therefore, to specify a term construct, we enrich the notion of arity with some set of types for typing and extension context for variable binding.

\begin{defn}[{\cite[p.~322]{Fiore2010}}]\label{def:binding-arity}
  A \emph{binding arity} with a set $T$ of types is an inhabitant of $\left(T^* \times T\right)^* \times T$ where $T^*$ is the set of lists over $T$.
  In a binding arity $(((\Delta_1, A_1), \ldots, (\Delta_n, A_n)), A)$, every $\Delta_i$ and $A_i$ refers to the \emph{extension context} and the \emph{type of the $i$-th argument}, respectively, and $A$ the \emph{target type}.
  For brevity, it is denoted by $\bargs \to A$ where $[\Delta_i]$ is omitted if empty.
\end{defn}

\begin{example}
  Observe that $\Rule{Abs}$ and $\Rule{App}$ rules in \Cref{fig:STLC-typing-derivations} can be read as
  \renewcommand{\eqnhighlightheight}{\vphantom{\Gamma}}
  \renewcommand{\eqnannotationfont}{\rmfamily\footnotesize} % package default
  \vspace{2em}
  \begin{equation*}
    \frac{\Gamma, \eqnmarkbox[blue]{ext1}{\isTerm{x} : A, \cdot} \vdash \isTerm{t} : \eqnmarkbox[red]{aty1}{B}}{\Gamma \vdash \isTerm{\lam{x}t} : \eqnmarkbox{ty1}{A \bto B}}
    \qquad\text{and}\qquad
    \frac{\Gamma, \eqnmarkbox[blue]{ext2}{\cdot} \vdash \isTerm{t} : \eqnmarkbox[red]{aty2}{A \bto B} \qquad \Gamma, \eqnmarkbox[blue]{ext3}{\cdot} \vdash \isTerm{u} : \eqnmarkbox[red]{aty3}{A}}{\Gamma \vdash \isTerm{t\;u} : \eqnmarkbox{ty2}{B}}
  \end{equation*}
  \annotate[yshift=1.5em]{above}{ext1,ext2,ext3}{extension context}
  \annotate[yshift=0.4em]{above,right}{aty1,aty2,aty3}{argument type}
  \annotate[yshift=0em]{below}{ty1,ty2}{target type}
  \vspace{.5em}
  \\
  if the empty extension context $\cdot$ is added verbosely, so they can be specified by binding arities ${[A]B} \to {(A \bto B)}$ and ${(A \bto B), A} \to {B}$ respectively.
\end{example}

%\begin{example}
%Consider the $\Rule{Abs}$ rule (\Cref{fig:STLC-typing-derivations}).
%Its binding arity has $\Type_{\Sigma_{\bto}}\{A, B\}$ as types and is ${[A]B} \to {(A \bto B)}$.
%This means that the $\Rule{Abs}$ rule for derivations of\/ $\Gamma |- \lam{x}t : A \bto B$ contains:
%\begin{inlineenum}
%  \item a derivation of\/ $\Gamma, x : A |- t : B$ as an argument of type $B$ in a typing context extended by a variable~$x$ of type~$A$;
%  \item the type $A\bto B$ for itself.
%\end{inlineenum}
%In a similar vein, the arity of the $\Rule{App}$ rule is denoted as ${(A \bto B), A} \to {B}$.
%This specifies that any derivation of\/ $\Gamma \vdash t\;u : B$ accepts derivations $\Gamma \vdash t : A \bto B$ and $\Gamma \vdash u : A$ as its arguments.
%These arguments have types $A \bto B$ and $A$ respectively and their extension contexts are empty.
%\end{example}

Next, akin to a signature, a binding signature $\Omega$ consists of a set of operation symbols along with their respective binding arities:
\begin{defn}\label{def:binding-signature}
  For a type signature $\Sigma$, a \emph{binding signature} $\Omega$ is a set $O$ with a function
  \[
    \arity \colon O \to \sm{\Xi : \UU} \left(\Type_\Sigma(\Xi)^* \times \Type_\Sigma(\Xi)\right)^* \times \Type_\Sigma(\Xi).
  \]
\end{defn}
That is, each inhabitant $o: O$  is associated with a set $\Xi$ of type variables and an arity $\arity(o)$ with $\Type_{\Sigma}(\Xi)$ as types denoted by $\bop$.

The set~$\Xi$ of type variables for each operation, called its \emph{local context}, plays an important role.
To use a rule like $\Rule{Abs}$ in an actual typing derivation, we need to substitute \emph{concrete types}, i.e.\ types without any type variables, for variables $A, B$.
In our formulation of substitution~(\ref{def:substitution}), we must first identify which type variables to substitute for.
As such, this information forms part of the arity of an operation, and typing derivations, defined subsequently, will include functions $\rho$ from~$\Xi$ to concrete types specifying how to instantiate typing rules by substitution.

By a \emph{simply typed language} $(\Sigma, \Omega)$, we mean a pair of a type signature~$\Sigma$ and a binding signature~$\Omega$.
Now, we define raw terms for $(\Sigma, \Omega)$ first.

\begin{figure}
  \centering
  \small
  \judgbox{V |-_{\Sigma, \Omega} \isTerm{t}}{$\isTerm{t}$ is a raw term for a language $(\Sigma, \Omega)$ with free variables in~$V$}
  \begin{mathpar}
    \inferrule{\isTerm{x} \in V}{V |-_{\Sigma, \Omega} \isTerm{x}}\,\Rule{Var}
    \and
    \inferrule{\cdot |-_{\Sigma} A \\ V |-_{\Sigma, \Omega}\isTerm{t}}{V |-_{\Sigma, \Omega} \isTerm{t \annotatecolon A}}\,\Rule{Anno}
    \\
    \inferrule{V, \vars{x}_\isTerm{1} |-_{\Sigma, \Omega} \isTerm{t_1} \\ \cdots \\ V, \vars{x}_\isTerm{n} |-_{\Sigma, \Omega} \isTerm{t_n} } {V |-_{\Sigma, \Omega} \tmOpts }\,\Rule{Op} \and
    \text{for $\bop$ in $\Omega$}
  \end{mathpar}
  \caption{Raw terms}
  \label{fig:raw-terms}
\end{figure}


\begin{defn}
  For a simply typed language $(\Sigma, \Omega)$, the family of sets of \emph{raw terms} indexed by a list~$V$ of variables consists of
  \begin{inlineenum}
    \item variables in $V$,
    \item annotations $\isTerm{t \annotatecolon A}$ for some raw term $t$ in $V$ and a type $A$, and
    \item a construct $\tmOpts$ for some $\bop$ in~$O$, where $\vars{x}_{\isTerm{i}}$'s are lists of variables whose length is equal to the length of~$\Delta_i$, and $t_i$'s are raw terms in the variable list $V, \vars{x}_i$.
  \end{inlineenum}
  These correspond to rules $\Rule{Var}$, $\Rule{Anno}$, and $\Rule{Op}$ in \Cref{fig:raw-terms} respectively.
\end{defn}

Before defining typing derivations, we need a definition of typing contexts.

\begin{defn}
A \emph{typing context} $\Gamma \colon \Cxt_{\Sigma}$ is formed by $\cdot$ for the empty context and $\Gamma, x : A$ for an additional variable~$x$ with a concrete type $A : \Type_{\Sigma}(\emptyset)$.
The list of variables in~$\Gamma$ is denoted~$\erase\Gamma$.
%The substitution $\simsub{\Gamma}{\rho}$ of a context $\Gamma$ is defined by applying substitution to each $A$ in $\Gamma$.
\end{defn}

The definition of typing derivations is a bit more involved.
We need some information to compare types on the object level during type synthesis and substitute those type variables in a typing derivation $\Gamma \vdash \tmOpts : A$ for an operation $o$ in $\Omega$ at some point.
Here we choose to include a substitution $\rho$ from the local context $\Xi$ to $\emptyset$ as part of its typing derivation explicitly:

\begin{defn}\label{def:typing-derivations}
  For a simply typed language $(\Sigma, \Omega)$, the family of sets of \emph{typing derivations} of $\Gamma \vdash \isTerm{t} : A$, indexed by a typing context $\Gamma : \Cxt_\Sigma$, a raw term~$\isTerm{t}$ with free variables in $\erase{\Gamma}$, and a type $A : \Type_\Sigma(\emptyset)$, consists of 
  \begin{enumerate}
    \item a derivation of $\Gamma |-_{\Sigma, \Omega} x : A$ if $x : A$ is in $\Gamma$,
    \item a derivation of $\Gamma |-_{\Sigma, \Omega} (t \annotatecolon A) : A$ if $\Gamma \vdash_{\Sigma, \Omega} \isTerm{t} : A$ has a derivation, and
    \item a derivation of $\Gamma |-_{\Sigma, \Omega} \tmOpts : \simsub{A_0}{\rho}$ for an operation $\bop$ if there are $\rho\colon \Xi \to \Type_{\Sigma}(\emptyset)$ and a derivation of $\Gamma, \vars{x}_{\isTerm{i}} : \simsub{\Delta_i}{\rho} |-_{\Sigma, \Omega} \isTerm{t_i} : \simsub{A_i}{\rho}$ for each $i$,
  \end{enumerate}
  corresponding to rules $\Rule{Var}$, $\Rule{Anno}$, and $\Rule{Op}$ in \Cref{fig:extrinsic-typing} respectively.
\end{defn}


\begin{example}
Raw terms (\Cref{fig:STLC-raw-terms}) and typing derivations (\Cref{fig:STLC-typing-derivations}) for simply typed $\lambda$-calculus can be specified by the type signature $\Sigma_{\bto}$ (\Cref{ex:type-signature-for-function-type}) and the binding signature consisting of $\aritysymbol{\mathsf{app}}{A, B}{(A \bto B), A}{B}$ and $\aritysymbol{\mathsf{abs}}{A , B}{[A]B}{(A \bto B)}$.
Rules $\Rule{Abs}$ and $\Rule{App}$ in simply typed $\lambda$-calculus are subsumed by the $\Rule{Op}$ rule schema, as applications $t\;u$ and abstractions $\lam{x}t$ can be introduced uniformly as $\tmOp_{\mathsf{app}}(t, u)$ and $\tmOp_{\mathsf{abs}}(x.t)$, respectively.
\end{example}

\subsection{Bidirectional binding signatures and bidirectional type systems} \label{subsec:bidirectional-system}
For a bidirectional type system, typing judgements appear in two forms: $\Gamma |- t \syn A$ and $\Gamma |- t \chk A$, but these two typing judgements can be considered as a single typing judgement $\Gamma |- t :^\dir{d} A$, additionally indexed by a \emph{mode} $d : \Mode$---which can be either $\syn$ or $\chk$.
Therefore, to define a bidirectional type system, we enrich the concept of binding arity to \emph{bidirectional binding arity}, which further specifies the mode for each of its arguments and for the conclusion:

\begin{defn} \label{def:bidirectional-binding-signature}
  A \emph{bidirectional binding arity} with a set $T$ of types is an inhabitant of
  \[
    \left(T^* \times T \times \Mode \right)^* \times T \times \Mode.
  \]
  For clarity, an arity is denoted by $\biarity$.
\end{defn}

\begin{example}
Consider the $\ChkRule{Abs}$ rule (\Cref{fig:STLC-bidirectional-typing-derivations}) for $\lam{x}t$.
It has the arity ${[A]B^{\chk}}\to{(A \bto B)^{\chk}}$, indicating additionally that both $\lam{x}t$ and its argument $t$ are checking.
Likewise, the $\SynRule{App}$ rule has the arity ${(A \bto B)^{\syn}, A^{\chk}} \to {B^{\syn}}$.
\end{example}

\begin{defn}
  For a type signature $\Sigma$, a \emph{bidirectional binding signature} $\Omega$ is a set $O$ with
  \[
    \mathit{ar}\colon O \to \sum_{\Xi : \UU} \left(\Type_{\Sigma}(\Xi)^* \times \Type_{\Sigma}(\Xi) \times {\Mode}\right)^* \times \Type_{\Sigma}(\Xi) \times {\Mode}.
  \]
\end{defn}
We write $\biop$ for an operation $o$ with a variable set $\Xi$ and its bidirectional binding arity with $\Type_{\Sigma}(\Xi)$ as types.
We call it \emph{checking} if $d$ is ${\chk}$ or \emph{synthesising} if $d$ is ${\syn}$; similarly its $i$-th argument is checking if $d_i$ is $\chk$ and synthesising if $d_i$ is $\syn$.
A bidirectional type system $(\Sigma, \Omega)$ refers to a pair of a type signature $\Sigma$ and a bidirectional binding signature $\Omega$.


\begin{defn}\label{def:bidirectional-typing-derivations}\label{def:mode-derivations}
  For a bidirectional type system $(\Sigma, \Omega)$,
  \begin{itemize}
    \item the set of \emph{bidirectional typing derivations} of $\Gamma \vdash_{\Sigma, \Omega} t :^\dir{d} A$, indexed by a typing context $\Gamma$, a raw term $\isTerm{t}$ under $\erase{\Gamma}$, a mode $\dir{d}$, and a type $A$, is defined in \Cref{fig:bidirectional-typing-derivations} and particularly
          \[
            \Gamma |-_{\Sigma, \Omega} \tmOpts :^{\dir{d}} \simsub{A_0}{\rho}
          \]
          has a derivation for $\biop$ in $\Omega$ if there is $\rho\colon \Xi \to \Type_{\Sigma}(\emptyset)$ and a derivation of $\Gamma, \vars{x}_{\isTerm{i}} : \simsub{\Delta_i}{\rho} \vdash_{\Sigma, \Omega} \isTerm{t_i} :^{\dir{d_i}} \simsub{A_i}{\rho}$ for each $i$;
    \item the set of \emph{mode derivations} of $V |-_{\Sigma, \Omega} t^\dir{d}$, indexed by a list $V$ of variables, a raw term $\isTerm{t}$ under $V$, and a mode $d$, is defined in \Cref{fig:mode-derivations}.
  \end{itemize}
  The two judgements {\small$\boxed{\Gamma \vdash_{\Sigma, \Omega} \isTerm{t} \syn A}$} and {\small$\boxed{\Gamma \vdash_{\Sigma, \Omega} \isTerm{t} \chk A}$} stand for ${\Gamma \vdash_{\Sigma, \Omega} \isTerm{t} :^{\syn} A}$ and ${\Gamma \vdash_{\Sigma, \Omega} \isTerm{t} :^{\chk} A}$, respectively.
  A typing rule is \emph{checking} if its conclusion mode is $\chk$ or \emph{synthesising} otherwise.
\end{defn}

Every bidirectional binding signature $\Omega$ gives rise to a binding signature $\erase{\Omega}$ if we erase modes from $\Omega$, called the \emph{(mode) erasure} of $\Omega$.
Hence a bidirectional type system $(\Sigma, \Omega)$ also specifies a simply typed language $(\Sigma, \erase{\Omega})$, including raw terms and typing derivations.

\begin{example}\label{ex:signature-simply-typed-lambda}
Having established generic definitions, we can now specify simply typed $\lambda$-calculus and its bidirectional type system---including raw terms, (bidirectional) typing derivations, and mode derivations---using just a pair of signatures $\Sigma_{\bto}$ (\Cref{ex:type-signature-for-function-type}) and $\Omega^{\Leftrightarrow}_\Lambda$ which consists of 
\[
  \aritysymbol{\mathsf{abs}}{A , B}{[A]B^{\chk}}{(A \bto B)^{\chk}}
  \quad\text{and}\quad
  \aritysymbol{\mathsf{app}}{A, B}{(A \bto B)^{\syn}, A^{\chk}}{B^{\syn}}.
\]
\end{example}
More importantly, we are able to reason about constructions and properties that hold for any simply typed language with a bidirectional type system once and for all by quantifying over $(\Sigma, \Omega)$.

\begin{figure}
  \centering
  \small
  \judgbox{\Gamma |-_{\Sigma, \Omega} \isTerm{t} : A}{$\isTerm{t}$ has a concrete type $A$ under $\Gamma$ for a language $(\Sigma, \Omega)$}
  \begin{mathpar}
%    \mprset{sep=0.5em}
    \inferrule{(\isTerm{x} : A) \in \Gamma}{\Gamma |-_{\Sigma, \Omega} \isTerm{x} : A}\,\Rule{Var}
    \and
    \inferrule{\Gamma |-_{\Sigma, \Omega} \isTerm{t} : A}{\Gamma |-_{\Sigma, \Omega} (\isTerm{t \annotatecolon A}) : A}\,\Rule{Anno}
    \and
    \inferrule{\rho : \Sub{\Xi}{\emptyset} \and  \Gamma, \vec{\isTerm{x}}_\isTerm{1} : \simsub{\Delta_{1}}{\rho} |-_{\Sigma, \Omega} \isTerm{t_1} : \simsub{A_{1}}{\rho} \; \cdots \; \Gamma, \vec{\isTerm{x}}_\isTerm{n} : \simsub{\Delta_{n}}{\rho} |-_{\Sigma, \Omega} \isTerm{t_n} : \simsub{A_{n}}{\rho}}
    {\Gamma |-_{\Sigma, \Omega} \tmOpts : \simsub{A_0}{\rho}}\,\Rule{Op}
    \\
  \text{for $\bop$ in $\Omega$}
  \end{mathpar}
  \caption{Typing derivations}
  \label{fig:extrinsic-typing}
\end{figure}
\begin{figure}
  \centering
  \small
  \judgbox{\Gamma |-_{\Sigma, \Omega} \isTerm{t} :^\dir{d} A}{$\isTerm{t}$ has a type $A$ in mode $\dir{d}$ under $\Gamma$ for a bidirectional system $(\Sigma, \Omega)$} 
  \begin{mathpar}
    \inferrule{(x : A) \in \Gamma}{\Gamma |-_{\Sigma, \Omega} \isTerm{x} :^{\syn} A}\,\SynRule{Var}
    \and
    \inferrule{\Gamma |-_{\Sigma, \Omega} \isTerm{t} :^{\chk} A}{\Gamma |-_{\Sigma, \Omega} (\isTerm{t \annotatecolon A}) :^{\syn}  A}\,\SynRule{Anno}
    \and
    \inferrule{\Gamma |-_{\Sigma, \Omega} \isTerm{t} :^{{\syn}} B \\ B = A}{\Gamma |-_{\Sigma, \Omega} \isTerm{t} :^{\chk} A}\,\ChkRule{Sub}
    \\
    \inferrule{\rho\colon \Sub{\Xi}{\emptyset} \\ \Gamma, \simsub{\vars{x}_\isTerm{1} : \Delta_{1}}{\rho} |-_{\Sigma, \Omega} \isTerm{t_{1}} \mathrel{:^\dir{d_1}} \simsub{A_1}{\rho} \and \cdots \and \Gamma, \simsub{\vars{x}_\isTerm{n} : \Delta_{n}}{\rho} |-_{\Sigma, \Omega} \isTerm{t_n} \mathrel{:^{\dir{d_n}}} \simsub{A_{n}}{\rho}}
    {\Gamma |-_{\Sigma, \Omega} \tmOpts :^{\dir{d}} \simsub{A_0}{\rho}} \,\Rule{Op}
    \\ \text{for $\biop$ in $\Omega$}
  \end{mathpar}
  \caption{Bidirectional typing derivations}
  \label{fig:bidirectional-typing-derivations}
\end{figure}

\begin{figure}
  \centering
  \small
  \judgbox{V |-_{\Sigma, \Omega} \isTerm{t}^\dir{d}}{$\isTerm{t}$ is in mode $d$ with free variables in $V$ for $(\Sigma, \Omega)$}
  \begin{mathpar}
    \inferrule{x \in V}{V |-_{\Sigma, \Omega} \isTerm{x}^{\syn}}\,\SynRule{Var}
    \and
    \inferrule{\cdot |-_{\Sigma} A \\ V |-_{\Sigma, \Omega}\isTerm{t}^{\chk}}{V |-_{\Sigma, \Omega} (\isTerm{t \annotatecolon A})^{\syn}}\,\SynRule{Anno}
    \and
    \inferrule{V |-_{\Sigma, \Omega} \isTerm{t}^{\syn}}{V |-_{\Sigma, \Omega} \isTerm{t}^{\chk}}\,\ChkRule{Sub}
    \and
    \inferrule{V, \vars{x}_1 |-_{\Sigma, \Omega} \isTerm{t_1}^\dir{d_1} \\ \cdots \\ V, \vars{x}_{n} |-_{\Sigma, \Omega} \isTerm{t_n}^\dir{d_n}}
    {V |-_{\Sigma, \Omega} \tmOpts^\dir{d}}\,\Rule{Op}
    \and \text{for $\biop$}
  \end{mathpar}
  \caption{Mode derivations}
  \label{fig:mode-derivations}
\end{figure}

\input{S4-pre-synthesis}
%!TEX root = BiSig.tex

\section{Bidirectional type synthesis and checking} \label{sec:type-synthesis}
This section focuses on defining mode-correctness and deriving bidirectional type synthesis for any mode-correct bidirectional type system $(\Sigma, \Omega)$.
We start with \Cref{sec:mode-correctness} by defining mode-correctness and showing the uniqueness of synthesised types.
This uniqueness means that any two synthesised types for the same raw term $t$ under the same context $\Gamma$ have to be equal.
It will be used especially in \Cref{subsec:bidirectional-synthesis-checking} for the proof of the decidability of bidirectional type synthesis and checking.
Then, we conclude this section with the trichotomy on raw terms in \Cref{subsec:trichotomy}.

\subsection{Mode correctness}\label{sec:mode-correctness}
As \citet{Dunfield2021} outlined, mode-correctness for a bidirectional typing rule means that 
\begin{inlineenum}
\item each `input' type variable in a premise must be an `output' variable in `earlier' premises, or provided by the conclusion if the rule is checking;
\item each `output' type variable in the conclusion should be some `output' variable in a premise if the rule is synthesising.
\end{inlineenum}
Here `input' variables refer to variables in an extension context and in a checking premise.
It is important to note that the order of premises in a bidirectional typing rule also matters, since synthesised type variables are instantiated incrementally during type synthesis.

Consider the rule $\ChkRule{Abs}$ (\Cref{fig:STLC-bidirectional-typing-derivations}) as an example.
This rule is mode-correct, as the type variables $A$ and $B$ in its only premise are already provided by its conclusion $A \bto B$.
Likewise, the rule $\SynRule{App}$ for an application term $\isTerm{t\;u}$ is mode-correct because:
\begin{inlineenum}
\item the type $A \bto B$ of the first argument $t$ is synthesised, thereby ensuring type variables $A$ and $B$ must be known if successfully synthesised;
\item the type of the second argument $u$ is checked against $A$, which has been synthesised earlier;
\item as a result, the type of an application $t\;u$ can be synthesised.
\end{inlineenum}

Now let us define mode-correctness rigorously.
As we have outlined, the condition of mode-correctness for a synthesising rule is different from that of a checking rule, and the argument order also matters.
Defining the condition directly for a rule, and thus in our setting for an operation, can be somewhat intricate.
Instead, we choose to define the conditions for the argument list---more specifically, triples $\biargvec$ of an extension context $\Delta_i$, a type $A_i$, and a mode $\dir{d_i}$---pertaining to an operation, for an operation, and subsequently for a signature.
We also need some auxiliary definitions for the subset of variables of a type and of an extension context, and the set of variables that have been synthesised:
\begin{defn}
  The finite subsets\footnote{%
  There are various definitions for finite subsets of a set within type theory, but for our purposes the choice among these definitions is not a matter of concern.}
  of \emph{(free) variables} of a type $A$ and of variables in an extension context~$\Delta$ are denoted by $\fv(A)$ and $\fv(\Delta)$ respectively.
  For an argument list $\biargs$, the set of type variables $A_{i}^{\dir{d_i}}$ with $d_i$ being $\syn$ is denoted by $\synvar(\biargs)$, i.e.\ $\synvar$ gives the set of type variables that will be synthesised during type synthesis.
\end{defn}

\begin{defn}\label{def:mode-correctness-args}
  The \emph{mode-correctness} $\MCas\left(\biargvec\right)$ for an argument list $\biargs$ with respect to a subset $S$ of $\Xi$ is a predicate defined by
  {\small
  \begin{align*}
    \MCas( \cdot ) & = \top \\
    \MCas\left(\biargvec, \chkbiarg[n]\right)
                                  & = \fv(\Delta_n, A_n) \subseteq \left( S \cup \synvar\left(\biargvec\right)\right) \land \MCas\left(\biargvec\right) \\
    \label{eq:MC2} \MCas\left(\biargvec, \synbiarg[n]\right) 
                                  & = \phantom{, A_n} \fv(\Delta_n) \subseteq \left( S \cup \synvar\left(\biargvec\right)\right) \land  \MCas\left(\biargvec\right)
  \end{align*}}
  where $\MCas( \cdot ) = \top$ means that an empty list is always mode-correct.
\end{defn}
This definition encapsulates the idea that every `input' type variable, possibly derived from an extension context~$\Delta_n$ or a checking argument~$A_n$, must be an `output' variable from $\synvar(\biargvec)$ or, if the rule is checking, belong to the set $S$ of `input' variables in its conclusion.
This condition must also be met for every tail of the argument list to ensure that `output' variables accessible at each argument are from preceding arguments only, hence an inductive definition.
\begin{defn}\label{def:mode-correctness}
  An arity $\biarity$ is \emph{mode-correct} if 
  \begin{enumerate}
    \item either $d$ is $\chk$, its argument list is mode-correct with respect to $\fv(A_0)$, and the union $\fv(A_0) \cup \synvar(\biargvec)$ contains every inhabitant of $\Xi$;
    \item or $d$ is $\syn$, its argument list is mode-correct with respect to $\emptyset$, and $\synvar(\biargvec)$ contains every inhabitant of $\Xi$ and, particularly, $\fv(A_0)$.
  \end{enumerate}
  A bidirectional binding signature $\Omega$ is \emph{mode-correct} if every operation's arity is mode-correct.
\end{defn}
For a checking operation, an `input' variable of an argument could be derived from~$A_0$, as these are known during type checking as an input.
Since every inhabitant of~$\Xi$ can be located in either~$A_0$ or synthesised variables, we can determine a concrete type for each inhabitant of~$\Xi$ during type synthesis.
On the other hand, for a synthesising operation, we do not have any known variables at the onset of type synthesis, so the argument list should be mode-correct with respect to~$\emptyset$.
Also, the set of synthesised variables alone should include every type variable in~$\Xi$ and particularly in~$A_n$.

\begin{remark}
  Mode-correctness is fundamentally a condition for bidirectional typing \emph{rules}, not for derivations.
  Thus, this property cannot be formulated without treating rules as some mathematical object such as those general definitions in \Cref{sec:defs}.
  This contrasts with the properties in \Cref{sec:pre-synthesis}, which can still be specified for individual systems in the absence of a general definition.
\end{remark}

It is easy to check the bidirectional type system $(\Sigma_{\bto}, \Omega^{\Leftrightarrow}_{\Lambda}$) for simply typed $\lambda$-calculus is mode-correct by definition or by the following lemma:
\begin{lemma}\label{lem:decidability-mode-correctness}
  For any bidirectional binding arity $\biarity$, it is decidable whether it is mode-correct.
\end{lemma}


Now, we set out to show the uniqueness of synthesised types for a mode-correct bidirectional type system.
For a specific system, its proof is typically a straightforward induction on the typing derivations.
However, since mode-correctness is inductively defined on the argument list, our proof proceeds by induction on both the typing derivations and the argument list:
\begin{lemma}[Uniqueness of synthesised types]\label{thm:unique-syn}
  In a mode-correct bidirectional type system $(\Sigma, \Omega)$, the synthesised types of any two derivations
  \[
    \Gamma |-_{\Sigma, \Omega} \isTerm{t} \syn A
    \quad\text{and}\quad
    \Gamma |-_{\Sigma, \Omega} \isTerm{t} \syn B
  \]
  for the same term $t$ must be equal, i.e.\ $A = B$.
\end{lemma}
\begin{proof}%[Proof of \Cref{thm:unique-syn}]
  We prove the statement by induction on derivations $d_1$ and $d_2$ for $\Gamma |-_{\Sigma, \Omega} \isTerm{t} \syn A$ and $\Gamma |-_{\Sigma, \Omega} \isTerm{t} \syn B$.
  Our system is syntax-directed, so $d_1$ and $d_2$ must be derived from the same rule: 
  \begin{itemize}
    \item $\SynRule{Var}$ follows from the fact that each variable as a raw term refers to the same variable in its context.
    \item $\SynRule{Anno}$ holds trivially, since the synthesised type $A$ is from the term $t \annotatecolon A$ in question.
    \item $\Rule{Op}$: Recall that a derivation of $\Gamma |- \tmOpts \syn A$ contains a substitution $\rho$ from the local context $\Xi$ to concrete types.
      To prove that any two typing derivations has the same synthesised type, it suffices to show that those substitutions $\rho_1$~and~$\rho_2$ of $d_1$~and~$d_2$, respectively, agree on variables in $\synvar(\biargs)$ so that $\simsub{A_0}{\rho_1} = \simsub{A_0}{\rho_2}$.
      We prove it by induction on the argument list:
      \begin{enumerate}
        \item For the empty list, the statement is vacuously true.
        \item If $\dir{d_{i+1}}$ is~$\chk$, then the statement holds by induction hypothesis.
        \item If $\dir{d_{i+1}}$ is~$\syn$, then $\simsub{\Delta_{i+1}}{\rho_1} = \simsub{\Delta_{i+1}}{\rho_2}$ by induction hypothesis (of the list).
          Therefore, under the same context $\Gamma, \simsub{\Delta_{i+1}}{\rho_1} = \Gamma, \simsub{\Delta_{i+1}}{\rho_2}$ the term $t_{i+1}$ must have the same synthesised type $\simsub{A_{i+1}}{\rho_1} = \simsub{A_{i+1}}{\rho_1}$ by induction hypothesis (of the typing derivation), so $\rho_1$ and $\rho_2$ agree on $\fv(A_{i+1})$ in addition to $\synvar(\biargs)$.
      \end{enumerate}
  \end{itemize}\qed
\end{proof}

%Uniqueness of the synthesised types is a prevalent property in bidirectional type systems, although the specific proofs can vary depending on the constructs in the system.
%For instance, for derivations of $\Gamma |- t\;u \syn B_i$ for $i = 1, 2$ in simply typed $\lambda$-calculus, the hypothesis is applied to their sub-derivations $\Gamma |- t \syn A_i \bto B_i$ to conclude that $A_1 \bto B_1 = A_2 \bto B_2$ and thus $B_1 = B_2$.
%On the other hand, our proof is based on mode-correctness and need not consider specific sub-derivations.

\subsection{Decidability of bidirectional type synthesis and checking}\label{subsec:bidirectional-synthesis-checking}

We have arrived at the main technical contribution of this paper.

\begin{theorem}\label{thm:bidirectional-type-synthesis-checking}
  In a mode-correct bidirectional type system $(\Sigma, \Omega)$,
  \begin{enumerate}
    \item if $\erase{\Gamma} |-_{\Sigma, \Omega} \isTerm{t}^{\syn}$, then it is decidable whether $\Gamma |-_{\Sigma, \Omega} \isTerm{t} \syn A$ for some $A$;
    \item if $\erase{\Gamma} |-_{\Sigma, \Omega} \isTerm{t}^{\chk}$, then it is decidable for any~$A$ whether $\Gamma |-_{\Sigma, \Omega} \isTerm{t} \chk A$.
  \end{enumerate}
\end{theorem}

The interesting part of the theorem is the case for the $\Rule{Op}$ rule.
We shall give its insight first instead of jumping into the details.
Recall that a typing derivation for $\tmOpts$ contains a substitution $\rho\colon \Xi \to \Type_{\Sigma}(\emptyset)$.
The goal of type synthesis for this case is exactly to define such a substitution~$\rho$, and we have to start with an `accumulating' substitution: a substitution $\rho_0$ that is partially defined on $\fv(A_0)$ if $d$ is $\chk$ or otherwise nowhere.
By mode-correctness, the accumulating substitution~$\rho_i$ will be defined on enough synthesised variables so that type synthesis or checking can be performed on $t_{i}$ with the context $\Gamma, \vars{x}_{i} : \simsub{\Delta_{i}}{\rho_{i}}$ based on its mode derivation $\erase{\Gamma}, \vars{x}_i |-_{\Sigma, \Omega} t_i^{\dir{d_i}}$.
If we visit a synthesising argument $\synbiarg[i+1]$, then we may extend the domain of $\rho_i$ to $\rho_{i+1}$ with the synthesised variables $\fv(A_{i + 1})$, provided that type synthesis is successful and that the synthesised type can be \emph{unified with $A_{i+ 1}$}.
If we go through every $t_i$ successfully, then we will have a total substitution $\rho_n$ by mode-correctness and a derivation of $\Gamma, \vars{x}_i : \Delta_i |-_{\Sigma, \Omega} t_i :^{\dir{d_i}} \simsub{A}{\rho_n}$ for each sub-term~$t_i$.

\begin{remark}
To make the argument above sound, it is necessary to compare types and solve a unification problem.
Hence, we assume that the set~$\Xi$ of type variables has a decidable equality, thereby ensuring that the set $\Type_{\Sigma}(\Xi)$ of types also has a decidable equality.\footnote{%
To simplify our choice, we may confine $\Xi$ to any set within the family of sets $\Fin(n)$ of naturals less than~$n$, given that these sets have a decidable equality and the arity of a type construct is finite.
Indeed, in our formalisation, we adopt $\Fin(n)$ as the set of type variables in the definition of $\Type_{\Sigma}$\ifarxiv (see \Cref{sec:formalisation} for details)\fi.
For the sake of clarity in presentation, though, we just use named variables and assume that $\Xi$ has a decidable equality.}
\end{remark}
We need some auxiliary definitions for the notion of extension to state the unification problem:
\begin{defn}
By an \emph{extension} $\sigma \geq \rho$ of a partial substitution $\rho$ we mean that the domain $\dom(\sigma)$ of $\sigma$ contains the domain of $\rho$ and $\sigma(x) = \rho(x)$ for every $x$ in $\dom(\rho)$.
  By a \emph{minimal extension} $\bar{\rho}$ of $\rho$ satisfying $P$ we mean an extension $\bar{\rho} \geq \rho$ with $P(\bar{\rho})$ such that $\sigma \geq \bar{\rho}$ whenever $\sigma \geq \rho$ and $P(\sigma)$.
\end{defn}
\begin{lemma}\label{lem:unify}
  For any $A$ of $\Type_{\Sigma}(\Xi)$, $B$ of $\Type_{\Sigma}(\emptyset)$, and a partial substitution $\rho \colon \Xi \to \Type_{\Sigma}(\emptyset)$, either
  \begin{enumerate}
    \item there is a minimal extension $\bar{\rho}$ of $\rho$ such that $\simsub{A}{\bar{\rho}} = B$, or
    \item there is no extension $\sigma$ of $\rho$ such that $\simsub{A}{\sigma} = B$
  \end{enumerate}
\end{lemma}
This lemma can be derived from the correctness of first-order unification~\citep{McBride2003,McBride2003a}, or be proved directly without unification.
We are now ready for \cref{thm:bidirectional-type-synthesis-checking}:

\begin{proof}[of {\Cref{thm:bidirectional-type-synthesis-checking}}]
  We prove this statement by induction on the mode derivation $\erase{\Gamma} |-_{\Sigma, \Omega} \isTerm{t}^{\dir{d}}$.
  The two cases \SynRule{Var} and \SynRule{Anno} are straightforward and independent of mode-correctness.
  The case \ChkRule{Sub} invokes the uniqueness of synthesised types to refute the case that $\Gamma |-_{\Sigma, \Omega} \isTerm{t} \syn B$ but $A \neq B$ for a given type $A$.
  The first three cases follow essentially the same reasoning provided by \citet{Wadler2022}, so we only detail the last case \Rule{Op}, which is new (but has been discussed informally above).
  For brevity we omit the subscript $(\Sigma, \Omega)$.

      For a mode derivation of $\erase{\Gamma} |- \tmOpts^{\dir{d}}$, we first claim:
      \begin{claim}\label{lem:args-induction}
        For an argument list $\biargs$ and any partial substitution $\rho$ from $\Xi$ to $\emptyset$,
        either
        \begin{enumerate}
          \item there is a minimal extension $\ext{\rho}$ of $\rho$ such that 
            \begin{equation} \label{eq:claim}
              \dom(\ext{\rho}) \supseteq \synvar(\biargs)
              \;\text{and}\;
%              \text{the domain of $\ext{\rho}$ contains $\synvar(\biargs)$ and\/} \quad
              \Gamma, \vars{x}_\isTerm{i} : \simsub{\Delta_i}{\ext{\rho}} |- \isTerm{t_i} \colon \simsub{A_i}{\ext{\rho}}^{\dir{d_i}}
            \end{equation}
            for $i = 1, \ldots, n$, or
          \item there is no extension $\sigma$ of $\rho$ such that \eqref{eq:claim} holds.
        \end{enumerate}
      \end{claim}

      Then, we proceed with a case analysis on $\dir{d}$ in the mode derivation:
      \begin{itemize}
        \item $\dir{d}$ is $\syn$: We apply our claim with the partial substitution $\rho_0$ defined nowhere.
          \begin{enumerate}
            \item If there is no $\sigma \geq \rho$ such that \eqref{eq:claim} holds but $\Gamma |- \tmOpts \syn A$ for some $A$, then by inversion we have $\rho\colon \Sub{\Xi}{\emptyset}$ such that
              \[
                \Gamma, \vars{x}_\isTerm{i} : \simsub{\Delta_i}{\rho} |- \isTerm{t_i} \colon \simsub{A_i}{\rho}^{\dir{d_i}}
              \]
              for every $i$.
              Obviously, $\rho \geq \rho_0$ and ${\Gamma, \vars{x}_\isTerm{i} : \simsub{\Delta_i}{\rho} |- \isTerm{t_i} \colon \simsub{A_i}{\rho}^{\dir{d_i}}}$ for every $i$,  which contradict the assumption that no such extension exists.

            \item If there exists a minimal $\ext{\rho} \geq \rho_0$ defined on $\synvar(\biargs)$ such that \eqref{eq:claim} holds, then by mode-correctness $\ext{\rho}$ is total, and thus
              \[
                \Gamma |- \tmOpts \syn \simsub{A_0}{\ext{\rho}}.
              \]
          \end{enumerate}

        \item $\dir{d}$ is $\chk$: Let $A$ be a type and apply \Cref{lem:unify} with $\rho_0$ defined nowhere.
          \begin{enumerate}
            \item If there is no $\sigma \geq \rho_0$ s.t.\ $\simsub{A_0}{\sigma} = A$ but $\Gamma |- \tmOpts \chk A$, then inversion gives us a substitution $\rho$ s.t.\ $A = \simsub{A_0}{\rho}$---a contradiction.
            \item If there is a minimal $\ext{\rho} \geq \rho_0$ s.t.\ $\simsub{A_0}{\ext{\rho}} = A$, then apply our claim with $\ext{\rho}$:
              \begin{enumerate}
                \item If no $\sigma \geq \ext{\rho}$ satisfies \eqref{eq:claim} but $\Gamma |- \tmOpts \chk A$, then by inversion there is $\gamma$ s.t.\ $\simsub{A_0}{\gamma} = A$ and ${\Gamma, \vars{x}_\isTerm{i} : \simsub{\Delta_i}{\gamma} |- \isTerm{t_i} \colon \simsub{A_i}{\gamma}^{\dir{d_i}}}$ for every $i$.
                  Given that $\ext{\rho} \geq \rho$ is minimal s.t.\ $\simsub{A_0}{\ext{\rho}} = A$, it follows that $\gamma$ is an extension of $\ext{\rho}$, but by assumption no such an extension satisfying ${\Gamma, \vars{x}_\isTerm{i} : \simsub{\Delta_i}{\gamma} |- \isTerm{t_i} \colon \simsub{A_i}{\gamma}^{\dir{d_i}}}$ exists, thus a contradiction.
                
                \item If there is a minimal $\ext{\ext{\rho}} \geq \ext{\rho}$ s.t.\ \eqref{eq:claim}, then by mode-correctness $\ext{\ext{\rho}}$ is total and
                  \[
                    \Gamma |- \tmOpts \chk \simsub{A_0}{\ext{\ext{\rho}}}
                  \]
                  where $\simsub{A_0}{\ext{\ext{\rho}}} = \simsub{A_0}{\ext{\rho}} = A$ since $\ext{\ext{\rho}}(x) = \ext{\rho}$ for every $x$ in $\dom(\ext{\rho})$.
              \end{enumerate}
          \end{enumerate}
      \end{itemize}
      We have proved the decidability by induction on the derivation $\erase{\Gamma} |-_{\Sigma, \Omega} t^{\dir{d}}$ assuming the claim, which remains to be proved.
      \begin{proof}[of Claim]
        We prove it by induction on the list $\biargs$:
        \begin{enumerate}
          \item For the empty list, $\rho$ is the minimal extension of $\rho$ itself satisfying \eqref{eq:claim} trivially. 
          \item For $\biargvec, \biarg[m+1]$, by induction hypothesis on the list, we have two cases:
            \begin{enumerate}
              \item If there is no $\sigma \geq \rho$ s.t.\ \eqref{eq:claim} holds for all $1 \leq i \leq m$ but a minimal $\gamma \geq \rho$ such that~\eqref{eq:claim} holds for all $1 \leq i \leq m + 1$, then we have a contradiction.
              \item There is a minimal $\ext{\rho} \geq \rho$ s.t.\ \eqref{eq:claim} holds for $1 \leq i \leq m$.
                By case analysis on $\dir{d_{m+1}}$:
                \begin{itemize}
                  \item $\dir{d_{m+1}}$ is $\chk$: By mode-correctness, $\simsub{\Delta_{m+1}}{\ext{\rho}}$ and $\simsub{A_{m+1}}{\ext{\rho}}$ are defined.
                    By the ind.\ hyp.\ $ \Gamma, \vars{x}_\isTerm{m+1} : \simsub{\Delta_{m+1}}{\ext{\rho}} |- \isTerm{t_{m+1}} \chk \simsub{A_{m+1}}{\ext{\rho}}$ is decidable.
                    Clearly, if $\Gamma, \vars{x}_\isTerm{m+1} : \simsub{\Delta_{m+1}}{\ext{\rho}} |- \isTerm{t_{m+1}} \chk \simsub{A_{m+1}}{\ext{\rho}}$ then the desired statement is proved; otherwise we easily derive a contradiction.

                  \item $\dir{d_{m+1}}$ is $\syn$: By mode-correctness, $\simsub{\Delta_{m+1}}{\ext{\rho}}$ is defined.
                    By the ind.\ hyp., `$\Gamma, \vars{x}_\isTerm{m+1} : \simsub{\Delta_{m+1}}{\ext{\rho}} |- \isTerm{t_{m+1}} \syn A$ for some $A$' is decidable:
                    \begin{enumerate}
                      \item If $\Gamma, \vars{x}_\isTerm{m+1} : \simsub{\Delta_{m+1}}{\ext{\rho}} |/- \isTerm{t_{m+1}} \syn A$ for any $A$ but there is $\gamma \geq \rho$ s.t.\ \eqref{eq:claim} holds for $1 \leq i \leq m+1$, then $\gamma \geq \ext{\rho}$.
                        Therefore $\simsub{\Delta_{m+1}}{\ext{\rho}} = \simsub{\Delta_{m+1}}{\gamma}$, and we derive a contradiction because $\Gamma, \vars{x}_\isTerm{m+1} : \simsub{\Delta_{m+1}}{\ext{\rho}} |- \isTerm{t_{m+1}} \syn \simsub{A_{m+1}}{\gamma}$.
                      \item If $\Gamma, \vars{x}_\isTerm{m+1} : \simsub{\Delta_{m+1}}{\ext{\rho}} |- \isTerm{t_{m+1}} \syn A$ for some $A$, then \Cref{lem:unify} gives the following two cases:
                        \begin{itemize}
                          \item Suppose no $\sigma \geq \ext{\rho}$ s.t.\ $\simsub{A_{m+1}}{\sigma} = A$ but an extension $\gamma \geq \rho$ s.t.\ \eqref{eq:claim} holds for $1 \leq i \leq m + 1$. 
                            Then, $\gamma \geq \ext{\rho}$ by the minimality of $\ext{\rho}$ and thus
                            $\Gamma, \vars{x}_\isTerm{m+1} : \simsub{\Delta_{m+1}}{\ext{\rho}} |- \isTerm{t_{m+1}} \syn \simsub{A_{m+1}}{\gamma}$.
                            However, by \Cref{thm:unique-syn}, the synthesised type $\simsub{A_{m+1}}{\gamma}$ must be unique, so $\gamma$ is an extension of $\ext{\rho}$ s.t.\ $\simsub{A_{m+1}}{\gamma} = A$, i.e.\ a contradiction.
                          \item If there is a minimal $\ext{\ext{\rho}} \geq \ext{\rho}$ such that $\simsub{A_{m+1}}{\ext{\ext{\rho}}} = A$, then it is not hard to show that $\ext{\ext{\rho}}$ is also the minimal extension of $\rho$ such that \eqref{eq:claim} holds for all $1 \leq i \leq m + 1$.
                        \end{itemize}
                    \end{enumerate}
                \end{itemize}
            \end{enumerate}
        \end{enumerate}
        We have proved our claim for any argument list by induction.\hfill
        $\blacksquare$
      \end{proof}
%  \end{itemize}
      We have completed the proof of \cref{thm:bidirectional-type-synthesis-checking}. \qed
\end{proof}

The formal counterpart of the above proof in \Agda functions as two top-level programs for type checking and synthesis.
These programs provide either the typing derivation or its negation proof.
Each case analysis branches depending on the outcomes of bidirectional type synthesis and checking for each sub-term, as well as the unification process.
If a negation proof is not of interest in practice, these programs can be simplified by discarding the cases that yield negation proofs.
Alternatively, we could consider generalising typing derivations instead, like our generalised mode derivations (\Cref{fig:generalised-mode-derivations}), to reformulate negation proofs positively to deliver more informative error messages.
This would assist programmers in resolving issues with ill-typed terms, rather than returning a blatant `no'.

\subsection{Trichotomy on raw terms by type synthesis} \label{subsec:trichotomy}

Combining the bidirectional type synthesiser with the mode decorator, soundness, and completeness from \cref{sec:pre-synthesis}, we derive a type synthesiser parameterised by~$(\Sigma, \Omega)$, generalising \cref{thm:implementation}.

\begin{corollary}[Trichotomy on raw terms]\label{cor:trichotomy}
  For any mode-correct bidirectional type system $(\Sigma, \Omega)$, 
  exactly one of the following holds:
  \begin{enumerate}
    \item $\erase{\Gamma} |-_{\Sigma, \Omega} \isTerm{t}^{\syn}$ and $\Gamma |-_{\Sigma, \erase{\Omega}} \isTerm{t} : A$ for some type~$A$,
    \item $\erase{\Gamma} |-_{\Sigma, \Omega} \isTerm{t}^{\syn}$ but $\Gamma |/-_{\Sigma, \erase{\Omega}} \isTerm{t} : A$ for any type~$A$, or 
    \item $\erase{\Gamma} |/-_{\Sigma, \Omega} \isTerm{t}^{\syn}$.
  \end{enumerate}
\end{corollary}

%!TEX root = BiSig.tex

\section{Examples}\label{sec:example}
To exhibit the applicability of our approach, we discuss two more examples:
one has infinitely many operations and the other includes many more constructs than simply typed $\lambda$-calculus, exhibiting the practical side of a general treatment.

\subsection{Spine application}\label{subsec:spine}
A spine application $t\;u_1\;\ldots\;u_n$ is a form of application that consists of a head term $t$ and an indeterminate number of arguments $u_1\;u_2\;\dots\;u_n$.
This arrangement allows direct access to the head term, making it practical in various applications, and has been used by \Agda's core language. %employs this form of application, as does its reflected syntax used for metaprogramming.

At first glance, accommodating this form of application may seem impossible, given that the number of arguments for a construct is finite and has to be fixed.
Nonetheless, the total number of operation symbols a signature can have need not be finite, allowing us to establish a corresponding construct for each number $n$ of arguments, i.e.\ viewing the following rule
\bgroup
\small
  \begin{mathpar}
    \inferrule{\Gamma \vdash t \syn A_1 \bto \left(A_2 \bto \left(\dots \bto \left(A_n \bto B\right)\ldots\right)\right) \and \Gamma \vdash u_1 \chk A_1 \and \!\!\cdots\!\! \and \Gamma \vdash u_n \chk A_n}{\Gamma \vdash t\;u_1\;\ldots\;u_n \syn B}
  \end{mathpar}
\egroup
as a rule schema parametrised by $n$, so the signature $\Omega_{\Lambda}^{\Leftrightarrow}$ can be extended with 
\[
  \aritysymbol{\mathsf{app}_n}{A_1,\ldots, A_n, B}{A_1 \bto \left(A_2 \bto \left(\dots \bto \left(A_n \bto B\right)\ldots\right)\right)^{\syn}, A_1^{\chk}, \ldots, A_n^{\chk}}{B}
\]
Each application $t\;u_1\;\ldots\;u_n$ can be introduced as $\tmOp_{\mathsf{app}_n}(t; u_1; \ldots; u_n)$, thereby exhibiting the necessity of having an arbitrary set for operation symbols.

\subsection{Computational calculi}\label{subsec:PCF}
Implementing a stand-alone type synthesiser for a simply typed language is typically a straightforward task.
However, the code size increases proportionally to the number of type constructs and of arguments associated with each term construct.
When dealing with a fixed number $n$ of type constructs, for each synthesising construct, there are two cases for a checking argument but $n + 1$ cases for each synthesising argument---the successful synthesis of the expected type, an instance where it fails, or $n-1$ cases where the expected type does not match; similarly for a checking construct---making the task tedious.
Thus, having a generator is helpful and can significantly reduce the effort for implementation.

For illustrative purposes, consider a computational calculus~\cite{Moggi1989} with additional constructs listed in \cref{tab:computational-calculus}.
The extended language has `only' 15 constructs, far fewer than a realistic programming language would have, but there are nearly 100 possible cases to consider in bidirectional type synthesis.

On the other hand, similar to a parser generator, a type-synthesiser generator only needs a specification $(\Sigma, \Omega)$ from the user to produce a corresponding synthesiser. 
In more detail, such a specification can be derived by extending $\Sigma_{\bto}$ with additional type constructs with $\mathsf{nat}$, $\mathsf{prod}$, $\mathsf{sum}$, and $\mathsf{T}$ such that
\[
  \arity({\mathsf{nat}}) = 0\qquad
  \arity({\mathsf{T}}) = 1\qquad
  \arity({\mathsf{prod}}) = \arity({\mathsf{sum}}) = 2.
\]
Types $\mathtt{nat}$, $A \times B$, $A + B$, and $T(A)$ are given as $\tyOp_{\mathsf{nat}}$, $\tyOp_{\mathsf{prod}}(A, B)$, $\tyOp_{\mathsf{sum}}(A, B)$, and $\tyOp_{\mathsf{T}}(A)$ respectively.
The signature $\Omega_{\Lambda}^{\Leftrightarrow}$ is then extended with operations listed in \cref{tab:computational-calculus}.
Mode-correctness can be derived by \cref{lem:decidability-mode-correctness} and its type synthesiser by \cref{cor:trichotomy} with the specification~$(\Sigma, \Omega)$ directly.%
\footnote{For a demonstration in \Agda, see \cref{sec:formalisation}.}


%!TEX root = BiSig.tex

\section{Discussion} \label{sec:future}
We believe that our formal treatment lays a foundation for further investigation, as the essential aspects of bidirectional typing have been studied rigorously. 
While our current development is based on simply typed languages to highlight the core ideas, it is evident that many concepts and extensions remain unexplored.
We will discuss related work and outline potential directions for future research.
\subsubsection{Language formalisation frameworks}
Many general definitions of logics and type theories and frameworks for defining them have been proposed.

\citet{Harper1993a,Harper2007}
\cite{Uemura2021}
\cite{Bauer2020,Haselwarter2021,Bauer2022a}
Our general definitions are akin to the definitions used in other language formalisation frameworks~\citep{Ahrens2022,Allais2021,Fiore2022} in a proof assistant---the concept of a signature is at object level.

%\subsubsection{Generic bidirectional type synthesis beyond simple types}
%Clearly, bidirectional type synthesis becomes prominent because it can scale up to deal with more complex types than simple types.
%As for a general treatment for another class of languages, one has to devise a general definition of languages, which may have been used by existing language-formalisation frameworks.
%Then, one could enrich the general definition with modes and adopt the definition of mode-correctness accordingly. 
%Soundness and completeness of our formulation should carry over, since they are only about the separation and combination of mode and typing information for a given raw term (in a syntax-directed formulation).
%Mode decoration is about annotating a raw term with modes and marking missing annotations, used for guiding the direction during bidirectional type synthesis, so it should also carry over.
%What makes the general development of bidirectional type synthesis challenging is inherent to the tension between expressiveness and usability for type synthesis and the general treatment for other language facilities.
%We discusses the cases for dependent types and for polymorphic types below.
%
%\paragraph{Dependent types} Type synthesis for a dependent type theory involves term equality check for type equality problem, so a generic decidable bidirectional type synthesis relies on the decidability of the term equality check which in turn requires the ability of specifying computation rules within a language specification.
%\varcitet{Felicissimo2023}{'s computational LF} is capable of defining dependent type theories with computational rules like \textsc{Dedukti}~\cite{Assaf2016} and generic bidirectional type synthesis.
%Apart from their more general definition targeting dependent type theories, generic bidirectional type synthesis \cite[Theorem~3.12]{Felicissimo2023} that is decidable, sound, and complete (with respect to decorated terms) is possible under the assumption that a given language specification is mode-correct (or well-typed \cite[Section~3.2]{Felicissimo2023}) and equality check is well-behaved: the specified computation rules are type-preserving, confluent, and strongly normalising.
%Yet, the last assumption is hard to establish even for a specific language and still lacks a general consideration.
%
%\paragraph{Polymorphic types} For \SystemF and other languages where type-level variable binding is allowed, we can start with the notion of polymorphic signature $(\Sigma, \Omega)$ by~\citet{Hamana2011} whereas the arity of each type construct $\sigma$ in $\Sigma$ becomes a binding arity with a single sort $*$ and accordingly a term construct can specify a pair of extension contexts~$\left<\Xi\right>[\Delta]$ for term variables $\Delta$ and type variables $\Xi$.
%It is not hard to see how general definitions for bidirectional typing rules and mode derivations can be extended from \varcitet[Section~3.4]{Hamana2011}{'s definitions}.
%For example, the universal type $\forall \alpha.\, A$ and type abstraction in \SystemF can then be specified as operations
%\[
%  \mathsf{all}\colon *\rhd [*]* \to *
%  \quad\text{and}\quad
%  \mathsf{tabs}\colon [*]A \rhd \left< * \right>A^{\chk} \to
%  \tyOp_{\mathsf{all}}(\alpha.\, A)^{\chk}
%\]
%\Cref{thm:bidirectional-type-synthesis-checking} should also carry over, since no equation is imposed on types and no guess work is needed.
%However, explicit type application is unusable in practice but implicit application results the \emph{instantiation problem}~\cite[Section~5]{Dunfield2021} which amounts to replacing type equality check with checking the subtyping relation $A <: B$ in rule $\ChkRule{Sub}$ and type synthesis with the guess of $B$ in $\forall \alpha.\, A <: A[B/\alpha]$. 
%A satisfactory theory for bidirectional typing with polymorphism and subtyping should be neutral and thus accommodate various solutions to the instantiation problem, which is still under development.
%
\subsubsection{Generic bidirectional type synthesis beyond simple types}

Bidirectional type synthesis plays a crucial role in handling more complex types than simple types.
To provide a formal treatment for a broader class of languages, we need a general definition of languages in question, which may have been used in existing theories or frameworks.
Then, this general definition can be augmented with modes and the concept of mode-correctness can be introduced.
Soundness and completeness (\cref{lem:soundness-completeness}) should naturally extend, as they pertain solely to the separation and combination of mode and typing information within a given raw term in a syntax-directed formulation.
Mode decoration (\cref{sec:mode-decoration}) involves annotating a raw term with modes and marking missing annotations should also work.
The challenge in developing bidirectional type synthesis on a more general case comes from striking the right balance between expressiveness and usability for type synthesis while considering other language features.
We discuss cases involving dependent types and polymorphic types below.

\paragraph{Dependent Types}
In the context of dependent type theory, type synthesis entails a term equality check for type equality.
Achieving generic decidable bidirectional type synthesis depends on the decidability of term equality, which, in turn, relies on the ability to specify computation rules within a language specification.

\varcitet{Felicissimo2023}{'s computation LF} is capable of defining dependent type theories with computational rules, similar to Dedukti, and enabling generic bidirectional type synthesis.
Generic bidirectional type synthesis, as presented in Felicissimo's work, is decidable, sound, and complete with respect to decorated terms, provided that the underlying language specification is mode-correct, and the equality check is well-behaved, i.e.\ type-preserving, confluent, and strongly normalising.
However, the last assumption is challenging to establish, even for specific languages and lacks a comprehensive general understanding.

\paragraph{Polymorphic Types}
In the case of languages like \SystemF and others that permit type-level variable binding, we can start with the notion of a polymorphic signature, as introduced by Hamana---each type construct in a signature is specified by a binding arity with a single sort $*$ and a term construct can employ a pair of extension contexts for term variables and type variables.

Extending general definitions for bidirectional typing and mode derivations from \citeauthor{Hamana2011}'s work is straightforward. 
For example, the universal type $\forall \alpha.\, A$ and type abstraction in \SystemF can be specified as operations:
\[
\mathsf{all} : * \rhd [*] * \to *
\quad\text{and}\quad
\mathsf{tabs} : [*] A \rhd \left< * \right> A^{\chk} \to \tyOp_{\mathsf{all}}(\alpha. A)^{\chk}
\]
The decidability of bidirectional type synthesis checking (\cref{thm:bidirectional-type-synthesis-checking}) should also carry over, as no equations are imposed on types and no guessing is required.

However, explicit type application is impractical but implicit application results in the \emph{instantiation problem} which amounts to replacing type equality checks $A = B$ with checking subtyping $A <: B$ and guessing $B$ in $\forall \alpha. A <: A[B/\alpha]$ in the presence of universal quantification.
A general theory for bidirectional typing with polymorphism and subtyping that accommodates various solutions to the instantiation problem is still under development.

\subsubsection{More characteristics of bidirectional typing}

\subsubsection{Conclusion}
There are still many concepts that are possible to articulate such as, for example, Pfenning's recipe for bidirectionalising typing rules.
In contrast, there are concepts that may be hard to pin down, notably `annotation character'~\cite{Dunfield2021}, which is roughly about how easy it is for the user to write annotated programs.
These considerations may continue to require creativity from the designer of type systems whereas its implementation could be automated entirely.


\bibliographystyle{splncs04nat}
\bibliography{ref}

\ifarxiv

\appendix
\subfile{S7-formalisation}

%!TEX root = BiSig.tex

\section{Detailed definitions}\label{sec:defs-proofs}

\begin{definition}\label{def:syn-var}
  The subset $\fv(\Delta)$ of variables in an extension context~$\Delta$ is defined by\/ $\fv(\cdot) = \emptyset$ and\/ $\fv(\Delta, A) = \fv(\Delta) \cup \fv(A)$.
  For an argument list $\biargs$, the set of \emph{synthesised type variables} is defined by 
  \begin{align*}
    \synvar(\cdot)                   & = \emptyset  \\
    \synvar(\biargs, \chkbiarg[n+1]) & = \phantom{\fv(A_{n+1}) \cup {}} \synvar(\biargs) \\
    \synvar(\biargs, \synbiarg[n+1]) & = \fv(A_{n+1}) \cup           \synvar(\biargs).
  \end{align*}
\end{definition}

\fi

\end{document}
