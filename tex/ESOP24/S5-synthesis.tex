%!TEX root = BiSig.tex

\section{Bidirectional type synthesis and checking} \label{sec:type-synthesis}
This section focuses on defining mode-correctness and deriving bidirectional type synthesis for any mode-correct bidirectional type system $(\Sigma, \Omega)$.
We start with \Cref{sec:mode-correctness} by defining mode-correctness and showing the uniqueness of synthesised types.
This uniqueness means that any two synthesised types for the same raw term $t$ under the same context $\Gamma$ have to be equal.
It will be used especially in \Cref{subsec:bidirectional-synthesis-checking} for the proof of the decidability of bidirectional type synthesis and checking.
Then, we conclude this section with the trichotomy on raw terms in \Cref{subsec:trichotomy}.

\subsection{Mode correctness}\label{sec:mode-correctness}
As \citet{Dunfield2021} outlined, mode-correctness for a bidirectional typing rule means that 
\begin{inlineenum}
\item each `input' type variable in a premise must be an `output' variable in `earlier' premises, or provided by the conclusion if the rule is checking;
\item each `output' type variable in the conclusion should be some `output' variable in a premise if the rule is synthesising.
\end{inlineenum}
Here `input' variables refer to variables in an extension context and in a checking premise.
It is important to note that the order of premises in a bidirectional typing rule also matters, since synthesised type variables are instantiated incrementally during type synthesis.

Consider the rule $\ChkRule{Abs}$ (\Cref{fig:STLC-bidirectional-typing-derivations}) as an example.
This rule is mode-correct, as the type variables $A$ and $B$ in its only premise are already provided by its conclusion $A \bto B$.
Likewise, the rule $\SynRule{App}$ for an application term $\isTerm{t\;u}$ is mode-correct because:
\begin{inlineenum}
\item the type $A \bto B$ of the first argument $t$ is synthesised, thereby ensuring type variables $A$ and $B$ must be known if successfully synthesised;
\item the type of the second argument $u$ is checked against $A$, which has been synthesised earlier;
\item as a result, the type of an application $t\;u$ can be synthesised.
\end{inlineenum}

Now let us define mode-correctness rigorously.
As we have outlined, the condition of mode-correctness for a synthesising rule is different from that of a checking rule, and the argument order also matters.
Defining the condition directly for a rule, and thus in our setting for an operation, can be somewhat intricate.
Instead, we choose to define the conditions for the argument list---more specifically, triples $\biargvec$ of an extension context $\Delta_i$, a type $A_i$, and a mode $\dir{d_i}$---pertaining to an operation, for an operation, and subsequently for a signature.
We also need some auxiliary definitions for the subset of variables of a type and of an extension context, and the set of variables that have been synthesised:
\begin{defn}
  The finite subsets\footnote{%
  There are various definitions for finite subsets of a set within type theory, but for our purposes the choice among these definitions is not a matter of concern.}
  of \emph{(free) variables} of a type $A$ and of variables in an extension context~$\Delta$ are denoted by $\fv(A)$ and $\fv(\Delta)$ respectively.
  For an argument list $\biargs$ the set of type variables $A_{i}^{\dir{d_i}}$ with $d_i$ being $\syn$ is denoted by $\synvar(\biargs)$, i.e.\ $\synvar$ gives the set of type variables that will be synthesised during type synthesis.
\end{defn}

\begin{defn}\label{def:mode-correctness-args}
  The \emph{mode-correctness} $\MCas\left(\biargvec\right)$ for an argument list $\biargs$ with respect to a subset $S$ of $\Xi$ is a predicate defined by
  {\small
  \begin{align*}
    \MCas( \cdot ) & = \top \\
    \MCas\left(\biargvec, \chkbiarg[n]\right)
                                  & = \fv(\Delta_n, A_n) \subseteq \left( S \cup \synvar\left(\biargvec\right)\right) \land \MCas\left(\biargvec\right) \\
    \label{eq:MC2} \MCas\left(\biargvec, \synbiarg[n]\right) 
                                  & = \phantom{, A_n} \fv(\Delta_n) \subseteq \left( S \cup \synvar\left(\biargvec\right)\right) \land  \MCas\left(\biargvec\right)
  \end{align*}}
  where $\MCas( \cdot ) = \top$ means that an empty list is always mode-correct.
\end{defn}
This definition encapsulates the idea that every `input' type variable, possibly derived from an extension context~$\Delta_n$ or a checking argument~$A_n$, must be an `output' variable from $\synvar(\biargvec)$ or, if the rule is checking, belong to the set $S$ of `input' variables in its conclusion.
This condition must also be met for every tail of the argument list to ensure that `output' variables accessible at each argument are from preceding arguments only, hence an inductive definition.
\begin{defn}\label{def:mode-correctness}
  An arity $\biarity$ is \emph{mode-correct} if 
  \begin{enumerate}
    \item either $d$ is $\chk$, its argument list is mode-correct with respect to $\fv(A_0)$, and the union $\fv(A_0) \cup \synvar(\biargvec)$ contains every inhabitant of $\Xi$;
    \item or $d$ is $\syn$, its argument list is mode-correct with respect to $\emptyset$, and $\synvar(\biargvec)$ contains every inhabitant of $\Xi$ and, particularly, $\fv(A_0)$.
  \end{enumerate}
  A bidirectional binding signature $\Omega$ is \emph{mode-correct} if every operation's arity is mode-correct.
\end{defn}
For a checking operation, an `input' variable of an argument could be derived from~$A_0$, as these are known during type checking as an input.
Since every inhabitant of~$\Xi$ can be located in either~$A_0$ or synthesised variables, we can determine a concrete type for each inhabitant of~$\Xi$ during type synthesis.
On the other hand, for a synthesising operation, we do not have any known variables at the onset of type synthesis, so the argument list should be mode-correct with respect to~$\emptyset$.
Also, the set of synthesised variables alone should include every type variable in~$\Xi$ and particularly in~$A_n$.

\begin{remark}
  Mode-correctness is fundamentally a condition for bidirectional typing \emph{rules}, not for derivations.
  Thus, this property cannot be formulated without treating rules as some mathematical object such as those general definitions in \Cref{sec:defs}.
  This contrasts with the properties in \Cref{sec:pre-synthesis}, which can still be specified for individual systems in the absence of a general definition.
\end{remark}

It is easy to check the bidirectional type system $(\Sigma_{\bto}, \Omega^{\Leftrightarrow}_{\Lambda}$) for the simply typed $\lambda$-calculus is mode-correct by definition or by the following lemma:
\begin{lemma}\label{lem:decidability-mode-correctness}
  For any bidirectional binding arity $\biarity$, it is decidable whether it is mode-correct.
\end{lemma}


Now, we set out to show the uniqueness of synthesised types for a mode-correct bidirectional type system.
For a specific system, its proof is typically a straightforward induction on the typing derivations.
However, since mode-correctness is inductively defined on the argument list, our proof proceeds by induction on both the typing derivations and the argument list:
\begin{lemma}[Uniqueness of synthesised types]\label{thm:unique-syn}
  In a mode-correct bidirectional type system $(\Sigma, \Omega)$, the synthesised types of any two derivations
  \[
    \Gamma |-_{\Sigma, \Omega} \isTerm{t} \syn A
    \quad\text{and}\quad
    \Gamma |-_{\Sigma, \Omega} \isTerm{t} \syn B
  \]
  for the same term $t$ must be equal, i.e.\ $A = B$.
\end{lemma}
\begin{proof}%[Proof of \Cref{thm:unique-syn}]
  We prove the statement by induction on derivations $d_1$ and $d_2$ for $\Gamma |-_{\Sigma, \Omega} \isTerm{t} \syn A$ and $\Gamma |-_{\Sigma, \Omega} \isTerm{t} \syn B$.
  Our system is syntax-directed, so $d_1$ and $d_2$ must be derived from the same rule: 
  \begin{itemize}
    \item $\SynRule{Var}$ follows from that each variable as a raw term refers to the same variable in its context.
    \item $\SynRule{Anno}$ holds trivially, since the synthesised type $A$ is from the term $t \annotatecolon A$ in question.
    \item $\Rule{Op}$: Recall that a derivation of\/ $\Gamma |- \tmOpts \syn A$ contains a substitution $\rho$ from the local context $\Xi$ to concrete types.
      To prove that any two typing derivations has the same synthesised type, it suffices to show that those substitutions $\rho_1$~and~$\rho_2$ of $d_1$~and~$d_2$, respectively, agree on variables in $\synvar(\biargs)$ so that $\simsub{A_0}{\rho_1} = \simsub{A_0}{\rho_2}$.
      We prove it by induction on the argument list:
      \begin{enumerate}
        \item For the empty list, the statement is vacuously true.
        \item If $\dir{d_{i+1}}$ is~$\chk$, then the statement holds by induction hypothesis.
        \item If $\dir{d_{i+1}}$ is~$\syn$, then $\simsub{\Delta_{i+1}}{\rho_1} = \simsub{\Delta_{i+1}}{\rho_2}$ and induction hypothesis (of the list).
          Therefore, under the same context $\Gamma, \simsub{\Delta_{i+1}}{\rho_1} = \Gamma, \simsub{\Delta_{i+1}}{\rho_2}$ the term $t_{i+1}$ must have the same synthesised type $\simsub{A_{i+1}}{\rho_1} = \simsub{A_{i+1}}{\rho_1}$ by induction hypothesis (of the typing derivation), so $\rho_1$ and $\rho_2$ agree on $\fv(A_{i+1})$ in addition to $\synvar(\biargs)$.
      \end{enumerate}
  \end{itemize}\qed
\end{proof}

%Uniqueness of the synthesised types is a prevalent property in bidirectional type systems, although the specific proofs can vary depending on the constructs in the system.
%For instance, for derivations of $\Gamma |- t\;u \syn B_i$ for $i = 1, 2$ in simply typed $\lambda$-calculus, the hypothesis is applied to their sub-derivations $\Gamma |- t \syn A_i \bto B_i$ to conclude that $A_1 \bto B_1 = A_2 \bto B_2$ and thus $B_1 = B_2$.
%On the other hand, our proof is based on mode-correctness and need not consider specific sub-derivations.

\subsection{Decidability of bidirectional type synthesis and checking}\label{subsec:bidirectional-synthesis-checking}

We have arrived at the main technical contribution of this paper.

\begin{theorem}\label{thm:bidirectional-type-synthesis-checking}
  In a mode-correct bidirectional type system $(\Sigma, \Omega)$,
  \begin{enumerate}
    \item if\/ $\erase{\Gamma} |-_{\Sigma, \Omega} \isTerm{t}^{\syn}$, then it is decidable whether $\Gamma |-_{\Sigma, \Omega} \isTerm{t} \syn A$ for some $A$;
    \item if\/ $\erase{\Gamma} |-_{\Sigma, \Omega} \isTerm{t}^{\chk}$, then it is decidable for any~$A$ whether $\Gamma |-_{\Sigma, \Omega} \isTerm{t} \chk A$.
  \end{enumerate}
\end{theorem}

The interesting part of the theorem is the case for the $\Rule{Op}$ rule and it is rather complex.
Here we shall give its insight first instead of jumping into the details.
Recall that a typing derivation for $\tmOpts$ contains a substitution $\rho\colon \Xi \to \Type_{\Sigma}(\emptyset)$.
The goal of type synthesis is exactly to define such a substitution $\rho$, so we have to start with an `accumulating' substitution: a substitution $\rho_0$ that is partially defined on $\fv(A_0)$ if $d$ is $\chk$ or otherwise nowhere.
By mode-correctness, the accumulating substitution~$\rho_i$ will be defined on enough synthesised variables so that type synthesis or checking can be performed on $t_{i}$ with the context $\Gamma, \vars{x}_{i} : \simsub{\Delta_{i}}{\rho_{i}}$ based on its mode derivation $\erase{\Gamma}, \vars{x}_i |-_{\Sigma, \Omega} t_i^{\dir{d_i}}$.
If we visit a synthesising argument $\synbiarg[i+1]$, then we may extend the domain of $\rho_i$ to include the synthesised variables $\fv(A_{i + 1})$ if type synthesis is successful and also that the synthesised type can be \emph{unified with $A_{i+ 1}$} and thereby \emph{extend} $\rho_i$ to $\bar{\rho_i} = \rho_{i+1}$ with the unifier.
Then, if we go through every $t_i$ successfully, we will have a total substitution $\rho_n$ by mode-correctness and a derivation of $\Gamma, \vars{x}_i : \Delta_i |-_{\Sigma, \Omega} t_i :^{\dir{d_i}} \simsub{A}{\rho_n}$ for each sub-term $t_i$.

\begin{remark}
To make the argument above sound, it is necessary to compare types and solve a unification problem.
Hence, we assume that the set~$\Xi$ of type variables has a decidable equality, thereby ensuring that the set $\Type_{\Sigma}(\Xi)$ of types also has a decidable equality.\footnote{%
To simplify our choice, we may confine $\Xi$ to any set within the family of sets $\Fin(n)$ of naturals less than~$n$, given that these sets have a decidable equality and the arity of a type construct is finite.
Indeed, in our formalisation, we adopt $\Fin(n)$ as the set of type variables in the definition of $\Type_{\Sigma}$\ifarxiv (see \Cref{sec:formalisation} for details)\fi.
For the sake of clarity in presentation, though, we use named variables and just assume that $\Xi$ has a decidable equality.}
\end{remark}
We need some auxiliary definitions for the notion of extension to state the unification problem:
\begin{defn}
By an \emph{extension}\/ $\sigma \geq \rho$ of a partial substitution $\rho$ we mean that the domain $\dom(\sigma)$ of $\sigma$ contains the domain of $\rho$ and $\sigma(x) = \rho(x)$ for every\/ $x$ in $\dom(\rho)$.
  By a \emph{minimal extension}\/ $\bar{\rho}$ of $\rho$ satisfying $P$ we mean an extension $\bar{\rho} \geq \rho$ with $P(\bar{\rho})$ such that $\sigma \geq \bar{\rho}$ whenever $\sigma \geq \rho$ and $P(\sigma)$.
\end{defn}
\begin{lemma}\label{lem:unify}
  For any\/ $A$ of\/ $\Type_{\Sigma}(\Xi)$, $B$ of\/ $\Type_{\Sigma}(\emptyset)$, and a partial substitution\/ $\rho \colon \Xi \to \Type_{\Sigma}(\emptyset)$, 
  \begin{enumerate}
    \item either there is a minimal extension\/ $\bar{\rho}$ of\/ $\rho$ such that\/ $\simsub{A}{\bar{\rho}} = B$,
    \item or there is no extension\/ $\sigma$ of\/ $\rho$ such that\/ $\simsub{A}{\sigma} = B$
  \end{enumerate}
\end{lemma}
This lemma can be inferred from the correctness of first-order unification~\citep{McBride2003,McBride2003a}, or be proved directly without unification.
We are now ready for \cref{thm:bidirectional-type-synthesis-checking}:

\begin{proof}[of {\Cref{thm:bidirectional-type-synthesis-checking}}]
  We prove this statement by induction on the mode derivation\/ $\erase{\Gamma} |-_{\Sigma, \Omega} \isTerm{t}^{\dir{d}}$.
  The two cases \SynRule{Var} and \SynRule{Anno} are straightforward and independent of mode-correctness.
  The case \ChkRule{Sub} invokes the uniqueness of synthesised types to refute the case that $\Gamma |-_{\Sigma, \Omega} \isTerm{t} \syn B$ but $A \neq B$ for a given type $A$.
  The first three cases follow essentially the same reasoning provided by \citet{Wadler2022}, so we only detail the last case \Rule{Op} which is new and has been discussed informally above.
  For brevity we omit the subscript $(\Sigma, \Omega)$.

      For a mode derivation of $\erase{\Gamma} |- \tmOpts^{\dir{d}}$, we first claim:
      \begin{claim}\label{lem:args-induction}
        For an argument list $\biargs$ and any partial substitution $\rho$ from $\Xi$ to $\emptyset$
        \begin{enumerate}
          \item either there is a minimal extension $\ext{\rho}$ of $\rho$ such that 
            \begin{equation} \label{eq:claim}
              \dom(\ext{\rho}) \supseteq \synvar(\biargs)
              \;\text{and}\;
%              \text{the domain of $\ext{\rho}$ contains $\synvar(\biargs)$ and\/} \quad
              \Gamma, \vars{x}_\isTerm{i} : \simsub{\Delta_i}{\ext{\rho}} |- \isTerm{t_i} \colon \simsub{A_i}{\ext{\rho}}^{\dir{d_i}}
            \end{equation}
            for $i = 1, \ldots, n$
          \item or there is no extension $\sigma$ of $\rho$ such that \eqref{eq:claim} holds.
        \end{enumerate}
      \end{claim}

      Then, we proceed with a case analysis on $\dir{d}$ in the mode derivation:
      \begin{itemize}
        \item $\dir{d}$ is $\syn$: We apply our claim with the partial substitution $\rho_0$ defined nowhere.
          \begin{enumerate}
            \item If there is no $\sigma \geq \rho$ such that \eqref{eq:claim} holds but $\Gamma |- \tmOpts \syn A$ for some $A$, then by inversion we have $\rho\colon \Sub{\Xi}{\emptyset}$ such that
              \[
                \Gamma, \vars{x}_\isTerm{i} : \simsub{\Delta_i}{\rho} |- \isTerm{t_i} \colon \simsub{A_i}{\rho}^{\dir{d_i}}
              \]
              for every $i$.
              Obviously, $\rho \geq \rho_0$ and ${\Gamma, \vars{x}_\isTerm{i} : \simsub{\Delta_i}{\rho} |- \isTerm{t_i} \colon \simsub{A_i}{\rho}^{\dir{d_i}}}$ for every $i$ and it contradicts the assumption that no such an extension exists.

            \item If there exists a minimal $\ext{\rho} \geq \rho_0$ defined on $\synvar(\biargs)$ such that \eqref{eq:claim} holds, then by mode-correctness $\ext{\rho}$ is total and thus
              \[
                \Gamma |- \tmOpts \syn \simsub{A_0}{\ext{\rho}}.
              \]
          \end{enumerate}

        \item $\dir{d}$ is $\chk$: Let $A$ be a type and apply \Cref{lem:unify} with $\rho_0$ defined nowhere.
          \begin{enumerate}
            \item If there is no $\sigma \geq \rho_0$ s.t.\ $\simsub{A_0}{\sigma} = A$ but $\Gamma |- \tmOpts \chk A$, then inversion gives us a substitution $\rho$ s.t.\ $A = \simsub{A_0}{\rho}$---a contradiction.
            \item If there is a minimal $\ext{\rho} \geq \rho_0$ s.t.\ $\simsub{A_0}{\ext{\rho}} = A$, then apply our claim with $\ext{\rho}$:
              \begin{enumerate}
                \item If there is no $\sigma \geq \ext{\rho}$ satisfying \eqref{eq:claim} but $\Gamma |- \tmOpts \chk A$, then by inversion there is $\gamma$ s.t.\ $\simsub{A_0}{\gamma} = A$ and, for every $i$, ${\Gamma, \vars{x}_\isTerm{i} : \simsub{\Delta_i}{\gamma} |- \isTerm{t_i} \colon \simsub{A_i}{\gamma}^{\dir{d_i}}}$.
                  Given that $\ext{\rho} \geq \rho$ is minimal s.t.\ $\simsub{A_0}{\ext{\rho}} = A$, then $\gamma$ is an extension of $\ext{\rho}$ but by assumption no such an extension satisfying ${\Gamma, \vars{x}_\isTerm{i} : \simsub{\Delta_i}{\gamma} |- \isTerm{t_i} \colon \simsub{A_i}{\gamma}^{\dir{d_i}}}$ exists, thus a contradiction.
                
                \item If there is a minimal $\ext{\ext{\rho}} \geq \ext{\rho}$ s.t.\ \eqref{eq:claim}, then by mode-correctness $\ext{\ext{\rho}}$ is total and
                  \[
                    \Gamma |- \tmOpts \chk \simsub{A_0}{\ext{\ext{\rho}}}
                  \]
                  where $\simsub{A_0}{\ext{\ext{\rho}}} = \simsub{A_0}{\ext{\rho}} = A$ since $\ext{\ext{\rho}}(x) = \ext{\rho}$ for every $x$ in $\dom(\ext{\rho})$.
              \end{enumerate}
          \end{enumerate}
      \end{itemize}
      \begin{proof}[of Claim]
        We prove it by induction on the list $\biargs$:
        \begin{enumerate}
          \item For the empty list, $\rho$ is the minimal extension of $\rho$ itself satisfying \eqref{eq:claim} trivially. 
          \item For $\biargvec, \biarg[m+1]$, by induction hypothesis on the list, we have two cases:
            \begin{enumerate}
              \item If there is no $\sigma \geq \rho$ s.t.\ \eqref{eq:claim} holds for all $1 \leq i \leq m$ but a minimal $\gamma \geq \rho$ such that~\eqref{eq:claim} holds for all $1 \leq i \leq m + 1$, then we have a contradiction.
              \item There is a minimal $\ext{\rho} \geq \rho$ s.t.\ \eqref{eq:claim} holds for $1 \leq i \leq m$.
                By case analysis on $\dir{d_{m+1}}$:
                \begin{itemize}
                  \item $\dir{d_{m+1}}$ is $\chk$: By mode-correctness, $\simsub{\Delta_{m+1}}{\ext{\rho}}$ and $\simsub{A_{m+1}}{\ext{\rho}}$ are defined.
                    By the ind.\ hyp.\ $ \Gamma, \vars{x}_\isTerm{m+1} : \simsub{\Delta_{m+1}}{\ext{\rho}} |- \isTerm{t_{m+1}} \chk \simsub{A_{m+1}}{\ext{\rho}}$ is decidable.
                    Clearly, if $\Gamma, \vars{x}_\isTerm{m+1} : \simsub{\Delta_{m+1}}{\ext{\rho}} |- \isTerm{t_{m+1}} \chk \simsub{A_{m+1}}{\ext{\rho}}$ then the desired statement is proved; otherwise we easily derive a contradiction.

                  \item $\dir{d_{m+1}}$ is $\syn$: By mode-correctness, $\simsub{\Delta_{m+1}}{\ext{\rho}}$ is defined.
                    By the ind.\ hyp., `$\Gamma, \vars{x}_\isTerm{m+1} : \simsub{\Delta_{m+1}}{\ext{\rho}} |- \isTerm{t_{m+1}} \syn A$ for some $A$' is decidable:
                    \begin{enumerate}
                      \item If $\Gamma, \vars{x}_\isTerm{m+1} : \simsub{\Delta_{m+1}}{\ext{\rho}} |/- \isTerm{t_{m+1}} \syn A$ for any $A$ but there is $\gamma \geq \rho$ s.t.\ \eqref{eq:claim} holds for $1 \leq i \leq m+1$, then $\gamma \geq \ext{\rho}$.
                        Therefore $\simsub{\Delta_{m+1}}{\ext{\rho}} = \simsub{\Delta_{m+1}}{\gamma}$ and we derive a contradiction because $\Gamma, \vars{x}_\isTerm{m+1} : \simsub{\Delta_{m+1}}{\ext{\rho}} |- \isTerm{t_{m+1}} \syn \simsub{A_{m+1}}{\gamma}$.
                      \item If $\Gamma, \vars{x}_\isTerm{m+1} : \simsub{\Delta_{m+1}}{\ext{\rho}} |- \isTerm{t_{m+1}} \syn A$ for some $A$, then \Cref{lem:unify} gives following two cases:
                        \begin{itemize}
                          \item Suppose no $\sigma \geq \ext{\rho}$ s.t.\ $\simsub{A_{m+1}}{\sigma} = A$ but an extension $\gamma \geq \rho$ s.t.\ \eqref{eq:claim} holds for $1 \leq i \leq m + 1$. 
                            Then, $\gamma \geq \ext{\rho}$ by the minimality of $\ext{\rho}$ and thus
                            $\Gamma, \vars{x}_\isTerm{m+1} : \simsub{\Delta_{m+1}}{\ext{\rho}} |- \isTerm{t_{m+1}} \syn \simsub{A_{m+1}}{\gamma}$.
                            However, by \Cref{thm:unique-syn}, the synthesised type $\simsub{A_{m+1}}{\gamma}$ must be unique, so $\gamma$ is an extension of $\ext{\rho}$ s.t.\ $\simsub{A_{m+1}}{\gamma} = A$, i.e.\ a contradiction.
                          \item If there is a minimal $\ext{\ext{\rho}} \geq \ext{\rho}$ such that $\simsub{A_{m+1}}{\ext{\ext{\rho}}} = A$, then it is not hard to show that $\ext{\ext{\rho}}$ is also the minimal extension of $\rho$ such that \eqref{eq:claim} holds for all $1 \leq i \leq m + 1$.
                        \end{itemize}
                    \end{enumerate}
                \end{itemize}
            \end{enumerate}
        \end{enumerate}
        Therefore, we have proved our claim for any argument list by induction.\hfill
        $\blacksquare$
      \end{proof}
%  \end{itemize}
  We have completed the proof by induction on the derivation $\erase{\Gamma} |-_{\Sigma, \Omega} t^{\dir{d}}$.\qed
\end{proof}

The formal counterpart of the above proof in \Agda functions as two top-level programs for type checking and synthesis.
These programs provide either the typing derivation or its negation proof.
Each case analysis branches depending on the outcomes of bidirectional type synthesis and checking for each sub-term, as well as the unification process.
If a negation proof is not of interest in practice, these programs can be simplified by discarding the cases that yield negation proofs.
Alternatively, we could consider generalising typing derivations instead, like our generalised mode derivations (\Cref{fig:generalised-mode-derivations}), to reformulate negation proofs positively to deliver more informative error messages.
This would assist programmers in resolving issues with ill-typed terms, rather than returning a blatant `no'.

\subsection{Trichotomy on raw terms by type synthesis} \label{subsec:trichotomy}

Combining the bidirectional type synthesiser with the mode decorator, soundness, and completeness from \cref{sec:pre-synthesis}, we derive a type synthesiser parameterised by~$(\Sigma, \Omega)$, generalising \cref{thm:implementation}.

\begin{corollary}[Trichotomy on raw terms]\label{cor:trichotomy}
  For any mode-correct bidirectional type system $(\Sigma, \Omega)$, 
  exactly one of the following holds:
  \begin{enumerate}
    \item $\erase{\Gamma} |-_{\Sigma, \Omega} \isTerm{t}^{\syn}$ and $\Gamma |-_{\Sigma, \erase{\Omega}} \isTerm{t} : A$ for some type~$A$,
    \item $\erase{\Gamma} |-_{\Sigma, \Omega} \isTerm{t}^{\syn}$ but $\Gamma |/-_{\Sigma, \erase{\Omega}} \isTerm{t} : A$ for any type~$A$, or 
    \item $\erase{\Gamma} |/-_{\Sigma, \Omega} \isTerm{t}^{\syn}$.
  \end{enumerate}
\end{corollary}
