%!TEX root = BiSig.tex

\documentclass[BiSig.tex]{subfiles}


%% ODER: format ==         = "\mathrel{==}"
%% ODER: format /=         = "\neq "
%
%
\makeatletter
\@ifundefined{lhs2tex.lhs2tex.sty.read}%
  {\@namedef{lhs2tex.lhs2tex.sty.read}{}%
   \newcommand\SkipToFmtEnd{}%
   \newcommand\EndFmtInput{}%
   \long\def\SkipToFmtEnd#1\EndFmtInput{}%
  }\SkipToFmtEnd

\newcommand\ReadOnlyOnce[1]{\@ifundefined{#1}{\@namedef{#1}{}}\SkipToFmtEnd}
\usepackage{amstext}
\usepackage{amssymb}
\usepackage{stmaryrd}
\DeclareFontFamily{OT1}{cmtex}{}
\DeclareFontShape{OT1}{cmtex}{m}{n}
  {<5><6><7><8>cmtex8
   <9>cmtex9
   <10><10.95><12><14.4><17.28><20.74><24.88>cmtex10}{}
\DeclareFontShape{OT1}{cmtex}{m}{it}
  {<-> ssub * cmtt/m/it}{}
\newcommand{\texfamily}{\fontfamily{cmtex}\selectfont}
\DeclareFontShape{OT1}{cmtt}{bx}{n}
  {<5><6><7><8>cmtt8
   <9>cmbtt9
   <10><10.95><12><14.4><17.28><20.74><24.88>cmbtt10}{}
\DeclareFontShape{OT1}{cmtex}{bx}{n}
  {<-> ssub * cmtt/bx/n}{}
\newcommand{\tex}[1]{\text{\texfamily#1}}	% NEU

\newcommand{\Sp}{\hskip.33334em\relax}


\newcommand{\Conid}[1]{\mathit{#1}}
\newcommand{\Varid}[1]{\mathit{#1}}
\newcommand{\anonymous}{\kern0.06em \vbox{\hrule\@width.5em}}
\newcommand{\plus}{\mathbin{+\!\!\!+}}
\newcommand{\bind}{\mathbin{>\!\!\!>\mkern-6.7mu=}}
\newcommand{\rbind}{\mathbin{=\mkern-6.7mu<\!\!\!<}}% suggested by Neil Mitchell
\newcommand{\sequ}{\mathbin{>\!\!\!>}}
\renewcommand{\leq}{\leqslant}
\renewcommand{\geq}{\geqslant}
\usepackage{polytable}

%mathindent has to be defined
\@ifundefined{mathindent}%
  {\newdimen\mathindent\mathindent\leftmargini}%
  {}%

\def\resethooks{%
  \global\let\SaveRestoreHook\empty
  \global\let\ColumnHook\empty}
\newcommand*{\savecolumns}[1][default]%
  {\g@addto@macro\SaveRestoreHook{\savecolumns[#1]}}
\newcommand*{\restorecolumns}[1][default]%
  {\g@addto@macro\SaveRestoreHook{\restorecolumns[#1]}}
\newcommand*{\aligncolumn}[2]%
  {\g@addto@macro\ColumnHook{\column{#1}{#2}}}

\resethooks

\newcommand{\onelinecommentchars}{\quad-{}- }
\newcommand{\commentbeginchars}{\enskip\{-}
\newcommand{\commentendchars}{-\}\enskip}

\newcommand{\visiblecomments}{%
  \let\onelinecomment=\onelinecommentchars
  \let\commentbegin=\commentbeginchars
  \let\commentend=\commentendchars}

\newcommand{\invisiblecomments}{%
  \let\onelinecomment=\empty
  \let\commentbegin=\empty
  \let\commentend=\empty}

\visiblecomments

\newlength{\blanklineskip}
\setlength{\blanklineskip}{0.66084ex}

\newcommand{\hsindent}[1]{\quad}% default is fixed indentation
\let\hspre\empty
\let\hspost\empty
\newcommand{\NB}{\textbf{NB}}
\newcommand{\Todo}[1]{$\langle$\textbf{To do:}~#1$\rangle$}

\EndFmtInput
\makeatother
%
%
%
%
%
%
% This package provides two environments suitable to take the place
% of hscode, called "plainhscode" and "arrayhscode". 
%
% The plain environment surrounds each code block by vertical space,
% and it uses \abovedisplayskip and \belowdisplayskip to get spacing
% similar to formulas. Note that if these dimensions are changed,
% the spacing around displayed math formulas changes as well.
% All code is indented using \leftskip.
%
% Changed 19.08.2004 to reflect changes in colorcode. Should work with
% CodeGroup.sty.
%
\ReadOnlyOnce{polycode.fmt}%
\makeatletter

\newcommand{\hsnewpar}[1]%
  {{\parskip=0pt\parindent=0pt\par\vskip #1\noindent}}

% can be used, for instance, to redefine the code size, by setting the
% command to \small or something alike
\newcommand{\hscodestyle}{}

% The command \sethscode can be used to switch the code formatting
% behaviour by mapping the hscode environment in the subst directive
% to a new LaTeX environment.

\newcommand{\sethscode}[1]%
  {\expandafter\let\expandafter\hscode\csname #1\endcsname
   \expandafter\let\expandafter\endhscode\csname end#1\endcsname}

% "compatibility" mode restores the non-polycode.fmt layout.

\newenvironment{compathscode}%
  {\par\noindent
   \advance\leftskip\mathindent
   \hscodestyle
   \let\\=\@normalcr
   \let\hspre\(\let\hspost\)%
   \pboxed}%
  {\endpboxed\)%
   \par\noindent
   \ignorespacesafterend}

\newcommand{\compaths}{\sethscode{compathscode}}

% "plain" mode is the proposed default.
% It should now work with \centering.
% This required some changes. The old version
% is still available for reference as oldplainhscode.

\newenvironment{plainhscode}%
  {\hsnewpar\abovedisplayskip
   \advance\leftskip\mathindent
   \hscodestyle
   \let\hspre\(\let\hspost\)%
   \pboxed}%
  {\endpboxed%
   \hsnewpar\belowdisplayskip
   \ignorespacesafterend}

\newenvironment{oldplainhscode}%
  {\hsnewpar\abovedisplayskip
   \advance\leftskip\mathindent
   \hscodestyle
   \let\\=\@normalcr
   \(\pboxed}%
  {\endpboxed\)%
   \hsnewpar\belowdisplayskip
   \ignorespacesafterend}

% Here, we make plainhscode the default environment.

\newcommand{\plainhs}{\sethscode{plainhscode}}
\newcommand{\oldplainhs}{\sethscode{oldplainhscode}}
\plainhs

% The arrayhscode is like plain, but makes use of polytable's
% parray environment which disallows page breaks in code blocks.

\newenvironment{arrayhscode}%
  {\hsnewpar\abovedisplayskip
   \advance\leftskip\mathindent
   \hscodestyle
   \let\\=\@normalcr
   \(\parray}%
  {\endparray\)%
   \hsnewpar\belowdisplayskip
   \ignorespacesafterend}

\newcommand{\arrayhs}{\sethscode{arrayhscode}}

% The mathhscode environment also makes use of polytable's parray 
% environment. It is supposed to be used only inside math mode 
% (I used it to typeset the type rules in my thesis).

\newenvironment{mathhscode}%
  {\parray}{\endparray}

\newcommand{\mathhs}{\sethscode{mathhscode}}

% texths is similar to mathhs, but works in text mode.

\newenvironment{texthscode}%
  {\(\parray}{\endparray\)}

\newcommand{\texths}{\sethscode{texthscode}}

% The framed environment places code in a framed box.

\def\codeframewidth{\arrayrulewidth}
\RequirePackage{calc}

\newenvironment{framedhscode}%
  {\parskip=\abovedisplayskip\par\noindent
   \hscodestyle
   \arrayrulewidth=\codeframewidth
   \tabular{@{}|p{\linewidth-2\arraycolsep-2\arrayrulewidth-2pt}|@{}}%
   \hline\framedhslinecorrect\\{-1.5ex}%
   \let\endoflinesave=\\
   \let\\=\@normalcr
   \(\pboxed}%
  {\endpboxed\)%
   \framedhslinecorrect\endoflinesave{.5ex}\hline
   \endtabular
   \parskip=\belowdisplayskip\par\noindent
   \ignorespacesafterend}

\newcommand{\framedhslinecorrect}[2]%
  {#1[#2]}

\newcommand{\framedhs}{\sethscode{framedhscode}}

% The inlinehscode environment is an experimental environment
% that can be used to typeset displayed code inline.

\newenvironment{inlinehscode}%
  {\(\def\column##1##2{}%
   \let\>\undefined\let\<\undefined\let\\\undefined
   \newcommand\>[1][]{}\newcommand\<[1][]{}\newcommand\\[1][]{}%
   \def\fromto##1##2##3{##3}%
   \def\nextline{}}{\) }%

\newcommand{\inlinehs}{\sethscode{inlinehscode}}

% The joincode environment is a separate environment that
% can be used to surround and thereby connect multiple code
% blocks.

\newenvironment{joincode}%
  {\let\orighscode=\hscode
   \let\origendhscode=\endhscode
   \def\endhscode{\def\hscode{\endgroup\def\@currenvir{hscode}\\}\begingroup}
   %\let\SaveRestoreHook=\empty
   %\let\ColumnHook=\empty
   %\let\resethooks=\empty
   \orighscode\def\hscode{\endgroup\def\@currenvir{hscode}}}%
  {\origendhscode
   \global\let\hscode=\orighscode
   \global\let\endhscode=\origendhscode}%

\makeatother
\EndFmtInput
%
%
\ReadOnlyOnce{agda.fmt}%


\RequirePackage[T1]{fontenc}
\RequirePackage[utf8]{inputenc}
\RequirePackage{amsfonts}

\providecommand\mathbbm{\mathbb}

% TODO: Define more of these ...
\DeclareUnicodeCharacter{737}{\textsuperscript{l}}
\DeclareUnicodeCharacter{8718}{\ensuremath{\blacksquare}}
\DeclareUnicodeCharacter{8759}{::}
\DeclareUnicodeCharacter{9669}{\ensuremath{\triangleleft}}
\DeclareUnicodeCharacter{8799}{\ensuremath{\stackrel{\scriptscriptstyle ?}{=}}}
\DeclareUnicodeCharacter{10214}{\ensuremath{\llbracket}}
\DeclareUnicodeCharacter{10215}{\ensuremath{\rrbracket}}
\DeclareUnicodeCharacter{27E6}{\ensuremath{\llbracket}}
\DeclareUnicodeCharacter{27E7}{\ensuremath{\rrbracket}}
\DeclareUnicodeCharacter{2200}{\ensuremath{\forall}}
\DeclareUnicodeCharacter{2218}{\ensuremath{\circ}}
\DeclareUnicodeCharacter{2286}{\ensuremath{\subseteq}}
\DeclareUnicodeCharacter{207A}{\ensuremath{^+}}
\DeclareUnicodeCharacter{207B}{\ensuremath{^-}}
\DeclareUnicodeCharacter{222A}{\ensuremath{\cup}}


\DeclareUnicodeCharacter{2294}{\ensuremath{\sqcup}}

\DeclareUnicodeCharacter{019B}{\ensuremath{{\mkern0.75mu\mathchar '26\mkern -9.75mu\lambda}}}

\DeclareUnicodeCharacter{1D43}{\ensuremath{^a}}
\DeclareUnicodeCharacter{1D9C}{\ensuremath{^c}}
\DeclareUnicodeCharacter{02B3}{\ensuremath{^r}}
\DeclareUnicodeCharacter{02E2}{\ensuremath{^s}}
\DeclareUnicodeCharacter{209C}{\ensuremath{_t}}

\DeclareUnicodeCharacter{2080}{\ensuremath{_0}}
\DeclareUnicodeCharacter{2081}{\ensuremath{_1}}
\DeclareUnicodeCharacter{2082}{\ensuremath{_2}}
\DeclareUnicodeCharacter{2083}{\ensuremath{_3}}
\DeclareUnicodeCharacter{2084}{\ensuremath{_4}}

\DeclareUnicodeCharacter{2115}{\ensuremath{\mathbb{N}}}
\DeclareUnicodeCharacter{2208}{\ensuremath{\in}}
\DeclareUnicodeCharacter{2236}{:}
\DeclareUnicodeCharacter{2261}{\ensuremath{\equiv}}
\DeclareUnicodeCharacter{2237}{\ensuremath{\mathrel{::}}}
\DeclareUnicodeCharacter{2283}{\ensuremath{\supset}}
\DeclareUnicodeCharacter{2982}{\ensuremath{\bbcolon}}

\DeclareUnicodeCharacter{0393}{\ensuremath{\Gamma}}
\DeclareUnicodeCharacter{0394}{\ensuremath{\Delta}}
\DeclareUnicodeCharacter{0398}{\ensuremath{\Theta}}
\DeclareUnicodeCharacter{03A3}{\ensuremath{\Sigma}}
\DeclareUnicodeCharacter{039B}{\ensuremath{\Lambda}}
\DeclareUnicodeCharacter{039E}{\ensuremath{\Xi}}

\DeclareUnicodeCharacter{03B9}{\ensuremath{\iota}}
\DeclareUnicodeCharacter{03BB}{\ensuremath{\lambda}}
\DeclareUnicodeCharacter{03C0}{\ensuremath{\pi}}
\DeclareUnicodeCharacter{03C1}{\ensuremath{\rho}}
\DeclareUnicodeCharacter{03C3}{\ensuremath{\sigma}}
\DeclareUnicodeCharacter{03C9}{\ensuremath{\omega}}

\DeclareUnicodeCharacter{2032}{\ensuremath{{}^\prime}}
\DeclareUnicodeCharacter{2113}{\ensuremath{\ell}}
\DeclareUnicodeCharacter{2203}{\ensuremath{\exists}}
\DeclareUnicodeCharacter{2207}{\ensuremath{\nabla}}
\DeclareUnicodeCharacter{220B}{\ensuremath{\ni}}
\DeclareUnicodeCharacter{228E}{\ensuremath{\uplus}}
\DeclareUnicodeCharacter{2264}{\ensuremath{\leq}}
\DeclareUnicodeCharacter{21A3}{\ensuremath{\rightarrowtail}}
\DeclareUnicodeCharacter{21D0}{\ensuremath{\Leftarrow}}
\DeclareUnicodeCharacter{21D2}{\ensuremath{\Rightarrow}}
\DeclareUnicodeCharacter{21D4}{\ensuremath{\Leftrightarrow}}
\DeclareUnicodeCharacter{25B7}{\ensuremath{\rhd}}
\DeclareUnicodeCharacter{22A2}{\ensuremath{\vdash}}
\DeclareUnicodeCharacter{22A4}{\ensuremath{\top}}
\DeclareUnicodeCharacter{22A5}{\ensuremath{\bot}}

\DeclareUnicodeCharacter{1D57}{\ensuremath{^t}}

% TODO: This is in general not a good idea.
\providecommand\textepsilon{$\epsilon$}
\providecommand\textmu{$\mu$}


%Actually, varsyms should not occur in Agda output.

% TODO: Make this configurable. IMHO, italics doesn't work well
% for Agda code.

\renewcommand\Varid[1]{\mathord{\textsf{#1}}}
\let\Conid\Varid
\newcommand\Keyword[1]{\textsf{\textbf{#1}}}

\EndFmtInput




\begin{document}

\section{Demonstration} \label{sec:formalisation}
As our theory is developed with \Agda constructively, the formal counterparts of our development can be used as programs directly.
We sketch their use with our running example, the simply typed $\lambda$-calculus, by
\begin{inlineenum}
  \item specifying the language $(\Sigma_{\bto}, \Omega^{\Leftrightarrow})$,
  \item showing it mode-correct, and then
  \item instantiating its type synthesiser.
\end{inlineenum}

\paragraph{Specifying a language}
Signatures \ensuremath{\Conid{ΛₜD}} and \ensuremath{\Conid{Λ⇔D}} are defined for $\Sigma_{\bto}$ and $\Omega^{\Leftrightarrow}$: 
\begin{hscode}\SaveRestoreHook
\column{B}{@{}>{\hspre}l<{\hspost}@{}}%
\column{3}{@{}>{\hspre}l<{\hspost}@{}}%
\column{9}{@{}>{\hspre}l<{\hspost}@{}}%
\column{10}{@{}>{\hspre}l<{\hspost}@{}}%
\column{25}{@{}>{\hspre}l<{\hspost}@{}}%
\column{71}{@{}>{\hspre}l<{\hspost}@{}}%
\column{E}{@{}>{\hspre}l<{\hspost}@{}}%
\>[B]{}\Keyword{data}\;\Conid{ΛₜOp}\;\mathbin{:}\;\Conid{Set}\;\Keyword{where}\;\Varid{base}\;\Varid{imp}\;\mathbin{:}\;\Conid{ΛₜOp}{}\<[E]%
\\[\blanklineskip]%
\>[B]{}\Conid{ΛₜD}\;\mathbin{:}\;\Conid{S.SigD}{}\<[E]%
\\
\>[B]{}\Conid{ΛₜD}\;\mathrel{=}\;\Varid{sigd}\;\Conid{ΛₜOp}\;\Varid{λ}\;\Keyword{where}\;\Varid{base}\;\Varid{→}\;\Varid{0};\Varid{imp}\;\Varid{→}\;\Varid{2}{}\<[E]%
\\[\blanklineskip]%
\>[B]{}\Keyword{open}\;\Keyword{import}\;\Conid{Syntax.BiTyped.Signature}\;\Conid{ΛₜD}{}\<[E]%
\\[\blanklineskip]%
\>[B]{}\Keyword{data}\;\Conid{ΛOp}\;\mathbin{:}\;\Conid{Set}\;\Keyword{where}\;\iden{\text{\textasciigrave}}\Varid{app}\;\iden{\text{\textasciigrave}}\Varid{abs}\;\mathbin{:}\;\Conid{ΛOp}{}\<[E]%
\\[\blanklineskip]%
\>[B]{}\Conid{Λ⇔D}\;\mathbin{:}\;\Conid{SigD}{}\<[E]%
\\
\>[B]{}\Conid{Λ⇔D}\;\Varid{.Op}\;{}\<[10]%
\>[10]{}\mathrel{=}\;\Conid{ΛOp}{}\<[E]%
\\
\>[B]{}\Conid{Λ⇔D}\;\Varid{.ar}\;{}\<[10]%
\>[10]{}\mathrel{=}\;\Varid{λ}\;\Keyword{where}{}\<[E]%
\\
\>[B]{}\hsindent{3}{}\<[3]%
\>[3]{}\iden{\text{\textasciigrave}}\Varid{app}\;{}\<[9]%
\>[9]{}\Varid{→}\;\Varid{2}\;\Varid{▷}\;\rho[\;{}\<[25]%
\>[25]{}[]\;{\vdash}\![\;\Conid{Chk}\;\mskip1.5mu]\;\text{\textasciigrave}\;\Varid{1}\;\mskip1.5mu]\;\rho[\;[]\;{\vdash}\![\;\Conid{Syn}\;\mskip1.5mu]\;\text{\textasciigrave}\;\Varid{1}\;\Varid{⊃}\;\text{\textasciigrave}\;\Varid{0}\;\mskip1.5mu]\;{}\<[71]%
\>[71]{}[]\;\Varid{⇒}\;\text{\textasciigrave}\;\Varid{0}{}\<[E]%
\\
\>[B]{}\hsindent{3}{}\<[3]%
\>[3]{}\iden{\text{\textasciigrave}}\Varid{abs}\;{}\<[9]%
\>[9]{}\Varid{→}\;\Varid{2}\;\Varid{▷}\;\rho[\;\text{\textasciigrave}\;\Varid{1}\;\Varid{∷}\;{}\<[25]%
\>[25]{}[]\;{\vdash}\![\;\Conid{Chk}\;\mskip1.5mu]\;\text{\textasciigrave}\;\Varid{0}\;\mskip1.5mu]\;{}\<[71]%
\>[71]{}[]\;\Varid{⇐}\;\text{\textasciigrave}\;\Varid{1}\;\Varid{⊃}\;\text{\textasciigrave}\;\Varid{0}{}\<[E]%
\ColumnHook
\end{hscode}\resethooks
where
\begin{inlineenum}
  \item \ensuremath{\Conid{S.SigD}} is the type of type signatures,
  \item \ensuremath{\Conid{SigD}} is the type of bidirectional binding signatures,
  \item \ensuremath{\Varid{2}} indicates the number of type variable variables in an operation,
  \item $`i$ the $i$-th type variable, 
\end{inlineenum}
and the definitions of decidable equality for \ensuremath{\Conid{ΛₜOp}} and \ensuremath{\Conid{ΛOp}} are omitted above.


\paragraph{Proving mode-correctness}
To prove that the specified language $(\Sigma_{\bto}, \Omega^{\Leftrightarrow})$ is mode-correct, we simply invoke \cref{lem:decidability-mode-correctness} for each construct:
\begin{hscode}\SaveRestoreHook
\column{B}{@{}>{\hspre}l<{\hspost}@{}}%
\column{14}{@{}>{\hspre}l<{\hspost}@{}}%
\column{E}{@{}>{\hspre}l<{\hspost}@{}}%
\>[B]{}\Keyword{open}\;\Keyword{import}\;\Conid{Theory.ModeCorrectness.Decidability}\;\Conid{ΛₜD}{}\<[E]%
\\[\blanklineskip]%
\>[B]{}\Varid{mcΛ⇔D}\;\mathbin{:}\;\Conid{ModeCorrect}\;\Conid{Λ⇔D}{}\<[E]%
\\
\>[B]{}\Varid{mcΛ⇔D}\;\iden{\text{\textasciigrave}}\Varid{app}\;{}\<[14]%
\>[14]{}\mathrel{=}\;\Varid{from-yes}\;(\Conid{ModeCorrectᶜ?}\;(\Conid{Λ⇔D}\;\Varid{.ar}\;\iden{\text{\textasciigrave}}\Varid{app})){}\<[E]%
\\
\>[B]{}\Varid{mcΛ⇔D}\;\iden{\text{\textasciigrave}}\Varid{abs}\;{}\<[14]%
\>[14]{}\mathrel{=}\;\Varid{from-yes}\;(\Conid{ModeCorrectᶜ?}\;(\Conid{Λ⇔D}\;\Varid{.ar}\;\iden{\text{\textasciigrave}}\Varid{abs})){}\<[E]%
\ColumnHook
\end{hscode}\resethooks
where \ensuremath{\Varid{from-yes}} extracts the positive witness from an inhabitant of the \ensuremath{\Conid{Dec}} type.

\paragraph{Instantiating a type synthesiser}
Now we have the definition \ensuremath{\Conid{Λ⇔D}} for the bidirectional type system $(\Sigma_{\bto}, \Omega^{\Leftrightarrow})$ with \ensuremath{\Varid{mcΛ⇔D}} for its mode-correctness, so we can instantiate its type synthesiser (\cref{cor:trichotomy}) just by importing the module
\begin{hscode}\SaveRestoreHook
\column{B}{@{}>{\hspre}l<{\hspost}@{}}%
\column{E}{@{}>{\hspre}l<{\hspost}@{}}%
\>[B]{}\Keyword{open}\;\Keyword{import}\;\Conid{Theory.Trichotomy}\;\Conid{Λ⇔D}\;\Varid{mcΛ⇔D}{}\<[E]%
\ColumnHook
\end{hscode}\resethooks
where the type synthesiser is defined:
\begin{hscode}\SaveRestoreHook
\column{B}{@{}>{\hspre}l<{\hspost}@{}}%
\column{3}{@{}>{\hspre}l<{\hspost}@{}}%
\column{E}{@{}>{\hspre}l<{\hspost}@{}}%
\>[B]{}\Varid{synthesise}\;{}\<[E]%
\\
\>[B]{}\hsindent{3}{}\<[3]%
\>[3]{}\mathbin{:}\;(\Conid{Γ}\;\mathbin{:}\;\Conid{Cxt})\;(\Varid{r}\;\mathbin{:}\;\Conid{Raw}\;(\Varid{length}\;\Conid{Γ}))\;{}\<[E]%
\\
\>[B]{}\hsindent{3}{}\<[3]%
\>[3]{}\Varid{→}\;\Conid{Dec}\;(\Varid{∃[}\;\Conid{A}\;\mskip1.5mu]\;\Conid{Γ}\;\Varid{⊢}\;\Varid{r}\;\Varid{⦂}\;\Conid{A})\;\Varid{⊎}\;\Varid{¬}\;\Conid{Pre}\;\Conid{Syn}\;\Varid{r}{}\<[E]%
\ColumnHook
\end{hscode}\resethooks
Every statement in our development so far has been formally proved constructively, so the proof \ensuremath{\Varid{synthesise}}r can actually compute as a program!

\paragraph{Running a type synthesiser}
The type of raw terms for $(\Sigma_{\bto}, \Omega^{\Leftrightarrow})$ are defined in the module \ensuremath{\Conid{Syntax.Typed.Raw.Term}} with the mode-erased signature \ensuremath{\Varid{erase}\;\Conid{Λ⇔D}}:
\begin{hscode}\SaveRestoreHook
\column{B}{@{}>{\hspre}l<{\hspost}@{}}%
\column{E}{@{}>{\hspre}l<{\hspost}@{}}%
\>[B]{}\Keyword{open}\;\Keyword{import}\;\Conid{Theory.Erasure}{}\<[E]%
\\
\>[B]{}\Keyword{open}\;\Keyword{import}\;\Conid{Syntax.Typed.Raw.Term}\;(\Varid{erase}\;\Conid{Λ⇔D}){}\<[E]%
\ColumnHook
\end{hscode}\resethooks

As raw terms for operations are defined generically using a single constructor $\ensuremath{\Varid{op}}$, it is not so convenient to manipulate them directly.
Hence, we use pattern synonyms and define raw terms in a form close to the informal presentation:
\begin{hscode}\SaveRestoreHook
\column{B}{@{}>{\hspre}l<{\hspost}@{}}%
\column{15}{@{}>{\hspre}l<{\hspost}@{}}%
\column{19}{@{}>{\hspre}l<{\hspost}@{}}%
\column{E}{@{}>{\hspre}l<{\hspost}@{}}%
\>[B]{}\Keyword{infixl}\;\Varid{8}\;\Varid{\char95 ·\char95 }{}\<[E]%
\\
\>[B]{}\Keyword{infixr}\;\Varid{7}\;\Varid{ƛ\char95 }{}\<[E]%
\\[\blanklineskip]%
\>[B]{}\Varid{pattern}\;\Varid{\char95 ·\char95 }\;{}\<[15]%
\>[15]{}\Varid{r}\;\Varid{s}\;\mathrel{=}\;\Varid{op}\;(\iden{\text{\textasciigrave}}\Varid{app}\;\Varid{,}\;\Varid{s}\;\Varid{,}\;\Varid{r}\;\Varid{,}\;\anonymous ){}\<[E]%
\\
\>[B]{}\Varid{pattern}\;\Varid{ƛ\char95 }\;{}\<[15]%
\>[15]{}\Varid{r}\;{}\<[19]%
\>[19]{}\mathrel{=}\;\Varid{op}\;(\iden{\text{\textasciigrave}}\Varid{abs}\;\Varid{,}\;\Varid{r}\;\Varid{,}\;\anonymous ){}\<[E]%
\\[\blanklineskip]%
\>[B]{}\Conid{S}\;\mathbin{:}\;\Conid{Raw}\;\Varid{n}{}\<[E]%
\\
\>[B]{}\Conid{S}\;\mathrel{=}\;\Varid{ƛ}\;\Varid{ƛ}\;\Varid{ƛ}\;\text{\textasciigrave}\;\Varid{suc}\;(\Varid{suc}\;\Varid{zero})\;\Varid{·}\;\text{\textasciigrave}\;\Varid{zero}\;\Varid{·}\;(\text{\textasciigrave}\;\Varid{suc}\;\Varid{zero}\;\Varid{·}\;\text{\textasciigrave}\;\Varid{zero}){}\<[E]%
\ColumnHook
\end{hscode}\resethooks
Then, invoking the program \ensuremath{\Varid{synthesise}} with \ensuremath{\Conid{S}} and its required type annotation gives us a typing derivation as expected:
\begin{hscode}\SaveRestoreHook
\column{B}{@{}>{\hspre}l<{\hspost}@{}}%
\column{E}{@{}>{\hspre}l<{\hspost}@{}}%
\>[B]{}\Varid{⊢S?}\;\mathrel{=}\;\Varid{synthesise}\;[]\;(((\Varid{b}\;\Varid{⊃}\;\Varid{b}\;\Varid{⊃}\;\Varid{b})\;\Varid{⊃}\;(\Varid{b}\;\Varid{⊃}\;\Varid{b})\;\Varid{⊃}\;\Varid{b}\;\Varid{⊃}\;\Varid{b})\;\Varid{∋}\;\Conid{S}){}\<[E]%
\ColumnHook
\end{hscode}\resethooks

%\begin{code}
%inl
%(yes
% (op
%  (imp ,
%   op
%   (imp ,
%    op (base , []) ∷
%    op
%    (imp ,
%     op (base , []) ∷
%     op (base , []) ∷ [])
%    ∷ [])
%   ∷
%   op
%   (imp ,
%    op
%    (imp ,
%     op (base , []) ∷
%     op (base , []) ∷ [])
%    ∷
%    op
%    (imp ,
%     op (base , []) ∷
%     op (base , []) ∷ [])
%    ∷ [])
%   ∷ [])
%  ,
%  (op
%   (imp ,
%    op
%    (imp ,
%     op (base , []) ∷
%     op
%     (imp ,
%      op (base , []) ∷
%      op (base , []) ∷ [])
%     ∷ [])
%    ∷
%    op
%    (imp ,
%     op
%     (imp ,
%      op (base , []) ∷
%      op (base , []) ∷ [])
%     ∷
%     op
%     (imp ,
%      op (base , []) ∷
%      op (base , []) ∷ [])
%     ∷ [])
%    ∷ [])
%   ∋
%   op
%   ((λ x →
%       extend (sigd ΛₜOp (λ { base → 0 ; imp → 2 }))
%       (extend (sigd ΛₜOp (λ { base → 0 ; imp → 2 }))
%        (λ ()) tt
%        (op
%         (imp ,
%          op
%          (imp ,
%           op (base , []) ∷
%           op (base , []) ∷ [])
%          ∷
%          op
%          (imp ,
%           op (base , []) ∷
%           op (base , []) ∷ [])
%          ∷ [])))
%       ((λ ()) , tt)
%       (op
%        (imp ,
%         op (base , []) ∷
%         op
%         (imp ,
%          op (base , []) ∷
%          op (base , []) ∷ [])
%         ∷ []))
%       (∪-⊆⁺ (λ { (here eq) → here eq ; (there ()) })
%        (λ x∈ →
%           there
%           (∪-⊆⁺ (λ { (here eq) → here eq ; (there ()) })
%            (λ x∈₁ → there ((λ ()) x∈₁)) x∈
%            | (∈-∪⁻ (cons zero [] tt) x∈ | inl tt)))
%        (∪-⊆⁺ (λ x₁ → x₁) []⊆xs
%         (∀-cons (λ _ → there (here refl))
%          (λ x₁ → ∀-cons (λ _ → here refl) (λ x₂ {}) x₁) x tt₀)
%         | (∈-∪⁻ (cons (suc zero) (cons zero [] tt) ((λ ()) , tt))
%            (∀-cons (λ _ → there (here refl))
%             (λ x₁ → ∀-cons (λ _ → here refl) (λ x₂ {}) x₁) x tt₀)
%            | inl tt))
%        | (∈-∪⁻ (cons (suc zero) [] tt)
%           (∪-⊆⁺ (λ x₁ → x₁) []⊆xs
%            (∀-cons (λ _ → there (here refl))
%             (λ x₁ → ∀-cons (λ _ → here refl) (λ x₂ {}) x₁) x tt₀)
%            | (∈-∪⁻ (cons (suc zero) (cons zero [] tt) ((λ ()) , tt))
%               (∀-cons (λ _ → there (here refl))
%                (λ x₁ → ∀-cons (λ _ → here refl) (λ x₂ {}) x₁) x tt₀)
%               | inl tt))
%           | inl ((λ ()) , tt))))
%    ,
%    refl ,
%    op
%    ((λ x →
%        extend (sigd ΛₜOp (λ { base → 0 ; imp → 2 }))
%        (extend (sigd ΛₜOp (λ { base → 0 ; imp → 2 }))
%         (λ ()) tt
%         (op
%          (imp ,
%           op (base , []) ∷
%           op (base , []) ∷ [])))
%        ((λ ()) , tt)
%        (op
%         (imp ,
%          op (base , []) ∷
%          op (base , []) ∷ []))
%        (∪-⊆⁺ (λ { (here eq) → here eq ; (there ()) })
%         (λ x∈ →
%            there
%            (∪-⊆⁺ (λ { (here eq) → here eq ; (there ()) })
%             (λ x∈₁ → there ((λ ()) x∈₁)) x∈
%             | (∈-∪⁻ (cons zero [] tt) x∈ | inl tt)))
%         (∪-⊆⁺ (λ x₁ → x₁) []⊆xs
%          (∀-cons (λ _ → there (here refl))
%           (λ x₁ → ∀-cons (λ _ → here refl) (λ x₂ {}) x₁) x tt₀)
%          | (∈-∪⁻ (cons (suc zero) (cons zero [] tt) ((λ ()) , tt))
%             (∀-cons (λ _ → there (here refl))
%              (λ x₁ → ∀-cons (λ _ → here refl) (λ x₂ {}) x₁) x tt₀)
%             | inl tt))
%         | (∈-∪⁻ (cons (suc zero) [] tt)
%            (∪-⊆⁺ (λ x₁ → x₁) []⊆xs
%             (∀-cons (λ _ → there (here refl))
%              (λ x₁ → ∀-cons (λ _ → here refl) (λ x₂ {}) x₁) x tt₀)
%             | (∈-∪⁻ (cons (suc zero) (cons zero [] tt) ((λ ()) , tt))
%                (∀-cons (λ _ → there (here refl))
%                 (λ x₁ → ∀-cons (λ _ → here refl) (λ x₂ {}) x₁) x tt₀)
%                | inl tt))
%            | inl ((λ ()) , tt))))
%     ,
%     refl ,
%     op
%     ((λ x →
%         extend (sigd ΛₜOp (λ { base → 0 ; imp → 2 }))
%         (extend (sigd ΛₜOp (λ { base → 0 ; imp → 2 }))
%          (λ ()) tt (op (base , [])))
%         ((λ ()) , tt) (op (base , []))
%         (∪-⊆⁺ (λ { (here eq) → here eq ; (there ()) })
%          (λ x∈ →
%             there
%             (∪-⊆⁺ (λ { (here eq) → here eq ; (there ()) })
%              (λ x∈₁ → there ((λ ()) x∈₁)) x∈
%              | (∈-∪⁻ (cons zero [] tt) x∈ | inl tt)))
%          (∪-⊆⁺ (λ x₁ → x₁) []⊆xs
%           (∀-cons (λ _ → there (here refl))
%            (λ x₁ → ∀-cons (λ _ → here refl) (λ x₂ {}) x₁) x tt₀)
%           | (∈-∪⁻ (cons (suc zero) (cons zero [] tt) ((λ ()) , tt))
%              (∀-cons (λ _ → there (here refl))
%               (λ x₁ → ∀-cons (λ _ → here refl) (λ x₂ {}) x₁) x tt₀)
%              | inl tt))
%          | (∈-∪⁻ (cons (suc zero) [] tt)
%             (∪-⊆⁺ (λ x₁ → x₁) []⊆xs
%              (∀-cons (λ _ → there (here refl))
%               (λ x₁ → ∀-cons (λ _ → here refl) (λ x₂ {}) x₁) x tt₀)
%              | (∈-∪⁻ (cons (suc zero) (cons zero [] tt) ((λ ()) , tt))
%                 (∀-cons (λ _ → there (here refl))
%                  (λ x₁ → ∀-cons (λ _ → here refl) (λ x₂ {}) x₁) x tt₀)
%                 | inl tt))
%             | inl ((λ ()) , tt))))
%      ,
%      refl ,
%      op
%      ((λ x →
%          extend (sigd ΛₜOp (λ { base → 0 ; imp → 2 }))
%          (extend (sigd ΛₜOp (λ { base → 0 ; imp → 2 }))
%           (λ ()) tt (op (base , [])))
%          ((λ ()) , tt) (op (base , []))
%          (∪-⊆⁺
%           (λ x∈ →
%              ∪-⊆⁺ (λ { (here eq) → here eq ; (there ()) })
%              (λ x∈₁ →
%                 there
%                 (∪-⊆⁺ (λ { (here eq) → here eq ; (there ()) })
%                  (λ x∈₂ → there ((λ ()) x∈₂)) x∈₁
%                  | (∈-∪⁻ (cons zero [] tt) x∈₁ | inl tt)))
%              x∈
%              | (∈-∪⁻ (cons (suc zero) [] tt) x∈ | inl ((λ ()) , tt)))
%           (λ x₁ → there (there ((λ ()) ([]⊆xs x₁))))
%           (∀-cons (λ _ → there (here refl))
%            (λ x₁ → F.∀-cons (λ _ → here refl) (λ x₂ {}) x₁) x tt₀)
%           | (∈-∪⁻ (cons (suc zero) (cons zero [] tt) ((λ ()) , tt))
%              (∀-cons (λ _ → there (here refl))
%               (λ x₁ → F.∀-cons (λ _ → here refl) (λ x₂ {}) x₁) x tt₀)
%              | inl tt)))
%       ,
%       refl ,
%       op
%       ((λ x →
%           extend (sigd ΛₜOp (λ { base → 0 ; imp → 2 }))
%           (extend (sigd ΛₜOp (λ { base → 0 ; imp → 2 }))
%            (λ ()) tt (op (base , [])))
%           ((λ ()) , tt) (op (base , []))
%           (∪-⊆⁺
%            (λ x∈ →
%               ∪-⊆⁺ (λ { (here eq) → here eq ; (there ()) })
%               (λ x∈₁ →
%                  there
%                  (∪-⊆⁺ (λ { (here eq) → here eq ; (there ()) })
%                   (λ x∈₂ → there ((λ ()) x∈₂)) x∈₁
%                   | (∈-∪⁻ (cons zero [] tt) x∈₁ | inl tt)))
%               x∈
%               | (∈-∪⁻ (cons (suc zero) [] tt) x∈ | inl ((λ ()) , tt)))
%            (λ x₁ → there (there ((λ ()) ([]⊆xs x₁))))
%            (∀-cons (λ _ → there (here refl))
%             (λ x₁ → ∀-cons (λ _ → here refl) (λ x₂ {}) x₁) x tt₀)
%            | (∈-∪⁻ (cons (suc zero) (cons zero [] tt) ((λ ()) , tt))
%               (∀-cons (λ _ → there (here refl))
%                (λ x₁ → ∀-cons (λ _ → here refl) (λ x₂ {}) x₁) x tt₀)
%               | inl tt)))
%        ,
%        refl ,
%        var (here refl) refl ,
%        var (there (here refl)) refl , tt)
%       ,
%       _⊢_⦂_.op
%       ((λ x →
%           extend (sigd ΛₜOp (λ { base → 0 ; imp → 2 }))
%           (extend (sigd ΛₜOp (λ { base → 0 ; imp → 2 }))
%            (λ ()) tt
%            (op
%             (imp ,
%              op (base , []) ∷
%              op (base , []) ∷ [])))
%           ((λ ()) , tt) (op (base , []))
%           (∪-⊆⁺
%            (λ x∈ →
%               ∪-⊆⁺ (λ { (here eq) → here eq ; (there ()) })
%               (λ x∈₁ →
%                  there
%                  (∪-⊆⁺ (λ { (here eq) → here eq ; (there ()) })
%                   (λ x∈₂ → there ((λ ()) x∈₂)) x∈₁
%                   | (∈-∪⁻ (cons zero [] tt) x∈₁ | inl tt)))
%               x∈
%               | (∈-∪⁻ (cons (suc zero) [] tt) x∈ | inl ((λ ()) , tt)))
%            (λ x₁ → there (there ((λ ()) ([]⊆xs x₁))))
%            (∀-cons (λ _ → there (here refl))
%             (λ x₁ → ∀-cons (λ _ → here refl) (λ x₂ {}) x₁) x tt₀)
%            | (∈-∪⁻ (cons (suc zero) (cons zero [] tt) ((λ ()) , tt))
%               (∀-cons (λ _ → there (here refl))
%                (λ x₁ → F.∀-cons (λ _ → here refl) (λ x₂ {}) x₁) x tt₀)
%               | inl tt)))
%        ,
%        refl ,
%        var (here refl) refl ,
%        var (there (there (here refl))) refl , tt)
%       , tt)
%      , tt)
%     , tt)
%    , tt))))
%\end{code}

\end{document}
