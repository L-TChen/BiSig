%!TEX root = BiSig.tex

\section{Discussion} \label{sec:future}
We believe that our formal treatment lays a foundation for further investigation, as the essential aspects of bidirectional typing have been studied rigorously. 
While our current development is based on simply typed languages to highlight the core ideas, it is evident that many concepts and aspects remain untouched.
%In this section we discuss related work and potential directions for future research.
\subsubsection{Language formalisation frameworks}

The study of presenting logics universally at least date back to universal algebra and model theory where structures are studied for certain notions of arities and signatures.
In programming language theory, \varcitet{Aczel1978}{'s binding signature} is an example which has been used to prove a general confluence theorem.
Many general definitions and frameworks for defining logics and type theories have been proposed, and we classify them into two groups by where signatures reside: the meta level or the object level of a meta-language:
\begin{enumerate}
  \item \varcitet{Harper1993a}{'s logical framework \LF} and its family of variants~\cite{Harper2007,Assaf2016,Felicissimo2023} are \emph{extensions} of Martin-L\"of type theory where signatures are on the \emph{meta level} and naturally capable of specifying dependent type theories;
\item general dependent type theories~\cite{Bauer2020,Haselwarter2021,Bauer2022a,Uemura2021}, categorical semantics~\cite{Fiore1999,Tanaka2006,Tanaka2006a,Fiore2010,Hamana2011,Fiore2013,Arkor2020,Fiore2022} (which includes the syntactic model as a special case), and frameworks for substructural systems~\cite{Tanaka2006,Tanaka2006a,Wood2022} are developed \emph{within} a meta-theory (set theory or type theory) where signatures are on the \emph{object level} and their expressiveness varies depending on their target languages.
\end{enumerate}

The \LF approach is naturally expressive but each extension requires a different framework and a different implementation to check formal \LF proofs.
Formalising \LF and its variants is at least as hard as formalising a dependent type theory, and they are mostly implemented separately from their theory and unverified.

For the second group, theories developed in set theory can often be reformulated in type theory and thus manageable for formalisation in a type-theoretic proof assistant. 
Such examples include frameworks developed by \citet{Allais2021,Ahrens2022,Fiore2022} in \Agda, which are relatively correct to their meta-language, although their expressiveness is limited to simply typed theories.
Our work belongs to the second group, as we aim for a formalism in a type theory to minimise the gap between theory and implementation.

\subsubsection{Beyond simple types}

Bidirectional type synthesis plays a crucial role in handling more complex types than simple types, and we sketch how our theory can be extended to treat a broader class of languages.
First, we need a general definition of languages in question (\cref{subsec:simple-types,subsec:binding-sig}).
Then, this definition can be augmented with modes (\cref{subsec:bidirectional-system}) and the definition of mode-correctness (\cref{def:mode-correctness}) can be adapted accordingly.
Soundness and completeness (\cref{lem:soundness-completeness}) should still hold, as they amount to the separation and combination of mode and typing information for a given raw term (in a syntax-directed formulation).
Mode decoration (\cref{sec:mode-decoration}), which annotates a raw term with modes and marks missing annotations, should also work.
As for the decidability of bidirectional type synthesis, we discuss two cases involving polymorphic types and dependent types below.

\paragraph{Polymorphic types}
In the case of languages like \SystemF and others that permit type-level variable binding, we can start with the notion of polymorphic signature, as introduced by \citet{Hamana2011}---\begin{inlineenum}
  \item each type construct in a signature is specified by a binding arity with only one type $*$, and
  \item a term construct can employ a pair of extension contexts for term variables and type variables.
\end{inlineenum}

Extending general definitions for bidirectional typing and mode derivations from \citeauthor{Hamana2011}'s work is straightforward. 
\ifarxiv
For example, the universal type $\forall \alpha.\, A$ and type abstraction in \SystemF can be specified as operations
\[
  \mathsf{all} : * \rhd [*] * \to *
  \quad\text{and}\quad
  \mathsf{tabs} : [*] A \rhd \left< * \right> A^{\chk} \to \tyOp_{\mathsf{all}}(\alpha. A)^{\chk}.
\]
\else
For example, the universal type $\forall \alpha.\, A$ and type abstraction in \SystemF can be specified as operations $\mathsf{all} : * \rhd [*] * \to *$ and $\mathsf{tabs} : [*] A \rhd \left< * \right> A^{\chk} \to \tyOp_{\mathsf{all}}(\alpha. A)^{\chk}$.
\fi
The decidability of bidirectional type synthesis (\cref{thm:bidirectional-type-synthesis-checking}) should also carry over, as no equations are imposed on types and no guessing (for type application) is required.
Adding subtyping $A \mathrel{<:} B$ to languages can be done by replacing type equality with a subtyping relation $\mathrel{<:}$ and type equality check with subtyping check, so polymorphically typed languages with subtyping such as \SystemFsub can be specified.
The main idea of bidirectional typing does not change, so it should be possible to extend the formal theory without further assumptions too.

\ifarxiv
However, explicit type application in \SystemF and \SystemFsub is impractical but its implicit version results in a \emph{stationary rule}~\cite{Leivant1986}:
\[
  \inferrule{\Gamma \vdash t \syn \forall \alpha. A}{\Gamma \vdash t \syn A[B/\alpha]}
\]
\else
However, explicit type application in \SystemF and \SystemFsub is impractical but its implicit version results in a \emph{stationary rule}~\cite{Leivant1986}
\fi
which is not syntax-directed.
By translating the rule to subtyping, we have the \emph{instantiation problem} which amounts to guessing $B$ in $\forall \alpha. A <: A[B/\alpha]$.
A theory that accommodates various solutions to the problem is left as future work.

\paragraph{Dependent types}
In the context of dependent type theory, type synthesis entails a term equality check for type equality.
Achieving decidable bidirectional type synthesis depends on the decidability of term equality, which, in turn, relies on the ability to specify computation rules within a language specification.

Extending the logical framework \Dedukti~\cite{Assaf2016}, \varcitet{Felicissimo2023}{'s recent \textsf{CompLF}} is capable of defining dependent type theories with computational rules and enables generic bidirectional type synthesis.
They propose a notion of \LF signature with modes, a definition of mode-correctness (called \emph{well-typed} op.\ cit.), and an algorithm of bidirectional type synthesis.
The algorithm is decidable, sound and complete with respect to mode-decorated terms, provided that the specified language is mode-correct, and that the set of computational rules is type-preserving, confluent, and strongly normalising.
The last assumption, however, is challenging to establish even for specific languages and lacks a general understanding.

\subsubsection{Beyond syntax-directedness}

To explore general settings, an extreme possibility is to independently specify an ordinary and a bidirectional type system over the same raw terms and then investigate relationships between the two systems.

Yet, we could first focus on simpler cases that the ordinary typing part is still syntax-directed, but each typing rule is refined to multiple bidirectional variants, including different orders of its premises.
In such cases, the mode decorator would need to find all mode derivations, but the type synthesiser should still work in a syntax-directed manner on each mode derivation.
Completeness could still take the simple form presented in this paper too.

The next step is to consider non-syntax-directed systems where each typing rule has multiple bidirectional variants.
For this setting, the mode decorator and the bidirectional type synthesiser will have to backtrack.
It is still possible to treat soundness as the separation of mode and type information, but completeness will pose a problem.
For every raw term, a mode derivation chooses a mode assignment while a typing derivation chooses a typing rule, but there may not be a bidirectional typing rule for this particular combination.
To retain the mode decoration, one could make the mode decorator produce all possible mode derivations, and refine completeness to say that any typing derivation can be combined with one of mode derivations into a bidirectional typing derivation.
%It should be possible to fix this problem while retaining the mode decoration phase.


\subsubsection{Towards a richer formal theory}

There are more principles and techniques in bidirectional typing that could be formally studied in general, with one notable example being the Pfenning recipe for bidirectionalising typing rules~\citep[Section~4]{Dunfield2021}.
There are also concepts that may be hard to fully formalise, for example `annotation character'~\citep[Section~3.4]{Dunfield2021}, which is roughly about how easy it is for the user to write annotated programs, but it would be interesting to explore to what extent such concepts can be formalised.

\begin{credits}
\subsubsection{Acknowledgements.}
We thank Kuen-Bang {Hou (Favonia)} and anonymous reviewers for their comments and suggestions.
The work was supported by the National Science and Technology Council of Taiwan under grant NSTC 112-2221-E-001-003-MY3.
\end{credits}
