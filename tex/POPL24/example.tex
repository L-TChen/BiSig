%!TEX root = BiSig.tex

\section{Examples}\label{sec:example}
Our theory works for any language that can be specified by a bidirectional binding signature.
To showcase its general applicability, we present two additional examples:
\begin{enumerate}
\item \PCF with products and let bindings, and
\item spine calculus, a simply typed $\lambda$-calculus with applications in spine form.
\end{enumerate}
The first is a standard exercise in implementing a type synthesiser.
The second example includes not only infinitely many operations but also a type which is not formed by solely one type construct~$\tyOp$.

\subsection{PCF with Products and Let Bindings}\label{subsec:PCF}
Although we have developed a theory for presenting a bidirectional type system and a condition for the decidability of bidirectional type synthesis, designing a bidirectional type system, or its signature, remains a creative activity.\footnote{This has also been the case for parsing: while there are grammars and parser generators, one still has to design a grammar to parse or disambiguate an ambiguous grammar.}
While the focus of this paper is not to design bidirectional type systems, we simply adopt the example of a bidirectional \PCF from \citet{Wadler2022}, which extends bidirectional simply typed $\lambda$-calculus in \Cref{sec:key-ideas} by adding the following rules:
\bgroup
\small
  \begin{mathpar}
    \inferrule{ }{\Gamma \vdash \mathtt{z} \chk \mathtt{nat}}
    \and
    \inferrule{\Gamma \vdash t \chk \mathtt{nat}}{\Gamma \vdash \mathtt{s}(t) \chk \mathtt{nat}}
    \and
    \inferrule{\Gamma \vdash t \syn \mathtt{nat} \and \Gamma \vdash t_0 \chk A \and \Gamma, x : \mathtt{nat} \vdash t_1 \chk A}{\Gamma \vdash \mathtt{ifz}(t_0; x.t_1)(t) \chk A}
    \and
    \inferrule{\Gamma \vdash t \chk A \and \Gamma \vdash u \chk B}{\Gamma \vdash (t, u) \chk A \times B}
    \and
    \inferrule{\Gamma \vdash t \syn A \times B}{\Gamma \vdash \mathtt{proj}_1(t) \syn A}
    \and
    \inferrule{\Gamma \vdash u \syn A \times B}{\Gamma \vdash \mathtt{proj}_2(u) \syn A}
    \\
    \inferrule{\Gamma, x : A \vdash t \chk A}{\Gamma \vdash \mu x.\, t \chk A}
    \and
    \inferrule{\Gamma \vdash t \syn A \and \Gamma, x : A \vdash u \chk B}{\Gamma \vdash \mathtt{let}\;x = t\;\mathtt{in}\;u\chk B}
  \end{mathpar}
\egroup

%In this language, there are additional type constructs other than $\bto$, so first we extend $\Sigma_{\bto}$ (\Cref{ex:type-signature-for-function-type}) with operations
In this language, we introduce additional type constructs apart from $\bto$.
To do so, we extend $\Sigma_{\bto}$ (as shown in \Cref{ex:type-signature-for-function-type}) with the operations $\mathsf{nat}$ and $\mathsf{times}$ such that
\[
  \arity({\mathsf{nat}}) = 0 \quad\text{and}\quad \arity({\mathsf{times}}) = 2
\]
introducing $\mathtt{nat}$ and $A \times B$ as $\tyOp_{\mathsf{nat}}$ and $\tyOp_{\mathsf{times}}(A, B)$, respectively.
Following this, we also extend the bidirectional binding signature $\Omega_{\Lambda}^{\Leftrightarrow}$ with an operation for each rule above:
\begin{flalign*}
  &  \aritysymbol{\mathsf{z}}{\cdot}{\cdot}{\mathtt{nat}^{\chk}}
  && \aritysymbol{\mathsf{s}}{\cdot}{\mathtt{nat}^{\chk}}{\mathtt{nat}^{\chk}}
  && \aritysymbol{\mathsf{ifz}}{A}{\mathtt{nat}^{\syn}, A^{\chk}, A^{\chk}}{A^{\chk}} \\
  &  \aritysymbol{\mathsf{pair}}{A, B}{A^{\chk}, B^{\chk}}{A \times B^{\chk}}
  && \aritysymbol{\mathsf{proj}_1}{A, B}{A \times B^{\syn}}{A^{\syn}}
  && \aritysymbol{\mathsf{proj}_2}{A, B}{A \times B^{\syn}}{B^{\syn}} \\
  &  \aritysymbol{\mathsf{mu}}{A}{[A]A^{\chk}}{A^{\chk}}
  && \aritysymbol{\mathsf{let}}{A, B}{A^{\syn},[A]B^{\chk}}{B^{\chk}}
\end{flalign*}
With these operations, a straightforward check shows that every rule is mode-correct, therefore this is a mode-correct bidirectional type system.
As a result, we can instantiate the type synthesiser presented in \Cref{sec:type-synthesis} for this system, eliminating the need for re-implementation.

\subsection{Spine Calculus}\label{subsec:spine}
A spine application, denoted as $t\;u_1\;\ldots\;u_n$, is a variant of application that consists of a head term $t$ and an indeterminate number of arguments $u_1\;u_2\;\dots\;u_n$.
This arrangement allows direct access to the head term, making it practical in various applications.
For instance, \Agda's core language employs this form of application, as does its reflected syntax used for metaprogramming.

At first glance, accommodating this form of application may seem impossible, given that the number of arguments a construct can accept has to be fixed in a signature.
Nonetheless, there is no constraint on the total number of operation symbols a signature can have, allowing us to establish corresponding constructs for each number $n$ of arguments, thereby exhibiting the necessity of having an arbitrary set (with decidable equality) for operation symbols.

To illustrate this, let us consider the following rule:
\bgroup
\small
  \begin{mathpar}
    \inferrule{\Gamma \vdash t \syn A_1 \bto \left(A_2 \bto \left(\dots \bto \left(A_n \bto B\right)\ldots\right)\right) \and \Gamma \vdash u_1 \chk A_1 \and \dots \and \Gamma \vdash u_n \chk A_n}{\Gamma \vdash t\;u_1\;\ldots\;u_n \syn B}
  \end{mathpar}
\egroup
We use the same type signature $\Sigma_{\bto}$ but extend the bidirectional binding signature $\Omega_{\Lambda}^{\Leftrightarrow}$ with 
\[
  \aritysymbol{\mathsf{app}_n}{A_1,\ldots, A_n, B}{A_1 \bto \left(A_2 \bto \left(\dots \bto \left(A_n \bto B\right)\ldots\right)\right)^{\syn}, A_1^{\chk}, \ldots, A_n^{\chk}}{B}
\]
for each $n : \N$.
Then each spine application $t\;u_1\;\ldots\;u_n$ can be introduced as $\tmOp_{\mathsf{app}_n}(t; u_1; \ldots; u_n)$.
