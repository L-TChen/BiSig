%!TEX root = BiSig.tex

% requires bbold package
\DeclareSymbolFont{bbsymbol}{U}{bbold}{m}{n}
\DeclareMathSymbol{\bbcolon}{\mathrel}{bbsymbol}{"3A}

\newcommand{\mathsc}[1]{\textnormal{\textsc{#1}}}

\newcommand{\Agda}{\textsc{Agda}\xspace}
\newcommand{\SystemF}{{System~\textsf{F}}\xspace}
\newcommand{\PCF}{\textsc{PCF}\xspace}
\newcommand{\PoplMark}{\textsc{PoplMark}\xspace}
\newcommand{\AxiomK}{Axiom~\textsf{K}\xspace}


\newcommand{\arity}{\mathit{ar}}

\newcommand{\xto}[1]{\xrightarrow{#1}}
  
\newcommand{\fv}{\mathit{fv}}

\newcommand{\tmOp}{\mathsf{op}}
\newcommand{\tyOp}{\mathsf{op}}

\mathchardef\mhyphen="2D % hyphen in mathmode

\newcommand{\Type}{\mathsf{Ty}}
\newcommand{\Term}{\mathsf{Tm}}
\newcommand{\Cxt}{\mathsf{Cxt}}
\newcommand{\Mode}{\mathsf{Mode}}

\newcommand{\bto}{\mathbin{\bm{\supset}}}
\newcommand{\btimes}{\mathbin{\bm{\wedge}}}

\newcommand{\simsub}[2]{{#1}\!\left<{#2}\right>}

\newcommand{\Identifier}{\mathsf{Id}}

\newcommand{\annotate}{\bbcolon}
\newcommand{\subsum}{\mathord{\uparrow}}

\newcommand{\Sub}[2]{\mathsf{Sub}_{\Sigma}(#1,#2)}
\newcommand{\PSub}[2]{\mathsf{PSub}_{\Sigma}(#1,#2)}
\newcommand{\Ren}[2]{\mathsf{Ren}(#1, #2)}

\definecolor{dRed}{rgb}{0.45, 0.0, 0.0}
\definecolor{dBlue}{rgb}{0.0, 0.0, 0.65}
\definecolor{dPurple}{rgb}{0.45, 0.0, 0.65}
\definecolor{dDark}{rgb}{0.2, 0.2, 0.2}

\newcommand{\isTerm}[1]{{\textcolor{dDark}{#1}}}
\newcommand{\isType}[1]{#1}
\newcommand{\isCxt}[1]{#1}
\newcommand{\isDir}[1]{\mathrel{#1}}

\newcommand{\dir}[1]{{\color{dPurple}{#1}}}
\newcommand{\chk}{\mathrel{\color{dBlue}{\Leftarrow}}}
\newcommand{\syn}{\mathrel{\color{dRed}{\Rightarrow}}}
\newcommand{\Rule}[1]{\ensuremath{\mathsc{#1}}}
\newcommand{\SynRule}[1]{\ensuremath{\mathsc{#1}^{\syn}}}
\newcommand{\ChkRule}[1]{\ensuremath{\mathsc{#1}^{\chk}}}

\newcommand{\ext}[1]{\bar{#1}}
\newcommand{\erase}[1]{\left|#1\right|}

\newcommand{\var}{\mathsf{var}}
\newcommand{\synvar}{\var_{\syn}}

\newcommand{\abs}{\mathsf{abs}}
\newcommand{\app}{\mathsf{app}}

\mathlig{|-}{\mathrel{\vdash}} 
\newcommand{\judgbox}[2]{{\raggedright $\boxed{#1}$ \quad \text{#2}}}

\newcommand{\aritysymbol}[4]{#1\colon #2 \rhd #3 \to #4}
% Some shortcuts
\newcommand{\vars}[1]{\vec{\isTerm{#1}}}
\newcommand{\tmOpts}{\isTerm{\tmOp}_o(\vars{x}_\isTerm{1}.\, \isTerm{t_1}; \ldots;\vars{x}_\isTerm{n}.\, \isTerm{t_n})}
\newcommand{\tmOpus}{\isTerm{\tmOp}_o(\vars{x}_\isTerm{1}.\, \isTerm{u_1}; \ldots;\vars{x}_\isTerm{n}.\, \isTerm{u_n})}
