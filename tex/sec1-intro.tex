%!TEX root = BiSig.tex

\section{Introduction}\label{sec:intro}

Points to elaborate:
\begin{enumerate}
  \item Type inference and checking as a transition from (parsed) abstract syntax trees to well-typed terms.
  \item Bidirectional typing is by now the only solution that scales well.
  \item Not yet a general theory but by now a design principle.
    No `type checker generator' as each type checker is implemented on a case-by-case basis, but an algorithm can usually be read off from syntax-directed typing rules.
    A hint to a general pattern in syntax-directedness.
  \item The existing work of bidirectional typing does not take account of practical aspects such as missing annotation.
    Lightly or rarely discussed.
  \item A meta-problem: existing work is based on classical logic (or left unspecified at least).
    Soundness of completeness of a type checking algorithm in a less straightforward way.
  \item A review of what has been done towards this line of research.
  \item Finally, our responses (contributions of this paper).
\end{enumerate}


\begin{enumerate}
  \item Use application in spin form to justify our design choice: type expression of arbitrary rank and the number of constructs in a calculus.
  \item Argue that syntax-directed form is essential and practical.
  \item Pfenning's recipe is only of design guide not a technical requirement (there are many exceptions).
\end{enumerate}

\subsection{Related work}

\citep{Xie2018}
\subsubsection{Language formalization and its frameworks}
\cite{Wadler2022}
PoplMark Challenge~\citep{Aydemir2005}

\cite{Cimini2020,Cimini2022}

\citep{Ahrens2018,Fiore2022,Gheri2020,Ahrens2022}
\cite{Allais2021}

\subsubsection{Theories of abstract syntax with variable binding}
\cite{Fiore1999,Hirschowitz2010,Ahrens2018,Fiore2022,Ahrens2021,Arkor2020,Hirschowitz2022}
\cite{Fiore2013,Hamana2011,Hamana2022}


\subsubsection{Type checker generation}
\cite{Gast2004,Grewe2015,Pacak2020,Cimini2020}

\subsubsection{Bidirectional typing}

\cite{Pierce2000}
\cite{Peyton-Jones2007}
\cite{Dunfield2021}


\subsection{Plan and Contributions of this paper}

Our assumptions and contributions:
\begin{enumerate}
  \item Assumption: Syntax-directed typing rules
  \item Practical: bidirectionalisation over annotatability, decidability over soundness and completeness.
  \item Generic: Type checking once and for all bidirectional type systems for simple types.
  \item Correct and Constructive: Fully formalised in Martin-Löf type theory (with Axiom K).
    LEM is not assumed, so a decidable statement has both computational and non-trivial logical readings.
\end{enumerate}

Plan of this paper:
\begin{enumerate}
  \item Key ideas (\Cref{sec:key-ideas})
  \item Definitions for bidirectional type systems (bidirectional binding signature, bidirectional type systems, signature erasure and mode annotation) (\Cref{sec:defs})
  \item Soundness, completeness, bidirectionalisation, and annotatability (\Cref{sec:annotatability})
    \Josh{Erasure from bidirectional typing derivation to raw terms with a mode}
  \item Bidirectional type inference (\Cref{sec:type-synthesis})
  \item Formalisation with further examples (\Cref{sec:formalisation})
  \item Future work (\Cref{sec:future})
\end{enumerate}
  \LT{better title?}



