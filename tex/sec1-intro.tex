%!TEX root = BiSig.tex

\section{Introduction}\label{sec:intro}

Type checking and inference serve as the transition to type-checked programs, also known as \emph{well-typed terms}, from parsed abstract syntax trees, referred as \emph{raw terms} by contrast.
Type inference algorithms were conceived to ascertain the type of any raw term without any type annotations.
However, it was later found that full parametric polymorphism leads to undecidability in type inference, as do dependent types~\citep{Wells1999,Dowek1993}.
In light of these limitations, bidirectional type synthesis based on bidirectional typing emerged as a viable alternative.
It presents decidable algorithms for determining the types of \emph{suitably annotated} programs in languages with simple types, polymorphic types, dependent types, gradual types, among others.

The core idea of bidirectional type synthesis is simple.
It begins with bidirectional typing, extending typing judgements with two modes that direct the flow of types --- the origin of its name.
Thus, typing judgements come in two forms:
\begin{enumerate*}
  \item $\Gamma |- \isTerm{t} \syn A$ for synthesis, and 
  \item $\Gamma |- \isTerm{t} \chk A$ for checking.
\end{enumerate*}
The former indicates that the type of a raw term is synthesized, using both the term and its context as input.
The latter allows a term to be checked against a given type.
For instance, the type of a variable $x$ is synthesized via its context, and the type of an annotated term $t \annotate A$ is synthesized as $A$ after $t$ has been checked against $A$.
From this, a bidirectional type synthesis algorithm can be `read off' based on the type information flow, provided that all outputs are definitively determined by any inputs.
\citeauthor{Dunfield2021}'s recent survey~\citeyearpar{Dunfield2021} provides a summary of its essential criteria, design principles, and applications that we know so far.

%Type annotations may be seen burdensome at first, but in fact they are useful for programmers to understand the purpose of a program.
%This claim was put forth by~\citet{Pierce2000} and supported by software engineering practices, since type annotations are often included as part of top-level definitions and of the documentation even for languages such as variants of Hindley--Damas--Milner type system where (complete) type inference is possible.
%As a result, the focus of type-checking technology has shifted from type inference to bidirectional type synthesis, considering both theoretical limitations and practical considerations.

Distinct from the traditional technique of type inference such as Damas--Milner type inference, bidirectional type synthesis does \emph{not} need unification.
As argued by~\citet{Pierce2000}, missing annotations are recovered only using the information within adjacent nodes of a syntax tree.
Imposing long-distance unification constraints on bidirectional type synthesis undermines its essence of locality.
Further, adding type annotations for top-level definitions can improve clarity, making the purpose of a program more comprehensible, so this process should not be seen as a burden but rather as a practical compromise.
It is fair to say that bidirectional type synthesis is the foundation of type-checking technology across diverse languages.

Nevertheless, unlike parsing, which has a comprehensive theory and widely applicable practical techniques, bidirectional typing has been developed on a case-by-case basis without a general theory.
While it is relatively easy to derive a bidirectional type synthesis algorithm, a theory providing logical specifications with rigorously proven properties is still lacking.
As a result, we can only present informal design principles and considerations that have been discovered from individual systems.
Moreover, unlike the plethora of available parser generators, no `type-checker generators' exist, and each type checker must be independently built from scratch.

To tackle these challenges, this paper presents a theory of bidirectional typing and a verified implementation of bidirectional type synthesis for simple type theories.
There are multiple methods and extensions for designing bidirectional typing rules, such as those related to polymorphic types~\citep{Pierce2000,Peyton-Jones2007,Dunfield2013,Xie2018}.
Yet, once the definition of a bidirectional type system is established, deriving an algorithm becomes uniform.
Accordingly, our theory's objective is not to formulate rules for various language features, but to introduce 'grammars' for bidirectional typing, called \emph{bidirectional binding signatures}.
We also characterise essential criteria including \emph{completeness}, \emph{annotatability}, and \emph{mode-correctness} that are sufficient to derive a bidirectional type synthesis algorithm, as informally outlined by~\citet{Dunfield2021}.
We confine our discussion to simple type theories and their bidirectional type systems that have \emph{syntax-directed} typing rules for simplicity and because of the technical difficulty of presenting more advanced type theories in general.

\subsection{Related work}
As we will formalise our theory for not only simple type theories but also in a manner distinct from existing formulations in bidirectional typing, it is pertinent to discuss related work, apart from bidirectional typing, upon which this work is built.

\paragraph{Language formalisation and its frameworks}
The vision of formalising the metatheory of every programming language was initiated by the \PoplMark challenge~\citep{Aydemir2005}.
This is has been further exemplified in a textbook by~\citet{Wadler2022}, where language concepts are formally addressed using the proof assistant~\Agda, including bidirectional type synthesis for~\PCF.

It was observed~\citep{Altenkirch1993} that rudimentary meta-operations and meta-properties for variable binding and substitution constitute the bulk of language formalisation, inspiring a number of language-formalisation frameworks~\citep{Ahrens2018,Fiore2022,Gheri2020,Ahrens2022,Allais2021} to establish substitution, term traversal, and their meta-properties once and for a range of abstract syntax with variable binding.
One of the core gadgets of these frameworks is the concept of \emph{descriptions} (coined by~\citet{Chapman2010} for datatype-generic programming) or \emph{binding signatures} (coined by~\citet{Aczel1978} in line with the term \emph{signature} in universal algebra) for specifying typing rules of a language.
It is noteworthy that these state-of-the-art frameworks cannot fully define polymorphic types nor dependent types yet.

\begin{remark}\label{re:type-synthesis-as-decidability-proof}
The algorithm of bidirectional type synthesis as provided by~\citeauthor{Wadler2022} differs significantly from existing literature:
\begin{enumerate*}
  \item The algorithm is presented as a proof of \emph{(logical) decidability} as to whether a bidirectionally decorated raw term can be bidirectionally typed, equivalent to a program that returns a typing derivation for $\isTerm{t}$, if one exists, or the evidence of a contradiction otherwise.
  \item In contrast, an algorithm typically found in the literature is presented as \emph{algorithmic system} relations such as $\Gamma |- \isTerm{t} \syn A \mapsto \isTerm{t}'$, denoting that type annotations can be added to $\isTerm{t}$ in the external language to produce $\isTerm{t}'$ of type $A$ in the internal language.
    Such an algorithm is often accompanied with \emph{soundness} and \emph{completeness} assertions.
\end{enumerate*}
\end{remark}

\paragraph{Theories of abstract syntax with variable binding}
The aforementioned frameworks except \citeauthor{Gheri2020}'s are based on or inspired by the categorical analysis of initial semantics for abstract syntax with variable binding~\cite{Fiore1999}.
Its main idea is that the set of (untyped) abstract syntax trees for a given language can be interpreted as the \emph{initial algebra} for its \emph{signature} endofunctor on a family of sets $X_{\Gamma}$, indexed by contexts~$\Gamma$.
Simultaneous renaming of variables for some function $\sigma\colon \Gamma \to \Delta$ acts as a functorial map, mapping elements $t$ of $X_{\Gamma}$ to $\simsub{t}{\rho}$ of $X_{\Delta}$, forming a (covariant) \emph{presheaf}.
The signature endofunctor encapsulates the collection of language constructs, excluding the variable rule, while the initiality amounts to structural recursion, i.e.\ \emph{term traversal}.

Substitution can also be modelled categorically, but it does not play a role in this paper, though.

\begin{remark}
Constructing the initial algebra of syntax trees can be as simple as defining an inductive type with a few constructors that align with the variable rule and the rule scheme for algebraic operations specified by a signature~\citep{Allais2021,Fiore2022} (without taking the colimit of a sequence of objects iteratively constructed by the signature functor).
\end{remark}

\begin{remark} \label{re:type-signature}
The majority of existing theories treat types independently of terms, thereby excluding them from binding signatures.
To the best of our knowledge, the only exception to this approach is found in the work of~\citet{Arkor2020}, which incorporates signatures and equations for both terms and types.
Interestingly, this inclusion is critical for type synthesis when comparing a concrete type $N \bto N$ with an abstract type $A \bto B$, where $A$ and $B$ are type variables in a typing rule.
\end{remark}

While algebraic approaches for polymorphic type theories have been developed~\citep{Fiore2013,Hamana2011}, these approaches do not take subtyping into account.
Subtyping is essential for formulating important concepts in type synthesis such as \emph{principal types}.
On the other hand, in the realm of dependent types, \citeauthor{Cartmell1986}'s generalized algebraic theories~\citeyearpar{Cartmell1986} can handle a wide variety of dependent type theories.
\citet{Bezem2021} investigate the notion of presentation (extending signature) in the context of generalized algebraic theories.
Nonetheless, type synthesis for dependent types requires normalisation (or a form of conversion) to check type equality.
Normalisation in its generic form remains elusive and recent advances in this line of research are discussed in the doctoral thesis by~\citet{Valliappan2023}.

\paragraph{Type Checker Generation}
While there have been some efforts to generate type checkers grounded in unification~\citep{Gast2004,Grewe2015}, it should be noted that approaches solely based on unification are not suited to handling more complex type theories.

\cite{Pacak2020,Cimini2020}. 

\subsection{Plan and Contributions of this paper}

Moreover, given this opportunity of re-examining elements of bidirectional typing, we begin our theory from a general idea of type synthesis independent of bidirectional type synthesis and spot subtleties that are rarely, if any, discussed in the literature on bidirectional typing. 

Our approach involves constructing the theory within Martin-L\"of type theory and formalising it using the \Agda proof assistant.
This formalisation provides a generic type synthesis algorithm that is applicable to all calculi employing bidirectional typing for simple types, hence essentially a verified type-checker generator.
By rigorously defining its core principles, we aim to lay the foundation for further advancements in bidirectional typing.

%  Our work addresses the problem with existing methodologies, setting the stage for further advances towards a more general theory of type synthesis.

  %By formalizing our theory of bidirectional type synthesise in \Agda, we provide a framework capable of handling bidirectional type systems of simple types uniformly.
  %Our work addresses the problem with existing methodologies, 

Points to elaborate:
\begin{enumerate}
  \item The existing work of bidirectional typing does not take account of practical aspects such as missing annotation.
    Lightly or rarely discussed.
  \item A meta-problem: existing work is based on classical logic (or left unspecified at least).
    Soundness of completeness of a type checking algorithm in a less straightforward way.
\end{enumerate}
Assumption to justify:
\begin{enumerate}
  \item Use application in spin form to justify our design choice: type expression of arbitrary rank and the number of constructs in a calculus.
  \item Argue that syntax-directed form is essential and practical.
  \item Pfenning's recipe is only of design guide not a technical requirement (there are many exceptions).
\end{enumerate}

Our contributions:
\begin{enumerate}
  \item Generic: Type checking for simple type theories;
  \item Computational formulation: Bidirectionalisation over annotatability; Decidability over soundness and completeness;
  \item Verified and Constructive: Fully formalised in Martin-Löf type theory (with Axiom K).
    LEM is not assumed, so a decidable statement has both computational and non-trivial logical readings.
\end{enumerate}

Plan of this paper:
\begin{enumerate}
  \item Introduction (\Cref{sec:intro}) 
    \LT{1+3 pp}
  \item Key ideas (\Cref{sec:key-ideas})
    \Josh{2.5 pp}
  \item Definitions for bidirectional type systems (bidirectional binding signature, bidirectional type systems, signature erasure and mode annotation) (\Cref{sec:defs})
    \LT{4 pp}
  \item Soundness, completeness, bidirectionalisation, and annotatability (\Cref{sec:pre-synthesis})
    \Josh{Erasure from bidirectional typing derivation to raw terms with a mode}
    \Josh{3 pp}
  \item Bidirectional type inference (\Cref{sec:type-synthesis})
    \LT{4 pp}
  \item Formalisation and further examples (\Cref{sec:formalisation})
    \LT{5.5pp}
  \item Concluding remarks (\Cref{sec:future})
    \LT{1 p} \Josh{1p}
\end{enumerate}
  \LT{better title?}




