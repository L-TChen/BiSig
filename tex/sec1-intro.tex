%!TEX root = BiSig.tex

\section{Introduction}\label{sec:intro}

Type checking and inference serve as the transition from parsed abstract syntax trees, referred to as \emph{raw terms}, to well-typed programs, known as \emph{well-typed terms}.
Initially, type inference techniques focused on complete type inference, aiming to infer the type of any unannotated raw term.
However, it was soon realised that once polymorphism is introduced, complete type inference becomes undecidable~\citep{Wells1999}.

In light of this, \citet{Pierce2000} argued that type annotations are valuable for programmers in understanding the purpose of a program.
The prevalence of type annotations in current software engineering practices supports this claim.
Acknowledging the undecidability of complete type inference, the concept of bidirectional type synthesis emerged as a decidable algorithm to infer the type of suitably annotated programs in polymorphic programming languages.

Bidirectional typing~\citep{Pierce2000} extends the notion of typing rules by introducing two modes that govern the flow of types, thus earning its name.
This shift from (complete) type inference to type synthesis allows the use of existing type annotations to guide the flow of type information and determine the type of a term.
This is achieved through typing judgements in two modes:
  \[
    (\text{synthesising})\quad \Gamma \vdash \isTerm{t} \syn A
    \quad\text{and}\quad
    (\text{checking})\quad \Gamma \vdash \isTerm{t} \chk A
  \]
  In bidirectional type synthesis, the type of a term can be synthesized from the information in its context and the term itself or it can be checked against a given type depending on \emph{the mode of a raw term}.
For example, it may involve looking up the variable in its context; using a type annotation provided in the term after checking against the given type first, ensuring the validity of the provided annotation.
An algorithm for bidirectional type synthesis can often be `read off' directly from a calculus presented in bidirectional typing, provided that it is syntax-directed and mode-correct, as further explained later.

Bidirectional type synthesis is not limited to polymorphic languages; it is also applicable to dependently typed languages and has become an integral part of type checkers for modern programming languages.
A recent survey has summarised the design principles of bidirectional typing and progress so far on bidirectional typing.
 In fact, bidirectional type synthesis has demonstrated such significance that certain traditional techniques like unification are no longer as essential in comparison.
 As a result, bidirectional type synthesis can be regarded as the cornerstone of type-checking technology.

Bidirectional typing has proven to be an valuable set of design principles.
However, unlike parsing, which has both theoretical foundations and practical implementations, bidirectional typing has been developed on a case-by-case basis, lacking a general theory.
While it is rather easy to implement a specific type synthesizer using bidirectional typing, a theory that provides explicit specifications with rigorously proven properties is still absent.
Instead, we can only offer general expectations and considerations when designing bidirectional typing for a specific system, as summarised by~\citet{Dunfield2021}.
Consequently, there is no readily available `type-checker generator', and each type checker based on bidirectional typing must be implemented independently.

In light of the extensive adoption and scalability of bidirectional typing in programming languages, this paper aims to initiate a general theory of bidirectional typing and type synthesis. 
We will explore the key characteristics of bidirectional typing~\citep{Dunfield2021}, which are sufficient to derive a sound, complete, and decidable type synthesis algorithm for simple type theories.
Our approach involves constructing the theory within Martin-L\"of type theory and formalising it using the \Agda proof assistant.
This formalisation provides a generic type synthesis algorithm that is applicable to all calculi employing bidirectional typing for simple types, hence essentially a verified type-checker generator.
By rigorously defining its core principles, we aim to lay the foundation for further advancements in bidirectional typing.

\subsection{Related work}
Before proceeding with the main development, let us review what has been done towards this line of research.

\citep{Xie2018}
\paragraph{Language formalization and its frameworks}
\cite{Wadler2022}
PoplMark Challenge~\citep{Aydemir2005}

\cite{Cimini2020,Cimini2022}

\citep{Ahrens2018,Fiore2022,Gheri2020,Ahrens2022}
\cite{Allais2021}

\paragraph{Theories of abstract syntax with variable binding}
\cite{Fiore1999,Hirschowitz2010,Ahrens2018,Fiore2022,Ahrens2021,Arkor2020,Hirschowitz2022}
\cite{Fiore2013,Hamana2011,Hamana2022}


\paragraph{Type checker generation}
\cite{Gast2004,Grewe2015,Pacak2020,Cimini2020}

\paragraph{Bidirectional typing}

\cite{Pierce2000}
\cite{Peyton-Jones2007}
\cite{Dunfield2021}


\subsection{Plan and Contributions of this paper}

%  Our work addresses the problem with existing methodologies, setting the stage for further advances towards a more general theory of type synthesis.

  %In short, this paper is a small but first step towards a general theory of type checking and inference.
  %By formalizing our theory of bidirectional type synthesise in \Agda, we provide a framework capable of handling bidirectional type systems of simple types uniformly.
  %Our work addresses the problem with existing methodologies, 

\LT{
Points to elaborate:
\begin{enumerate}
  \item The existing work of bidirectional typing does not take account of practical aspects such as missing annotation.
    Lightly or rarely discussed.
  \item A meta-problem: existing work is based on classical logic (or left unspecified at least).
    Soundness of completeness of a type checking algorithm in a less straightforward way.
  \item A review of what has been done towards this line of research.
  \item Finally, our responses (contributions of this paper) and some example.
\end{enumerate}
Assumption to justify:
\begin{enumerate}
  \item Use application in spin form to justify our design choice: type expression of arbitrary rank and the number of constructs in a calculus.
  \item Argue that syntax-directed form is essential and practical.
  \item Pfenning's recipe is only of design guide not a technical requirement (there are many exceptions).
\end{enumerate}
}
Our restrictions:
\begin{enumerate}
  \item Assumptions: Why only simple types? Various ways to deal with polymorphism; consistent on simple types.
  \item Assumption: Syntax-directed typing rules
\end{enumerate}

Our contributions:
\begin{enumerate}
  \item Generic: Type checking for simple type theories;
  \item Computational formulation: Bidirectionalisation over annotatability; Decidability over soundness and completeness;
  \item Verified and Constructive: Fully formalised in Martin-Löf type theory (with Axiom K).
    LEM is not assumed, so a decidable statement has both computational and non-trivial logical readings.
\end{enumerate}

Plan of this paper:
\begin{enumerate}
  \item Introduction (\Cref{sec:intro}) 
    \LT{1+3 pp}
  \item Key ideas (\Cref{sec:key-ideas})
    \Josh{2.5 pp}
  \item Definitions for bidirectional type systems (bidirectional binding signature, bidirectional type systems, signature erasure and mode annotation) (\Cref{sec:defs})
    \LT{4 pp}
  \item Soundness, completeness, bidirectionalisation, and annotatability (\Cref{sec:pre-synthesis})
    \Josh{Erasure from bidirectional typing derivation to raw terms with a mode}
    \Josh{3 pp}
  \item Bidirectional type inference (\Cref{sec:type-synthesis})
    \LT{4 pp}
  \item Formalisation and further examples (\Cref{sec:formalisation})
    \LT{5.5pp}
  \item Concluding remarks (\Cref{sec:future})
    \LT{1 p} \Josh{1p}
\end{enumerate}
  \LT{better title?}




