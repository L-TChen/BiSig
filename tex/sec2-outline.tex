%!TEX root = BiSig.tex

\section{The Key Ideas} \label{sec:key-ideas}
\begin{figure}
  \small
  \bgroup
  \renewcommand{\arraystretch}{1.5}
  \begin{tabular}{ r l }
    $\boxed{\Gamma |-_{\Sigma, \Omega} \isTerm{t} \syn A}$ & A raw term $\isTerm{t}$ synthesises a type $A$ under $\Gamma$ \\
    $\boxed{\Gamma |-_{\Sigma, \Omega} \isTerm{t} \chk A}$ & A raw term $\isTerm{t}$ checks against a type $A$ under $\Gamma$
  \end{tabular}
  \egroup
  \centering
  \begin{mathpar}
    \inferrule{(\isTerm{x} : A) \in \Gamma}{\Gamma \vdash \isTerm{x} \syn A}\,\SynRule{Var}
    \and
    \inferrule{\Gamma \vdash \isTerm{t} \chk A}{\Gamma \vdash \isTerm{t \annotate A}\syn A}\,\SynRule{Anno}
    \and
    \inferrule{\Gamma \vdash \isTerm{t} \syn B \\ B = A}{\Gamma \vdash \isTerm{t} \chk A}\,\ChkRule{Sub}
    \and
    \inferrule{\Gamma \vdash \isTerm{t} \syn A \bto B \\ \Gamma \vdash \isTerm{u} \chk A}{\Gamma \vdash \isTerm{t\;u} \syn B}\,\SynRule{App}
    \and
    \inferrule{\Gamma, \isTerm{x} : A \vdash \isTerm{t} \chk B}{\Gamma \vdash \isTerm{\lam{x}t} \chk A \bto B}\,\ChkRule{Abs}
  \end{mathpar}
  \caption{Bidirectional simply typed $\lambda$-calculus $\Lambda^{\Leftrightarrow}_{\bto}$}
  \label{fig:bi-stlc}
\end{figure}
\begin{remark}
  We avoid using the term `function type' and its conventional notation $\to$ at the object level on purpose, as it may be confused with function types in our type-theoretic meta-language.
\end{remark}


