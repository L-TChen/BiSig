%!TEX root = BiSig.tex

\section{The Key Ideas} \label{sec:key-ideas}
\begin{figure}
  \centering
  \small
  \begin{mathpar}
    \boxed{\Gamma \vdash \isTerm{t} \syn A}
    \quad\text{and}\quad \boxed{\Gamma \vdash \isTerm{t} \chk A}
    \\
    \inferrule{(x : A) \in \Gamma}{\Gamma \vdash \isTerm{x} \syn A}\;(\text{var}^\syn)
    \and
    \inferrule{\Gamma \vdash \isTerm{t} \chk B \\ A = B}{\Gamma \vdash \isTerm{t : B}\syn A}\;(\text{anno})
    \and
    \inferrule{\Gamma \vdash \isTerm{t} \syn B \\ A = B}{\Gamma \vdash \isTerm{t\subsum} \chk A}\;(\text{sub})
    \and
    \inferrule{\Gamma \vdash \isTerm{t} \syn A \bto B \\ \Gamma \vdash \isTerm{u} \chk A}{\Gamma \vdash \isTerm{t\;u} \syn B}\;(\text{app})
    \and
    \inferrule{\Gamma, \isTerm{x} : A \vdash \isTerm{t} \chk B}{\Gamma \vdash \isTerm{\lam{x}t} \chk A \bto B}\;(\text{abs})
  \end{mathpar}
  \caption{Bidirectional simply typed $\lambda$-calculus $\Lambda^{\Leftrightarrow}_{\bto}$}
  \label{fig:bi-stlc}
\end{figure}
\begin{remark}
  We avoid using the term `function type' and its conventional notation $\to$ at the object level on purpose, as it may be confused with function types in our type-theoretic meta-language.
\end{remark}


