%!TEX root = BiSig.tex

\section{Simple Type Theories and Bidirectional Type Systems}\label{sec:defs}

\subsection{Simple Types}
\begin{definition}
  A \emph{(simple type) signature} $\Sigma$ consists of a set $I$ and an \emph{arity} function $\arity\colon I \to \mathbb{N}$.
  An inhabitant $i : I$ is meant to represent the \emph{$i$-th operation} $\tyOp_i$ and its \emph{number} $ \arity(i)$ of arguments.

  The judgements of \emph{$\Sigma$-types} and \emph{$\Sigma$-contexts} over a variable set $\Theta$ are defined inductively by \Cref{fig:simple-type,fig:simple-context} respectively.
We write $A : \Type_{\Sigma}(\Theta)$ for $\Theta |-_{\Sigma} A$ and $\Gamma \colon \Cxt_{\Sigma}(\Theta)$ for $\Theta |-_{\Sigma} \Gamma$ to disambiguate $\Sigma$-types and $\Sigma$-contexts in the case of any confusion.
\end{definition}

\begin{figure}
  \begin{minipage}[b]{.55\textwidth}
    \centering
    \small
    \judgbox{\Theta|-_{\Sigma} A}{$A$ is well-formed with type variables in $\Theta$}
    \begin{mathpar}
      \inferrule{\Theta \ni X_i}{\Theta |-_\Sigma X_i} \and
    \inferrule{\Theta |-_{\Sigma} A_1 \\ \cdots \\ \Theta |-_{\Sigma} A_{\arity(o)}}{\Theta |-_{\Sigma} \tyOp_o(A_1, \ldots, A_{\arity(o)})}
    \end{mathpar}
    \caption{Type formation rules}
    \label{fig:simple-type}
  \end{minipage}
  \begin{minipage}[b]{.4\textwidth}
    \centering
    \small
    \judgbox{\Theta |-_{\Sigma} \Gamma}{}
    \begin{mathpar}
      \inferrule{ }{\Theta |-_{\Sigma} \cdot }\and
      \inferrule{\Theta |-_{\Sigma} A \\ \Theta|-_{\Sigma} \Gamma}{\Theta |-_{\Sigma} \Gamma, A}
    \end{mathpar}
    \caption{Context formation rules}
  \label{fig:simple-context}
  \end{minipage}
\end{figure}

\begin{example}\label{ex:implication}
  Simply typed $\lambda$-calculus $\Lambda_{\bto}$ typically includes a binary implication type denoted by $\bto$ and some base types to ensure that the set of all types is non-empty.
  The type signature $\Sigma_{\bto}$ of simply typed $\lambda$-calculus consists of a binary operation $\bto$ and a base type (nullary operation) $N$.
  In the case of simply typed $\lambda$-calculus $\Lambda_{\bto, \btimes}$ with binary products, we can extend $\Sigma_{\bto}$ by adding a binary operation $\btimes$ to represent the binary product type, denoted as $\Sigma_{\bto, \btimes}$.
\end{example}

\subsection{Binding Signatures} \label{subsec:binding-sig}
\begin{definition}\label{def:binding-signature}
  For a type signature $\Sigma$, a \emph{binding signature} $\Omega$ consists of a set $O$ and a function
  \[
    \mathit{ar}\colon O \to \sum_{\Xi : \N} \left(\Cxt_\Sigma(\Xi) \times \Type_\Sigma(\Xi)\right)^* \times \Type_\Sigma(\Xi).
  \]
  Each inhabitant $o: O$ is meant to represent a term construct $\tmOp_o$ in a simple type theory with a triple $\left(\Xi, \left[\left(\Delta_1; A_1\right), \ldots, \left(\Delta_{n}; A_{n}\right) \right], A\right)$
  instead of a number as its arity $\arity_o$ where
  \begin{enumerate}
    \item $\Xi$ is the number of type variables, 
    \item $A : \Type_\Sigma(\Xi)$ is the target type of $\tmOp_o$, and
    \item $\left[\left(\Delta_1; A_{1}\right), \ldots, \left(\Delta_{n}; A_{n}\right) \right]$ is a list of pairs where
    \item the $i$-th pair $(\Delta_i, A_i) :\Cxt_{\Sigma}(\Xi) \times \Term_{\Sigma}(\Xi)$ are types of the binding variables and the type of the $i$-th argument of $\tmOp_{o}$.
  \end{enumerate}
  For brevity, we write $o \colon \Xi \rhd [\Delta_1]A_{1}, \ldots, \left[\Delta_{n}\right] A_{n} \to A$ to indicate an operation with its arity. 
\end{definition}
\begin{remark}[Terminology on binding signature]
  \cite{Aczel1978,Fiore2010}
\end{remark}


\begin{example} \label{ex:STLC-sig}
  Given the type signature $\Sigma_{\bto}$ for the implication type, the term signature $\Lambda_{\bto}$ for simply typed $\lambda$-calculus can be described by operations
  \begin{align*}
    \mathsf{app}\colon A, B \rhd (A \bto B), A \to B && \mathsf{abs}\colon A , B \rhd [A]B \to (A \bto B)
  \end{align*}
  or, verbosely, the signature $\Lambda_{\bto}$ consists of the type $O_\Lambda = \{\abs, \app\}$ with arities
  \begin{align*}
    \arity(\app) = (\{A, B\}, [(\cdot; A \bto B), (\cdot; A)], B)
    && 
    \arity(\abs) = (\{A, B\}, [(\cdot, A; B)], A \bto B)
  \end{align*}
  For the type signature $\Sigma_{\bto, \btimes}$, the term signature $\Omega_{\bto, \btimes}$ for simply typed $\lambda$-calculus with finite products has three additional operations
to $\Omega_{\bto}$:
  \begin{align*}
    \mathsf{pair}\colon A, B \rhd A, B \to A \btimes B
    && \fst \colon A, B \rhd A \btimes B \to A
    && \snd \colon A, B \rhd A \btimes B \to B
  \end{align*}
\end{example}

First, it is noteworthy that the set of term constructs in a type theory need not be finite.
For instance, a type theory may adopt a spine application which takes indefinitely many arguments---for each number $n$ of arguments an $(n+1)$-ary application construct can be introduced, so even if each operation has a definite number of arguments by definition a spine application is still expressible.

More importantly, the inclusion of the set of variables used in an operation is a salient feature of our binding signatures.
Instead of treating application as a family of constructs $\app_{A, B}$ indexed by all types $A$ and $B$, as done by \citet{Fiore2022}, we are able to identify them as a single construct $\app$.
This is not only for brevity but also necessary to compare the type equality during type synthesis and checking.

%\subsection{Intrinsically Typed Terms}
%
%Intrinsically typed terms for a type signature $\Sigma$ and a term signature $\Omega$ are nothing more than derivations of the intrinsic typing judgement $\Gamma |-_{\Sigma, \Omega} A$ constructed with only the rule $(\var)$ and the rule scheme $(\tmOp)$ displayed in \Cref{fig:intrinsic-typing}.
%
%The substitution $\rho\colon \Sub{\Xi}{\emptyset}$ is used to instantiate variables in the local context $\Xi$ with concrete types.
%Accordingly, $\rho$ has to be applied to types that appear in the arity to construct a term by $\tmOp$ to ensure all types are well-formed without any use of type variables.
%
%\begin{figure}
%  \centering
%  \small
%  \begin{mathpar}
%    \boxed{\Gamma |-_{\Sigma, \Omega} A} \\
%    \inferrule{A \in \Gamma}{\Gamma |-_{\Sigma, \Omega} A}\;(\var)
%    \and
%    \inferrule{\rho\colon \Sub{\Xi}{\emptyset}  \\ \Gamma, \simsub{\Delta_{1}}{\rho} |-_{\Sigma, \Omega} \simsub{A_{1}}{\rho} \quad\cdots\quad \Gamma, \simsub{\Delta_{n}}{\rho} |-_{\Sigma, \Omega} \simsub{A_{n}}{\rho}}
%    {\Gamma |-_{\Sigma, \Omega} \simsub{A}{\rho}}\;(\tmOp)
%    \and \text{for $o\colon \Xi \rhd [\Delta_1]A_1, \ldots, [\Delta_{n}]A_{n} \to A$ in $\Omega$}
%  \end{mathpar}
%  \caption{Intrinsic typing rules for a simple type theory $(\Sigma, \Omega)$}
%  \label{fig:intrinsic-typing}
%\end{figure}

\subsection{Simple Type Theories}

\begin{figure}
  \centering
  \small
  \judgbox{|-_{\Sigma, \Omega} \isTerm{t}}{$t$ is a raw term for the signature $(\Sigma, \Omega)$}
  \begin{mathpar}
    \inferrule{x : \Identifier}{|-_{\Sigma, \Omega} \isTerm{x}}\,\Rule{Var}
    \and
    \inferrule{\cdot |-_{\Sigma} A \\ |-_{\Sigma, \Omega}\isTerm{t}}{|-_{\Sigma, \Omega} \isTerm{t \annote A}}\,\Rule{Anno}
    \\
    \inferrule{|-_{\Sigma, \Omega} \isTerm{t_1} \quad \cdots \quad |-_{\Sigma, \Omega} \isTerm{t_n}}
    {|-_{\Sigma, \Omega} \isTerm{\tmOp_o(t_1, \ldots, t_n)}}\,\Rule{Op}_o
    \and \text{for $o \colon \Xi \rhd [\Delta_1]A_{1}, \ldots, [\Delta_{n}] A_{n} \to A$ in $\Omega$}
  \end{mathpar}
  \caption{Raw term formation rules}
\end{figure}

\begin{figure}
  \centering
  \small
  \judgbox{\Gamma |-_{\Sigma, \Omega} \isTerm{t} : A}{A raw term $t$ has a type $A$ (without type variables) under $\Gamma$ for the signature $(\Sigma, \Omega)$}
  \begin{mathpar}
    \inferrule{(x : A) \in \Gamma}{\Gamma |-_{\Sigma, \Omega} \isTerm{x} : A}\,\Rule{Var}
    \and
    \inferrule{\Gamma |- \isTerm{t} : A}{\Gamma |- (\isTerm{t \annote A}) : A}\,\Rule{Anno}
    \and
    \inferrule{\rho : \Sub{\Xi}{\emptyset} \\ \Gamma, \isTerm{\vec{x}_1} : \simsub{\Delta_{1}}{\rho} |-_{\Sigma, \Omega} \isTerm{t_1} : \simsub{A_{1}}{\rho} \quad\cdots\quad \Gamma, \isTerm{\vec{x}_n} : \simsub{\Delta_{n}}{\rho} |-_{\Sigma, \Omega} \isTerm{t_n} : \simsub{A_{n}}{\rho}}
    {\Gamma |-_{\Sigma, \Omega} \isTerm{\tmOp_o(\vec{x}_1.\,t_1; \ldots; \vec{x}_n.\,t_n)} : \simsub{A}{\rho}}\,\Rule{Op}
    \and \text{for $o\colon \Xi \rhd [\Delta_1]A_1, \ldots, [\Delta_{n}]A_{n} \to A$ in $\Omega$}
  \end{mathpar}
  \caption{Typing rules}
  \label{fig:extrinsic-typing}
\end{figure}

\subsection{Bidirectional Binding Signatures and Bidirectional Type Systems}

\begin{definition}
  For a type signature $\Sigma$, a \emph{bidirectional binding signature} $\Omega$ is a set $O$ with a function
  \[
    \mathit{ar}\colon O \to \sum_{\Xi : \N} \left(\Cxt_{\Sigma}(\Xi) \times \Type_{\Sigma}(\Xi) \times {\Mode}\right)^* \times \Type_{\Sigma}(\Xi) \times {\Mode}.
  \]
  where $\Mode$ consists of two inhabitants $\chk$ for checking and $\syn$ for synthesis.
  Bidirectional binding signatures are just binding signatures (\Cref{def:binding-signature}) augmented with a mode for each argument and its target of a construct $\tmOp_o$ in a bidirectional system, i.e.
  an arity is a $4$-tuple
  \[
    \left(\Xi, \left[\left(\Delta_1; A_1; d_1\right), \ldots, \left(\Delta_{n}; A_{n}; d_n\right) \right], A, d\right)
  \]
  where $d$ and $d_i$'s indicate the modes of a construct and its arguments respectively.

  For brevity, we write $o \colon \Xi \rhd [\Delta_1]A_{1}^{\dir{d_1}}, \ldots, [\Delta_{n}] A^{\dir{d_n}}_{n} \to A^{\dir{d}}$ to indicate an operation with its arity. 
\end{definition}

\begin{example}
  The term signature $\Lambda_{\bto}^{\Leftrightarrow}$ for bidirectional simply typed $\lambda$-calculus introduced in \Cref{subsec:binding-sig} can be specified by operations 
  \begin{align*}
    \mathsf{app}\colon A, B \rhd (A \bto B)^{\syn}, A^{\chk} \to B^{\syn} &&
    \mathsf{abs}\colon A , B \rhd [A]B^{\chk} \to (A \bto B)^{\chk}
  \end{align*}
  extending $\Lambda_{\bto}$ (\Cref{ex:STLC-sig}) with the mode information.
\end{example}

\LT{Some remark about signature erasure and annotation; introduce the notation $\erase{\Omega}$}

\begin{figure}
  \centering
  \small
  \judgbox{|-_{\Sigma, \Omega} \isTerm{t}^\dir{d}}{$t$ is a raw term for $(\Sigma, \erase{\Omega})$ in mode $d$ for  $(\Sigma, \Omega)$}
  \begin{mathpar}
    \inferrule{x : \Identifier}{|-_{\Sigma, \Omega} \isTerm{x}^{\syn}}\,\SynRule{Var}
    \and
    \inferrule{\cdot |-_{\Sigma} A \\ |-_{\Sigma, \Omega}\isTerm{t}^{\chk}}{|-_{\Sigma, \Omega} (\isTerm{t \annote A})^{\syn}}\,\SynRule{Anno}
    \and
    \inferrule{|-_{\Sigma, \Omega} \isTerm{t}^{\syn}}{|-_{\Sigma, \Omega} \isTerm{t}^{\chk}}\,\ChkRule{Sub}
  \end{mathpar}
  \begin{mathpar}
    \inferrule{|-_{\Sigma, \Omega} \isTerm{t_1}^\dir{d_1} \quad \cdots \quad |-_{\Sigma, \Omega} \isTerm{t_n}^\dir{d_n}}
    {|-_{\Sigma, \Omega} \isTerm{\tmOp_o(\vec{x}_1.\, t_1; \ldots;\vec{x}_n.\, t_n)}^\dir{d}}\,\Rule{Op}
    \and \text{for $o \colon \Xi \rhd [\Delta_1]A_{1}^{\dir{d_1}}, \ldots, [\Delta_{n}] A_{n}^{\dir{d_n}} \to A^{\dir{d}}$ in $\Omega$}
  \end{mathpar}
  \caption{Preprocessed raw terms}
\end{figure}

\begin{figure}
  \centering
  \small
  \judgbox{\Gamma |-_{\Sigma, \Omega} \isTerm{t} :^\dir{d} A}{A raw term $t$ has a type $A$ (without type variables) under $\Gamma$}
  \begin{mathpar}
    \inferrule{(x : A) \in \Gamma}{\Gamma |-_{\Sigma, \Omega} \isTerm{x} :^{\syn} A}\,\SynRule{Var}
    \and
    \inferrule{\Gamma |-_{\Sigma, \Omega} \isTerm{t} :^{\chk} A}{\Gamma |-_{\Sigma, \Omega} (\isTerm{t \annote B}) :^{\syn}  A}\,\SynRule{Anno}
    \and
    \inferrule{\Gamma |-_{\Sigma, \Omega} \isTerm{t} :^{{\syn}} B \\ A = B}{\Gamma |-_{\Sigma, \Omega} \isTerm{t} :^{\chk} A}\,\ChkRule{Sub}
    \\
    \inferrule{\rho\colon \Sub{\Xi}{\emptyset} \\ \Gamma, \simsub{\isTerm{\vec{x}_{1}} : \Delta_{1}}{\rho} |-_{\Sigma, \Omega} \isTerm{t_{1}} :^\dir{d_1} \simsub{A_1}{\rho} \\
      \cdots \\
    \Gamma, \simsub{\isTerm{\vec{x}_{n}} : \Delta_{n}}{\rho} |-_{\Sigma, \Omega} \isTerm{t_n} :^{\dir{d_n}} \simsub{A_{n}}{\rho}}
    {\Gamma |-_{\Sigma, \Omega} \isTerm{\tmOp_o(\vec{x}_1 .\, t_{1}; \ldots; \vec{x}_{n}.\, t_{n})} :^{\dir{d}} \simsub{A}{\rho}} \,\mathsf{Op}
    \and \text{for $o \colon \Xi \rhd [\Delta_1]A_{1}^{\dir{d_1}}, \ldots, [\Delta_{n}] A_{n}^{\dir{d_n}} \to A^\dir{d}$ in $\Omega$}
  \end{mathpar}
  \begin{minipage}{.15\linewidth}
  Abbreviations: 
  \end{minipage}
  \begin{minipage}{.8\linewidth}
    \begin{flushleft}
    $\boxed{\Gamma |-_{\Sigma, \Omega} \isTerm{t} \syn A} \defeq \boxed{\Gamma |-_{\Sigma, \Omega} \isTerm{t} :^{\syn} A}$ \quad A raw term $t$ synthesises a type $A$ under $\Gamma$ \\
    $\boxed{\Gamma |-_{\Sigma, \Omega} \isTerm{t} \chk A} \defeq \boxed{\Gamma |-_{\Sigma, \Omega} \isTerm{t} :^{\chk} A}$ \quad A raw term $t$ checks against a type $A$ under $\Gamma$
    \end{flushleft}
  \end{minipage}
  \caption{Bidirectional typing rules}
\end{figure}

\begin{example}
  
\end{example}



