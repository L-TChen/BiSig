%!TEX root = BiSig.tex

\section{Soundness, Completeness, and Mode Preprocessing}\label{sec:pre-synthesis}

\subsection{Soundness and Completeness}
\label{sec:soundness-and-completeness}

\Josh{Throughout this sub-section, we quantify over any bidirectional type system~$(\Sigma, \Omega)$, context $\Gamma : \Cxt_\Sigma(\emptyset)$, raw term $\erase\Gamma \vdash_{\Sigma, \erase\Omega} t$, mode~$d$, and type $A : \Type_\Sigma(\emptyset)$.}

Erasure of a bidirectional binding signature removes mode information and keeps everything else intact; this can be straightforwardly extended by induction to remove mode information from a bidirectional typing derivation and arrive at an ordinary typing derivation.

\begin{lemma}[Soundness]\label{thm:term-soundness}
If\/ $\Gamma \vdash_{\Sigma, \Omega} t :^\dir{d} A$, then $\Gamma \vdash_{\Sigma, \erase\Omega} t : A$.
\end{lemma}

\begin{proof}
Induction on the given derivation, mapping every bidirectional typing rule to its mode-less counterpart except $\ChkRule{Sub}$, in which case the induction hypothesis $\Gamma |-_{\Sigma,\erase\Omega} t : B$ suffices due to the premise $B = A$.
\end{proof}

We can also remove typing and retain mode information, arriving at a mode derivation instead.

\begin{proposition}\label{thm:typing-removal}
If\/ $\Gamma \vdash_{\Sigma, \Omega} t :^\dir{d} A$, then $\erase\Gamma \vdash_{\Sigma, \Omega} t^\dir{d}$.
\end{proposition}

\begin{proof}
Induction on the given derivation, mapping every rule to its counterpart.
\end{proof}

Conversely, if we have both mode and typing derivations for the same term, we can combine them and obtain a bidirectional typing derivation.

\begin{lemma}[Completeness]\label{thm:term-completeness}
If\/ $\erase\Gamma |-_{\Sigma, \Omega} t^\dir{d}$ and\/ $\Gamma |-_{\Sigma, \erase\Omega} t : A$, then $\Gamma |-_{\Sigma, \Omega} t :^\dir{d} A$.
\end{lemma}

\begin{proof}
Induction on the given mode derivation.
For \SynRule{Var}, \SynRule{Anno}, and \Rule{Op}, the outermost rule used in the given typing derivation must be the corresponding typing rule, so by the induction hypotheses we have bidirectional typing derivations for all the sub-terms, to which we can then apply the corresponding bidirectional typing rule.
The \ChkRule{Sub} case is similar but slightly simpler: the induction hypothesis directly gives us a derivation of $\Gamma |-_{\Sigma,\Omega} t \syn A$, to which we apply $\ChkRule{Sub}$.
\end{proof}

In short, soundness and completeness are no more than the separation and combination of mode and typing information carried by the three kinds of derivations while keeping their basic structure, which is directed by the same raw term.

\subsection{Mode Preprocessing}
\label{sec:mode-preprocessing}

\newcommand{\True}{\mathbf{T}}
\newcommand{\False}{\mathbf{F}}

\begin{figure}
  \centering
  \small
  \begin{tabular}{ r r l }
    & & is in (generalised) mode~$d$, \\
    $\smash{\boxed{V |-_{\Sigma, \Omega} \isTerm{t}^{\dir{d}\,v\,e}}}$
    & The raw term~$\isTerm{t}$\hspace{-.6em}
    & misses some type annotation iff $v = \False$, and \\
    & & has an outermost mode cast iff $e = \False$
  \end{tabular}
  \begin{mathpar}
    \inferrule{x \in V}{V |-_{\Sigma, \Omega} \isTerm{x}^{\syn\,\True\,\True}}\,\SynRule{Var}
    \and
    \inferrule{\cdot |-_{\Sigma} A \\ V |-_{\Sigma, \Omega}\isTerm{t}^{\chk\,v\,e}}{V |-_{\Sigma, \Omega} (\isTerm{t \annotate A})^{\syn\,v\,\True}}\,\SynRule{Anno}
    \and
    \inferrule{V |-_{\Sigma, \Omega}\isTerm{t}^{\chk\,v\,\True}}{V |-_{\Sigma, \Omega} \isTerm{t}^{\syn\,\False\,\False}}\,\SynRule{Missing}
    \and
    \inferrule{V |-_{\Sigma, \Omega} \isTerm{t}^{\syn\,v\,\True}}{V |-_{\Sigma, \Omega} \isTerm{t}^{\chk\,v\,\False}}\,\ChkRule{Sub}
    \and
    \inferrule{V, \vec x_1 |-_{\Sigma, \Omega} \isTerm{t_1}^{\dir{d_1}\,v_1\,e_1} \\ \cdots \\ V, \vec x_n |-_{\Sigma, \Omega} \isTerm{t_n}^{\dir{d_n}\,v_n\,e_n}}
    {V |-_{\Sigma, \Omega} \tmOpts^{\dir{d}\,(\bigwedge_i v_i)\,\True}}\,\Rule{Op}
  \end{mathpar}
  \caption{Generalised mode derivations}
\end{figure}

\begin{lemma}\label{thm:adjustment}
For any raw term $V |-_{\Sigma,\erase\Omega} t$, if
\[ V |- t^{\syn\,\True\,\True} ~~\vee~~ V |- t^{\chk\,\True\,\True} ~~\vee~~ \exists e.~V |- t^{\chk\,\False\,e} \]
then for any mode~$\dir{d}$, there is a derivation of\/ $V |-_{\Sigma,\Omega} t^{\dir{d}\,v\,e}$ for some boolean $v$~and~$e$.
\end{lemma}

Intuitively, the antecedent gives a precise classification of a term as either a `real' synthesising term (i.e.~without missing annotations or an outermost mode cast), a `real' checking term, or a term that has missing annotations.
The lemma only performs some slight adjustment to obtain a derivation with any required mode.

\begin{proof}[Proof of \cref{thm:adjustment}]
Case analysis on the antecedent and~$\dir{d}$.
\begin{itemize}
\item For the first disjunct with $\dir{d} = {\syn}$, the second disjunct with $\dir{d} = {\chk}$, and the third disjunct with $\dir{d} = {\chk}$:
The mode we have matches the mode to be established, so no adjustment is needed.
\item For the first disjunct with $\dir{d} = {\chk}$:
Apply $\ChkRule{Sub}$.
\item For the second disjunct with $\dir{d} = {\syn}$:
Apply $\SynRule{Missing}$.
\item For the third disjunct with $\dir{d} = {\syn}$:
If $e = \False$, then the outermost rule of the derivation must be $\ChkRule{Sub}$, removing which we get a derivation of $V |-_{\Sigma,\Omega} t^{\syn\,\False\,\True}$;
otherwise $e = \True$, in which case we can apply $\SynRule{Missing}$.
\vspace{-\topsep-\baselineskip}
\end{itemize}
\end{proof}

\begin{theorem}[Generalised Mode Preprocessing]\label{thm:generalised-mode-preprocessing}
For any raw term $V |-_{\Sigma,\erase\Omega} t$ and mode~$\dir{d}$, there is a derivation of\/ $V |-_{\Sigma,\Omega} t^{\dir{d}\,v\,e}$ for some boolean $v$~and~$e$.
\end{theorem}

\begin{proof}
The conclusion is the same as that of \cref{thm:adjustment}, so it suffices to prove the antecedent of \cref{thm:adjustment} by induction on~$t$.
\begin{itemize}
\item $\Rule{Var}$:
Establish the first disjunct using $\SynRule{Var}$.
\item $\Rule{Anno}$ ($t \bbcolon A$):
By the induction hypothesis and \cref{thm:adjustment}, there is a derivation of $V |-_{\Sigma,\Omega} t^{\chk\,v\,e}$ for some boolean $v$~and~$e$.
If $v = \False$, establish the third conjunct using $\SynRule{Anno}$ and $\ChkRule{Sub}$; otherwise, establish the first conjunct using $\ChkRule{Sub}$.
\item $\Rule{Op}$ ($\tmOpts$):
By the induction hypotheses and \cref{thm:adjustment}, there is a derivation of $V, \vec x_i |-_{\Sigma,\Omega} t_i^{\dir{d_i}\,v_i\,e_i}$ for each sub-term~$t_i$, and by applying $\Rule{Op}$ we get a derivation of $V |-_{\Sigma,\Omega} \tmOpts^{\dir d\,v\,\True}$ where $v = \bigwedge_i v_i$.
If $v = \True$, establish the first or second disjunct depending on~$\dir d$; otherwise, establish the third disjunct, applying $\ChkRule{Sub}$ if $\dir d = {\syn}$.
\vspace{-\topsep-\baselineskip}
\end{itemize}
\end{proof}

\begin{lemma}\label{thm:Pre?-true}
If\/ $V |-_{\Sigma,\Omega} t^{\dir{d}\,\True\,e}$, then $V |-_{\Sigma,\Omega} t^\dir{d}$.
\end{lemma}

\begin{lemma}\label{thm:Pre?-false}
If\/ $V |-_{\Sigma,\Omega} t^{\dir{d}\,\False\,e}$, then $V \not|-_{\Sigma,\Omega} t^\dir{d}$.
\end{lemma}

\begin{corollary}[Mode Preprocessing]\label{thm:mode-preprocessing}
  It is decidable for any raw term $V |-_{\Sigma,\erase\Omega} t$ and mode~$\dir{d}$ whether $V |-_{\Sigma,\Omega} \isTerm{t}^\dir{d}$.
\end{corollary}

\begin{proof}
By \cref{thm:generalised-mode-preprocessing}, there is a derivation of $V |-_{\Sigma,\erase\Omega} t^{\dir{d}\,v\,e}$ for some boolean $v$~and~$e$.
If $v = \True$, then $V |-_{\Sigma,\Omega} t^\dir{d}$ by \cref{thm:Pre?-true}; otherwise $V \not|-_{\Sigma,\Omega} t^\dir{d}$ by \cref{thm:Pre?-false}.
\end{proof}

\begin{corollary}
If\/ $V |-_{\Sigma,\Omega} t^\dir{d}$, then $V |-_{\Sigma,\Omega} t^{\dir{d}\,\True\,e}$ for some boolean~$e$.
\end{corollary}

\begin{corollary}
If\/ $V \not|-_{\Sigma,\Omega} t^\dir{d}$, then $V |-_{\Sigma,\Omega} t^{\dir{d}\,\False\,e}$ for some boolean~$e$.
\end{corollary}

\begin{proof}
By \cref{thm:generalised-mode-preprocessing}, there is a derivation of $V |-_{\Sigma,\erase\Omega} t^{\dir{d}\,v\,e}$ for some boolean $v$~and~$e$.
Either $v = \True$ or $v = \False$.
If $V |-_{\Sigma,\Omega} t^\dir{d}$, then $v$ has to be $\True$ since $v = \False$ leads to a contradiction with \cref{thm:Pre?-false}.
Symmetrically, if $V \not|-_{\Sigma,\Omega} t^\dir{d}$, then $v$ has to be $\False$ so as not to contradict \cref{thm:Pre?-true}.
\end{proof}

\Josh{($1\frac{1}{2}$~pages)

• Generalised mode derivations and bidirectionalisation

• Intuitive meaning of $v$~and~$e$

• Decidability of the original mode judgements

}

\subsection{Annotatability Is Not Completeness}
\label{sec:annotatability}

\Josh{$\frac{1}{2}$~pages}

\begin{figure}
  \centering\small
  \judgbox{\isTerm{t} \sqsupseteq \isTerm{u}}{A raw term $t$ is more annotated than $u$ (for some bidirectional type system $(\Sigma, \Omega)$)}
  \begin{mathpar}
    \inferrule{\isTerm{t} \sqsupseteq \isTerm{u}}
              {(\isTerm{t} \annotate A) \sqsupseteq \isTerm{u}}\;\Rule{More}
    \and
    \inferrule{\vphantom{x : \Identifier}}
              {\isTerm{x} \sqsupseteq \isTerm{x}}
    \and
    \inferrule{\isTerm{t} \sqsupseteq \isTerm{u}}
              {(\isTerm{t} \annotate A) \sqsupseteq (\isTerm{u} \annotate A)}
    \and
    \inferrule{\isTerm{t_1} \sqsupseteq \isTerm{u_1} \quad \cdots \quad \isTerm{t_n} \sqsupseteq{u_n}}
              {\tmOpts \sqsupseteq \tmOpus}
  \end{mathpar}
  
  \caption{Annotation ordering between raw terms}
  \label{fig:annotation-order}
\end{figure}

