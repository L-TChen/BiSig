%!TEX root = BiSig.tex

\section{Bidirectional Type Synthesis and Checking} \label{sec:type-synthesis}
This section focuses on defining mode-correctness and deriving bidirectional type synthesis for any mode-correct bidirectional type system $(\Sigma, \Omega)$.
We start with the statement of the decidability of bidirectional synthesis and the trichotomy on raw terms for mode-correct bidirectional type systems in \Cref{subsec:trichotomy}, but defer the definition of mode-correctness until \Cref{sec:mode-correctness} and the decidability proof until \Cref{subsec:bidirectional-synthesis-checking}.
We will then formalise mode-correctness and show the uniqueness of synthesized types in \Cref{sec:mode-correctness}.
This uniqueness property means that any two synthesised types for the same raw term $t$ under the same context $\Gamma$ have to be equal.
This will be used later in the proof of the decidability of bidirectional type synthesis and checking in \Cref{subsec:bidirectional-synthesis-checking}.

\subsection{Trichotomy on Raw Terms by Type Synthesis} \label{subsec:trichotomy}

\begin{theorem}[Decidability of Bidirectional Type Synthesis] \label{thm:bidirectional-type-synthesis}
  In a mode-correct bidirectional type system $(\Sigma, \Omega)$,
  it is decidable whether there exists some $A$ such that
  \[
    \Gamma |-_{\Sigma, \Omega} \isTerm{t} \syn A.
  \]
\end{theorem}

\begin{corollary}[Trichotomy on Raw Terms]\label{cor:trichotomy}
  Exactly one of the following holds:
  \begin{enumerate}
    \item $\isTerm{t}$ is synthesising and\/ $\Gamma |-_{\Sigma, \Omega} \isTerm{t} : A$ for some type $A$.
    \item $\isTerm{t}$ is synthesising but\/ $\Gamma \not|-_{\Sigma, \Omega} \isTerm{t} : A$ for any type $A$.
    \item $\isTerm{t}$ is not synthesising.
  \end{enumerate}
\end{corollary}
\begin{proof}
  Combine  \Cref{lem:soundness-completeness,thm:mode-preprocessing} with \Cref{thm:bidirectional-type-synthesis}.
  
\end{proof}

\subsection{Mode Correctness}\label{sec:mode-correctness}
As \citet{Dunfield2021} outlined, mode-correctness for a bidirectional type system means that for every typing rule:
\begin{enumerate*}
\item each `input' type variable in a premise must be synthesised from `earlier' premises or provided by the conclusion, if the rule is checking;
\item each `output' type variable in the conclusion should have been synthesised from premises already, if the rule is synthesis.
\end{enumerate*}
Here an `input' type variable refers to types in an extension context and the type of a premise in the checking mode.
It is important to note that the order of premises in a bidirectional typing rule also matters, since synthesised types are accumulated during type synthesis.

Consider the rule $\ChkRule{Abs}$ (\Cref{fig:STLC-bidirectional-typing-derivations}) as an example.
This rule is mode-correct, as the type variables $A$ and $B$ in its only premise are already provided by its conclusion $A \bto B$.
Likewise, the rule $\SynRule{App}$ for an application term $\isTerm{t\;u}$ is mode-correct because:
\begin{enumerate*}
\item the type $A \bto B$ of the first argument $t$ is synthesised, thereby ensuring type variables $A$ and $B$ must be known if successfully synthesised;
\item the type of the second argument $u$ is checked against $A$, which has been synthesised earlier;
\item as a result, the type of an application $t\;u$ can be synthesised.
\end{enumerate*}



Now we proceed with the definition of mode-correctness.
As we have outlined, the condition of mode-correctness for a synthesis rule is different from that of a checking rule, and the argument order also matters.
Defining the condition directly for a signature can be somewhat intricate.
Instead, we choose to define the condition in a few more steps, based on the list of arguments---more specifically, triples of an extension context $\Delta_i$, a type $A_i$, and a mode $\dir{d_i}$---pertaining to an operation, and subsequently on a signature.
We also need some auxiliary definitions for the subset of variables of a type and of a context and the set of variables that have been synthesised:
\begin{definition}
  The finite subset\footnote{%
  There are various definitions for finite subsets of a set within Martin-L\"{o}f type theory.
  However, for our purposes, the choice among these definitions is not a matter of concern.}
  of \emph{variables} which occur in a type $A$ is denoted by $\fv(A)$.
  The subset $\fv(\Gamma)$ of variables in a context $\Gamma$ is defined by $\fv(\cdot) = \emptyset$ and $\fv(\Gamma, A) = \fv(\Gamma) \cup \fv(A)$.
  For a list of \emph{arguments} $\overrightarrow{[\Delta_i]A_i^{d_i}}$, define the set of \emph{synthesis type variables} inductively as 
  \begin{align*}
    \synvar(\cdot)                                  & = \emptyset  \\
    \synvar(\overrightarrow{[\Delta_i]A_i^{\dir{d_i}}}, [\Delta_{n}] A_{n}^{\dir{\chk}}) & = 
    \synvar(\overrightarrow{[\Delta_i]A_i^{\dir{d_i}}}) \\
    \synvar(\overrightarrow{[\Delta_i]A_i^{\dir{d_i}}}, [\Delta_{n}] A_{n}^{\dir{\syn}}) & = 
    \synvar(\overrightarrow{[\Delta_i]A_i^{\dir{d_i}}}) \cup \fv(A_{n}).
  \end{align*}
\end{definition}
This subset contains type variables of a synthesising argument and they are exactly those type variables that would be synthesised during type synthesis.

\begin{definition}\label{def:mode-correctness}
  The \emph{mode-correctness for an argument list} $\Xi \rhd [\Delta_{1}] A_{1}^{\dir{d}}, \ldots, [\Delta_n]A_{n}^{\dir{d_n}}$  with respect to a subset $S$ of $\Xi$ is defined inductively as
  \begin{align*}
    \MC_{\mathit{as}}( \cdot ) ={} & \top \\
    \MC_{\mathit{as}}\left(\overrightarrow{[\Delta_i]A_i^{\dir{d_i}}}, [\Delta_{n}]A_{n}^{\syn}\right) 
    & = \MC_{\mathit{as}}\left(\overrightarrow{[\Delta_i]A_i^{\dir{d_i}}}\right)
    \land \fv(\Delta) \subseteq \left( S \cup \synvar\left(\overrightarrow{[\Delta_i]A_i^{\dir{d_i}}}\right) \right)\\
    \MC_{\mathit{as}}\left(\overrightarrow{[\Delta_i]A_i^{\dir{d_i}}}, [\Delta_{n}]A_{n}^{\chk}\right)
    & = \MC_{\mathit{as}}\left(\overrightarrow{[\Delta_i]A_i^{\dir{d_i}}}\right)
    \land \fv(\Delta, A_n) \subseteq \left( S \cup \synvar\left(\overrightarrow{[\Delta_i]A_i^{\dir{d_i}}}\right)\right).
  \end{align*}
\end{definition}
This definition encapsulates the idea that every `input' type variable, possibly deriving from a context $\Delta$ or a checking argument $A_n$, must be an `output' variable from $\synvar(\overrightarrow{[\Delta_i]A_i^{\dir{d_i}}})$ or, if the rule is checking, belong to the set $S$ of `input' variables in its conclusion.
This condition must be met for every tail of the argument list as well to ensure that `output' variables accessible at each argument position are from preceding arguments only, hence an inductive definition.
\begin{definition}
  An operation $o\colon \Xi \rhd [\Delta_{1}] A_{1}^{\dir{d}}, \ldots, [\Delta_n]A_{n}^{\dir{d_n}} \to A_0^{\dir{d}}$ is \emph{mode-correct} provided that
  \begin{enumerate}
    \item $d$ is $\chk$, its argument list is mode-correct w.r.t.\ $\fv(A_0)$, and the union $\fv(A_0) \cup \synvar(\overrightarrow{[\Delta_i]A_i^{\dir{d_i}}})$ contains every inhabitant of $\Xi$;
    \item $d$ is $\syn$, its argument list is mode-correct w.r.t.\ $\emptyset$, and the finite subset $\synvar(\overrightarrow{[\Delta_i]A_i^{\dir{d_i}}})$ contains every inhabitant of $\Xi$ and, particularly, $\fv(A_0)$.
  \end{enumerate}
  A bidirectional binding signature $\Omega$ is \emph{mode-correct} if its operations are all mode-correct.
\end{definition}
Finally, for a checking operation, an `input' variable of an argument could be derived from $A_0$, as these are known during type checking as an input.
Since every inhabitant of $\Xi$ can be located in either $A_0$ or synthesis variables, we can determine a concrete type for each inhabitant of $\Xi$ during type synthesis.
On the other hand, for a synthesis operation, we do not have any known variables at the onset of type synthesis, so the argument list should be mode-correct with respect to $\emptyset$.
Also, the set of synthesis variables alone should include every type variable in $\Xi$ and particularly in $A_n$.

\begin{remark}
  As this is a condition for \emph{typing rules} and not for raw terms or typing derivations, this property cannot be formalised without the concept of bidirectional binding signature presented in \Cref{sec:defs}.
  This contrasts with the properties in \Cref{sec:pre-synthesis}, which are properties that can still be expressed for individual systems even without a general definition of bidirectional type systems.
  Now that we have the notion of signature available, we can articulate mode-correctness above.
\end{remark}

\begin{lemma}[Uniqueness of the Synthesised Types]\label{thm:unique-syn}
  In a mode-correct bidirectional type system $(\Sigma, \Omega)$, the synthesised types of any two derivations
  \[
    \Gamma |-_{\Sigma, \Omega} \isTerm{t} \syn A
    \quad\text{and}\quad
    \Gamma |-_{\Sigma, \Omega} \isTerm{t} \syn B
  \]
  for the same term $t$ must be equal, i.e.\ $A = B$.
\end{lemma}
 
\begin{proof}
  
\end{proof}



\subsection{Decidability of Bidirectional Type Synthesis and Checking}\label{subsec:bidirectional-synthesis-checking}

We assume further that the set $\Xi$ of variables has a \emph{decidable equality}, rather than being an arbitrary set, so that the set $\Type_{\Sigma}(\Xi)$ of types also has a decidable equality.
This requirement for decidable equality is necessary, as type comparisons are ubiquitous throughout type synthesis.
To simplify our choice, we could just restrict $\Xi$ to any set within the family of sets $\Fin_n$ of natural numbers less than $n$, as defined by \citet{Dybjer1994}.
Given that this set has a decidable equality and the arity of a type construct is always finite, it would be a viable choice.
For clarity in presentation, however, we will continue to use named variables.

\begin{theorem}[Decidability of Bidirectional Type Synthesis and Checking] \label{thm:bidirectional-type-synthesis-checking}
  Let $(\Sigma, \Omega)$ be a mode-correct bidirectional type system.
  For every context $\Gamma$ and raw term $\isTerm{t}$ in mode $d$, 
  \begin{enumerate}
    \item if $\isTerm{t}$ is synthesising, then it is decidable whether there is a type~$A$ and a derivation of
      \[
        \Gamma |-_{\Sigma, \Omega} \isTerm{t} \syn A.
      \]
    \item if $\isTerm{t}$ is checking, then every type $A$ decides whether there is a derivation of
      \[
        \Gamma |-_{\Sigma, \Omega} \isTerm{t} \chk A.
      \]
  \end{enumerate}
\end{theorem}

\begin{proof}[Proof of {\Cref{thm:bidirectional-type-synthesis}}]
  We prove this statement by induction on the derivation of $|-_{\Sigma, \Omega} \isTerm{t}^{d}$ (instead of $\isTerm{t}$ alone).
  The first two cases \SynRule{Var} and \SynRule{Anno} are straightforward and have nothing to do with mode-correctness directly.
  The third case \ChkRule{Sub} invokes the uniqueness of synthesised types to refute the case that $\Gamma |- \isTerm{t} \syn B$ but $A \neq B$ for a given type $A$ to check.
  The last case \Rule{Op} is the interesting part of this theorem.
  \begin{description}
    \item[\SynRule{Var}:] If $\isTerm{t}$ is a variable $\isTerm{x}$ in the context $\Gamma$, then there exists a type $A$ such that $\Gamma |- \isTerm{x} \syn A$.
      Otherwise, if $\isTerm{x}$ is not in $\Gamma$ but $\Gamma |- \isTerm{x} \syn A$ is derivable, then by inversion $\isTerm{x}$ must be in $\Gamma$, a contradiction.
    \item[\SynRule{Anno}:] For a raw term of the form $\isTerm{t} \annotate A$ in synthesising mode, it is decidable that $\Gamma |- \isTerm{t} \chk A$ is derivable or not by induction hypothesis.
      \begin{enumerate}
        \item If $\Gamma |- \isTerm{t} \chk A$ is derivable, then $\Gamma |- \isTerm{t \annotate A} \syn A$ is derivable.
        \item If $\Gamma |- t \chk A$ is not derivable but $\Gamma |- \isTerm{t \annotate A} \syn$ is derivable, then by inversion $\Gamma |- t \chk A$ is derivable, thus a contradiction.
      \end{enumerate}
      
    \item[\ChkRule{Sub}:] If $t$ is in checking mode because of the subsumption rule, then by induction hypothesis it is decidable whether $\Gamma |- t \syn B$ for some $B$:
      \begin{enumerate}
        \item If $\Gamma |- t \syn B$ is derivable for some $B$, then we have to consider any type $A$ and it is decidable if $A$ is equal to $B$:
          \begin{enumerate}
            \item if $A = B$ then we have arrived a derivation of $\Gamma |- t \chk A$.
            \item if $A \neq B$ but $\Gamma |- t \chk A$ is derivable, then by inversion $\Gamma |- t \syn A$.
              However, by \Cref{thm:unique-syn}, synthesised types $A$ and $B$ must be equal, so it is a contradiction.
          \end{enumerate}
        \item if $\Gamma |- t \syn C$ is not derivable for any type $C$ but $\Gamma |- t \chk A$ is derivable, then by inversion $\Gamma |- t \syn B$ for some $B = A$, thus a contradiction.
      \end{enumerate}
      
%    \item For every mode-correct operation $o \colon \Xi \rhd [\Delta_1]A_{1}^{d_1}, \ldots, [\Delta_{n}] A^{d_n}_{n} \to A^{d}$, a context $\Gamma$, and a raw term
%      \[
%        \isTerm{\tmOp_{o}(\vec{x}_1.t_1,\ldots, \vec{x}_n.t_n)}, 
%      \]
%      with a mode $d$, it is decidable that there exists a substitution $\rho$ from $\Xi$ to $\emptyset$ and a typing derivation of
%      \[
%        \Gamma |-_{\Sigma, \Omega} \isTerm{\tmOp_{o}(\vec{x}_1.t_1,\ldots, \vec{x}_n.t_n)} : \simsub{A}{\rho}^{d}.
%      \]
    \item[\Rule{Op}:]
      Suppose that $t$ is of them $\tmOpts$ for some mode-correct operation $o$ in $\Omega$.
      No matter $t$ is checking or synthesising, we need to build derivations for its arguments incrementally and maintain a set of unifiers, i.e.\ a partial substitution $\rho\colon \PSub{\Xi}{\emptyset}$ is incrementally defined which is initially defined nowhere if $t$ is synthesising or on variables of the target $A$ if $t$ is checking. 

  \begin{claim}\label{lem:args-induction}
    For every set $\Xi$ of type variables, list 
    \[
      \mathit{as} = \left([\Delta_i] A_{i}^{\dir{d_i}}\right)_{i = 1}^{n}
    \]
    list of raw terms $t_i$'s in mode $d_i$, context $\Gamma$, and \emph{partial} substitution $\rho$ from $\Xi$ to $\emptyset$
    \begin{enumerate}
      \item either there is a minimal extension $\ext{\rho}$ of $\rho$ defined on $\synvar(\mathit{as})$ such that all of following judgements are derivable
        \[
          \Gamma, \vars{x}_\isTerm{i} : \simsub{\Delta_i}{\ext{\rho}} |-_{\Sigma, \Omega} \isTerm{t_i} \colon \simsub{A_i}{\ext{\rho}}^{\dir{d_i}}
        \]

      \item or there is no extension $\sigma$ of $\rho$ such that all $\Gamma, \vars{x}_\isTerm{i} : \simsub{\Delta_i}{\sigma} |-_{\Sigma, \Omega} \isTerm{t_i} \colon \simsub{A_i}{\sigma}^{\dir{d_i}}$ has a typing derivation. 
    \end{enumerate}
  \end{claim}
  \end{description}
\end{proof}
