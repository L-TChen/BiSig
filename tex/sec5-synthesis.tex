%!TEX root = BiSig.tex

\section{Bidirectional Type Synthesis and Checking} \label{sec:type-synthesis}
This section focuses on defining mode-correctness and deriving bidirectional type synthesis for any mode-correct bidirectional type system $(\Sigma, \Omega)$.
We start with \Cref{sec:mode-correctness} by defining mode-correctness and showing the uniqueness of synthesized types.
This uniqueness means that any two synthesised types for the same raw term $t$ under the same context $\Gamma$ have to be equal.
It will be used especially in \Cref{subsec:bidirectional-synthesis-checking} for the proof of the decidability of bidirectional type synthesis and checking.
Then, we conclude this section with the trichotomy on raw terms that has been illustrated at the beginning in \Cref{subsec:trichotomy}.

\subsection{Mode Correctness}\label{sec:mode-correctness}
As \citet{Dunfield2021} outlined, mode-correctness for a bidirectional typing rule means that 
\begin{enumerate*}
\item each `input' type variable in a premise must be an `output` variable in `earlier' premises or provided by the conclusion, if the rule is checking;
\item each `output' type variable in the conclusion should be some `output' variable in premises, if the rule is synthesis.
\end{enumerate*}
Here `input' variables refers to variables in an extension context and in the type of a premise in the checking mode.
It is important to note that the order of premises in a bidirectional typing rule also matters, since synthesis type variables are instantiated incrementally during type synthesis.

Consider the rule $\ChkRule{Abs}$ (\Cref{fig:STLC-bidirectional-typing-derivations}) as an example.
This rule is mode-correct, as the type variables $A$ and $B$ in its only premise are already provided by its conclusion $A \bto B$.
Likewise, the rule $\SynRule{App}$ for an application term $\isTerm{t\;u}$ is mode-correct because:
\begin{enumerate*}
\item the type $A \bto B$ of the first argument $t$ is synthesised, thereby ensuring type variables $A$ and $B$ must be known if successfully synthesised;
\item the type of the second argument $u$ is checked against $A$, which has been synthesised earlier;
\item as a result, the type of an application $t\;u$ can be synthesised.
\end{enumerate*}

Now let us define mode-correctness rigorously.
As we have outlined, the condition of mode-correctness for a synthesis rule is different from that of a checking rule, and the argument order also matters.
Defining the condition directly for a rule, and thus in our setting for an operation, can be somewhat intricate.
Instead, we choose to define the conditions for the argument list---more specifically, triples $\biargvec$ of an extension context $\Delta_i$, a type $A_i$, and a mode $\dir{d_i}$---pertaining to an operation, for an operation, and subsequently for a signature.
We also need some auxiliary definitions for the subset of variables of a type and of a context and the set of variables that have been synthesised:
\begin{definition}
  The finite subset\footnote{%
  There are various definitions for finite subsets of a set within Martin-L\"{o}f type theory.
  However, for our purposes, the choice among these definitions is not a matter of concern.}
  of \emph{(free) variables} of a type $A$ is denoted by $\fv(A)$.
  The subset $\fv(\Gamma)$ of variables in a context $\Gamma$ is defined by\/ $\fv(\cdot) = \emptyset$ and\/ $\fv(\Gamma, A) = \fv(\Gamma) \cup \fv(A)$.
  For an argument list $\biargvec$,\LT{Eww...} define the set of \emph{synthesis type variables} inductively as 
  \begin{align}
    \label{eq:synvar1}\synvar(\cdot)                   & = \emptyset  \\
    \label{eq:synvar2}\synvar(\biargvec, \chkbiarg[n]) & = \phantom{\fv(A_{n}) \cup {}} \synvar(\biargvec) \\
    \label{eq:synvar3}\synvar(\biargvec, \synbiarg[n]) & = \fv(A_{n}) \cup           \synvar(\biargvec).
  \end{align}
\end{definition}
This subset contains type variables of a synthesis argument and they are exactly those type variables that would be synthesised during type synthesis.

\begin{definition}\label{def:mode-correctness-args}
  The \emph{mode-correctness} $\MCas\left(\biargvec\right)$ for an argument list $\biargs$ with respect to a subset $S$ of $\Xi$ is defined as
  \begin{align}
    \label{eq:MC1} \MCas( \cdot ) & = \top \\
    \label{eq:MC3} \MCas\left(\biargvec, \chkbiarg[n]\right)
                                  & = \fv(\Delta, A_n) \subseteq \left( S \cup \synvar\left(\biargvec\right)\right) \land \MCas\left(\biargvec\right) \\
    \label{eq:MC2} \MCas\left(\biargvec, \synbiarg[n]\right) 
                                  & = \phantom{, A_n} \fv(\Delta) \subseteq \left( S \cup \synvar\left(\biargvec\right)\right) \land  \MCas\left(\biargvec\right)
  \end{align}
  where \eqref{eq:MC1} means an empty list is always mode-correct.
\end{definition}
This definition encapsulates the idea that every `input' type variable, possibly deriving from a context $\Delta$ or a checking argument $A_n$, must be an `output' variable from $\synvar(\biargvec)$ or, if the rule is checking, belong to the set $S$ of `input' variables in its conclusion.
This condition must be met for every tail of the argument list as well to ensure that `output' variables accessible at each argument position are from preceding arguments only, hence an inductive definition.
\begin{definition}\label{def:mode-correctness}
  An operation $\biop$ is \emph{mode-correct} if 
  \begin{enumerate}
    \item either $d$ is $\chk$, its argument list is mode-correct w.r.t.\ $\fv(A_0)$, and the union $\fv(A_0) \cup \synvar(\biargvec)$ contains every inhabitant of $\Xi$;
    \item or $d$ is $\syn$, its argument list is mode-correct w.r.t.\ $\emptyset$, and the finite subset $\synvar(\biargvec)$ contains every inhabitant of $\Xi$ and, particularly, $\fv(A_0)$.
  \end{enumerate}
  A bidirectional binding signature $\Omega$ is \emph{mode-correct} if its operations are all mode-correct.
\end{definition}
For a checking operation, an `input' variable of an argument could be derived from $A_0$, as these are known during type checking as an input.
Since every inhabitant of $\Xi$ can be located in either $A_0$ or synthesis variables, we can determine a concrete type for each inhabitant of $\Xi$ during type synthesis.
On the other hand, for a synthesis operation, we do not have any known variables at the onset of type synthesis, so the argument list should be mode-correct with respect to $\emptyset$.
Also, the set of synthesis variables alone should include every type variable in $\Xi$ and particularly in $A_n$.
It is also easy to check the bidirectional type system $(\Sigma_{\bto}, \Omega^{\Leftrightarrow}_{\Lambda}$) in \Cref{ex:signature-simply-typed-lambda} for simply typed $\lambda$-calculus is mode-correct according to our definition.

\begin{remark}
  Mode-correctness is fundamentally a condition for typing \emph{rules} themselves, not for derivations.
  Thus, this property cannot be formalised without treating rules as some mathematical object, such as the notion of bidirectional binding signature presented in \Cref{sec:defs}.
  This contrasts with the properties in \Cref{sec:pre-synthesis}, which can still be specified for individual systems even in the absence of a generic definition of bidirectional type systems.
\end{remark}

Now, we set out to show the uniqueness of the synthesised types for a mode-correct bidirectional type system.
For a specific system, its proof is typically a straightforward induction.
However, given the mode-correctness---which incorporates an inductive definition on the argument list---we establish our proof through induction, both on typing derivations and on the argument list:
\begin{lemma}[Uniqueness of the Synthesised Types]\label{thm:unique-syn}
  In a mode-correct bidirectional type system $(\Sigma, \Omega)$, the synthesised types of any two derivations
  \[
    \Gamma |-_{\Sigma, \Omega} \isTerm{t} \syn A
    \quad\text{and}\quad
    \Gamma |-_{\Sigma, \Omega} \isTerm{t} \syn B
  \]
  for the same term $t$ must be equal, i.e.\ $A = B$.
\end{lemma}
\begin{proof}%[Proof of \Cref{thm:unique-syn}]
  We prove the statement by induction on derivations $d_1$ and $d_2$ for $\Gamma |-_{\Sigma, \Omega} \isTerm{t} \syn A$ and $\Gamma |-_{\Sigma, \Omega} \isTerm{t} \syn B$.
  Our system is syntax-directed, so $d_1$ and $d_2$ must be derived from the same rule: 
  \begin{itemize}
    \item $\SynRule{Var}$ follows from that each variable as a raw term refers to the same variable in its context.
    \item $\SynRule{Anno}$ holds trivially, since the synthesised type $A$ is from the term $t \annotate A$ in question.
    \item $\Rule{Op}$: Recall that a derivation of\/ $\Gamma |- \tmOpts \syn A$ contains a substitution $\rho$ from the local context $\Xi$ to concrete types.
      To prove that any two typing derivations has the same synthesised type, it suffices to show that those substitutions $\rho_1$ and $\rho_2$ of $d_1$ and $d_2$, respectively, agree on variables in $\synvar\left(\biargs\right)$ so that $\simsub{A_0}{\rho_1} = \simsub{A_0}{\rho_2}$.
      We prove it by induction on the argument list:
      \begin{enumerate}
        \item For the empty list, the statement is vacuously true by~\eqref{eq:synvar1}.
        \item If $\dir{d_{i+1}}$ is checking, then the statement holds for~\eqref{eq:synvar2} by induction hypothesis.
        \item If $\dir{d_{i+1}}$ is synthesis, then $\simsub{\Delta_{i+1}}{\rho_1} = \simsub{\Delta_{i+1}}{\rho_2}$ by mode-correctness~\eqref{eq:MC2} and induction hypothesis (of the list).
          Therefore, under the same context $\Gamma, \simsub{\Delta_{i+1}}{\rho_1} = \Gamma, \simsub{\Delta_{i+1}}{\rho_2}$ the term $t_{i+1}$ must have the same synthesised type $\simsub{A_{i+1}}{\rho_1} = \simsub{A_{i+1}}{\rho_1}$ by induction hypothesis (of the typing derivation), so $\rho_1$ and $\rho_2$ agree on $\fv(A_{i+1})$ in addition to $\synvar\left(\biargs\right)$, as required for the case \eqref{eq:synvar3} in the definition of $\synvar$.
      \end{enumerate}
  \end{itemize}
\end{proof}

%Uniqueness of the synthesised types is a prevalent property in bidirectional type systems, although the specific proofs can vary depending on the constructs in the system.
%For instance, for derivations of $\Gamma |- t\;u \syn B_i$ for $i = 1, 2$ in simply typed $\lambda$-calculus, the hypothesis is applied to their sub-derivations $\Gamma |- t \syn A_i \bto B_i$ to conclude that $A_1 \bto B_1 = A_2 \bto B_2$ and thus $B_1 = B_2$.
%On the other hand, our proof is based on mode-correctness and need not consider specific sub-derivations.

\subsection{Decidability of Bidirectional Type Synthesis and Checking}\label{subsec:bidirectional-synthesis-checking}

We are here for the main technical contribution of this paper.

\begin{theorem}[Decidability of Bidirectional Type Synthesis and Checking] \label{thm:bidirectional-type-synthesis-checking}
  In a mode-correct bidirectional type system $(\Sigma, \Omega)$,
  \begin{enumerate}
    \item if\/ $\erase{\Gamma} |-_{\Sigma, \Omega} \isTerm{t}^{\syn}$, then it is decidable whether there is~$A$ such that
      \[
        \Gamma |-_{\Sigma, \Omega} \isTerm{t} \syn A;
      \]
    \item if\/ $\erase{\Gamma} |-_{\Sigma, \Omega} \isTerm{t}^{\chk}$, then it is decidable for any~$A$ whether
      \[
        \Gamma |-_{\Sigma, \Omega} \isTerm{t} \chk A.
      \]
  \end{enumerate}
\end{theorem}

The interesting part of the proof is the case for the $\Rule{Op}$ rule, so we shall give the insight first instead of jumping into the details.
Recall that a typing derivation for $\tmOpts$ contains a substitution $\rho\colon \Xi \to \Type_{\Sigma}(\emptyset)$.
The goal of type synthesis is exactly to define such a substitution $\rho$, so we have to start with a \emph{partial} substitution $\rho_0$, as an `accumulating argument', defined on $\fv(A_0)$ if $d$ is $\chk$ or otherwise nowhere.
Then, as we visit each synthesis argument $\synbiarg[i+1]$, we may be able to extend the domain of $\rho_i$ to include the synthesis variables $\fv(A_{i + 1})$ if type synthesis is successful and also that the synthesised type can be \emph{unified with $A_{i+ 1}$} and \emph{extending} $\rho_i$ to $\bar{\rho_i} = \rho_{i+1}$.
By mode-correctness, the accumulating substitution~$\rho_i$ would be defined on enough synthesis variables so that type synthesis or checking can be performed on $t_{i}$ with the context $\Gamma, \simsub{\Delta_{i+1}}{\rho_i}$ based on its mode derivation $\erase{\Gamma}, \vars{x}_i |-_{\Sigma, \Omega} t_i^{\dir{d_i}}$.
Then, if we go through every argument $t_i$ successfully, we will have a substitution $\rho_n$ defined everywhere by mode-correctness.

\begin{remark}
To make the argument above sound, it is necessary to compare types and solve a unification problem.
Hence, we assume that the set\/ $\Xi$ of variables has a decidable equality, thereby ensuring that the set $\Type_{\Sigma}(\Xi)$ of types also has a decidable equality.\footnote{%
To simplify our choice, we could simply confine $\Xi$ to any set within the family of sets $\Fin(n)$ of naturals less than~$n$, given that these sets have a decidable equality and the arity of a type construct is finite.
Indeed, in our formalisation, we adopt $\Fin(n)$ as the set of type variables in the definition of $\Type_{\Sigma}$ (see \Cref{sec:formalisation} for details).
For the sake of clarity in presentation, though, we keep using named variables and just assume that $\Xi$ has a decidable equality.}
\end{remark}
We need some auxiliary definitions for the notion of extension to state the unification problem:
\begin{definition}
  By an \emph{extension}\/ $\sigma \geq \rho$ of a partial substitution $\rho$ we mean that the domain of $\sigma$ contains the domain $\rho$ and\/ $\sigma(x) = \rho(x)$ for every\/ $x$ in the domain of $\rho$.
  By a \emph{minimal extension}\/ $\bar{\rho}$ of $\rho$ satisfying $P$ we mean an extension $\bar{\rho} \geq \rho$ with $P(\bar{\rho})$ such that $\sigma \geq \bar{\rho}$ whenever $\sigma \geq \rho$ and $P(\sigma)$.
\end{definition}
\begin{lemma}\label{lem:unify}
  For any\/ $A$ of\/ $\Type_{\Sigma}(\Xi)$, $B$ of\/ $\Type_{\Sigma}(\emptyset)$, and a partial substitution\/ $\rho \colon \Xi \to \Type_{\Sigma}(\emptyset)$, 
  \begin{enumerate}
    \item either there is a minimal extension\/ $\bar{\rho}$ of\/ $\rho$ such that\/ $\simsub{A}{\bar{\rho}} = B$, or 
    \item there is no extension\/ $\sigma$ of\/ $\rho$ such that\/ $\simsub{A}{\sigma} = B$
  \end{enumerate}
\end{lemma}
This lemma can be inferred from the correctness of first-order unification~\citep{McBride2003,McBride2003a}, or be proved directly without unification.
We are now ready for the decidability proof.

\begin{proof}[Proof of {\Cref{thm:bidirectional-type-synthesis-checking}}]
  We prove this statement by induction on the mode derivation\/ $\erase{\Gamma} |-_{\Sigma, \Omega} \isTerm{t}^{\dir{d}}$.
  The two cases \SynRule{Var} and \SynRule{Anno} are straightforward and have nothing to do with mode-correctness.
  The case \ChkRule{Sub} invokes the uniqueness of synthesised types to refute the case that $\Gamma |-_{\Sigma, \Omega} \isTerm{t} \syn B$ but $A \neq B$ for a given type $A$.
  The first three cases follow essentially the same argument provided by \citet{Wadler2022}, but we still give the arguments here for the sake of completeness.
  The last case \Rule{Op} is new and has been discussed above.
  (For brevity we omit the subscript $(\Sigma, \Omega)$.)
  \begin{itemize}
    \item \SynRule{Var}: As $\erase{\Gamma} |- \isTerm{t}$, there exists a type $A$ such that $\Gamma |- \isTerm{x} \syn A$.

    \item \SynRule{Anno}: For $\erase{\Gamma} |- (\isTerm{t} \annotate A)^{\syn}$, it is decidable whether $\Gamma |- \isTerm{t} \chk A$ by induction hypothesis.
      \begin{itemize}
        \item If $\Gamma |- \isTerm{t} \chk A$, then $\Gamma |- \isTerm{t \annotate A} \syn A$.
        \item If $\Gamma |/- t \chk A$ but $\Gamma |- \isTerm{t \annotate A} \syn$, then by inversion $\Gamma |- t \chk A$, leading to a contradiction.
      \end{itemize}
      
    \item \ChkRule{Sub}: If $\erase{\Gamma} |-t^{\chk}$ by the rule $\ChkRule{Sub}$, then $\Gamma |- t^{\syn}$ by inversion.
      By induction hypothesis, it is decidable whether $\Gamma |- t \syn B$ for some $B$:
      \begin{itemize}
        \item If $\Gamma |/- t \syn C$ for any $C$ but $\Gamma |- t \chk A$, then by inversion $\Gamma |- t \syn B$ for some $B = A$, thus a contradiction.
        \item If $\Gamma |- t \syn B$ for some $B$, then either $A = B$ or $A \neq B$ by the decidable equality on $\Type_{\Sigma}(\Xi)$: 
          \begin{itemize}
            \item if $A = B$ then we have\/ $\Gamma |- t \chk A$;
            \item if $A \neq B$ but $\Gamma |- t \chk A$, then by inversion $\Gamma |- t \syn A$.
              However, by \Cref{thm:unique-syn}, synthesised types $A$ and $B$ must be equal, so we derive a contradiction.
          \end{itemize}
      \end{itemize}
    \item \Rule{Op}:
      Consider the mode derivation of $\erase{\Gamma} |- \tmOpts^{\dir{d}}$.
      First we claim:
      \begin{claim}\label{lem:args-induction}
        For an argument list $\as = \biargs$ and any partial substitution $\rho$ from $\Xi$ to $\emptyset$
        \begin{enumerate}
          \item either there is a minimal extension $\ext{\rho}$ of $\rho$ such that 
            \begin{equation} \label{eq:claim}
              \text{$\ext{\rho}$ is defined on $\synvar(\as)$ and\/} \quad
              \Gamma, \vars{x}_\isTerm{i} : \simsub{\Delta_i}{\ext{\rho}} |- \isTerm{t_i} \colon \simsub{A_i}{\ext{\rho}}^{\dir{d_i}}
              \quad\text{for $i = 1, \ldots, n$};
            \end{equation}
          \item or there is no extension $\sigma$ of $\rho$ such that \eqref{eq:claim} holds.
        \end{enumerate}
      \end{claim}
      Then, we proceed with a case analysis on $\dir{d}$:
      \begin{itemize}
        \item $\dir{d}$ is $\syn$: We apply our claim with the partial substitution $\rho_0$ defined nowhere.
          \begin{enumerate}
            \item If there is no $\sigma \geq \rho$ such that \eqref{eq:claim} holds but $\Gamma |- \tmOpts \syn A$ for some $A$, then by inversion we have a substitution $\rho\colon \Sub{\Xi}{\emptyset}$ such that
              \[
                \Gamma, \vars{x}_\isTerm{i} : \simsub{\Delta_i}{\rho} |- \isTerm{t_i} \colon \simsub{A_i}{\rho}^{\dir{d_i}}
              \]
              for every $i$.
              Obviously, $\rho \geq \rho_0$ and ${\Gamma, \vars{x}_\isTerm{i} : \simsub{\Delta_i}{\rho} |- \isTerm{t_i} \colon \simsub{A_i}{\rho}^{\dir{d_i}}}$ for every $i$ and it contradicts the assumption that no such an extension exists.

            \item If there exists a minimal $\ext{\rho} \geq \rho_0$ defined on $\synvar(\biargs)$ such that~\eqref{eq:claim} holds, then by mode-correctness $\ext{\rho}$ is total and thus
              \[
                \Gamma |- \tmOpts \syn \simsub{A_0}{\ext{\rho}}.
              \]
          \end{enumerate}

        \item $\dir{d}$ is $\chk$: Let $A$ be a type and apply \Cref{lem:unify} with $\rho_0$ defined nowhere.
          \begin{enumerate}
            \item If there is no $\sigma \geq \rho_0$ such that $\simsub{A_0}{\sigma} = A$ but $\Gamma |- \tmOpts \chk A$, then by inversion there is a substitution $\rho$ such that $A = \simsub{A_0}{\rho}$, thus a contradiction.
            \item If there is a minimal $\ext{\rho} \geq \rho_0$ such that $\simsub{A_0}{\ext{\rho}} = A$, then apply our claim with $\ext{\rho}$:
              \begin{enumerate}
                \item If there is no $\sigma \geq \ext{\rho}$ satisfying \eqref{eq:claim} but $\Gamma |- \tmOpts \chk A$, then by inversion there is $\gamma$ such that $\simsub{A_0}{\gamma} = A$ and also ${\Gamma, \vars{x}_\isTerm{i} : \simsub{\Delta_i}{\gamma} |- \isTerm{t_i} \colon \simsub{A_i}{\gamma}^{\dir{d_i}}}$ for every $i$.
                  Given that $\ext{\rho} \geq \rho$ is minimal such that $\simsub{A_0}{\ext{\rho}} = A$, then $\gamma$ is an extension of $\ext{\rho}$ but by assumption no such an extension satisfying ${\Gamma, \vars{x}_\isTerm{i} : \simsub{\Delta_i}{\gamma} |- \isTerm{t_i} \colon \simsub{A_i}{\gamma}^{\dir{d_i}}}$ exists, thus a contradiction.
                
                \item If there is a minimal $\ext{\ext{\rho}} \geq \ext{\rho}$ such that \eqref{eq:claim}, then by mode-correctness $\ext{\ext{\rho}}$ is total and
                  \[
                    \Gamma |- \tmOpts \chk \simsub{A_0}{\ext{\ext{\rho}}}
                  \]
                  where $\simsub{A_0}{\ext{\ext{\rho}}} = \simsub{A_0}{\ext{\rho}} = A$ since $\ext{\ext{\rho}}(x) = \ext{\rho}$ for every $x$ in the domain of $\ext{\rho}$.
              \end{enumerate}
          \end{enumerate}
      \end{itemize}
      \begin{claimproof}
        We prove it by induction on the list $\biargs$:
        \begin{enumerate}
          \item For the empty list, $\rho$ is the minimal extension of $\rho$ itself satisfying \eqref{eq:claim} trivially. 
          \item For $\biargvec, \biarg[m+1]$, by induction hypothesis on the list, we have two cases:
            \begin{enumerate}
              \item If there is no $\sigma \geq \rho$ such that \eqref{eq:claim} holds for all $1 \leq i \leq m$ but a minimal $\gamma \geq \rho$ such that~\eqref{eq:claim} holds for all $1 \leq i \leq m + 1$, then we clearly have a contradiction.
              \item There is a minimal $\ext{\rho} \geq \rho$ such that \eqref{eq:claim} holds for all $1 \leq i \leq m$.
                By case analysis on $\dir{d_{m+1}}$:
                \begin{itemize}
                  \item $\dir{d_{m+1}}$ is $\chk$: By mode-correctness, $\simsub{\Delta_{m+1}}{\ext{\rho}}$ and $\simsub{A_{m+1}}{\ext{\rho}}$ are defined and by the induction hypothesis it is decidable whether $\Gamma, \vars{x}_\isTerm{m+1} : \simsub{\Delta_{m+1}}{\ext{\rho}} |- \isTerm{t_{m+1}} \chk \simsub{A_{m+1}}{\ext{\rho}}^{\dir{d_{m+1}}}$.
                    Clearly, if $\Gamma, \vars{x}_\isTerm{m+1} : \simsub{\Delta_{m+1}}{\ext{\rho}} |- \isTerm{t_{m+1}} \chk \simsub{A_{m+1}}{\ext{\rho}}^{\dir{d_{m+1}}}$ then the desired statement is proved; otherwise we can also easily derive a contradiction.

                  \item $\dir{d_{m+1}}$ is $\syn$: By mode-correctness, $\simsub{\Delta_{m+1}}{\ext{\rho}}$ is defined and by the induction hypothesis, it is decidable that $\Gamma, \vars{x}_\isTerm{m+1} : \simsub{\Delta_{m+1}}{\ext{\rho}} |- \isTerm{t_{m+1}} \syn A$ for some $A$.
                    \begin{enumerate}
                      \item If $\Gamma, \vars{x}_\isTerm{m+1} : \simsub{\Delta_{m+1}}{\ext{\rho}} |/- \isTerm{t_{m+1}} \syn A$ for any $A$ but there is $\gamma \geq \rho$ s.t.\ \eqref{eq:claim} holds for $1 \leq i \leq m+1$, then $\gamma \geq \ext{\rho}$.
                        Therefore $\simsub{\Delta_{m+1}}{\ext{\rho}} = \simsub{\Delta_{m+1}}{\gamma}$ and we derive a contradiction because $\Gamma, \vars{x}_\isTerm{m+1} : \simsub{\Delta_{m+1}}{\ext{\rho}} |- \isTerm{t_{m+1}} \syn \simsub{A_{m+1}}{\gamma}$.
                      \item If $\Gamma, \vars{x}_\isTerm{m+1} : \simsub{\Delta_{m+1}}{\ext{\rho}} |- \isTerm{t_{m+1}} \syn A$ for some $A$, then by \Cref{lem:unify} we have two cases: %we unify $A$ with $A_{m+1}$ extending $\ext{\rho}$.
                        \begin{itemize}
                          \item Suppose no $\sigma \geq \ext{\rho}$ such that $\simsub{A_{m+1}}{\sigma} = A$ but an extension $\gamma \geq \rho$ such that \eqref{eq:claim} holds for $1 \leq i \leq m + 1$. 
                            Then, $\gamma \geq \ext{\rho}$ by the minimality of $\ext{\rho}$ and thus
                            $\Gamma, \vars{x}_\isTerm{m+1} : \simsub{\Delta_{m+1}}{\ext{\rho}} |- \isTerm{t_{m+1}} \syn \simsub{A_{m+1}}{\gamma}$.
                            However, by \Cref{thm:unique-syn}, the synthesised type $\simsub{A_{m+1}}{\gamma}$ must be unique, so $\gamma$ is an extension of $\ext{\rho}$ such that $\simsub{A_{m+1}}{\gamma} = A$, leading to a contradiction.
                          \item If there is a minimal $\ext{\ext{\rho}} \geq \ext{\rho}$ such that $\simsub{A_{m+1}}{\ext{\ext{\rho}}} = A$, then it is not hard to show that $\ext{\ext{\rho}}$ is the minimal extension of $\rho$ such that \eqref{eq:claim} holds for all $1 \leq i \leq m + 1$.
                        \end{itemize}
                    \end{enumerate}
                \end{itemize}
            \end{enumerate}
        \end{enumerate}
        \hfill{\footnotesize$\blacksquare$}
      \end{claimproof}
      We now have completed the decidability proof.
  \end{itemize}
\end{proof}

The formal counterpart of the above proof in \Agda essentially functions as two programs for type checking and synthesis.
These programs either compute the typing derivation or provide a proof of contradiction.
Each case analysis simply branches depending on the outcomes of bidirectional type synthesis and checking for each subterm, as well as the unification process.
If a contradiction proof is not of interest for implementation, these programs can be simplified by disregarding the cases that yield such contradiction proofs.
Alternatively, we could consider generalising typing derivations like our generalised mode derivations (\Cref{fig:generalised-mode-derivations}).
This could accommodate ill-typed cases and improve contradiction proofs to deliver more informative error messages.
This would assist users in resolving issues with ill-typed terms, rather than simply returning a blatant 'no'.

\subsection{Trichotomy on Raw Terms by Type Synthesis} \label{subsec:trichotomy}

\begin{corollary}[Trichotomy on Raw Terms]\label{cor:trichotomy}
  For any mode-correct bidirectional type system $(\Sigma, \Omega)$, 
  exactly one of the following holds:
  \begin{enumerate}
    \item $\erase{\Gamma} |-_{\Sigma, \Omega} \isTerm{t}^{\syn}$ and\/ $\Gamma |-_{\Sigma, \Omega} \isTerm{t} : A$ for some type $A$.
    \item $\erase{\Gamma} |-_{\Sigma, \Omega} \isTerm{t}^{\syn}$ but\/ $\Gamma |/-_{\Sigma, \Omega} \isTerm{t} : A$.
    \item $\erase{\Gamma} |/-_{\Sigma, \Omega} \isTerm{t}^{\syn}$.
  \end{enumerate}
\end{corollary}
\begin{proof}
  Combine \Cref{lem:soundness-completeness,thm:mode-preprocessing} with \Cref{thm:bidirectional-type-synthesis-checking}.
  
\end{proof}
