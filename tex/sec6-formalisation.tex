%!TEX root = BiSig.tex

\documentclass[BiSig.tex]{subfiles}


%% ODER: format ==         = "\mathrel{==}"
%% ODER: format /=         = "\neq "
%
%
\makeatletter
\@ifundefined{lhs2tex.lhs2tex.sty.read}%
  {\@namedef{lhs2tex.lhs2tex.sty.read}{}%
   \newcommand\SkipToFmtEnd{}%
   \newcommand\EndFmtInput{}%
   \long\def\SkipToFmtEnd#1\EndFmtInput{}%
  }\SkipToFmtEnd

\newcommand\ReadOnlyOnce[1]{\@ifundefined{#1}{\@namedef{#1}{}}\SkipToFmtEnd}
\usepackage{amstext}
\usepackage{amssymb}
\usepackage{stmaryrd}
\DeclareFontFamily{OT1}{cmtex}{}
\DeclareFontShape{OT1}{cmtex}{m}{n}
  {<5><6><7><8>cmtex8
   <9>cmtex9
   <10><10.95><12><14.4><17.28><20.74><24.88>cmtex10}{}
\DeclareFontShape{OT1}{cmtex}{m}{it}
  {<-> ssub * cmtt/m/it}{}
\newcommand{\texfamily}{\fontfamily{cmtex}\selectfont}
\DeclareFontShape{OT1}{cmtt}{bx}{n}
  {<5><6><7><8>cmtt8
   <9>cmbtt9
   <10><10.95><12><14.4><17.28><20.74><24.88>cmbtt10}{}
\DeclareFontShape{OT1}{cmtex}{bx}{n}
  {<-> ssub * cmtt/bx/n}{}
\newcommand{\tex}[1]{\text{\texfamily#1}}	% NEU

\newcommand{\Sp}{\hskip.33334em\relax}


\newcommand{\Conid}[1]{\mathit{#1}}
\newcommand{\Varid}[1]{\mathit{#1}}
\newcommand{\anonymous}{\kern0.06em \vbox{\hrule\@width.5em}}
\newcommand{\plus}{\mathbin{+\!\!\!+}}
\newcommand{\bind}{\mathbin{>\!\!\!>\mkern-6.7mu=}}
\newcommand{\rbind}{\mathbin{=\mkern-6.7mu<\!\!\!<}}% suggested by Neil Mitchell
\newcommand{\sequ}{\mathbin{>\!\!\!>}}
\renewcommand{\leq}{\leqslant}
\renewcommand{\geq}{\geqslant}
\usepackage{polytable}

%mathindent has to be defined
\@ifundefined{mathindent}%
  {\newdimen\mathindent\mathindent\leftmargini}%
  {}%

\def\resethooks{%
  \global\let\SaveRestoreHook\empty
  \global\let\ColumnHook\empty}
\newcommand*{\savecolumns}[1][default]%
  {\g@addto@macro\SaveRestoreHook{\savecolumns[#1]}}
\newcommand*{\restorecolumns}[1][default]%
  {\g@addto@macro\SaveRestoreHook{\restorecolumns[#1]}}
\newcommand*{\aligncolumn}[2]%
  {\g@addto@macro\ColumnHook{\column{#1}{#2}}}

\resethooks

\newcommand{\onelinecommentchars}{\quad-{}- }
\newcommand{\commentbeginchars}{\enskip\{-}
\newcommand{\commentendchars}{-\}\enskip}

\newcommand{\visiblecomments}{%
  \let\onelinecomment=\onelinecommentchars
  \let\commentbegin=\commentbeginchars
  \let\commentend=\commentendchars}

\newcommand{\invisiblecomments}{%
  \let\onelinecomment=\empty
  \let\commentbegin=\empty
  \let\commentend=\empty}

\visiblecomments

\newlength{\blanklineskip}
\setlength{\blanklineskip}{0.66084ex}

\newcommand{\hsindent}[1]{\quad}% default is fixed indentation
\let\hspre\empty
\let\hspost\empty
\newcommand{\NB}{\textbf{NB}}
\newcommand{\Todo}[1]{$\langle$\textbf{To do:}~#1$\rangle$}

\EndFmtInput
\makeatother
%
%
%
%
%
%
% This package provides two environments suitable to take the place
% of hscode, called "plainhscode" and "arrayhscode". 
%
% The plain environment surrounds each code block by vertical space,
% and it uses \abovedisplayskip and \belowdisplayskip to get spacing
% similar to formulas. Note that if these dimensions are changed,
% the spacing around displayed math formulas changes as well.
% All code is indented using \leftskip.
%
% Changed 19.08.2004 to reflect changes in colorcode. Should work with
% CodeGroup.sty.
%
\ReadOnlyOnce{polycode.fmt}%
\makeatletter

\newcommand{\hsnewpar}[1]%
  {{\parskip=0pt\parindent=0pt\par\vskip #1\noindent}}

% can be used, for instance, to redefine the code size, by setting the
% command to \small or something alike
\newcommand{\hscodestyle}{}

% The command \sethscode can be used to switch the code formatting
% behaviour by mapping the hscode environment in the subst directive
% to a new LaTeX environment.

\newcommand{\sethscode}[1]%
  {\expandafter\let\expandafter\hscode\csname #1\endcsname
   \expandafter\let\expandafter\endhscode\csname end#1\endcsname}

% "compatibility" mode restores the non-polycode.fmt layout.

\newenvironment{compathscode}%
  {\par\noindent
   \advance\leftskip\mathindent
   \hscodestyle
   \let\\=\@normalcr
   \let\hspre\(\let\hspost\)%
   \pboxed}%
  {\endpboxed\)%
   \par\noindent
   \ignorespacesafterend}

\newcommand{\compaths}{\sethscode{compathscode}}

% "plain" mode is the proposed default.
% It should now work with \centering.
% This required some changes. The old version
% is still available for reference as oldplainhscode.

\newenvironment{plainhscode}%
  {\hsnewpar\abovedisplayskip
   \advance\leftskip\mathindent
   \hscodestyle
   \let\hspre\(\let\hspost\)%
   \pboxed}%
  {\endpboxed%
   \hsnewpar\belowdisplayskip
   \ignorespacesafterend}

\newenvironment{oldplainhscode}%
  {\hsnewpar\abovedisplayskip
   \advance\leftskip\mathindent
   \hscodestyle
   \let\\=\@normalcr
   \(\pboxed}%
  {\endpboxed\)%
   \hsnewpar\belowdisplayskip
   \ignorespacesafterend}

% Here, we make plainhscode the default environment.

\newcommand{\plainhs}{\sethscode{plainhscode}}
\newcommand{\oldplainhs}{\sethscode{oldplainhscode}}
\plainhs

% The arrayhscode is like plain, but makes use of polytable's
% parray environment which disallows page breaks in code blocks.

\newenvironment{arrayhscode}%
  {\hsnewpar\abovedisplayskip
   \advance\leftskip\mathindent
   \hscodestyle
   \let\\=\@normalcr
   \(\parray}%
  {\endparray\)%
   \hsnewpar\belowdisplayskip
   \ignorespacesafterend}

\newcommand{\arrayhs}{\sethscode{arrayhscode}}

% The mathhscode environment also makes use of polytable's parray 
% environment. It is supposed to be used only inside math mode 
% (I used it to typeset the type rules in my thesis).

\newenvironment{mathhscode}%
  {\parray}{\endparray}

\newcommand{\mathhs}{\sethscode{mathhscode}}

% texths is similar to mathhs, but works in text mode.

\newenvironment{texthscode}%
  {\(\parray}{\endparray\)}

\newcommand{\texths}{\sethscode{texthscode}}

% The framed environment places code in a framed box.

\def\codeframewidth{\arrayrulewidth}
\RequirePackage{calc}

\newenvironment{framedhscode}%
  {\parskip=\abovedisplayskip\par\noindent
   \hscodestyle
   \arrayrulewidth=\codeframewidth
   \tabular{@{}|p{\linewidth-2\arraycolsep-2\arrayrulewidth-2pt}|@{}}%
   \hline\framedhslinecorrect\\{-1.5ex}%
   \let\endoflinesave=\\
   \let\\=\@normalcr
   \(\pboxed}%
  {\endpboxed\)%
   \framedhslinecorrect\endoflinesave{.5ex}\hline
   \endtabular
   \parskip=\belowdisplayskip\par\noindent
   \ignorespacesafterend}

\newcommand{\framedhslinecorrect}[2]%
  {#1[#2]}

\newcommand{\framedhs}{\sethscode{framedhscode}}

% The inlinehscode environment is an experimental environment
% that can be used to typeset displayed code inline.

\newenvironment{inlinehscode}%
  {\(\def\column##1##2{}%
   \let\>\undefined\let\<\undefined\let\\\undefined
   \newcommand\>[1][]{}\newcommand\<[1][]{}\newcommand\\[1][]{}%
   \def\fromto##1##2##3{##3}%
   \def\nextline{}}{\) }%

\newcommand{\inlinehs}{\sethscode{inlinehscode}}

% The joincode environment is a separate environment that
% can be used to surround and thereby connect multiple code
% blocks.

\newenvironment{joincode}%
  {\let\orighscode=\hscode
   \let\origendhscode=\endhscode
   \def\endhscode{\def\hscode{\endgroup\def\@currenvir{hscode}\\}\begingroup}
   %\let\SaveRestoreHook=\empty
   %\let\ColumnHook=\empty
   %\let\resethooks=\empty
   \orighscode\def\hscode{\endgroup\def\@currenvir{hscode}}}%
  {\origendhscode
   \global\let\hscode=\orighscode
   \global\let\endhscode=\origendhscode}%

\makeatother
\EndFmtInput
%
%
\ReadOnlyOnce{agda.fmt}%


\RequirePackage[T1]{fontenc}
\RequirePackage[utf8]{inputenc}
\RequirePackage{amsfonts}

\providecommand\mathbbm{\mathbb}

% TODO: Define more of these ...
\DeclareUnicodeCharacter{737}{\textsuperscript{l}}
\DeclareUnicodeCharacter{8718}{\ensuremath{\blacksquare}}
\DeclareUnicodeCharacter{8759}{::}
\DeclareUnicodeCharacter{9669}{\ensuremath{\triangleleft}}
\DeclareUnicodeCharacter{8799}{\ensuremath{\stackrel{\scriptscriptstyle ?}{=}}}
\DeclareUnicodeCharacter{10214}{\ensuremath{\llbracket}}
\DeclareUnicodeCharacter{10215}{\ensuremath{\rrbracket}}
\DeclareUnicodeCharacter{27E6}{\ensuremath{\llbracket}}
\DeclareUnicodeCharacter{27E7}{\ensuremath{\rrbracket}}
\DeclareUnicodeCharacter{2200}{\ensuremath{\forall}}

\DeclareUnicodeCharacter{2294}{\ensuremath{\sqcup}}

\DeclareUnicodeCharacter{1D43}{\ensuremath{^a}}
\DeclareUnicodeCharacter{1D9C}{\ensuremath{^c}}
\DeclareUnicodeCharacter{02B3}{\ensuremath{^r}}
\DeclareUnicodeCharacter{02E2}{\ensuremath{^s}}

\DeclareUnicodeCharacter{2080}{\ensuremath{_0}}
\DeclareUnicodeCharacter{2081}{\ensuremath{_1}}
\DeclareUnicodeCharacter{2082}{\ensuremath{_2}}
\DeclareUnicodeCharacter{2083}{\ensuremath{_3}}
\DeclareUnicodeCharacter{2084}{\ensuremath{_4}}

\DeclareUnicodeCharacter{2115}{\ensuremath{\mathbb{N}}}
\DeclareUnicodeCharacter{2208}{\ensuremath{\in}}
\DeclareUnicodeCharacter{2236}{:}
\DeclareUnicodeCharacter{2261}{\ensuremath{\equiv}}
\DeclareUnicodeCharacter{2237}{\ensuremath{\mathrel{::}}}
\DeclareUnicodeCharacter{2982}{\ensuremath{\bbcolon}}

\DeclareUnicodeCharacter{0393}{\ensuremath{\Gamma}}
\DeclareUnicodeCharacter{0394}{\ensuremath{\Delta}}
\DeclareUnicodeCharacter{0398}{\ensuremath{\Theta}}
\DeclareUnicodeCharacter{03A3}{\ensuremath{\Sigma}}
\DeclareUnicodeCharacter{039B}{\ensuremath{\Lambda}}
\DeclareUnicodeCharacter{039E}{\ensuremath{\Xi}}

\DeclareUnicodeCharacter{03B9}{\ensuremath{\iota}}
\DeclareUnicodeCharacter{03BB}{\ensuremath{\lambda}}
\DeclareUnicodeCharacter{03C0}{\ensuremath{\pi}}
\DeclareUnicodeCharacter{03C3}{\ensuremath{\sigma}}
\DeclareUnicodeCharacter{03C9}{\ensuremath{\omega}}

\DeclareUnicodeCharacter{2032}{\ensuremath{{}^\prime}}
\DeclareUnicodeCharacter{2113}{\ensuremath{\ell}}
\DeclareUnicodeCharacter{2207}{\ensuremath{\nabla}}
\DeclareUnicodeCharacter{220B}{\ensuremath{\ni}}
\DeclareUnicodeCharacter{2264}{\ensuremath{\leq}}
\DeclareUnicodeCharacter{21D0}{\ensuremath{\Leftarrow}}
\DeclareUnicodeCharacter{21D2}{\ensuremath{\Rightarrow}}
\DeclareUnicodeCharacter{22A2}{\ensuremath{\vdash}}
\DeclareUnicodeCharacter{22A4}{\ensuremath{\top}}
\DeclareUnicodeCharacter{22A5}{\ensuremath{\bot}}

\DeclareUnicodeCharacter{1D57}{\ensuremath{^t}}

% TODO: This is in general not a good idea.
\providecommand\textepsilon{$\epsilon$}
\providecommand\textmu{$\mu$}


%Actually, varsyms should not occur in Agda output.

% TODO: Make this configurable. IMHO, italics doesn't work well
% for Agda code.

\renewcommand\Varid[1]{\mathord{\textsf{#1}}}
\let\Conid\Varid
\newcommand\Keyword[1]{\textsf{\textbf{#1}}}

\EndFmtInput




\begin{document}

\section{Formalisation} \label{sec:formalisation}
As we have mentioned in \Cref{sec:intro}, our theory was initially developed with \Agda using \AxiomK and has been later translated into the mathematical vernacular.
While the translation is reasonably straightforward, understanding the design of the formalisation itself could pose some difficulty.
If the reader is already comfortable with the informal theory presented so far and assured by the existence of its formalisation, they may choose to skip this section.

\paragraph{Revisiting language formalisation frameworks}
Unlike prior frameworks~\citep{Allais2021,Fiore2022,Ahrens2022} that have primarily focused on meta-properties centred around substitution and term traversal for intrinsically-typed terms, our theory of bidirectional type synthesis does not require term substitution but structural induction for extrinsically-typed terms.
The formal definitions of extrinsic typing in bidirectional type systems, including bidirectional typing derivations and mode derivations, are more complex than their intrinsic counterparts.
For a specific language such as \PCF, \citet{Wadler2022} noted that `extrinsically-typed terms require about 1.6 times as much code as intrinsically-typed' for their formalisation of type safety.
Take the formal type-theoretic definition of typing derivations $\Gamma \vdash_{\Sigma, \Omega} \isTerm{t} : A$ (\Cref{fig:extrinsic-typing}) as an example.
Its intrinsic definition consists of of just one family of sets of intrinsically-typed terms indexed by $A$ and $\Gamma$, but its extrinsic counterpart is a generic family of sets indexed by additionally a generic raw term $t$, involving constructions of two different layers.

\paragraph{Category-theoretic analysis of well-typed terms}
\citet{Fiore1999}{'s} theory of abstract syntax and variable binding forms the foundation of \citet{Fiore2022}'s framework and inspires frameworks by \citet{Allais2021,Ahrens2022}. 
We may sketch the idea of their analysis as follows.
The set of (untyped) abstract syntax trees for a language can be understood as
\begin{enumerate*}
  \item a family of sets $\Term_{\Gamma}$ of well-scoped terms under a context~$\Gamma$ with
  \item variable renaming for a function $\sigma\colon \Gamma \to \Delta$ between variables acting as a functorial map from $\Term_{\Gamma}$ to $\Term_{\Delta}$, i.e.\ a presheaf $\Term\colon \mathbb{F} \to \mathsf{Set}$, and
  \item an \emph{initial} algebra $[\mathsf{v}, \mathsf{op}]$ on~$\Term$ given by the variable rule as a map $\mathsf{v}$ from the presheaf $V\colon \mathbb{F} \hookrightarrow \mathsf{Set}$ of variables to $\Term$ and other constructs as $\mathsf{op}\colon \mathbb{\Sigma}\Term \to \Term$ where $\mathbb{F}$ is the category of contexts, the functor $\mathbb{\Sigma}\colon \mathsf{Set}^\mathbb{F} \to \mathsf{Set}^\mathbb{F}$ encodes the binding arities of constructs, and the initiality amounts to structural recursion, i.e.\ \emph{term traversal}.
\end{enumerate*}
Substitution is modelled as some monoid multiplication with the notion of strength and a suitable monoidal structural on the category of presheaves.
To put it succinctly, it is the free $\mathbb{\Sigma}$-monoid over the presheaf~$V$ of variables.

\paragraph{Type-theoretic construction of well-typed terms}
Fortunately, in type theory, constructing the initial algebra of well-typed terms boils down to defining an inductive type with a few constructors that align primitive rules (such as $\Rule{Var}$) and a rule schema; term traversal can be defined by as usual pattern matching for the inductive type~\citep{Fiore2022}.

\paragraph{Whither a theory of extrinsic typing?}
In \citep{Fiore1999} or existing frameworks, extrinsic typing is not studied.
Our formalisation is inspired by \varcitet{Hermida1998}{'s} interpretation of structural induction as algebras on the category of predicates, viewing extrinsic definitions as endofunctors on the category of predicates.
However, a category-theoretic analysis of our formal constructions in line with \varcitet{Fiore1999}{'s} theory requires proficiency in category theory, which falls outside the scope of this paper.

As such, in this section, we will merely provide an overview of our design and explain the construction of simple types in \Cref{subsec:formal-simple-types} and bidirectional typing rules in \Cref{subsec:formal-extrinsic-typing} intuitively.
On the other hand, the formal proofs closely mirror their informal description presented so far, thus we will only illustrate the `if' part of \Cref{lem:soundness-completeness} as an example in \Cref{subsec:formal-proofs}. 

\subsection{Defining Simple Types}\label{subsec:formal-simple-types}

We begin the formal definition of signatures and simple types (\Cref{def:simple-signature}):
\begin{figure}[H]
  \small
  \begin{minipage}[t]{.45\textwidth}
    \begin{hscode}\SaveRestoreHook
\column{B}{@{}>{\hspre}l<{\hspost}@{}}%
\column{3}{@{}>{\hspre}l<{\hspost}@{}}%
\column{5}{@{}>{\hspre}l<{\hspost}@{}}%
\column{16}{@{}>{\hspre}l<{\hspost}@{}}%
\column{19}{@{}>{\hspre}l<{\hspost}@{}}%
\column{E}{@{}>{\hspre}l<{\hspost}@{}}%
\>[B]{}\Keyword{record}\;\Conid{Desc}\;\mathbin{:}\;\Conid{Set₁}\;\Keyword{where}{}\<[E]%
\\
\>[B]{}\hsindent{3}{}\<[3]%
\>[3]{}\Keyword{field}{}\<[E]%
\\
\>[3]{}\hsindent{2}{}\<[5]%
\>[5]{}\Conid{Op}\;{}\<[16]%
\>[16]{}\mathbin{:}\;{}\<[19]%
\>[19]{}\Conid{Set}{}\<[E]%
\\
\>[3]{}\hsindent{2}{}\<[5]%
\>[5]{}\{\kern-.9pt\vrule width .75pt height 7.125pt depth 1.975pt\kern-1.5pt\;\Varid{decEq}\;\kern-1.5pt\vrule width .75pt height 7.125pt depth 1.975pt\kern-.9pt\}\;{}\<[16]%
\>[16]{}\mathbin{:}\;{}\<[19]%
\>[19]{}\Conid{DecEq}\;\Conid{Op}{}\<[E]%
\\
\>[3]{}\hsindent{2}{}\<[5]%
\>[5]{}\Varid{rules}\;{}\<[16]%
\>[16]{}\mathbin{:}\;{}\<[19]%
\>[19]{}\Conid{Op}\;\Varid{→}\;\Conid{ℕ}{}\<[E]%
\ColumnHook
\end{hscode}\resethooks
  \end{minipage}
  \begin{minipage}[t]{.45\textwidth}
    \begin{hscode}\SaveRestoreHook
\column{B}{@{}>{\hspre}l<{\hspost}@{}}%
\column{3}{@{}>{\hspre}l<{\hspost}@{}}%
\column{8}{@{}>{\hspre}l<{\hspost}@{}}%
\column{24}{@{}>{\hspre}l<{\hspost}@{}}%
\column{E}{@{}>{\hspre}l<{\hspost}@{}}%
\>[B]{}\Varid{⟦\char95 ⟧}\;\mathbin{:}\;\Conid{Desc}\;\Varid{→}\;\Conid{Set}\;\Varid{→}\;\Conid{Set}{}\<[E]%
\\
\>[B]{}\Varid{⟦}\;\Conid{D}\;\Varid{⟧}\;\Conid{X}\;\mathrel{=}\;\Conid{Σ[}\;\Varid{i}\;\Varid{∈}\;\Conid{D}\;\Varid{.Op}\;\mskip1.5mu]\;\Conid{X}\;\hat{}\;(\Conid{D}\;\Varid{.rules}\;\Varid{i}){}\<[E]%
\\[\blanklineskip]%
\>[B]{}\Keyword{data}\;\Conid{Ty}\;(\Conid{Ξ}\;\mathbin{:}\;\Conid{ℕ})\;\mathbin{:}\;\Conid{Set}\;\Keyword{where}{}\<[E]%
\\
\>[B]{}\hsindent{3}{}\<[3]%
\>[3]{}\text{\textasciigrave}\anonymous \;{}\<[8]%
\>[8]{}\mathbin{:}\;\Conid{Fin}\;\Conid{Ξ}\;{}\<[24]%
\>[24]{}\Varid{→}\;\Conid{Ty}\;\Conid{Ξ}{}\<[E]%
\\
\>[B]{}\hsindent{3}{}\<[3]%
\>[3]{}\Varid{op}\;{}\<[8]%
\>[8]{}\mathbin{:}\;\Varid{⟦}\;\Conid{D}\;\Varid{⟧}\;(\Conid{Ty}\;\Conid{Ξ})\;{}\<[24]%
\>[24]{}\Varid{→}\;\Conid{Ty}\;\Conid{Ξ}{}\<[E]%
\ColumnHook
\end{hscode}\resethooks
  \end{minipage}
\end{figure}
This definition mirrors our informal counterpart, with the exception of \ensuremath{\Varid{⟦}\;\Conid{D}\;\Varid{⟧}} for a signature~\ensuremath{\Conid{D}}.
As we are defining simple types inductively, the arity \ensuremath{(\Varid{rules}\;\Varid{o})} is intended to denote the number of \ensuremath{\Conid{Ty}} arguments, referring back to what we are in the process of defining.
To circumvent this self-reference, we first define \ensuremath{\Varid{⟦}\;\Conid{D}\;\Varid{⟧}} on an arbitrary \ensuremath{\Conid{Set}} rather than \ensuremath{\Conid{Ty}} itself.
Consequently, the inductive type of types specified by a signature can be defined using two constructors, \ensuremath{\text{\textasciigrave}\anonymous } and \ensuremath{\Varid{op}}.
A variable is represented by \ensuremath{\text{\textasciigrave}\Varid{n}} for some inhabitant \ensuremath{\Varid{n}} of the type \ensuremath{\Conid{Fin}\;\Conid{Ξ}} of natural numbers less than~\ensuremath{\Conid{Ξ}}.
Each inhabitant \ensuremath{\Varid{op}\;(\Varid{i}\;\Varid{,}\;\Varid{ts})} consists of an operation symbol \ensuremath{\Varid{i}} and a $n$-tuple \ensuremath{\Varid{ts}}.

From a categorical perspective, \ensuremath{\Varid{⟦}\;\Conid{D}\;\Varid{⟧}} can be understood as the functor from \ensuremath{\Conid{Set}} to \ensuremath{\Conid{Set}} which maps $X$ to a $O$-indexed coproduct $\sum_{i \in O} X ^ {\arity(o)}$ of products.
The type \ensuremath{\Conid{Ty}} is then the free \ensuremath{\Varid{⟦}\;\Conid{D}\;\Varid{⟧}}-algebra over the type \ensuremath{\Conid{Fin}\;\Conid{Ξ}} or the initial algebra for the functor \ensuremath{\Conid{Fin}\;\Conid{Ξ}\;\Varid{+}\;\Varid{⟦}\;\Conid{D}\;\Varid{⟧}}.

In short, to define terms by a signature, we define additionally a \emph{signature functor} \ensuremath{\Varid{⟦}\;\Conid{D}\;\Varid{⟧}}.

\subsection{Defining Raw Terms}\label{subsec:formal-extrinsic-typing}

The following record types represent the type \ensuremath{\Conid{ArgD}} for arguments $[\Delta]A^{\dir{d}}$, the type \ensuremath{\Conid{ConD}} for operations $\biop$, and the type \ensuremath{\Conid{Desc}} for bidirectional binding signatures, respectively, verbatim.
Removing \ensuremath{\Varid{mode}} from these definitions recover the definitions of binding arguments, arities, and signatures.
\begin{figure}[H]
  \small
  \begin{minipage}[t]{.3\textwidth}
  \begin{hscode}\SaveRestoreHook
\column{B}{@{}>{\hspre}l<{\hspost}@{}}%
\column{3}{@{}>{\hspre}l<{\hspost}@{}}%
\column{5}{@{}>{\hspre}l<{\hspost}@{}}%
\column{19}{@{}>{\hspre}l<{\hspost}@{}}%
\column{E}{@{}>{\hspre}l<{\hspost}@{}}%
\>[B]{}\Keyword{record}\;\Conid{ArgD}\;(\Conid{Ξ}\;\mathbin{:}\;\Conid{ℕ})\;\mathbin{:}\;\Conid{Set}{}\<[E]%
\\
\>[B]{}\hsindent{3}{}\<[3]%
\>[3]{}\Keyword{where}{}\<[E]%
\\
\>[B]{}\hsindent{3}{}\<[3]%
\>[3]{}\Keyword{field}{}\<[E]%
\\
\>[3]{}\hsindent{2}{}\<[5]%
\>[5]{}\Varid{cxt}\;{}\<[19]%
\>[19]{}\mathbin{:}\;\Conid{Cxt}\;\Conid{Ξ}{}\<[E]%
\\
\>[3]{}\hsindent{2}{}\<[5]%
\>[5]{}\Varid{type}\;{}\<[19]%
\>[19]{}\mathbin{:}\;\Conid{Ty}\;\Conid{Ξ}{}\<[E]%
\\
\>[3]{}\hsindent{2}{}\<[5]%
\>[5]{}\dir{\Varid{mode}}\;{}\<[19]%
\>[19]{}\mathbin{:}\;\Conid{Mode}{}\<[E]%
\ColumnHook
\end{hscode}\resethooks
  \end{minipage}
  \begin{minipage}[t]{.33\textwidth}
  \begin{hscode}\SaveRestoreHook
\column{B}{@{}>{\hspre}l<{\hspost}@{}}%
\column{3}{@{}>{\hspre}l<{\hspost}@{}}%
\column{5}{@{}>{\hspre}l<{\hspost}@{}}%
\column{18}{@{}>{\hspre}l<{\hspost}@{}}%
\column{E}{@{}>{\hspre}l<{\hspost}@{}}%
\>[B]{}\Keyword{record}\;\Conid{ConD}\;\mathbin{:}\;\Conid{Set}\;\Keyword{where}{}\<[E]%
\\
\>[B]{}\hsindent{3}{}\<[3]%
\>[3]{}\Keyword{constructor}\;\Varid{ι}{}\<[E]%
\\
\>[B]{}\hsindent{3}{}\<[3]%
\>[3]{}\Keyword{field}{}\<[E]%
\\
\>[3]{}\hsindent{2}{}\<[5]%
\>[5]{}\{\mskip1.5mu \Varid{vars}\mskip1.5mu\}\;{}\<[18]%
\>[18]{}\mathbin{:}\;\Conid{ℕ}{}\<[E]%
\\
\>[3]{}\hsindent{2}{}\<[5]%
\>[5]{}\Varid{args}\;{}\<[18]%
\>[18]{}\mathbin{:}\;\Conid{List}\;(\Conid{ArgD}\;\Varid{vars}){}\<[E]%
\\
\>[3]{}\hsindent{2}{}\<[5]%
\>[5]{}\Varid{type}\;{}\<[18]%
\>[18]{}\mathbin{:}\;\Conid{Ty}\;\Varid{vars}{}\<[E]%
\\
\>[3]{}\hsindent{2}{}\<[5]%
\>[5]{}\dir{\Varid{mode}}\;{}\<[18]%
\>[18]{}\mathbin{:}\;\Conid{Mode}{}\<[E]%
\ColumnHook
\end{hscode}\resethooks
  \end{minipage}
  \begin{minipage}[t]{.3\textwidth}
   \begin{hscode}\SaveRestoreHook
\column{B}{@{}>{\hspre}l<{\hspost}@{}}%
\column{3}{@{}>{\hspre}l<{\hspost}@{}}%
\column{5}{@{}>{\hspre}l<{\hspost}@{}}%
\column{17}{@{}>{\hspre}l<{\hspost}@{}}%
\column{E}{@{}>{\hspre}l<{\hspost}@{}}%
\>[B]{}\Keyword{record}\;\Conid{Desc}\;\mathbin{:}\;\Conid{Set₁}\;\Keyword{where}{}\<[E]%
\\
\>[B]{}\hsindent{3}{}\<[3]%
\>[3]{}\Keyword{field}{}\<[E]%
\\
\>[3]{}\hsindent{2}{}\<[5]%
\>[5]{}\Conid{Op}\;{}\<[17]%
\>[17]{}\mathbin{:}\;\Conid{Set}{}\<[E]%
\\
\>[3]{}\hsindent{2}{}\<[5]%
\>[5]{}\Varid{rules}\;{}\<[17]%
\>[17]{}\mathbin{:}\;\Conid{Op}\;\Varid{→}\;\Conid{ConD}{}\<[E]%
\ColumnHook
\end{hscode}\resethooks
  \end{minipage}
\end{figure}
Our raw terms, being well-scoped and indexed by a list $V$ of free variables, requires another definition of a signature functor for their construction.
This involves mapping \ensuremath{\Conid{Fam}} to \ensuremath{\Conid{Fam}}, where \ensuremath{\Conid{Fam}\;\mathrel{=}\;\Conid{ℕ}\;\Varid{→}\;\Conid{Set}} represents the family of sets indexed by a number (representing the number of free variables).
The signature endofunctor for constructing raw terms shares similarities with the signature functor used for defining simple types, but rather than using a single functor, we employ four distinct functors that are defined separately for the extension context, arguments, constructs, and the coproduct indexed by constructs.
Each of these corresponds to a variant of \ensuremath{\Varid{⟦\char95 ⟧}}:
\begin{figure}[H]
  \small 
  \begin{minipage}[t]{.52\textwidth}
\begin{hscode}\SaveRestoreHook
\column{B}{@{}>{\hspre}l<{\hspost}@{}}%
\column{8}{@{}>{\hspre}l<{\hspost}@{}}%
\column{17}{@{}>{\hspre}l<{\hspost}@{}}%
\column{19}{@{}>{\hspre}l<{\hspost}@{}}%
\column{22}{@{}>{\hspre}l<{\hspost}@{}}%
\column{27}{@{}>{\hspre}l<{\hspost}@{}}%
\column{E}{@{}>{\hspre}l<{\hspost}@{}}%
\>[B]{}\Varid{⟦\char95 ⟧ᵃ}\;{}\<[8]%
\>[8]{}\mathbin{:}\;\Conid{TExps}\;\Conid{Ξ}\;{}\<[19]%
\>[19]{}\Varid{→}\;\Conid{Fam}\;\Varid{→}\;\Conid{Fam}{}\<[E]%
\\
\>[B]{}\Varid{⟦}\;\Conid{Δ}\;{}\<[17]%
\>[17]{}\Varid{⟧ᵃ}\;{}\<[22]%
\>[22]{}\Conid{X}\;\Varid{n}\;{}\<[27]%
\>[27]{}\mathrel{=}\;\Conid{X}\;(\Varid{length}\;\Conid{Δ}\;\Varid{ʳ+}\;\Varid{n}){}\<[E]%
\\[\blanklineskip]%
\>[B]{}\Varid{⟦\char95 ⟧ᵃˢ}\;{}\<[8]%
\>[8]{}\mathbin{:}\;\Conid{ArgsD}\;\Conid{Ξ}\;{}\<[19]%
\>[19]{}\Varid{→}\;\Conid{Fam}\;\Varid{→}\;\Conid{Fam}{}\<[E]%
\\
\>[B]{}\Varid{⟦}\;\Varid{[]}\;{}\<[17]%
\>[17]{}\Varid{⟧ᵃˢ}\;{}\<[22]%
\>[22]{}\anonymous \;\anonymous \;{}\<[27]%
\>[27]{}\mathrel{=}\;\Varid{⊤}{}\<[E]%
\\
\>[B]{}\Varid{⟦}\;(\Conid{Δ}\;\Varid{⊢}\;\Conid{A})\;\Varid{∷}\;\Conid{Ds}\;{}\<[17]%
\>[17]{}\Varid{⟧ᵃˢ}\;{}\<[22]%
\>[22]{}\Conid{X}\;\Varid{n}\;{}\<[27]%
\>[27]{}\mathrel{=}\;\Varid{⟦}\;\Conid{Δ}\;\Varid{⟧ᵃ}\;\Conid{X}\;\Varid{n}\;\Varid{×}\;\Varid{⟦}\;\Conid{Ds}\;\Varid{⟧ᵃˢ}\;\Conid{X}\;\Varid{n}{}\<[E]%
\ColumnHook
\end{hscode}\resethooks
  \end{minipage}
  \begin{minipage}[t]{.47\textwidth}
\begin{hscode}\SaveRestoreHook
\column{B}{@{}>{\hspre}l<{\hspost}@{}}%
\column{8}{@{}>{\hspre}l<{\hspost}@{}}%
\column{10}{@{}>{\hspre}l<{\hspost}@{}}%
\column{14}{@{}>{\hspre}l<{\hspost}@{}}%
\column{16}{@{}>{\hspre}l<{\hspost}@{}}%
\column{19}{@{}>{\hspre}l<{\hspost}@{}}%
\column{E}{@{}>{\hspre}l<{\hspost}@{}}%
\>[B]{}\Varid{⟦\char95 ⟧ᶜ}\;{}\<[8]%
\>[8]{}\mathbin{:}\;\Conid{ConD}\;{}\<[16]%
\>[16]{}\Varid{→}\;\Conid{Fam}\;\Varid{→}\;\Conid{Fam}{}\<[E]%
\\
\>[B]{}\Varid{⟦}\;\Varid{ι}\;\Conid{D}\;\anonymous \;{}\<[10]%
\>[10]{}\Varid{⟧ᶜ}\;{}\<[14]%
\>[14]{}\Conid{X}\;{}\<[19]%
\>[19]{}\mathrel{=}\;\Varid{⟦}\;\Conid{D}\;\Varid{⟧ᵃˢ}\;\Conid{X}{}\<[E]%
\\[\blanklineskip]%
\>[B]{}\Varid{⟦\char95 ⟧}\;{}\<[8]%
\>[8]{}\mathbin{:}\;\Conid{Desc}\;{}\<[16]%
\>[16]{}\Varid{→}\;\Conid{Fam}\;\Varid{→}\;\Conid{Fam}{}\<[E]%
\\
\>[B]{}\Varid{⟦}\;\Conid{D}\;{}\<[10]%
\>[10]{}\Varid{⟧}\;{}\<[14]%
\>[14]{}\Conid{X}\;\Varid{n}\;{}\<[19]%
\>[19]{}\mathrel{=}\;\Conid{Σ[}\;\Varid{i}\;\Varid{∈}\;\Conid{D}\;\Varid{.Op}\;\mskip1.5mu]\;\Varid{⟦}\;\Conid{D}\;\Varid{.rules}\;\Varid{i}\;\Varid{⟧ᶜ}\;\Conid{X}\;\Varid{n}{}\<[E]%
\ColumnHook
\end{hscode}\resethooks
  \end{minipage}
\end{figure}
The inductive type of raw terms indexed by a list of free variables can now be defined by 
\begin{figure}[H]
  \small
\begin{hscode}\SaveRestoreHook
\column{B}{@{}>{\hspre}l<{\hspost}@{}}%
\column{3}{@{}>{\hspre}l<{\hspost}@{}}%
\column{8}{@{}>{\hspre}l<{\hspost}@{}}%
\column{23}{@{}>{\hspre}l<{\hspost}@{}}%
\column{33}{@{}>{\hspre}l<{\hspost}@{}}%
\column{E}{@{}>{\hspre}l<{\hspost}@{}}%
\>[B]{}\Keyword{data}\;\Conid{Raw}\;\mathbin{:}\;\Conid{ℕ}\;\Varid{→}\;\Conid{Set}\;\Keyword{where}{}\<[E]%
\\
\>[B]{}\hsindent{3}{}\<[3]%
\>[3]{}\text{\textasciigrave}\anonymous \;{}\<[8]%
\>[8]{}\mathbin{:}\;\Conid{Fin}\;\Varid{n}\;{}\<[33]%
\>[33]{}\Varid{→}\;\Conid{Raw}\;\Varid{n}{}\<[E]%
\\
\>[B]{}\hsindent{3}{}\<[3]%
\>[3]{}\Varid{\char95 ∋\char95 }\;{}\<[8]%
\>[8]{}\mathbin{:}\;\Conid{Ty}\;{}\<[23]%
\>[23]{}\Varid{→}\;\Conid{Raw}\;\Varid{n}\;{}\<[33]%
\>[33]{}\Varid{→}\;\Conid{Raw}\;\Varid{n}{}\<[E]%
\\
\>[B]{}\hsindent{3}{}\<[3]%
\>[3]{}\Varid{op}\;{}\<[8]%
\>[8]{}\mathbin{:}\;\Varid{⟦}\;\Conid{D}\;\Varid{⟧}\;\Conid{Raw}\;\Varid{n}\;{}\<[33]%
\>[33]{}\Varid{→}\;\Conid{Raw}\;\Varid{n}{}\<[E]%
\ColumnHook
\end{hscode}\resethooks
\end{figure}
and mirrors our informal definition (\Cref{fig:raw-terms}) where \ensuremath{\Conid{A}\;\Varid{∋}\;\Varid{t}} corresponds to the raw term $t \annotate A$.

\subsection{Defining Extrinsic Bidirectional Typing Derivations}
\begin{figure}[H]
  \small
% bidirectionally typed terms
%Fam : (ℓ′ : Level) (X : R.Fam ℓ) → Set (ℓ ⊔ lsuc ℓ′)
%Fam ℓ′ X = (Γ : Cxt 0) → X (length Γ) → Mode → Ty → Set ℓ′
%
\begin{hscode}\SaveRestoreHook
\column{B}{@{}>{\hspre}l<{\hspost}@{}}%
\column{3}{@{}>{\hspre}l<{\hspost}@{}}%
\column{7}{@{}>{\hspre}l<{\hspost}@{}}%
\column{8}{@{}>{\hspre}l<{\hspost}@{}}%
\column{10}{@{}>{\hspre}l<{\hspost}@{}}%
\column{20}{@{}>{\hspre}l<{\hspost}@{}}%
\column{24}{@{}>{\hspre}l<{\hspost}@{}}%
\column{27}{@{}>{\hspre}l<{\hspost}@{}}%
\column{30}{@{}>{\hspre}l<{\hspost}@{}}%
\column{33}{@{}>{\hspre}l<{\hspost}@{}}%
\column{36}{@{}>{\hspre}l<{\hspost}@{}}%
\column{46}{@{}>{\hspre}l<{\hspost}@{}}%
\column{E}{@{}>{\hspre}l<{\hspost}@{}}%
\>[B]{}\Varid{⟦\char95 ⟧ᵃ}\;{}\<[7]%
\>[7]{}\mathbin{:}\;(\Conid{Δ}\;\mathbin{:}\;\Conid{TExps}\;\Conid{Ξ})\;(\Conid{X}\;\mathbin{:}\;\Conid{R.Fam})\;(\Conid{Y}\;\mathbin{:}\;(\Conid{Γ}\;\mathbin{:}\;\Conid{Cxt}\;\Varid{0})\;\Varid{→}\;\Conid{X}\;(\Varid{length}\;\Conid{Γ})\;\Varid{→}\;\Conid{Set})\;{}\<[E]%
\\
\>[7]{}\Varid{→}\;\Conid{TSub}\;\Conid{Ξ}\;\Varid{0}\;\Varid{→}\;(\Conid{Γ}\;\mathbin{:}\;\Conid{Cxt}\;\Varid{0})\;\Varid{→}\;\Conid{R.⟦}\;\Conid{Δ}\;\Varid{⟧ᵃ}\;\Conid{X}\;(\Varid{length}\;\Conid{Γ})\;\Varid{→}\;\Conid{Set}{}\<[E]%
\\
\>[B]{}\Varid{⟦}\;\Varid{[]}\;{}\<[10]%
\>[10]{}\Varid{⟧ᵃ}\;\Conid{X}\;\Conid{Y}\;\Varid{σ}\;\Conid{Γ}\;\Varid{t}\;{}\<[24]%
\>[24]{}\mathrel{=}\;\Conid{Y}\;\Conid{Γ}\;\Varid{t}{}\<[E]%
\\
\>[B]{}\Varid{⟦}\;\Conid{A}\;\Varid{∷}\;\Conid{Δ}\;{}\<[10]%
\>[10]{}\Varid{⟧ᵃ}\;\Conid{X}\;\Conid{Y}\;\Varid{σ}\;\Conid{Γ}\;\Varid{t}\;{}\<[24]%
\>[24]{}\mathrel{=}\;\Varid{⟦}\;\Conid{Δ}\;\Varid{⟧ᵃ}\;\Conid{X}\;\Conid{Y}\;\Varid{σ}\;((\Conid{A}\;\Varid{⟨}\;\Varid{σ}\;\Varid{⟩})\;\Varid{∷}\;\Conid{Γ})\;\Varid{t}{}\<[E]%
\\[\blanklineskip]%
\>[B]{}\Varid{⟦\char95 ⟧ᵃˢ}\;{}\<[8]%
\>[8]{}\mathbin{:}\;(\Conid{D}\;\mathbin{:}\;\Conid{ArgsD}\;\Conid{Ξ})\;(\Conid{X}\;\mathbin{:}\;\Conid{R.Fam})\;(\Conid{Y}\;\mathbin{:}\;\Conid{Fam}\;\Conid{X})\;{}\<[E]%
\\
\>[8]{}\Varid{→}\;\Conid{TSub}\;\Conid{Ξ}\;\Varid{0}\;\Varid{→}\;(\Conid{Γ}\;\mathbin{:}\;\Conid{Cxt}\;\Varid{0})\;\Varid{→}\;\Conid{R.⟦}\;\Varid{eraseᵃˢ}\;\Conid{D}\;\Varid{⟧ᵃˢ}\;\Conid{X}\;(\Varid{length}\;\Conid{Γ})\;\Varid{→}\;\Conid{Set}{}\<[E]%
\\
\>[B]{}\Varid{⟦}\;\Varid{[]}\;{}\<[20]%
\>[20]{}\Varid{⟧ᵃˢ}\;\anonymous \;{}\<[27]%
\>[27]{}\anonymous \;{}\<[30]%
\>[30]{}\anonymous \;{}\<[33]%
\>[33]{}\anonymous \;{}\<[36]%
\>[36]{}\anonymous \;{}\<[46]%
\>[46]{}\mathrel{=}\;\Varid{⊤}{}\<[E]%
\\
\>[B]{}\Varid{⟦}\;\Conid{Δ}\;\Varid{⊢[}\;\Varid{d}\;\mskip1.5mu]\;\Conid{A}\;\Varid{∷}\;\Conid{Ds}\;{}\<[20]%
\>[20]{}\Varid{⟧ᵃˢ}\;\Conid{X}\;{}\<[27]%
\>[27]{}\Conid{Y}\;{}\<[30]%
\>[30]{}\Varid{σ}\;{}\<[33]%
\>[33]{}\Conid{Γ}\;{}\<[36]%
\>[36]{}(\Varid{x}\;\Varid{,}\;\Varid{xs})\;{}\<[46]%
\>[46]{}\mathrel{=}\;{}\<[E]%
\\
\>[B]{}\hsindent{3}{}\<[3]%
\>[3]{}\Varid{⟦}\;\Conid{Δ}\;\Varid{⟧ᵃ}\;\Conid{X}\;(\Varid{λ}\;\iden{\Conid{Γ}^\prime}\;\iden{\Varid{x}^\prime}\;\Varid{→}\;\Conid{Y}\;\iden{\Conid{Γ}^\prime}\;\iden{\Varid{x}^\prime}\;\Varid{d}\;(\Conid{A}\;\Varid{⟨}\;\Varid{σ}\;\Varid{⟩}))\;\Varid{σ}\;\Conid{Γ}\;\Varid{x}\;\Varid{×}\;\Varid{⟦}\;\Conid{Ds}\;\Varid{⟧ᵃˢ}\;\Conid{X}\;\Conid{Y}\;\Varid{σ}\;\Conid{Γ}\;\Varid{xs}{}\<[E]%
\\[\blanklineskip]%
\>[B]{}\Varid{⟦\char95 ⟧ᶜ}\;\mathbin{:}\;(\Conid{D}\;\mathbin{:}\;\Conid{ConD})\;(\Conid{X}\;\mathbin{:}\;\Conid{R.Fam})\;(\Conid{Y}\;\mathbin{:}\;\Conid{Fam}\;\Conid{X})\;\Varid{→}\;\Conid{Fam}\;(\Conid{R.⟦}\;\Varid{eraseᶜ}\;\Conid{D}\;\Varid{⟧ᶜ}\;\Conid{X}){}\<[E]%
\\
\>[B]{}\Varid{⟦}\;\Varid{ι}\;\{\mskip1.5mu \Conid{Ξ}\mskip1.5mu\}\;\Varid{d}\;\Conid{B}\;\Conid{D}\;\Varid{⟧ᶜ}\;\Conid{X}\;\Conid{Y}\;\Conid{Γ}\;\Varid{xs}\;\Varid{d′}\;\Conid{A}\;\mathrel{=}\;{}\<[E]%
\\
\>[B]{}\hsindent{3}{}\<[3]%
\>[3]{}\Varid{d}\;\Varid{≡}\;\Varid{d′}\;\Varid{×}\;\Conid{Σ[}\;\Varid{σ}\;\Varid{∈}\;\Conid{TSub}\;\Conid{Ξ}\;\Varid{0}\;\mskip1.5mu]\;\Conid{B}\;\Varid{⟨}\;\Varid{σ}\;\Varid{⟩}\;\Varid{≡}\;\Conid{A}\;\Varid{×}\;\Varid{⟦}\;\Conid{D}\;\Varid{⟧ᵃˢ}\;\Conid{X}\;\Conid{Y}\;\Varid{σ}\;\Conid{Γ}\;\Varid{xs}{}\<[E]%
\\[\blanklineskip]%
\>[B]{}\Varid{⟦\char95 ⟧}\;\mathbin{:}\;(\Conid{D}\;\mathbin{:}\;\Conid{Desc})\;(\Conid{X}\;\mathbin{:}\;\Conid{R.Fam})\;(\Conid{Y}\;\mathbin{:}\;\Conid{Fam}\;\Conid{X})\;\Varid{→}\;\Conid{Fam}\;(\Conid{R.⟦}\;\Varid{erase}\;\Conid{D}\;\Varid{⟧}\;\Conid{X}){}\<[E]%
\\
\>[B]{}\Varid{⟦}\;\Conid{D}\;\Varid{⟧}\;\Conid{X}\;\Conid{Y}\;\Conid{Γ}\;(\Varid{i}\;\Varid{,}\;\Varid{xs})\;\Varid{d}\;\Conid{A}\;\mathrel{=}\;\Varid{⟦}\;\Conid{D}\;\Varid{.rules}\;\Varid{i}\;\Varid{⟧ᶜ}\;\Conid{X}\;\Conid{Y}\;\Conid{Γ}\;\Varid{xs}\;\Varid{d}\;\Conid{A}{}\<[E]%
\ColumnHook
\end{hscode}\resethooks
\end{figure}
%
\begin{figure}[H]
  \small 
\begin{minipage}[t]{.48\textwidth}
%
%  _⊢_⇐_ _⊢_⇒_
%    : (Γ : Cxt 0) → Raw (length Γ) → Ty → Set
%  Γ ⊢ r ⇐ A  = Γ ⊢ r [ Chk ] A
%  Γ ⊢ r ⇒ A  = Γ ⊢ r [ Syn ] A
%
\begin{hscode}\SaveRestoreHook
\column{B}{@{}>{\hspre}l<{\hspost}@{}}%
\column{3}{@{}>{\hspre}l<{\hspost}@{}}%
\column{5}{@{}>{\hspre}l<{\hspost}@{}}%
\column{9}{@{}>{\hspre}l<{\hspost}@{}}%
\column{E}{@{}>{\hspre}l<{\hspost}@{}}%
\>[B]{}\Keyword{mutual}{}\<[E]%
\\
\>[B]{}\hsindent{3}{}\<[3]%
\>[3]{}\Keyword{data}\;\Varid{\char95 ⊢\char95 [\char95 ]\char95 }\;\mathbin{:}\;\Conid{Fam₀}\;\Conid{Raw}\;\Keyword{where}{}\<[E]%
\\[\blanklineskip]%
\>[3]{}\hsindent{2}{}\<[5]%
\>[5]{}\Varid{var}\;\mathbin{:}\;(\Varid{i}\;\mathbin{:}\;\Conid{A}\;\Varid{∈}\;\Conid{Γ})\;{}\<[E]%
\\
\>[5]{}\hsindent{4}{}\<[9]%
\>[9]{}\Varid{→}\;\Conid{L.index}\;\Varid{i}\;\Varid{≡}\;\Varid{j}\;{}\<[E]%
\\
\>[5]{}\hsindent{4}{}\<[9]%
\>[9]{}\Varid{→}\;\Conid{Γ}\;\Varid{⊢}\;(\text{\textasciigrave}\;\Varid{j})\;\Varid{⇒}\;\Conid{A}{}\<[E]%
\\[\blanklineskip]%
\>[3]{}\hsindent{2}{}\<[5]%
\>[5]{}\Varid{\char95 ∋\char95 }\;\mathbin{:}\;(\Conid{A}\;\mathbin{:}\;\Conid{Ty})\;{}\<[E]%
\\
\>[5]{}\hsindent{4}{}\<[9]%
\>[9]{}\Varid{→}\;\Conid{Γ}\;\Varid{⊢}\;\Varid{r}\;\Varid{⇐}\;\Conid{A}\;{}\<[E]%
\\
\>[5]{}\hsindent{4}{}\<[9]%
\>[9]{}\Varid{→}\;\Conid{Γ}\;\Varid{⊢}\;(\Conid{A}\;\Varid{∋}\;\Varid{r})\;\Varid{⇒}\;\Conid{A}{}\<[E]%
\ColumnHook
\end{hscode}\resethooks
\end{minipage}
\begin{minipage}[t]{.48\textwidth}
\begin{hscode}\SaveRestoreHook
\column{B}{@{}>{\hspre}l<{\hspost}@{}}%
\column{5}{@{}>{\hspre}l<{\hspost}@{}}%
\column{9}{@{}>{\hspre}l<{\hspost}@{}}%
\column{E}{@{}>{\hspre}l<{\hspost}@{}}%
\>[5]{}\Varid{\char95 ↑\char95 }\;\mathbin{:}\;\Conid{Γ}\;\Varid{⊢}\;\Varid{r}\;\Varid{⇒}\;\Conid{B}\;{}\<[E]%
\\
\>[5]{}\hsindent{4}{}\<[9]%
\>[9]{}\Varid{→}\;\Conid{A}\;\Varid{≡}\;\Conid{B}\;{}\<[E]%
\\
\>[5]{}\hsindent{4}{}\<[9]%
\>[9]{}\Varid{→}\;\Conid{Γ}\;\Varid{⊢}\;\Varid{r}\;\Varid{⇐}\;\Conid{A}{}\<[E]%
\\[\blanklineskip]%
\>[5]{}\Varid{op}\;{}\<[9]%
\>[9]{}\mathbin{:}\;\Varid{⟦}\;\Conid{D}\;\Varid{⟧}\;\Conid{Raw}\;\Varid{\char95 ⊢\char95 [\char95 ]\char95 }\;\Conid{Γ}\;\Varid{rs}\;\Varid{d}\;\Conid{A}\;{}\<[E]%
\\
\>[9]{}\Varid{→}\;\Conid{Γ}\;\Varid{⊢}\;\Varid{op}\;\Varid{rs}\;[\mskip1.5mu \;\Varid{d}\;\mskip1.5mu]\;\Conid{A}{}\<[E]%
\ColumnHook
\end{hscode}\resethooks
  
\end{minipage}
\end{figure}
\subsection{Formal Proofs: Soundness as an Example}\label{subsec:formal-proofs}

\begin{figure}[H]
  \small
% soundness proof
\begin{minipage}{.48\textwidth}
\begin{hscode}\SaveRestoreHook
\column{B}{@{}>{\hspre}l<{\hspost}@{}}%
\column{3}{@{}>{\hspre}l<{\hspost}@{}}%
\column{5}{@{}>{\hspre}l<{\hspost}@{}}%
\column{26}{@{}>{\hspre}l<{\hspost}@{}}%
\column{30}{@{}>{\hspre}l<{\hspost}@{}}%
\column{33}{@{}>{\hspre}l<{\hspost}@{}}%
\column{46}{@{}>{\hspre}l<{\hspost}@{}}%
\column{E}{@{}>{\hspre}l<{\hspost}@{}}%
\>[B]{}\Keyword{mutual}{}\<[E]%
\\[\blanklineskip]%
\>[B]{}\hsindent{3}{}\<[3]%
\>[3]{}\Varid{soundness}\;\mathbin{:}\;\Conid{Γ}\;\Varid{⊢}\;\Varid{r}\;[\mskip1.5mu \;\Varid{d}\;\mskip1.5mu]\;\Conid{A}\;{}\<[30]%
\>[30]{}\Varid{→}\;{}\<[33]%
\>[33]{}\Conid{Γ}\;\Varid{⊢}\;\Varid{r}\;\Varid{⦂}\;\Conid{A}{}\<[E]%
\\
\>[B]{}\hsindent{3}{}\<[3]%
\>[3]{}\Varid{soundness}\;(\Varid{var}\;\Varid{i}\;\Varid{eq})\;{}\<[26]%
\>[26]{}\mathrel{=}\;\Varid{var}\;\Varid{i}\;\Varid{eq}{}\<[E]%
\\
\>[B]{}\hsindent{3}{}\<[3]%
\>[3]{}\Varid{soundness}\;(\Conid{A}\;\Varid{∋}\;\Varid{t})\;{}\<[26]%
\>[26]{}\mathrel{=}\;\Conid{A}\;\Varid{∋}\;\Varid{soundness}\;\Varid{t}{}\<[E]%
\\
\>[B]{}\hsindent{3}{}\<[3]%
\>[3]{}\Varid{soundness}\;(\Varid{t}\;\Varid{↑}\;\Varid{refl})\;{}\<[26]%
\>[26]{}\mathrel{=}\;\Varid{soundness}\;\Varid{t}{}\<[E]%
\\
\>[B]{}\hsindent{3}{}\<[3]%
\>[3]{}\Varid{soundness}\;(\Varid{op}\;\Varid{ts})\;{}\<[26]%
\>[26]{}\mathrel{=}\;{}\<[E]%
\\
\>[3]{}\hsindent{2}{}\<[5]%
\>[5]{}\Varid{op}\;(\Varid{soundnessᶜ}\;(\Conid{BD}\;\Varid{.rules}\;\anonymous )\;\Varid{ts}){}\<[E]%
\\[\blanklineskip]%
\>[B]{}\hsindent{3}{}\<[3]%
\>[3]{}\Varid{soundnessᶜ}\;{}\<[E]%
\\
\>[3]{}\hsindent{2}{}\<[5]%
\>[5]{}\mathbin{:}\;(\Conid{D}\;\mathbin{:}\;\Conid{ConD})\;{}\<[E]%
\\
\>[3]{}\hsindent{2}{}\<[5]%
\>[5]{}\Varid{→}\;\Varid{⟦}\;\Conid{D}\;\Varid{⟧ᶜ}\;\Conid{Raw}\;\Varid{\char95 ⊢\char95 [\char95 ]\char95 }\;\Conid{Γ}\;\Varid{rs}\;\Varid{d}\;\Conid{A}\;{}\<[E]%
\\
\>[3]{}\hsindent{2}{}\<[5]%
\>[5]{}\Varid{→}\;\Conid{T.⟦}\;\Varid{eraseᶜ}\;\Conid{D}\;\Varid{⟧ᶜ}\;\Conid{Raw}\;\Varid{\char95 ⊢\char95 ⦂\char95 }\;\Conid{Γ}\;\Varid{rs}\;\Conid{A}{}\<[E]%
\\
\>[B]{}\hsindent{3}{}\<[3]%
\>[3]{}\Varid{soundnessᶜ}\;(\Varid{ι}\;\anonymous \;\anonymous \;\Conid{Ds})\;(\anonymous \;\Varid{,}\;\Varid{σ}\;\Varid{,}\;\Varid{σ-eq}\;\Varid{,}\;\Varid{ts})\;{}\<[46]%
\>[46]{}\mathrel{=}\;{}\<[E]%
\\
\>[3]{}\hsindent{2}{}\<[5]%
\>[5]{}\Varid{σ}\;\Varid{,}\;\Varid{σ-eq}\;\Varid{,}\;\Varid{soundnessᵃˢ}\;\Conid{Ds}\;\Varid{ts}{}\<[E]%
\ColumnHook
\end{hscode}\resethooks
\end{minipage}
\begin{minipage}{.5\textwidth}
\begin{hscode}\SaveRestoreHook
\column{B}{@{}>{\hspre}l<{\hspost}@{}}%
\column{3}{@{}>{\hspre}l<{\hspost}@{}}%
\column{5}{@{}>{\hspre}l<{\hspost}@{}}%
\column{17}{@{}>{\hspre}l<{\hspost}@{}}%
\column{23}{@{}>{\hspre}l<{\hspost}@{}}%
\column{26}{@{}>{\hspre}l<{\hspost}@{}}%
\column{27}{@{}>{\hspre}l<{\hspost}@{}}%
\column{35}{@{}>{\hspre}l<{\hspost}@{}}%
\column{40}{@{}>{\hspre}l<{\hspost}@{}}%
\column{41}{@{}>{\hspre}l<{\hspost}@{}}%
\column{45}{@{}>{\hspre}l<{\hspost}@{}}%
\column{50}{@{}>{\hspre}l<{\hspost}@{}}%
\column{E}{@{}>{\hspre}l<{\hspost}@{}}%
\>[3]{}\Varid{soundnessᵃˢ}\;{}\<[E]%
\\
\>[3]{}\hsindent{2}{}\<[5]%
\>[5]{}\mathbin{:}\;(\Conid{Ds}\;\mathbin{:}\;\Conid{ArgsD}\;\Conid{Ξ})\;{}\<[E]%
\\
\>[3]{}\hsindent{2}{}\<[5]%
\>[5]{}\Varid{→}\;\Varid{⟦}\;\Conid{Ds}\;\Varid{⟧ᵃˢ}\;{}\<[27]%
\>[27]{}\Conid{Raw}\;\Varid{\char95 ⊢\char95 [\char95 ]\char95 }\;{}\<[40]%
\>[40]{}\Varid{σ}\;\Conid{Γ}\;\Varid{rs}\;{}\<[E]%
\\
\>[3]{}\hsindent{2}{}\<[5]%
\>[5]{}\Varid{→}\;\Conid{T.⟦}\;\Varid{eraseᵃˢ}\;\Conid{Ds}\;\Varid{⟧ᵃˢ}\;{}\<[27]%
\>[27]{}\Conid{Raw}\;\Varid{\char95 ⊢\char95 ⦂\char95 }\;{}\<[40]%
\>[40]{}\Varid{σ}\;\Conid{Γ}\;\Varid{rs}{}\<[E]%
\\
\>[3]{}\Varid{soundnessᵃˢ}\;\Varid{[]}\;{}\<[35]%
\>[35]{}\anonymous \;{}\<[45]%
\>[45]{}\mathrel{=}\;{}\<[E]%
\\
\>[3]{}\hsindent{2}{}\<[5]%
\>[5]{}\Varid{tt}{}\<[E]%
\\
\>[3]{}\Varid{soundnessᵃˢ}\;((\Conid{Δ}\;\Varid{⊢[}\;\anonymous \;\mskip1.5mu]\;\anonymous )\;\Varid{∷}\;\Conid{Ds})\;(\Varid{t}\;\Varid{,}\;\Varid{ts})\;{}\<[45]%
\>[45]{}\mathrel{=}\;{}\<[E]%
\\
\>[3]{}\hsindent{2}{}\<[5]%
\>[5]{}\Varid{soundnessᵃ}\;\Conid{Δ}\;\Varid{t}\;\Varid{,}\;\Varid{soundnessᵃˢ}\;\Conid{Ds}\;\Varid{ts}{}\<[E]%
\\[\blanklineskip]%
\>[3]{}\Varid{soundnessᵃ}\;{}\<[E]%
\\
\>[3]{}\hsindent{2}{}\<[5]%
\>[5]{}\mathbin{:}\;(\Conid{Δ}\;\mathbin{:}\;\Conid{TExps}\;\Conid{Ξ})\;{}\<[E]%
\\
\>[3]{}\hsindent{2}{}\<[5]%
\>[5]{}\Varid{→}\;\Varid{⟦}\;\Conid{Δ}\;\Varid{⟧ᵃ}\;{}\<[17]%
\>[17]{}\Conid{Raw}\;(\Varid{λ}\;\iden{\Conid{Γ}^\prime}\;\iden{\Varid{r}^\prime}\;\Varid{→}\;\iden{\Conid{Γ}^\prime}\;\Varid{⊢}\;\iden{\Varid{r}^\prime}\;[\mskip1.5mu \;\Varid{d}\;\mskip1.5mu]\;\Conid{A})\;{}\<[50]%
\>[50]{}\Varid{σ}\;\Conid{Γ}\;\Varid{r}\;{}\<[E]%
\\
\>[3]{}\hsindent{2}{}\<[5]%
\>[5]{}\Varid{→}\;\Conid{T.⟦}\;\Conid{Δ}\;\Varid{⟧ᵃ}\;{}\<[17]%
\>[17]{}\Conid{Raw}\;(\Varid{λ}\;\iden{\Conid{Γ}^\prime}\;\iden{\Varid{r}^\prime}\;\Varid{→}\;\iden{\Conid{Γ}^\prime}\;\Varid{⊢}\;\iden{\Varid{r}^\prime}\;\Varid{⦂}\;\Conid{A})\;{}\<[50]%
\>[50]{}\Varid{σ}\;\Conid{Γ}\;\Varid{r}{}\<[E]%
\\
\>[3]{}\Varid{soundnessᵃ}\;\Varid{[]}\;{}\<[23]%
\>[23]{}\Varid{t}\;{}\<[26]%
\>[26]{}\mathrel{=}\;\Varid{soundness}\;{}\<[41]%
\>[41]{}\Varid{t}{}\<[E]%
\\
\>[3]{}\Varid{soundnessᵃ}\;(\anonymous \;\Varid{∷}\;\Conid{Δ})\;{}\<[23]%
\>[23]{}\Varid{t}\;{}\<[26]%
\>[26]{}\mathrel{=}\;\Varid{soundnessᵃ}\;\Conid{Δ}\;\Varid{t}{}\<[E]%
\ColumnHook
\end{hscode}\resethooks
\end{minipage}

\end{figure}

\end{document}
